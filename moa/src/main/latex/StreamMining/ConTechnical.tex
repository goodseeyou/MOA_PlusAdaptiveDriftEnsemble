% Technical.tex
% Description of setting, ADWIN, and ADWIN2
%
%%%%%%%%%%%%%%%%%%%%%%%%%%%%%%%%%%%%%%%%%%%%%%%%%%%%%%%

\section{Maintaining Updated Windows of Varying Length}
\label{SMain}
In %\cite{bif-gav}
this section we describe algorithms 
for dynamically adjusting the length
of a data window, make a formal claim about its performance, and 
derive an efficient variation.
%evaluate it experimentally to confirm the theoretical predictions. 

We will use Hoeffding's bound in order to obtain formal guarantees, and %to be able to design 
a streaming algorithm.
However, other tests computing differences between window distributions may be used.
 

\BEGINOMIT
Machine learning algorithms that detect change, usually compare statistics of two windows.
There have been in the literature, some different window management strategies:
\begin{itemize}
\item Equal \& fixed size subwindows: Kifer et al.~\cite{kifer-detecting} compares one window of older data with a window with the same size keeping the most recent data.
\item Equal size adjacent subwindows: Dasu et al.~\cite{Dasu} compares two adjacent windows of the same size of recent data.
\item Total window against subwindow: Gama et al.~\cite{Gama} compares the window that contains all the data with a subwindow of data from the beginning until it detects that the accuracy of the algorithm decreases.
\end{itemize}

Figure~\ref{Fig:wms} shows these strategies.
Our proposal is to compare all the adjacent subwindows in which is possible to partition the window containing all the data.
It seems that this procedure may be the most accurate, since it looks at all possible subwindows partitions. On the other hand, time cost is the main disadvantage of this method. Considering this, we provided another version working in the strict conditions of the Data Stream model, namely low
memory and low processing per item.

\begin{figure}[ht]
Let $W= \fbox{101010110111111}$
\begin{itemize}
\item {Equal \& fixed size subwindows:} $ \fbox{1010}1011011\fbox{1111}$
%{\sl D. Kifer, S. Ben-David, and J. Gehrke}. Detecting change in data streams. 2004
%\hline

\item {Equal size adjacent subwindows:} $ 1010101\fbox{1011}\fbox{1111}$ 
%{\sl Dasu et al.}
%\hline

\item { Total window against subwindow:}
$ \fbox{\fbox{10101011011}1111}$
%{\sl J. Gama, P. Medas, G. Castillo, and P. Rodrigues.} Learning with drift detection. 2004

%\hline
%\end{columns}

\item {{\tt ADWIN:}  All adjacent subwindows:}
\begin{eqnarray*}
\fbox{1} \fbox{01010110111111} \\
\fbox{1010} \fbox{10110111111} \\
\fbox{1010101} \fbox{10111111} \\
\fbox{1010101101} \fbox{11111} \\
\fbox{10101011011111} \fbox{1} \\
\end{eqnarray*}

\end{itemize}
\caption{Different window management strategies}
\label{Fig:wms}
\end{figure}
\ENDOMIT

%%%%%%%%%%%%%%%%%%%%%%%%%%%%%%%%%%%%%%%%%%%%%%%%%%%%%%%

\subsection{Setting}%{Description}

The inputs to the algorithms are a confidence value $\delta\in (0,1)$ 
and a (possibly infinite) sequence of real values 
$x_1$, $x_2$, $x_3$, \dots, $x_t$, \dots{} 
The value of $x_t$ is available only at time $t$.
Each $x_t$ is generated according to some distribution~$D_t$, 
independently for every $t$. 
We denote with $\mu_t$ and $\sigma^2_t$ 
the expected value and the variance of  $x_t$ when it is drawn according to $D_t$. 
We assume that $x_t$ is always in $[0,1]$; by an easy
rescaling, we can handle any case in which we know an interval 
$[a,b]$ such that $a \le x_t \le b$ with probability $1$. 
Nothing else is known about the sequence of 
distributions $D_t$; in particular, 
$\mu_t$ and $\sigma_t^2$ are unknown for all $t$. 

\def\adwinz{{\tt ADWIN0 }}
\def\adwinzz{{\tt ADWIN0}}
\def\adwintwo{{\tt ADWIN }}
\def\adwintwoz{{\tt ADWIN}}
\subsection{First algorithm: \adwinz}
\label{Adwin0}
The first algorithm keeps a sliding window $W$ with the most recently read 
$x_i$. Let~$n$ denote the length of~$W$, 
$\hmu_W$ the (observed) average of the elements in~$W$, 
and~$\mu_W$ the (unknown) average of~$\mu_t$ for~$t\in W$. 
Strictly speaking, these quantities should be indexed by~$t$, 
but in general~$t$ will be clear from the context. 

Since the values of $\mu_t$ can oscillate wildly, 
there is no guarantee that~$\mu_{W}$ or~$\hmu_W$ will be anywhere close 
to the instantaneous value~$\mu_t$, even for long~$W$.  
However, $\mu_{W}$ is the expected value of $\hmu_{W}$, so 
$\mu_{W}$ and $\hmu_{W}$ {\em do} get close as~$W$ grows. 

Algorithm \adwinz is presented in Figure \ref{blnAlgorithm}. 
The idea is simple: whenever two ``large enough'' 
subwindows of $W$ exhibit ``distinct enough'' averages, 
one can conclude that the corresponding expected values
are different, and the older portion of the window is dropped.
In other words, $W$ is kept as long as possible
while the null hypothesis ``$\mu_t$ has remained
constant in $W$'' is sustainable up to confidence $\delta$.%
\footnote{It would easy to use instead the null hypothesis 
``there has been no change greater than $\epsilon$'', 
for a user-specified~$\epsilon$ expressing the smallest change 
that deserves reaction.}
``Large enough'' and ``distinct enough'' above are made precise
by choosing an appropriate statistical test for distribution change, 
which in general involves 
the value of $\delta$, the lengths of the subwindows, 
and their contents. We choose one particular statistical
test for our implementation, but this is not the essence 
of our proposal -- many other tests could be used. 
At every step, \adwinz simply outputs the value of $\hmu_W$
as an approximation to $\mu_W$. 
%It could additionally output some uncertainty margin for 
%$\hmu_W$, which would be useful in many applications, but we
%do not develop this possibility.  
%and a value $\eps_t$ which is a uncertainty margin for $\hmu_W$.  
%This margin, which is useful to know in many applications, 
%will be computed (and argued about) independently of $\epsc$. 

The value of $\epsc$ for a partition $W_0\cdot W_1$ of $W$ 
is computed as follows: Let $n_0$ and $n_1$ be the lengths
of $W_0$ and $W_1$ and $n$ be the length of $W$, so $n=n_0+n_1$. 
Let $\hmu_{W_0}$ and $\hmu_{W_1}$ be the averages of the values
in $W_0$ and $W_1$, and $\mu_{W_0}$ and $\mu_{W_1}$
their expected values. 
To obtain totally rigorous performance guarantees we define:
%
\begin{eqnarray*}
m &=& \frac{1}{1/n_0 + 1/n_1} \mbox{ (harmonic mean of $n_0$ and $n_1$),} \\
\delta'&=&\frac{\delta}{n}, \mbox{ and } \quad
\epsc = \sqrt{\frac{1}{2m} \cdot \ln\frac{4}{\delta'}} \;.
\end{eqnarray*}
%
Our statistical test for different distributions in~$W_0$ and~$W_1$ 
simply checks whether the observed average in both subwindows differs 
by more than the threshold~$\epsc$.
The role of $\delta'$ is to avoid problems with multiple hypothesis
testing (since we will be testing $n$ different possibilities 
for~$W_0$ and~$W_1$ and we want global error below~$\delta$). 
Later we will provide a more sensitive test based on the normal approximation 
that, although not 100\% rigorous, is perfectly valid in practice. 

\begin{figure}

\centering

\begin{codebox}
\Procname{$\proc{\adwinzz: Adaptive Windowing Algorithm}$}
\li Initialize Window $W$
\li \For each $t >0$ %
\li \Do $W \gets W \cup \{x_t\}$ (i.e., add $x_t$ to the head of $W$)
\li \Repeat Drop elements from the tail of $W$ 
\li \Until $|\hmu_{W_0}-\hmu_{W_1}|  < \epsc$ holds %\geq
\li \quad\quad for every split of $W$ into $W=W_0 \cdot W_1$ 
\li output $\hmu_{W}$ % the pair $\hmu_{W}$ and $\eps_t$ (as in the Theorem)
\end{codebox}
\caption{Algorithm \adwinzz.}\label{blnAlgorithm}
\end{figure}

Now we state our main technical result about the performance of \adwinzz:

\begin{theorem}
\label{ThBV}
At every time step we have 

\begin{enumerate}
\item {\em (False positive rate bound).} If $\mu_t$ remains constant within $W$, 
the probability that \adwinz shrinks the window 
at this step is at most $\delta$.

\item {\em (False negative rate bound).} 
Suppose that for {\em some} partition of $W$ in two parts $W_0W_1$ 
(where $W_1$ contains the most recent items) 
we have $|\mu_{W_0}-\mu_{W_1}| > 2\epsc$. 
Then with probability $1-\delta$ \adwinz
shrinks $W$ to $W_1$, or shorter.
\end{enumerate}
\end{theorem}

\noindent
%The proof of the theorem is given in Appendix A. 
%\onecolumn
%\appendix 
%\section{Appendix: Proof of Theorem 1}

\begin{proof}

{\bf{Part 1)}}
Assume $\muz=\muu=\mu_W$ as null hypothesis. We show that for any partition
$W$ as $W_0W_1$ we have probability at most $\delta/n$ that \adwinz
decides to shrink $W$ to $W_1$, or equivalently,
%
\begin{eqnarray*}
%\label{Edecomposeprobs0}
\Pr [\, |\hmuu - \hmuz| \ge  \epsc\, ] \le \delta/n.
\end{eqnarray*}
%
Since there are at most $n$ partitions $W_0W_1$, the claim follows by the union bound. 
Note that, for every real number $k\in (0,1)$, $|\hmuu - \hmuz|\ge  \epsc$ 
can be decomposed as
\begin{eqnarray*}
\label{Edecomposeprobs1}
\Pr [\, |\hmuu - \hmuz| \ge  \epsc \,]
&\le& \Pr [\, |\hmuu - \mu_W| \ge  k \epsc \,] + \Pr[\, |\mu_W - \hmuz| \ge  (1-k)\epsc )\,].
\end{eqnarray*} 
Applying the Hoeffding bound, we have then 
\begin{eqnarray*}
\label{Edecomposeprobs2}
\Pr [\, |\hmuu - \hmuz| \ge  \epsc \,]
&\le& 2\exp(-2 (k\,\epsc)^2 \,n_0) + 2\exp(-2 ((1-k)\,\epsc)^2 \,n_1)
\end{eqnarray*} 
To approximately minimize the sum, we choose the value of $k$ 
that makes both probabilities equal, i.e.\ such that 
$$
(k \,\epsc)^2 \,n_0 = ((1-k)\,\epsc)^2 \,n_1. 
$$
which is $k=\sqrt{n_1 / n_0} / (1+\sqrt{n_1 / n_0})$. 
For this $k$, we have precisely 
$$
(k \,\epsc)^2 \, n_0 
      = \frac{n_1 n_0}{(\sqrt{n_0}+\sqrt{n_1})^2}\; \epsc^2 
      \le \frac{n_1 n_0}{(n_0+n_1)} \; \epsc^2
      = m \; \epsc^2. 
$$
Therefore, in order to have 
$$
\Pr [\, |\hmuu - \hmuz| \ge  \epsc \,] \le \frac{\delta}{n}
$$
it suffices to have 
$$
4 \exp(-2m \; \epsc^2) \le \frac{\delta}{n}
$$
which is satisfied by 
$$
\epsc = \sqrt{\frac{1}{2m} \; \ln\frac{4n}{\delta}}\;.
$$
%%%%%%%%%%%%%%%%%%%%%%%%%%%%%%%%%%%%%%%%%%%%%%%%%%%%%%%
{\bf{Part 2)}}
Now assume $|\muz - \muu| > %\ge 
 2\epsc$. We want to show that
$\Pr [\, |\hmuu - \hmuz| \le  \epsc \,] \le \delta$, which means that 
with probability at least $1-\delta$ change is detected and 
the algorithm cuts $W$ to $W_1$. 
As before, for any $k\in (0,1)$, we can decompose $|\hmuz - \hmuu| \le  \epsc $ as
%
\begin{eqnarray*}
\Pr [\, |\hmuz - \hmuu| \le  \epsc \,] &\le& 
\Pr [\, (|\hmuz - \muz| \ge  k \epsc ) \cup (|\hmuu - \muu| \ge  (1-k)\epsc )\,]  \\   
&\le&\Pr [\, |\hmuz - \muz| \ge  k \epsc \,] +  
\Pr [\, |\hmuu - \muu| \ge  (1-k) \epsc \,]. 
\end{eqnarray*}
%
To see the first inequality, observe that if $|\hmuz - \hmuu| \le  \epsc$, 
$|\hmuz - \muz| \le  k \epsc$, 
and $|\hmuu - \muu| \le  (1-k) \epsc$
hold, by the triangle inequality we have 
$$
|\muz - \muu| \le |\hmuz + k\epsc - \hmuu + (1-k)\epsc| 
\le |\hmuz-\hmuu| + \epsc \le 2\epsc,$$
contradicting the hypothesis. 
Using the Hoeffding bound, we have then 
%
\begin{eqnarray*}
\Pr [\, |\hmuz - \hmuu| \ge  \epsc \,]   &\le&
2 \, \exp(-2(k\,\epsc)^2 \, n_0) + 2 \, \exp(-2((1-k)\,\epsc)^2 \, n_1). 
\end{eqnarray*} 
%
Now, choose $k$ as before to make both terms equal. By the calculations
in Part 1 we have 
%
\begin{eqnarray*}
\Pr [\, |\hmuz - \hmuu| \ge  \epsc \,]  &\le& 
4 \exp(-2 \, m \, \epsc^2) \le \frac{\delta}{n} \le \delta,
\end{eqnarray*} 
%
as desired. 

\end{proof}


%\hline

In practice, the definition of $\epsc$ as above is too conservative. 
Indeed, it is based on the Hoeffding bound, 
which is valid for all distributions but greatly overestimates the
probability of large deviations for distributions of small variance;
in fact, it is equivalent to assuming always the worst-case variance
$\sigma^2=1/4$. 
In practice, one can observe that 
$\mu_{W_0}-\mu_{W_1}$ tends to a normal distribution
for large window sizes, and use 
%
\begin{equation}
\label{Enewepsilon}
\epsc= \sqrt{\frac{2}{m} \cdot \sigma^2_W  \cdot \ln\frac{2}{\delta'} }
       \; + \; \frac{2}{3\,m} \, \ln\frac{2}{\delta'},
\end{equation}
%
where $\sigma^2_W$ is the observed variance of the elements in window~$W$. 
Thus, the term with the square root is essentially equivalent 
to setting~$\epsc$ to~$k$ times the standard deviation, for~$k$ depending
on the desired confidence~$\delta$, as is done in~\cite{Gama}. 
The extra additive term protects the
cases where the window sizes are too small to apply the normal approximation, 
as an alternative to the traditional use of requiring, say, 
sample size at least~$30$; 
it can be formally derived from the so-called Bernstein bound. 
Additionally, one (somewhat involved) argument shows that 
setting $\delta'=\delta/(\ln n)$ is enough in this context 
to protect from the multiple hypothesis testing problem; anyway, 
in the actual algorithm that we will run (\adwintwoz), 
only $O(\log n)$ subwindows are checked, 
which justifies using  $\delta'=\delta/(\ln n)$. %this value for~$\delta'$. 
Theorem \ref{ThBV} holds for this new value of $\epsc$, 
up to the error introduced by the normal approximation.
We have used these better bounds in all our implementations. 

\BEGINOMIT
One can justify, with a smaller degree of formality,
that the claims above should hold for all settings of the 
parameters that occur in practice if one sets the smaller values
$d=3+\ln(2/\delta)$ and 
$$
\epsc= \sqrt{\min\left\{{n \over 2 n_0 n_1},
             \frac{6 n \cdot (\min\{\hmu_W,1-\hmu_W\} + \theta)}{n_0 n_1} \right\}
             \cdot d}.
$$
in the definition of $\epsc$. 
We have used this variation in all our implementations.
\ENDOMIT 

%{\bf [THIS SECTION WILL BE EXTENDED AND FINISHED IN THE FINAL VERSION OF THIS THESIS]}

Let us consider how \adwinz behaves in two special cases: sudden (but infrequent)
changes, and slow gradual changes. Suppose that for a long time $\mu_t$ has remained
fixed at a value $\mu$, and that it suddenly jumps to a value $\mu'=\mu+\epsilon$. 
%TODO.........................................................
%{\tt 
%- grafiques per a adwin1, ilustrant the above.
%} 
\begin{figure}[t]
      \begin{center}
        % GNUPLOT: LaTeX picture
\setlength{\unitlength}{0.240900pt}
\ifx\plotpoint\undefined\newsavebox{\plotpoint}\fi
\begin{picture}(959,720)(0,0)
\sbox{\plotpoint}{\rule[-0.200pt]{0.400pt}{0.400pt}}%
\put(181,143){\makebox(0,0)[r]{$0.1$}}
\put(201.0,143.0){\rule[-0.200pt]{4.818pt}{0.400pt}}
\put(181,197){\makebox(0,0)[r]{$0.2$}}
\put(201.0,197.0){\rule[-0.200pt]{4.818pt}{0.400pt}}
\put(181,252){\makebox(0,0)[r]{$0.3$}}
\put(201.0,252.0){\rule[-0.200pt]{4.818pt}{0.400pt}}
\put(181,306){\makebox(0,0)[r]{$0.4$}}
\put(201.0,306.0){\rule[-0.200pt]{4.818pt}{0.400pt}}
\put(181,360){\makebox(0,0)[r]{$0.5$}}
\put(201.0,360.0){\rule[-0.200pt]{4.818pt}{0.400pt}}
\put(181,414){\makebox(0,0)[r]{$0.6$}}
\put(201.0,414.0){\rule[-0.200pt]{4.818pt}{0.400pt}}
\put(181,468){\makebox(0,0)[r]{$0.7$}}
\put(201.0,468.0){\rule[-0.200pt]{4.818pt}{0.400pt}}
\put(181,523){\makebox(0,0)[r]{$0.8$}}
\put(201.0,523.0){\rule[-0.200pt]{4.818pt}{0.400pt}}
\put(181,577){\makebox(0,0)[r]{$0.9$}}
\put(201.0,577.0){\rule[-0.200pt]{4.818pt}{0.400pt}}
\put(221.0,123.0){\rule[-0.200pt]{0.400pt}{4.818pt}}
\put(221,82){\makebox(0,0){$0$}}
\put(221.0,577.0){\rule[-0.200pt]{0.400pt}{4.818pt}}
\put(307.0,123.0){\rule[-0.200pt]{0.400pt}{4.818pt}}
%\put(307,82){\makebox(0,0){$500$}}
\put(307.0,577.0){\rule[-0.200pt]{0.400pt}{4.818pt}}
\put(393.0,123.0){\rule[-0.200pt]{0.400pt}{4.818pt}}
\put(393,82){\makebox(0,0){$1000$}}
\put(393.0,577.0){\rule[-0.200pt]{0.400pt}{4.818pt}}
\put(480.0,123.0){\rule[-0.200pt]{0.400pt}{4.818pt}}
%\put(480,82){\makebox(0,0){$1500$}}
\put(480.0,577.0){\rule[-0.200pt]{0.400pt}{4.818pt}}
\put(566.0,123.0){\rule[-0.200pt]{0.400pt}{4.818pt}}
\put(566,82){\makebox(0,0){$2000$}}
\put(566.0,577.0){\rule[-0.200pt]{0.400pt}{4.818pt}}
\put(652.0,123.0){\rule[-0.200pt]{0.400pt}{4.818pt}}
%\put(652,82){\makebox(0,0){$2500$}}
\put(652.0,577.0){\rule[-0.200pt]{0.400pt}{4.818pt}}
\put(738.0,123.0){\rule[-0.200pt]{0.400pt}{4.818pt}}
\put(738,82){\makebox(0,0){$3000$}}
\put(738.0,577.0){\rule[-0.200pt]{0.400pt}{4.818pt}}
\put(778,143){\makebox(0,0)[l]{ 0}}
\put(738.0,143.0){\rule[-0.200pt]{4.818pt}{0.400pt}}
\put(778,230){\makebox(0,0)[l]{ 500}}
\put(738.0,230.0){\rule[-0.200pt]{4.818pt}{0.400pt}}
\put(778,317){\makebox(0,0)[l]{ 1000}}
\put(738.0,317.0){\rule[-0.200pt]{4.818pt}{0.400pt}}
\put(778,403){\makebox(0,0)[l]{ 1500}}
\put(738.0,403.0){\rule[-0.200pt]{4.818pt}{0.400pt}}
\put(778,490){\makebox(0,0)[l]{ 2000}}
\put(738.0,490.0){\rule[-0.200pt]{4.818pt}{0.400pt}}
\put(778,577){\makebox(0,0)[l]{ 2500}}
\put(738.0,577.0){\rule[-0.200pt]{4.818pt}{0.400pt}}
\put(221.0,143.0){\rule[-0.200pt]{124.545pt}{0.400pt}}
\put(738.0,143.0){\rule[-0.200pt]{0.400pt}{104.551pt}}
\put(221.0,577.0){\rule[-0.200pt]{124.545pt}{0.400pt}}
\put(221.0,143.0){\rule[-0.200pt]{0.400pt}{104.551pt}}
\put(40,360){\makebox(0,0){$\mu$ axis}}
\put(967,360){\makebox(0,0){Width}}
\put(479,21){\makebox(0,0){$t$ axis}}
\put(479,659){\makebox(0,0){\adwinz}}
\put(578,537){\makebox(0,0)[r]{$\mu_t$}}
\put(598.0,537.0){\rule[-0.200pt]{24.090pt}{0.400pt}}
\put(223,523){\usebox{\plotpoint}}
\put(391.67,306){\rule{0.400pt}{52.275pt}}
\multiput(391.17,414.50)(1.000,-108.500){2}{\rule{0.400pt}{26.138pt}}
\put(223.0,523.0){\rule[-0.200pt]{40.712pt}{0.400pt}}
\put(393.0,306.0){\rule[-0.200pt]{82.629pt}{0.400pt}}
\put(578,496){\makebox(0,0)[r]{$\hat{\mu}_W$}}
\multiput(598,496)(20.756,0.000){5}{\usebox{\plotpoint}}
\put(698,496){\usebox{\plotpoint}}
\put(223,577){\usebox{\plotpoint}}
\put(223.00,577.00){\usebox{\plotpoint}}
\put(226.74,560.47){\usebox{\plotpoint}}
\put(229.88,545.17){\usebox{\plotpoint}}
\put(235.58,533.18){\usebox{\plotpoint}}
\put(242.22,547.57){\usebox{\plotpoint}}
\put(247.63,527.80){\usebox{\plotpoint}}
\put(254.85,519.54){\usebox{\plotpoint}}
\put(266.33,525.00){\usebox{\plotpoint}}
\put(281.51,532.26){\usebox{\plotpoint}}
\put(296.48,524.48){\usebox{\plotpoint}}
\put(314.37,525.18){\usebox{\plotpoint}}
\put(330.61,525.57){\usebox{\plotpoint}}
\put(347.85,529.43){\usebox{\plotpoint}}
\put(366.41,529.41){\usebox{\plotpoint}}
\put(384.08,529.00){\usebox{\plotpoint}}
\multiput(393,528)(1.036,-20.730){2}{\usebox{\plotpoint}}
\put(396.07,475.16){\usebox{\plotpoint}}
\multiput(397,464)(1.336,-20.712){2}{\usebox{\plotpoint}}
\multiput(399,433)(0.384,-20.752){2}{\usebox{\plotpoint}}
\multiput(400,379)(1.219,-20.720){2}{\usebox{\plotpoint}}
\put(408.06,340.06){\usebox{\plotpoint}}
\put(411.14,321.70){\usebox{\plotpoint}}
\put(415.15,309.71){\usebox{\plotpoint}}
\put(421.04,325.87){\usebox{\plotpoint}}
\put(431.65,328.96){\usebox{\plotpoint}}
\put(439.82,320.73){\usebox{\plotpoint}}
\put(450.69,314.69){\usebox{\plotpoint}}
\put(463.56,314.56){\usebox{\plotpoint}}
\put(480.04,318.89){\usebox{\plotpoint}}
\put(493.94,313.06){\usebox{\plotpoint}}
\put(510.53,306.00){\usebox{\plotpoint}}
\put(529.67,306.67){\usebox{\plotpoint}}
\put(547.73,304.00){\usebox{\plotpoint}}
\put(566.13,308.13){\usebox{\plotpoint}}
\put(585.04,307.00){\usebox{\plotpoint}}
\put(605.32,307.00){\usebox{\plotpoint}}
\put(624.31,307.00){\usebox{\plotpoint}}
\put(644.26,307.63){\usebox{\plotpoint}}
\put(664.04,309.00){\usebox{\plotpoint}}
\put(684.19,311.59){\usebox{\plotpoint}}
\put(704.38,312.00){\usebox{\plotpoint}}
\put(724.42,313.00){\usebox{\plotpoint}}
\put(736,312){\usebox{\plotpoint}}
\sbox{\plotpoint}{\rule[-0.400pt]{0.800pt}{0.800pt}}%
\sbox{\plotpoint}{\rule[-0.200pt]{0.400pt}{0.400pt}}%
\put(578,455){\makebox(0,0)[r]{$W$}}
\sbox{\plotpoint}{\rule[-0.400pt]{0.800pt}{0.800pt}}%
\put(598.0,455.0){\rule[-0.400pt]{24.090pt}{0.800pt}}
\put(223,145){\usebox{\plotpoint}}
\put(223,143.84){\rule{0.241pt}{0.800pt}}
\multiput(223.00,143.34)(0.500,1.000){2}{\rule{0.120pt}{0.800pt}}
\put(224,145.34){\rule{0.482pt}{0.800pt}}
\multiput(224.00,144.34)(1.000,2.000){2}{\rule{0.241pt}{0.800pt}}
\put(226,147.34){\rule{0.482pt}{0.800pt}}
\multiput(226.00,146.34)(1.000,2.000){2}{\rule{0.241pt}{0.800pt}}
\put(228,149.34){\rule{0.482pt}{0.800pt}}
\multiput(228.00,148.34)(1.000,2.000){2}{\rule{0.241pt}{0.800pt}}
\put(230,150.84){\rule{0.241pt}{0.800pt}}
\multiput(230.00,150.34)(0.500,1.000){2}{\rule{0.120pt}{0.800pt}}
\put(231,152.34){\rule{0.482pt}{0.800pt}}
\multiput(231.00,151.34)(1.000,2.000){2}{\rule{0.241pt}{0.800pt}}
\put(233,154.34){\rule{0.482pt}{0.800pt}}
\multiput(233.00,153.34)(1.000,2.000){2}{\rule{0.241pt}{0.800pt}}
\put(235,156.34){\rule{0.482pt}{0.800pt}}
\multiput(235.00,155.34)(1.000,2.000){2}{\rule{0.241pt}{0.800pt}}
\put(237,157.84){\rule{0.241pt}{0.800pt}}
\multiput(237.00,157.34)(0.500,1.000){2}{\rule{0.120pt}{0.800pt}}
\put(238,159.34){\rule{0.482pt}{0.800pt}}
\multiput(238.00,158.34)(1.000,2.000){2}{\rule{0.241pt}{0.800pt}}
\put(240,161.34){\rule{0.482pt}{0.800pt}}
\multiput(240.00,160.34)(1.000,2.000){2}{\rule{0.241pt}{0.800pt}}
\put(240.84,164){\rule{0.800pt}{0.482pt}}
\multiput(240.34,164.00)(1.000,1.000){2}{\rule{0.800pt}{0.241pt}}
\put(243,164.84){\rule{0.482pt}{0.800pt}}
\multiput(243.00,164.34)(1.000,1.000){2}{\rule{0.241pt}{0.800pt}}
\put(245,166.34){\rule{0.482pt}{0.800pt}}
\multiput(245.00,165.34)(1.000,2.000){2}{\rule{0.241pt}{0.800pt}}
\put(247,168.34){\rule{0.482pt}{0.800pt}}
\multiput(247.00,167.34)(1.000,2.000){2}{\rule{0.241pt}{0.800pt}}
\put(247.84,171){\rule{0.800pt}{0.482pt}}
\multiput(247.34,171.00)(1.000,1.000){2}{\rule{0.800pt}{0.241pt}}
\put(250,171.84){\rule{0.482pt}{0.800pt}}
\multiput(250.00,171.34)(1.000,1.000){2}{\rule{0.241pt}{0.800pt}}
\put(252,173.34){\rule{0.482pt}{0.800pt}}
\multiput(252.00,172.34)(1.000,2.000){2}{\rule{0.241pt}{0.800pt}}
\put(252.84,176){\rule{0.800pt}{0.482pt}}
\multiput(252.34,176.00)(1.000,1.000){2}{\rule{0.800pt}{0.241pt}}
\put(255,176.84){\rule{0.482pt}{0.800pt}}
\multiput(255.00,176.34)(1.000,1.000){2}{\rule{0.241pt}{0.800pt}}
\put(257,178.34){\rule{0.482pt}{0.800pt}}
\multiput(257.00,177.34)(1.000,2.000){2}{\rule{0.241pt}{0.800pt}}
\put(259,180.34){\rule{0.482pt}{0.800pt}}
\multiput(259.00,179.34)(1.000,2.000){2}{\rule{0.241pt}{0.800pt}}
\put(259.84,183){\rule{0.800pt}{0.482pt}}
\multiput(259.34,183.00)(1.000,1.000){2}{\rule{0.800pt}{0.241pt}}
\put(262,183.84){\rule{0.482pt}{0.800pt}}
\multiput(262.00,183.34)(1.000,1.000){2}{\rule{0.241pt}{0.800pt}}
\put(264,185.34){\rule{0.482pt}{0.800pt}}
\multiput(264.00,184.34)(1.000,2.000){2}{\rule{0.241pt}{0.800pt}}
\put(266,187.34){\rule{0.482pt}{0.800pt}}
\multiput(266.00,186.34)(1.000,2.000){2}{\rule{0.241pt}{0.800pt}}
\put(266.84,190){\rule{0.800pt}{0.482pt}}
\multiput(266.34,190.00)(1.000,1.000){2}{\rule{0.800pt}{0.241pt}}
\put(269,190.84){\rule{0.482pt}{0.800pt}}
\multiput(269.00,190.34)(1.000,1.000){2}{\rule{0.241pt}{0.800pt}}
\put(271,192.34){\rule{0.482pt}{0.800pt}}
\multiput(271.00,191.34)(1.000,2.000){2}{\rule{0.241pt}{0.800pt}}
\put(271.84,195){\rule{0.800pt}{0.482pt}}
\multiput(271.34,195.00)(1.000,1.000){2}{\rule{0.800pt}{0.241pt}}
\put(274,196.34){\rule{0.482pt}{0.800pt}}
\multiput(274.00,195.34)(1.000,2.000){2}{\rule{0.241pt}{0.800pt}}
\put(276,197.84){\rule{0.482pt}{0.800pt}}
\multiput(276.00,197.34)(1.000,1.000){2}{\rule{0.241pt}{0.800pt}}
\put(278,199.34){\rule{0.482pt}{0.800pt}}
\multiput(278.00,198.34)(1.000,2.000){2}{\rule{0.241pt}{0.800pt}}
\put(278.84,202){\rule{0.800pt}{0.482pt}}
\multiput(278.34,202.00)(1.000,1.000){2}{\rule{0.800pt}{0.241pt}}
\put(281,202.84){\rule{0.482pt}{0.800pt}}
\multiput(281.00,202.34)(1.000,1.000){2}{\rule{0.241pt}{0.800pt}}
\put(283,204.34){\rule{0.482pt}{0.800pt}}
\multiput(283.00,203.34)(1.000,2.000){2}{\rule{0.241pt}{0.800pt}}
\put(283.84,207){\rule{0.800pt}{0.482pt}}
\multiput(283.34,207.00)(1.000,1.000){2}{\rule{0.800pt}{0.241pt}}
\put(286,208.34){\rule{0.482pt}{0.800pt}}
\multiput(286.00,207.34)(1.000,2.000){2}{\rule{0.241pt}{0.800pt}}
\put(288,209.84){\rule{0.482pt}{0.800pt}}
\multiput(288.00,209.34)(1.000,1.000){2}{\rule{0.241pt}{0.800pt}}
\put(290,211.34){\rule{0.482pt}{0.800pt}}
\multiput(290.00,210.34)(1.000,2.000){2}{\rule{0.241pt}{0.800pt}}
\put(290.84,214){\rule{0.800pt}{0.482pt}}
\multiput(290.34,214.00)(1.000,1.000){2}{\rule{0.800pt}{0.241pt}}
\put(293,215.34){\rule{0.482pt}{0.800pt}}
\multiput(293.00,214.34)(1.000,2.000){2}{\rule{0.241pt}{0.800pt}}
\put(295,216.84){\rule{0.482pt}{0.800pt}}
\multiput(295.00,216.34)(1.000,1.000){2}{\rule{0.241pt}{0.800pt}}
\put(297,218.34){\rule{0.482pt}{0.800pt}}
\multiput(297.00,217.34)(1.000,2.000){2}{\rule{0.241pt}{0.800pt}}
\put(297.84,221){\rule{0.800pt}{0.482pt}}
\multiput(297.34,221.00)(1.000,1.000){2}{\rule{0.800pt}{0.241pt}}
\put(300,222.34){\rule{0.482pt}{0.800pt}}
\multiput(300.00,221.34)(1.000,2.000){2}{\rule{0.241pt}{0.800pt}}
\put(302,223.84){\rule{0.482pt}{0.800pt}}
\multiput(302.00,223.34)(1.000,1.000){2}{\rule{0.241pt}{0.800pt}}
\put(302.84,226){\rule{0.800pt}{0.482pt}}
\multiput(302.34,226.00)(1.000,1.000){2}{\rule{0.800pt}{0.241pt}}
\put(305,227.34){\rule{0.482pt}{0.800pt}}
\multiput(305.00,226.34)(1.000,2.000){2}{\rule{0.241pt}{0.800pt}}
\put(307,229.34){\rule{0.482pt}{0.800pt}}
\multiput(307.00,228.34)(1.000,2.000){2}{\rule{0.241pt}{0.800pt}}
\put(309,230.84){\rule{0.482pt}{0.800pt}}
\multiput(309.00,230.34)(1.000,1.000){2}{\rule{0.241pt}{0.800pt}}
\put(309.84,233){\rule{0.800pt}{0.482pt}}
\multiput(309.34,233.00)(1.000,1.000){2}{\rule{0.800pt}{0.241pt}}
\put(312,234.34){\rule{0.482pt}{0.800pt}}
\multiput(312.00,233.34)(1.000,2.000){2}{\rule{0.241pt}{0.800pt}}
\put(314,235.84){\rule{0.482pt}{0.800pt}}
\multiput(314.00,235.34)(1.000,1.000){2}{\rule{0.241pt}{0.800pt}}
\put(316,237.34){\rule{0.482pt}{0.800pt}}
\multiput(316.00,236.34)(1.000,2.000){2}{\rule{0.241pt}{0.800pt}}
\put(316.84,240){\rule{0.800pt}{0.482pt}}
\multiput(316.34,240.00)(1.000,1.000){2}{\rule{0.800pt}{0.241pt}}
\put(319,241.34){\rule{0.482pt}{0.800pt}}
\multiput(319.00,240.34)(1.000,2.000){2}{\rule{0.241pt}{0.800pt}}
\put(321,242.84){\rule{0.482pt}{0.800pt}}
\multiput(321.00,242.34)(1.000,1.000){2}{\rule{0.241pt}{0.800pt}}
\put(321.84,245){\rule{0.800pt}{0.482pt}}
\multiput(321.34,245.00)(1.000,1.000){2}{\rule{0.800pt}{0.241pt}}
\put(324,246.34){\rule{0.482pt}{0.800pt}}
\multiput(324.00,245.34)(1.000,2.000){2}{\rule{0.241pt}{0.800pt}}
\put(326,248.34){\rule{0.482pt}{0.800pt}}
\multiput(326.00,247.34)(1.000,2.000){2}{\rule{0.241pt}{0.800pt}}
\put(328,249.84){\rule{0.482pt}{0.800pt}}
\multiput(328.00,249.34)(1.000,1.000){2}{\rule{0.241pt}{0.800pt}}
\put(328.84,252){\rule{0.800pt}{0.482pt}}
\multiput(328.34,252.00)(1.000,1.000){2}{\rule{0.800pt}{0.241pt}}
\put(331,253.34){\rule{0.482pt}{0.800pt}}
\multiput(331.00,252.34)(1.000,2.000){2}{\rule{0.241pt}{0.800pt}}
\put(333,255.34){\rule{0.482pt}{0.800pt}}
\multiput(333.00,254.34)(1.000,2.000){2}{\rule{0.241pt}{0.800pt}}
\put(335,256.84){\rule{0.241pt}{0.800pt}}
\multiput(335.00,256.34)(0.500,1.000){2}{\rule{0.120pt}{0.800pt}}
\put(336,258.34){\rule{0.482pt}{0.800pt}}
\multiput(336.00,257.34)(1.000,2.000){2}{\rule{0.241pt}{0.800pt}}
\put(338,260.34){\rule{0.482pt}{0.800pt}}
\multiput(338.00,259.34)(1.000,2.000){2}{\rule{0.241pt}{0.800pt}}
\put(340,262.34){\rule{0.482pt}{0.800pt}}
\multiput(340.00,261.34)(1.000,2.000){2}{\rule{0.241pt}{0.800pt}}
\put(342,263.84){\rule{0.241pt}{0.800pt}}
\multiput(342.00,263.34)(0.500,1.000){2}{\rule{0.120pt}{0.800pt}}
\put(343,265.34){\rule{0.482pt}{0.800pt}}
\multiput(343.00,264.34)(1.000,2.000){2}{\rule{0.241pt}{0.800pt}}
\put(345,267.34){\rule{0.482pt}{0.800pt}}
\multiput(345.00,266.34)(1.000,2.000){2}{\rule{0.241pt}{0.800pt}}
\put(347,268.84){\rule{0.482pt}{0.800pt}}
\multiput(347.00,268.34)(1.000,1.000){2}{\rule{0.241pt}{0.800pt}}
\put(347.84,271){\rule{0.800pt}{0.482pt}}
\multiput(347.34,271.00)(1.000,1.000){2}{\rule{0.800pt}{0.241pt}}
\put(350,272.34){\rule{0.482pt}{0.800pt}}
\multiput(350.00,271.34)(1.000,2.000){2}{\rule{0.241pt}{0.800pt}}
\put(352,274.34){\rule{0.482pt}{0.800pt}}
\multiput(352.00,273.34)(1.000,2.000){2}{\rule{0.241pt}{0.800pt}}
\put(354,275.84){\rule{0.241pt}{0.800pt}}
\multiput(354.00,275.34)(0.500,1.000){2}{\rule{0.120pt}{0.800pt}}
\put(355,277.34){\rule{0.482pt}{0.800pt}}
\multiput(355.00,276.34)(1.000,2.000){2}{\rule{0.241pt}{0.800pt}}
\put(357,279.34){\rule{0.482pt}{0.800pt}}
\multiput(357.00,278.34)(1.000,2.000){2}{\rule{0.241pt}{0.800pt}}
\put(359,281.34){\rule{0.482pt}{0.800pt}}
\multiput(359.00,280.34)(1.000,2.000){2}{\rule{0.241pt}{0.800pt}}
\put(361,282.84){\rule{0.241pt}{0.800pt}}
\multiput(361.00,282.34)(0.500,1.000){2}{\rule{0.120pt}{0.800pt}}
\put(362,284.34){\rule{0.482pt}{0.800pt}}
\multiput(362.00,283.34)(1.000,2.000){2}{\rule{0.241pt}{0.800pt}}
\put(364,286.34){\rule{0.482pt}{0.800pt}}
\multiput(364.00,285.34)(1.000,2.000){2}{\rule{0.241pt}{0.800pt}}
\put(364.84,289){\rule{0.800pt}{0.482pt}}
\multiput(364.34,289.00)(1.000,1.000){2}{\rule{0.800pt}{0.241pt}}
\put(367,289.84){\rule{0.482pt}{0.800pt}}
\multiput(367.00,289.34)(1.000,1.000){2}{\rule{0.241pt}{0.800pt}}
\put(369,291.34){\rule{0.482pt}{0.800pt}}
\multiput(369.00,290.34)(1.000,2.000){2}{\rule{0.241pt}{0.800pt}}
\put(371,293.34){\rule{0.482pt}{0.800pt}}
\multiput(371.00,292.34)(1.000,2.000){2}{\rule{0.241pt}{0.800pt}}
\put(371.84,296){\rule{0.800pt}{0.482pt}}
\multiput(371.34,296.00)(1.000,1.000){2}{\rule{0.800pt}{0.241pt}}
\put(374,296.84){\rule{0.482pt}{0.800pt}}
\multiput(374.00,296.34)(1.000,1.000){2}{\rule{0.241pt}{0.800pt}}
\put(376,298.34){\rule{0.482pt}{0.800pt}}
\multiput(376.00,297.34)(1.000,2.000){2}{\rule{0.241pt}{0.800pt}}
\put(378,300.34){\rule{0.482pt}{0.800pt}}
\multiput(378.00,299.34)(1.000,2.000){2}{\rule{0.241pt}{0.800pt}}
\put(380,301.84){\rule{0.241pt}{0.800pt}}
\multiput(380.00,301.34)(0.500,1.000){2}{\rule{0.120pt}{0.800pt}}
\put(381,303.34){\rule{0.482pt}{0.800pt}}
\multiput(381.00,302.34)(1.000,2.000){2}{\rule{0.241pt}{0.800pt}}
\put(383,305.34){\rule{0.482pt}{0.800pt}}
\multiput(383.00,304.34)(1.000,2.000){2}{\rule{0.241pt}{0.800pt}}
\put(383.84,308){\rule{0.800pt}{0.482pt}}
\multiput(383.34,308.00)(1.000,1.000){2}{\rule{0.800pt}{0.241pt}}
\put(386,308.84){\rule{0.482pt}{0.800pt}}
\multiput(386.00,308.34)(1.000,1.000){2}{\rule{0.241pt}{0.800pt}}
\put(388,310.34){\rule{0.482pt}{0.800pt}}
\multiput(388.00,309.34)(1.000,2.000){2}{\rule{0.241pt}{0.800pt}}
\put(390,312.34){\rule{0.482pt}{0.800pt}}
\multiput(390.00,311.34)(1.000,2.000){2}{\rule{0.241pt}{0.800pt}}
\put(390.84,315){\rule{0.800pt}{0.482pt}}
\multiput(390.34,315.00)(1.000,1.000){2}{\rule{0.800pt}{0.241pt}}
\put(392.34,155){\rule{0.800pt}{39.026pt}}
\multiput(391.34,236.00)(2.000,-81.000){2}{\rule{0.800pt}{19.513pt}}
\put(395,154.34){\rule{0.482pt}{0.800pt}}
\multiput(395.00,153.34)(1.000,2.000){2}{\rule{0.241pt}{0.800pt}}
\put(397,154.34){\rule{0.482pt}{0.800pt}}
\multiput(397.00,155.34)(1.000,-2.000){2}{\rule{0.241pt}{0.800pt}}
\put(402,154.34){\rule{0.482pt}{0.800pt}}
\multiput(402.00,153.34)(1.000,2.000){2}{\rule{0.241pt}{0.800pt}}
\put(402.84,157){\rule{0.800pt}{0.482pt}}
\multiput(402.34,157.00)(1.000,1.000){2}{\rule{0.800pt}{0.241pt}}
\put(405,158.34){\rule{0.482pt}{0.800pt}}
\multiput(405.00,157.34)(1.000,2.000){2}{\rule{0.241pt}{0.800pt}}
\put(407,159.84){\rule{0.482pt}{0.800pt}}
\multiput(407.00,159.34)(1.000,1.000){2}{\rule{0.241pt}{0.800pt}}
\put(409,161.34){\rule{0.482pt}{0.800pt}}
\multiput(409.00,160.34)(1.000,2.000){2}{\rule{0.241pt}{0.800pt}}
\put(409.84,164){\rule{0.800pt}{0.482pt}}
\multiput(409.34,164.00)(1.000,1.000){2}{\rule{0.800pt}{0.241pt}}
\put(412,164.84){\rule{0.482pt}{0.800pt}}
\multiput(412.00,164.34)(1.000,1.000){2}{\rule{0.241pt}{0.800pt}}
\put(414,166.34){\rule{0.482pt}{0.800pt}}
\multiput(414.00,165.34)(1.000,2.000){2}{\rule{0.241pt}{0.800pt}}
\put(414.84,169){\rule{0.800pt}{0.482pt}}
\multiput(414.34,169.00)(1.000,1.000){2}{\rule{0.800pt}{0.241pt}}
\put(417,170.34){\rule{0.482pt}{0.800pt}}
\multiput(417.00,169.34)(1.000,2.000){2}{\rule{0.241pt}{0.800pt}}
\put(419,171.84){\rule{0.482pt}{0.800pt}}
\multiput(419.00,171.34)(1.000,1.000){2}{\rule{0.241pt}{0.800pt}}
\put(421,173.34){\rule{0.482pt}{0.800pt}}
\multiput(421.00,172.34)(1.000,2.000){2}{\rule{0.241pt}{0.800pt}}
\put(421.84,176){\rule{0.800pt}{0.482pt}}
\multiput(421.34,176.00)(1.000,1.000){2}{\rule{0.800pt}{0.241pt}}
\put(424,177.34){\rule{0.482pt}{0.800pt}}
\multiput(424.00,176.34)(1.000,2.000){2}{\rule{0.241pt}{0.800pt}}
\put(426,178.84){\rule{0.482pt}{0.800pt}}
\multiput(426.00,178.34)(1.000,1.000){2}{\rule{0.241pt}{0.800pt}}
\put(428,180.34){\rule{0.482pt}{0.800pt}}
\multiput(428.00,179.34)(1.000,2.000){2}{\rule{0.241pt}{0.800pt}}
\put(428.84,183){\rule{0.800pt}{0.482pt}}
\multiput(428.34,183.00)(1.000,1.000){2}{\rule{0.800pt}{0.241pt}}
\put(431,184.34){\rule{0.482pt}{0.800pt}}
\multiput(431.00,183.34)(1.000,2.000){2}{\rule{0.241pt}{0.800pt}}
\put(433,185.84){\rule{0.482pt}{0.800pt}}
\multiput(433.00,185.34)(1.000,1.000){2}{\rule{0.241pt}{0.800pt}}
\put(433.84,188){\rule{0.800pt}{0.482pt}}
\multiput(433.34,188.00)(1.000,1.000){2}{\rule{0.800pt}{0.241pt}}
\put(436,189.34){\rule{0.482pt}{0.800pt}}
\multiput(436.00,188.34)(1.000,2.000){2}{\rule{0.241pt}{0.800pt}}
\put(438,191.34){\rule{0.482pt}{0.800pt}}
\multiput(438.00,190.34)(1.000,2.000){2}{\rule{0.241pt}{0.800pt}}
\put(440,192.84){\rule{0.482pt}{0.800pt}}
\multiput(440.00,192.34)(1.000,1.000){2}{\rule{0.241pt}{0.800pt}}
\put(440.84,195){\rule{0.800pt}{0.482pt}}
\multiput(440.34,195.00)(1.000,1.000){2}{\rule{0.800pt}{0.241pt}}
\put(443,196.34){\rule{0.482pt}{0.800pt}}
\multiput(443.00,195.34)(1.000,2.000){2}{\rule{0.241pt}{0.800pt}}
\put(445,197.84){\rule{0.482pt}{0.800pt}}
\multiput(445.00,197.34)(1.000,1.000){2}{\rule{0.241pt}{0.800pt}}
\put(445.84,200){\rule{0.800pt}{0.482pt}}
\multiput(445.34,200.00)(1.000,1.000){2}{\rule{0.800pt}{0.241pt}}
\put(448,201.34){\rule{0.482pt}{0.800pt}}
\multiput(448.00,200.34)(1.000,2.000){2}{\rule{0.241pt}{0.800pt}}
\put(450,203.34){\rule{0.482pt}{0.800pt}}
\multiput(450.00,202.34)(1.000,2.000){2}{\rule{0.241pt}{0.800pt}}
\put(452,204.84){\rule{0.482pt}{0.800pt}}
\multiput(452.00,204.34)(1.000,1.000){2}{\rule{0.241pt}{0.800pt}}
\put(452.84,207){\rule{0.800pt}{0.482pt}}
\multiput(452.34,207.00)(1.000,1.000){2}{\rule{0.800pt}{0.241pt}}
\put(455,208.34){\rule{0.482pt}{0.800pt}}
\multiput(455.00,207.34)(1.000,2.000){2}{\rule{0.241pt}{0.800pt}}
\put(457,210.34){\rule{0.482pt}{0.800pt}}
\multiput(457.00,209.34)(1.000,2.000){2}{\rule{0.241pt}{0.800pt}}
\put(459,211.84){\rule{0.482pt}{0.800pt}}
\multiput(459.00,211.34)(1.000,1.000){2}{\rule{0.241pt}{0.800pt}}
\put(459.84,214){\rule{0.800pt}{0.482pt}}
\multiput(459.34,214.00)(1.000,1.000){2}{\rule{0.800pt}{0.241pt}}
\put(462,215.34){\rule{0.482pt}{0.800pt}}
\multiput(462.00,214.34)(1.000,2.000){2}{\rule{0.241pt}{0.800pt}}
\put(464,217.34){\rule{0.482pt}{0.800pt}}
\multiput(464.00,216.34)(1.000,2.000){2}{\rule{0.241pt}{0.800pt}}
\put(466,218.84){\rule{0.241pt}{0.800pt}}
\multiput(466.00,218.34)(0.500,1.000){2}{\rule{0.120pt}{0.800pt}}
\put(467,220.34){\rule{0.482pt}{0.800pt}}
\multiput(467.00,219.34)(1.000,2.000){2}{\rule{0.241pt}{0.800pt}}
\put(469,222.34){\rule{0.482pt}{0.800pt}}
\multiput(469.00,221.34)(1.000,2.000){2}{\rule{0.241pt}{0.800pt}}
\put(471,224.34){\rule{0.482pt}{0.800pt}}
\multiput(471.00,223.34)(1.000,2.000){2}{\rule{0.241pt}{0.800pt}}
\put(473,225.84){\rule{0.241pt}{0.800pt}}
\multiput(473.00,225.34)(0.500,1.000){2}{\rule{0.120pt}{0.800pt}}
\put(474,227.34){\rule{0.482pt}{0.800pt}}
\multiput(474.00,226.34)(1.000,2.000){2}{\rule{0.241pt}{0.800pt}}
\put(476,229.34){\rule{0.482pt}{0.800pt}}
\multiput(476.00,228.34)(1.000,2.000){2}{\rule{0.241pt}{0.800pt}}
\put(478,230.84){\rule{0.482pt}{0.800pt}}
\multiput(478.00,230.34)(1.000,1.000){2}{\rule{0.241pt}{0.800pt}}
\put(478.84,233){\rule{0.800pt}{0.482pt}}
\multiput(478.34,233.00)(1.000,1.000){2}{\rule{0.800pt}{0.241pt}}
\put(481,234.34){\rule{0.482pt}{0.800pt}}
\multiput(481.00,233.34)(1.000,2.000){2}{\rule{0.241pt}{0.800pt}}
\put(483,236.34){\rule{0.482pt}{0.800pt}}
\multiput(483.00,235.34)(1.000,2.000){2}{\rule{0.241pt}{0.800pt}}
\put(485,237.84){\rule{0.241pt}{0.800pt}}
\multiput(485.00,237.34)(0.500,1.000){2}{\rule{0.120pt}{0.800pt}}
\put(486,239.34){\rule{0.482pt}{0.800pt}}
\multiput(486.00,238.34)(1.000,2.000){2}{\rule{0.241pt}{0.800pt}}
\put(488,241.34){\rule{0.482pt}{0.800pt}}
\multiput(488.00,240.34)(1.000,2.000){2}{\rule{0.241pt}{0.800pt}}
\put(490,243.34){\rule{0.482pt}{0.800pt}}
\multiput(490.00,242.34)(1.000,2.000){2}{\rule{0.241pt}{0.800pt}}
\put(492,244.84){\rule{0.241pt}{0.800pt}}
\multiput(492.00,244.34)(0.500,1.000){2}{\rule{0.120pt}{0.800pt}}
\put(493,246.34){\rule{0.482pt}{0.800pt}}
\multiput(493.00,245.34)(1.000,2.000){2}{\rule{0.241pt}{0.800pt}}
\put(495,248.34){\rule{0.482pt}{0.800pt}}
\multiput(495.00,247.34)(1.000,2.000){2}{\rule{0.241pt}{0.800pt}}
\put(495.84,251){\rule{0.800pt}{0.482pt}}
\multiput(495.34,251.00)(1.000,1.000){2}{\rule{0.800pt}{0.241pt}}
\put(498,251.84){\rule{0.482pt}{0.800pt}}
\multiput(498.00,251.34)(1.000,1.000){2}{\rule{0.241pt}{0.800pt}}
\put(500,253.34){\rule{0.482pt}{0.800pt}}
\multiput(500.00,252.34)(1.000,2.000){2}{\rule{0.241pt}{0.800pt}}
\put(502,255.34){\rule{0.482pt}{0.800pt}}
\multiput(502.00,254.34)(1.000,2.000){2}{\rule{0.241pt}{0.800pt}}
\put(504,256.84){\rule{0.241pt}{0.800pt}}
\multiput(504.00,256.34)(0.500,1.000){2}{\rule{0.120pt}{0.800pt}}
\put(505,258.34){\rule{0.482pt}{0.800pt}}
\multiput(505.00,257.34)(1.000,2.000){2}{\rule{0.241pt}{0.800pt}}
\put(507,260.34){\rule{0.482pt}{0.800pt}}
\multiput(507.00,259.34)(1.000,2.000){2}{\rule{0.241pt}{0.800pt}}
\put(509,262.34){\rule{0.482pt}{0.800pt}}
\multiput(509.00,261.34)(1.000,2.000){2}{\rule{0.241pt}{0.800pt}}
\put(511,263.84){\rule{0.241pt}{0.800pt}}
\multiput(511.00,263.34)(0.500,1.000){2}{\rule{0.120pt}{0.800pt}}
\put(512,265.34){\rule{0.482pt}{0.800pt}}
\multiput(512.00,264.34)(1.000,2.000){2}{\rule{0.241pt}{0.800pt}}
\put(514,267.34){\rule{0.482pt}{0.800pt}}
\multiput(514.00,266.34)(1.000,2.000){2}{\rule{0.241pt}{0.800pt}}
\put(514.84,270){\rule{0.800pt}{0.482pt}}
\multiput(514.34,270.00)(1.000,1.000){2}{\rule{0.800pt}{0.241pt}}
\put(517,270.84){\rule{0.482pt}{0.800pt}}
\multiput(517.00,270.34)(1.000,1.000){2}{\rule{0.241pt}{0.800pt}}
\put(519,272.34){\rule{0.482pt}{0.800pt}}
\multiput(519.00,271.34)(1.000,2.000){2}{\rule{0.241pt}{0.800pt}}
\put(521,274.34){\rule{0.482pt}{0.800pt}}
\multiput(521.00,273.34)(1.000,2.000){2}{\rule{0.241pt}{0.800pt}}
\put(521.84,277){\rule{0.800pt}{0.482pt}}
\multiput(521.34,277.00)(1.000,1.000){2}{\rule{0.800pt}{0.241pt}}
\put(524,277.84){\rule{0.482pt}{0.800pt}}
\multiput(524.00,277.34)(1.000,1.000){2}{\rule{0.241pt}{0.800pt}}
\put(526,279.34){\rule{0.482pt}{0.800pt}}
\multiput(526.00,278.34)(1.000,2.000){2}{\rule{0.241pt}{0.800pt}}
\put(526.84,282){\rule{0.800pt}{0.482pt}}
\multiput(526.34,282.00)(1.000,1.000){2}{\rule{0.800pt}{0.241pt}}
\put(529,283.34){\rule{0.482pt}{0.800pt}}
\multiput(529.00,282.34)(1.000,2.000){2}{\rule{0.241pt}{0.800pt}}
\put(531,284.84){\rule{0.482pt}{0.800pt}}
\multiput(531.00,284.34)(1.000,1.000){2}{\rule{0.241pt}{0.800pt}}
\put(533,286.34){\rule{0.482pt}{0.800pt}}
\multiput(533.00,285.34)(1.000,2.000){2}{\rule{0.241pt}{0.800pt}}
\put(533.84,289){\rule{0.800pt}{0.482pt}}
\multiput(533.34,289.00)(1.000,1.000){2}{\rule{0.800pt}{0.241pt}}
\put(536,289.84){\rule{0.482pt}{0.800pt}}
\multiput(536.00,289.34)(1.000,1.000){2}{\rule{0.241pt}{0.800pt}}
\put(538,291.34){\rule{0.482pt}{0.800pt}}
\multiput(538.00,290.34)(1.000,2.000){2}{\rule{0.241pt}{0.800pt}}
\put(540,293.34){\rule{0.482pt}{0.800pt}}
\multiput(540.00,292.34)(1.000,2.000){2}{\rule{0.241pt}{0.800pt}}
\put(540.84,296){\rule{0.800pt}{0.482pt}}
\multiput(540.34,296.00)(1.000,1.000){2}{\rule{0.800pt}{0.241pt}}
\put(543,296.84){\rule{0.482pt}{0.800pt}}
\multiput(543.00,296.34)(1.000,1.000){2}{\rule{0.241pt}{0.800pt}}
\put(545,298.34){\rule{0.482pt}{0.800pt}}
\multiput(545.00,297.34)(1.000,2.000){2}{\rule{0.241pt}{0.800pt}}
\put(545.84,301){\rule{0.800pt}{0.482pt}}
\multiput(545.34,301.00)(1.000,1.000){2}{\rule{0.800pt}{0.241pt}}
\put(548,302.34){\rule{0.482pt}{0.800pt}}
\multiput(548.00,301.34)(1.000,2.000){2}{\rule{0.241pt}{0.800pt}}
\put(550,303.84){\rule{0.482pt}{0.800pt}}
\multiput(550.00,303.34)(1.000,1.000){2}{\rule{0.241pt}{0.800pt}}
\put(552,305.34){\rule{0.482pt}{0.800pt}}
\multiput(552.00,304.34)(1.000,2.000){2}{\rule{0.241pt}{0.800pt}}
\put(552.84,308){\rule{0.800pt}{0.482pt}}
\multiput(552.34,308.00)(1.000,1.000){2}{\rule{0.800pt}{0.241pt}}
\put(555,309.34){\rule{0.482pt}{0.800pt}}
\multiput(555.00,308.34)(1.000,2.000){2}{\rule{0.241pt}{0.800pt}}
\put(557,310.84){\rule{0.482pt}{0.800pt}}
\multiput(557.00,310.34)(1.000,1.000){2}{\rule{0.241pt}{0.800pt}}
\put(557.84,313){\rule{0.800pt}{0.482pt}}
\multiput(557.34,313.00)(1.000,1.000){2}{\rule{0.800pt}{0.241pt}}
\put(560,314.34){\rule{0.482pt}{0.800pt}}
\multiput(560.00,313.34)(1.000,2.000){2}{\rule{0.241pt}{0.800pt}}
\put(562,316.34){\rule{0.482pt}{0.800pt}}
\multiput(562.00,315.34)(1.000,2.000){2}{\rule{0.241pt}{0.800pt}}
\put(564,317.84){\rule{0.482pt}{0.800pt}}
\multiput(564.00,317.34)(1.000,1.000){2}{\rule{0.241pt}{0.800pt}}
\put(564.84,320){\rule{0.800pt}{0.482pt}}
\multiput(564.34,320.00)(1.000,1.000){2}{\rule{0.800pt}{0.241pt}}
\put(567,321.34){\rule{0.482pt}{0.800pt}}
\multiput(567.00,320.34)(1.000,2.000){2}{\rule{0.241pt}{0.800pt}}
\put(569,322.84){\rule{0.482pt}{0.800pt}}
\multiput(569.00,322.34)(1.000,1.000){2}{\rule{0.241pt}{0.800pt}}
\put(571,324.34){\rule{0.482pt}{0.800pt}}
\multiput(571.00,323.34)(1.000,2.000){2}{\rule{0.241pt}{0.800pt}}
\put(571.84,327){\rule{0.800pt}{0.482pt}}
\multiput(571.34,327.00)(1.000,1.000){2}{\rule{0.800pt}{0.241pt}}
\put(574,328.34){\rule{0.482pt}{0.800pt}}
\multiput(574.00,327.34)(1.000,2.000){2}{\rule{0.241pt}{0.800pt}}
\put(576,329.84){\rule{0.482pt}{0.800pt}}
\multiput(576.00,329.34)(1.000,1.000){2}{\rule{0.241pt}{0.800pt}}
\put(576.84,332){\rule{0.800pt}{0.482pt}}
\multiput(576.34,332.00)(1.000,1.000){2}{\rule{0.800pt}{0.241pt}}
\put(579,333.34){\rule{0.482pt}{0.800pt}}
\multiput(579.00,332.34)(1.000,2.000){2}{\rule{0.241pt}{0.800pt}}
\put(581,335.34){\rule{0.482pt}{0.800pt}}
\multiput(581.00,334.34)(1.000,2.000){2}{\rule{0.241pt}{0.800pt}}
\put(583,336.84){\rule{0.482pt}{0.800pt}}
\multiput(583.00,336.34)(1.000,1.000){2}{\rule{0.241pt}{0.800pt}}
\put(583.84,339){\rule{0.800pt}{0.482pt}}
\multiput(583.34,339.00)(1.000,1.000){2}{\rule{0.800pt}{0.241pt}}
\put(586,340.34){\rule{0.482pt}{0.800pt}}
\multiput(586.00,339.34)(1.000,2.000){2}{\rule{0.241pt}{0.800pt}}
\put(588,342.34){\rule{0.482pt}{0.800pt}}
\multiput(588.00,341.34)(1.000,2.000){2}{\rule{0.241pt}{0.800pt}}
\put(590,343.84){\rule{0.482pt}{0.800pt}}
\multiput(590.00,343.34)(1.000,1.000){2}{\rule{0.241pt}{0.800pt}}
\put(590.84,346){\rule{0.800pt}{0.482pt}}
\multiput(590.34,346.00)(1.000,1.000){2}{\rule{0.800pt}{0.241pt}}
\put(593,347.34){\rule{0.482pt}{0.800pt}}
\multiput(593.00,346.34)(1.000,2.000){2}{\rule{0.241pt}{0.800pt}}
\put(595,348.84){\rule{0.482pt}{0.800pt}}
\multiput(595.00,348.34)(1.000,1.000){2}{\rule{0.241pt}{0.800pt}}
\put(595.84,351){\rule{0.800pt}{0.482pt}}
\multiput(595.34,351.00)(1.000,1.000){2}{\rule{0.800pt}{0.241pt}}
\put(598,352.34){\rule{0.482pt}{0.800pt}}
\multiput(598.00,351.34)(1.000,2.000){2}{\rule{0.241pt}{0.800pt}}
\put(600,354.34){\rule{0.482pt}{0.800pt}}
\multiput(600.00,353.34)(1.000,2.000){2}{\rule{0.241pt}{0.800pt}}
\put(602,355.84){\rule{0.482pt}{0.800pt}}
\multiput(602.00,355.34)(1.000,1.000){2}{\rule{0.241pt}{0.800pt}}
\put(602.84,358){\rule{0.800pt}{0.482pt}}
\multiput(602.34,358.00)(1.000,1.000){2}{\rule{0.800pt}{0.241pt}}
\put(605,359.34){\rule{0.482pt}{0.800pt}}
\multiput(605.00,358.34)(1.000,2.000){2}{\rule{0.241pt}{0.800pt}}
\put(607,361.34){\rule{0.482pt}{0.800pt}}
\multiput(607.00,360.34)(1.000,2.000){2}{\rule{0.241pt}{0.800pt}}
\put(609,362.84){\rule{0.241pt}{0.800pt}}
\multiput(609.00,362.34)(0.500,1.000){2}{\rule{0.120pt}{0.800pt}}
\put(610,364.34){\rule{0.482pt}{0.800pt}}
\multiput(610.00,363.34)(1.000,2.000){2}{\rule{0.241pt}{0.800pt}}
\put(612,366.34){\rule{0.482pt}{0.800pt}}
\multiput(612.00,365.34)(1.000,2.000){2}{\rule{0.241pt}{0.800pt}}
\put(614,368.34){\rule{0.482pt}{0.800pt}}
\multiput(614.00,367.34)(1.000,2.000){2}{\rule{0.241pt}{0.800pt}}
\put(616,369.84){\rule{0.241pt}{0.800pt}}
\multiput(616.00,369.34)(0.500,1.000){2}{\rule{0.120pt}{0.800pt}}
\put(617,371.34){\rule{0.482pt}{0.800pt}}
\multiput(617.00,370.34)(1.000,2.000){2}{\rule{0.241pt}{0.800pt}}
\put(619,373.34){\rule{0.482pt}{0.800pt}}
\multiput(619.00,372.34)(1.000,2.000){2}{\rule{0.241pt}{0.800pt}}
\put(621,375.34){\rule{0.482pt}{0.800pt}}
\multiput(621.00,374.34)(1.000,2.000){2}{\rule{0.241pt}{0.800pt}}
\put(623,376.84){\rule{0.241pt}{0.800pt}}
\multiput(623.00,376.34)(0.500,1.000){2}{\rule{0.120pt}{0.800pt}}
\put(624,378.34){\rule{0.482pt}{0.800pt}}
\multiput(624.00,377.34)(1.000,2.000){2}{\rule{0.241pt}{0.800pt}}
\put(626,380.34){\rule{0.482pt}{0.800pt}}
\multiput(626.00,379.34)(1.000,2.000){2}{\rule{0.241pt}{0.800pt}}
\put(628,381.84){\rule{0.241pt}{0.800pt}}
\multiput(628.00,381.34)(0.500,1.000){2}{\rule{0.120pt}{0.800pt}}
\put(629,383.34){\rule{0.482pt}{0.800pt}}
\multiput(629.00,382.34)(1.000,2.000){2}{\rule{0.241pt}{0.800pt}}
\put(631,385.34){\rule{0.482pt}{0.800pt}}
\multiput(631.00,384.34)(1.000,2.000){2}{\rule{0.241pt}{0.800pt}}
\put(633,387.34){\rule{0.482pt}{0.800pt}}
\multiput(633.00,386.34)(1.000,2.000){2}{\rule{0.241pt}{0.800pt}}
\put(635,388.84){\rule{0.241pt}{0.800pt}}
\multiput(635.00,388.34)(0.500,1.000){2}{\rule{0.120pt}{0.800pt}}
\put(636,390.34){\rule{0.482pt}{0.800pt}}
\multiput(636.00,389.34)(1.000,2.000){2}{\rule{0.241pt}{0.800pt}}
\put(638,392.34){\rule{0.482pt}{0.800pt}}
\multiput(638.00,391.34)(1.000,2.000){2}{\rule{0.241pt}{0.800pt}}
\put(638.84,395){\rule{0.800pt}{0.482pt}}
\multiput(638.34,395.00)(1.000,1.000){2}{\rule{0.800pt}{0.241pt}}
\put(641,395.84){\rule{0.482pt}{0.800pt}}
\multiput(641.00,395.34)(1.000,1.000){2}{\rule{0.241pt}{0.800pt}}
\put(643,397.34){\rule{0.482pt}{0.800pt}}
\multiput(643.00,396.34)(1.000,2.000){2}{\rule{0.241pt}{0.800pt}}
\put(645,399.34){\rule{0.482pt}{0.800pt}}
\multiput(645.00,398.34)(1.000,2.000){2}{\rule{0.241pt}{0.800pt}}
\put(645.84,402){\rule{0.800pt}{0.482pt}}
\multiput(645.34,402.00)(1.000,1.000){2}{\rule{0.800pt}{0.241pt}}
\put(648,402.84){\rule{0.482pt}{0.800pt}}
\multiput(648.00,402.34)(1.000,1.000){2}{\rule{0.241pt}{0.800pt}}
\put(650,404.34){\rule{0.482pt}{0.800pt}}
\multiput(650.00,403.34)(1.000,2.000){2}{\rule{0.241pt}{0.800pt}}
\put(652,406.34){\rule{0.482pt}{0.800pt}}
\multiput(652.00,405.34)(1.000,2.000){2}{\rule{0.241pt}{0.800pt}}
\put(652.84,409){\rule{0.800pt}{0.482pt}}
\multiput(652.34,409.00)(1.000,1.000){2}{\rule{0.800pt}{0.241pt}}
\put(655,409.84){\rule{0.482pt}{0.800pt}}
\multiput(655.00,409.34)(1.000,1.000){2}{\rule{0.241pt}{0.800pt}}
\put(657,411.34){\rule{0.482pt}{0.800pt}}
\multiput(657.00,410.34)(1.000,2.000){2}{\rule{0.241pt}{0.800pt}}
\put(657.84,414){\rule{0.800pt}{0.482pt}}
\multiput(657.34,414.00)(1.000,1.000){2}{\rule{0.800pt}{0.241pt}}
\put(660,414.84){\rule{0.482pt}{0.800pt}}
\multiput(660.00,414.34)(1.000,1.000){2}{\rule{0.241pt}{0.800pt}}
\put(662,416.34){\rule{0.482pt}{0.800pt}}
\multiput(662.00,415.34)(1.000,2.000){2}{\rule{0.241pt}{0.800pt}}
\put(664,418.34){\rule{0.482pt}{0.800pt}}
\multiput(664.00,417.34)(1.000,2.000){2}{\rule{0.241pt}{0.800pt}}
\put(664.84,421){\rule{0.800pt}{0.482pt}}
\multiput(664.34,421.00)(1.000,1.000){2}{\rule{0.800pt}{0.241pt}}
\put(667,421.84){\rule{0.482pt}{0.800pt}}
\multiput(667.00,421.34)(1.000,1.000){2}{\rule{0.241pt}{0.800pt}}
\put(669,423.34){\rule{0.482pt}{0.800pt}}
\multiput(669.00,422.34)(1.000,2.000){2}{\rule{0.241pt}{0.800pt}}
\put(671,425.34){\rule{0.482pt}{0.800pt}}
\multiput(671.00,424.34)(1.000,2.000){2}{\rule{0.241pt}{0.800pt}}
\put(671.84,428){\rule{0.800pt}{0.482pt}}
\multiput(671.34,428.00)(1.000,1.000){2}{\rule{0.800pt}{0.241pt}}
\put(674,428.84){\rule{0.482pt}{0.800pt}}
\multiput(674.00,428.34)(1.000,1.000){2}{\rule{0.241pt}{0.800pt}}
\put(676,430.34){\rule{0.482pt}{0.800pt}}
\multiput(676.00,429.34)(1.000,2.000){2}{\rule{0.241pt}{0.800pt}}
\put(676.84,433){\rule{0.800pt}{0.482pt}}
\multiput(676.34,433.00)(1.000,1.000){2}{\rule{0.800pt}{0.241pt}}
\put(679,434.34){\rule{0.482pt}{0.800pt}}
\multiput(679.00,433.34)(1.000,2.000){2}{\rule{0.241pt}{0.800pt}}
\put(681,435.84){\rule{0.482pt}{0.800pt}}
\multiput(681.00,435.34)(1.000,1.000){2}{\rule{0.241pt}{0.800pt}}
\put(683,437.34){\rule{0.482pt}{0.800pt}}
\multiput(683.00,436.34)(1.000,2.000){2}{\rule{0.241pt}{0.800pt}}
\put(683.84,440){\rule{0.800pt}{0.482pt}}
\multiput(683.34,440.00)(1.000,1.000){2}{\rule{0.800pt}{0.241pt}}
\put(686,441.34){\rule{0.482pt}{0.800pt}}
\multiput(686.00,440.34)(1.000,2.000){2}{\rule{0.241pt}{0.800pt}}
\put(688,442.84){\rule{0.482pt}{0.800pt}}
\multiput(688.00,442.34)(1.000,1.000){2}{\rule{0.241pt}{0.800pt}}
\put(688.84,445){\rule{0.800pt}{0.482pt}}
\multiput(688.34,445.00)(1.000,1.000){2}{\rule{0.800pt}{0.241pt}}
\put(691,446.34){\rule{0.482pt}{0.800pt}}
\multiput(691.00,445.34)(1.000,2.000){2}{\rule{0.241pt}{0.800pt}}
\put(693,447.84){\rule{0.482pt}{0.800pt}}
\multiput(693.00,447.34)(1.000,1.000){2}{\rule{0.241pt}{0.800pt}}
\put(695,449.34){\rule{0.482pt}{0.800pt}}
\multiput(695.00,448.34)(1.000,2.000){2}{\rule{0.241pt}{0.800pt}}
\put(695.84,452){\rule{0.800pt}{0.482pt}}
\multiput(695.34,452.00)(1.000,1.000){2}{\rule{0.800pt}{0.241pt}}
\put(698,453.34){\rule{0.482pt}{0.800pt}}
\multiput(698.00,452.34)(1.000,2.000){2}{\rule{0.241pt}{0.800pt}}
\put(700,454.84){\rule{0.482pt}{0.800pt}}
\multiput(700.00,454.34)(1.000,1.000){2}{\rule{0.241pt}{0.800pt}}
\put(702,456.34){\rule{0.482pt}{0.800pt}}
\multiput(702.00,455.34)(1.000,2.000){2}{\rule{0.241pt}{0.800pt}}
\put(702.84,459){\rule{0.800pt}{0.482pt}}
\multiput(702.34,459.00)(1.000,1.000){2}{\rule{0.800pt}{0.241pt}}
\put(705,460.34){\rule{0.482pt}{0.800pt}}
\multiput(705.00,459.34)(1.000,2.000){2}{\rule{0.241pt}{0.800pt}}
\put(707,461.84){\rule{0.482pt}{0.800pt}}
\multiput(707.00,461.34)(1.000,1.000){2}{\rule{0.241pt}{0.800pt}}
\put(707.84,464){\rule{0.800pt}{0.482pt}}
\multiput(707.34,464.00)(1.000,1.000){2}{\rule{0.800pt}{0.241pt}}
\put(710,465.34){\rule{0.482pt}{0.800pt}}
\multiput(710.00,464.34)(1.000,2.000){2}{\rule{0.241pt}{0.800pt}}
\put(712,467.34){\rule{0.482pt}{0.800pt}}
\multiput(712.00,466.34)(1.000,2.000){2}{\rule{0.241pt}{0.800pt}}
\put(714,468.84){\rule{0.482pt}{0.800pt}}
\multiput(714.00,468.34)(1.000,1.000){2}{\rule{0.241pt}{0.800pt}}
\put(714.84,471){\rule{0.800pt}{0.482pt}}
\multiput(714.34,471.00)(1.000,1.000){2}{\rule{0.800pt}{0.241pt}}
\put(717,472.34){\rule{0.482pt}{0.800pt}}
\multiput(717.00,471.34)(1.000,2.000){2}{\rule{0.241pt}{0.800pt}}
\put(719,473.84){\rule{0.482pt}{0.800pt}}
\multiput(719.00,473.34)(1.000,1.000){2}{\rule{0.241pt}{0.800pt}}
\put(719.84,476){\rule{0.800pt}{0.482pt}}
\multiput(719.34,476.00)(1.000,1.000){2}{\rule{0.800pt}{0.241pt}}
\put(722,477.34){\rule{0.482pt}{0.800pt}}
\multiput(722.00,476.34)(1.000,2.000){2}{\rule{0.241pt}{0.800pt}}
\put(724,479.34){\rule{0.482pt}{0.800pt}}
\multiput(724.00,478.34)(1.000,2.000){2}{\rule{0.241pt}{0.800pt}}
\put(726,480.84){\rule{0.482pt}{0.800pt}}
\multiput(726.00,480.34)(1.000,1.000){2}{\rule{0.241pt}{0.800pt}}
\put(726.84,483){\rule{0.800pt}{0.482pt}}
\multiput(726.34,483.00)(1.000,1.000){2}{\rule{0.800pt}{0.241pt}}
\put(729,484.34){\rule{0.482pt}{0.800pt}}
\multiput(729.00,483.34)(1.000,2.000){2}{\rule{0.241pt}{0.800pt}}
\put(731,486.34){\rule{0.482pt}{0.800pt}}
\multiput(731.00,485.34)(1.000,2.000){2}{\rule{0.241pt}{0.800pt}}
\put(733,487.84){\rule{0.482pt}{0.800pt}}
\multiput(733.00,487.34)(1.000,1.000){2}{\rule{0.241pt}{0.800pt}}
\put(733.84,490){\rule{0.800pt}{0.482pt}}
\multiput(733.34,490.00)(1.000,1.000){2}{\rule{0.800pt}{0.241pt}}
\put(399.0,155.0){\usebox{\plotpoint}}
\sbox{\plotpoint}{\rule[-0.200pt]{0.400pt}{0.400pt}}%
\put(221.0,143.0){\rule[-0.200pt]{124.545pt}{0.400pt}}
\put(738.0,143.0){\rule[-0.200pt]{0.400pt}{104.551pt}}
\put(221.0,577.0){\rule[-0.200pt]{124.545pt}{0.400pt}}
\put(221.0,143.0){\rule[-0.200pt]{0.400pt}{104.551pt}}
\end{picture}

      \end{center}
\caption{Output of algorithm \adwinz with abrupt change.} 
  \label{fig:ADWIN-E}
\end{figure}
By part (2) of Theorem \ref{ThBV} and Equation \ref{Enewepsilon}, one can derive that 
the window will start shrinking after $O(\mu \ln(1/\delta) / \epsilon^2)$ steps, and 
in fact will be shrunk to the point where only $O(\mu \ln(1/\delta) / \epsilon^2)$ examples
prior to the change are left. From then on, if no further changes occur, no more examples
will be dropped so the window will expand unboundedly.


In case of a gradual change with slope $\alpha$ following
a long stationary period at $\mu$, 
\begin{figure}[t]
	\begin{center}
		<<<<<<< HEAD
% GNUPLOT: LaTeX picture
\setlength{\unitlength}{0.240900pt}
\ifx\plotpoint\undefined\newsavebox{\plotpoint}\fi
\begin{picture}(959,720)(0,0)
\sbox{\plotpoint}{\rule[-0.200pt]{0.400pt}{0.400pt}}%
\put(181,143){\makebox(0,0)[r]{$0.1$}}
\put(201.0,143.0){\rule[-0.200pt]{4.818pt}{0.400pt}}
\put(181,197){\makebox(0,0)[r]{$0.2$}}
\put(201.0,197.0){\rule[-0.200pt]{4.818pt}{0.400pt}}
\put(181,252){\makebox(0,0)[r]{$0.3$}}
\put(201.0,252.0){\rule[-0.200pt]{4.818pt}{0.400pt}}
\put(181,306){\makebox(0,0)[r]{$0.4$}}
\put(201.0,306.0){\rule[-0.200pt]{4.818pt}{0.400pt}}
\put(181,360){\makebox(0,0)[r]{$0.5$}}
\put(201.0,360.0){\rule[-0.200pt]{4.818pt}{0.400pt}}
\put(181,414){\makebox(0,0)[r]{$0.6$}}
\put(201.0,414.0){\rule[-0.200pt]{4.818pt}{0.400pt}}
\put(181,468){\makebox(0,0)[r]{$0.7$}}
\put(201.0,468.0){\rule[-0.200pt]{4.818pt}{0.400pt}}
\put(181,523){\makebox(0,0)[r]{$0.8$}}
\put(201.0,523.0){\rule[-0.200pt]{4.818pt}{0.400pt}}
\put(181,577){\makebox(0,0)[r]{$0.9$}}
\put(201.0,577.0){\rule[-0.200pt]{4.818pt}{0.400pt}}
\put(221.0,123.0){\rule[-0.200pt]{0.400pt}{4.818pt}}
\put(221,82){\makebox(0,0){$0$}}
\put(221.0,577.0){\rule[-0.200pt]{0.400pt}{4.818pt}}
\put(286.0,123.0){\rule[-0.200pt]{0.400pt}{4.818pt}}
%\put(286,82){\makebox(0,0){$500$}}
\put(286.0,577.0){\rule[-0.200pt]{0.400pt}{4.818pt}}
\put(350.0,123.0){\rule[-0.200pt]{0.400pt}{4.818pt}}
\put(350,82){\makebox(0,0){$1000$}}
\put(350.0,577.0){\rule[-0.200pt]{0.400pt}{4.818pt}}
\put(415.0,123.0){\rule[-0.200pt]{0.400pt}{4.818pt}}
%\put(415,82){\makebox(0,0){$1500$}}
\put(415.0,577.0){\rule[-0.200pt]{0.400pt}{4.818pt}}
\put(479.0,123.0){\rule[-0.200pt]{0.400pt}{4.818pt}}
\put(479,82){\makebox(0,0){$2000$}}
\put(479.0,577.0){\rule[-0.200pt]{0.400pt}{4.818pt}}
\put(544.0,123.0){\rule[-0.200pt]{0.400pt}{4.818pt}}
%\put(544,82){\makebox(0,0){$2500$}}
\put(544.0,577.0){\rule[-0.200pt]{0.400pt}{4.818pt}}
\put(609.0,123.0){\rule[-0.200pt]{0.400pt}{4.818pt}}
\put(609,82){\makebox(0,0){$3000$}}
\put(609.0,577.0){\rule[-0.200pt]{0.400pt}{4.818pt}}
\put(673.0,123.0){\rule[-0.200pt]{0.400pt}{4.818pt}}
%\put(673,82){\makebox(0,0){$3500$}}
\put(673.0,577.0){\rule[-0.200pt]{0.400pt}{4.818pt}}
\put(738.0,123.0){\rule[-0.200pt]{0.400pt}{4.818pt}}
\put(738,82){\makebox(0,0){$4000$}}
\put(738.0,577.0){\rule[-0.200pt]{0.400pt}{4.818pt}}
\put(778,143){\makebox(0,0)[l]{ 0}}
\put(738.0,143.0){\rule[-0.200pt]{4.818pt}{0.400pt}}
\put(778,230){\makebox(0,0)[l]{ 500}}
\put(738.0,230.0){\rule[-0.200pt]{4.818pt}{0.400pt}}
\put(778,317){\makebox(0,0)[l]{ 1000}}
\put(738.0,317.0){\rule[-0.200pt]{4.818pt}{0.400pt}}
\put(778,403){\makebox(0,0)[l]{ 1500}}
\put(738.0,403.0){\rule[-0.200pt]{4.818pt}{0.400pt}}
\put(778,490){\makebox(0,0)[l]{ 2000}}
\put(738.0,490.0){\rule[-0.200pt]{4.818pt}{0.400pt}}
\put(778,577){\makebox(0,0)[l]{ 2500}}
\put(738.0,577.0){\rule[-0.200pt]{4.818pt}{0.400pt}}
\put(221.0,143.0){\rule[-0.200pt]{124.545pt}{0.400pt}}
\put(738.0,143.0){\rule[-0.200pt]{0.400pt}{104.551pt}}
\put(221.0,577.0){\rule[-0.200pt]{124.545pt}{0.400pt}}
\put(221.0,143.0){\rule[-0.200pt]{0.400pt}{104.551pt}}
\put(40,360){\makebox(0,0){$\mu$ axis}}
\put(967,360){\makebox(0,0){Width}}
\put(479,21){\makebox(0,0){$t$ axis}}
\put(479,659){\makebox(0,0){\adwinz}}
\put(578,537){\makebox(0,0)[r]{$\mu_t$}}
\put(598.0,537.0){\rule[-0.200pt]{24.090pt}{0.400pt}}
\put(222,523){\usebox{\plotpoint}}
\put(349.67,520){\rule{0.400pt}{0.723pt}}
\multiput(349.17,521.50)(1.000,-1.500){2}{\rule{0.400pt}{0.361pt}}
\put(351.17,516){\rule{0.400pt}{0.900pt}}
\multiput(350.17,518.13)(2.000,-2.132){2}{\rule{0.400pt}{0.450pt}}
\put(352.67,513){\rule{0.400pt}{0.723pt}}
\multiput(352.17,514.50)(1.000,-1.500){2}{\rule{0.400pt}{0.361pt}}
\put(353.67,510){\rule{0.400pt}{0.723pt}}
\multiput(353.17,511.50)(1.000,-1.500){2}{\rule{0.400pt}{0.361pt}}
\put(355.17,507){\rule{0.400pt}{0.700pt}}
\multiput(354.17,508.55)(2.000,-1.547){2}{\rule{0.400pt}{0.350pt}}
\put(356.67,504){\rule{0.400pt}{0.723pt}}
\multiput(356.17,505.50)(1.000,-1.500){2}{\rule{0.400pt}{0.361pt}}
\put(357.67,500){\rule{0.400pt}{0.964pt}}
\multiput(357.17,502.00)(1.000,-2.000){2}{\rule{0.400pt}{0.482pt}}
\put(358.67,497){\rule{0.400pt}{0.723pt}}
\multiput(358.17,498.50)(1.000,-1.500){2}{\rule{0.400pt}{0.361pt}}
\put(360.17,494){\rule{0.400pt}{0.700pt}}
\multiput(359.17,495.55)(2.000,-1.547){2}{\rule{0.400pt}{0.350pt}}
\put(361.67,491){\rule{0.400pt}{0.723pt}}
\multiput(361.17,492.50)(1.000,-1.500){2}{\rule{0.400pt}{0.361pt}}
\put(362.67,488){\rule{0.400pt}{0.723pt}}
\multiput(362.17,489.50)(1.000,-1.500){2}{\rule{0.400pt}{0.361pt}}
\put(364.17,485){\rule{0.400pt}{0.700pt}}
\multiput(363.17,486.55)(2.000,-1.547){2}{\rule{0.400pt}{0.350pt}}
\put(365.67,481){\rule{0.400pt}{0.964pt}}
\multiput(365.17,483.00)(1.000,-2.000){2}{\rule{0.400pt}{0.482pt}}
\put(366.67,478){\rule{0.400pt}{0.723pt}}
\multiput(366.17,479.50)(1.000,-1.500){2}{\rule{0.400pt}{0.361pt}}
\put(368.17,475){\rule{0.400pt}{0.700pt}}
\multiput(367.17,476.55)(2.000,-1.547){2}{\rule{0.400pt}{0.350pt}}
\put(369.67,472){\rule{0.400pt}{0.723pt}}
\multiput(369.17,473.50)(1.000,-1.500){2}{\rule{0.400pt}{0.361pt}}
\put(370.67,469){\rule{0.400pt}{0.723pt}}
\multiput(370.17,470.50)(1.000,-1.500){2}{\rule{0.400pt}{0.361pt}}
\put(371.67,466){\rule{0.400pt}{0.723pt}}
\multiput(371.17,467.50)(1.000,-1.500){2}{\rule{0.400pt}{0.361pt}}
\put(373.17,462){\rule{0.400pt}{0.900pt}}
\multiput(372.17,464.13)(2.000,-2.132){2}{\rule{0.400pt}{0.450pt}}
\put(374.67,459){\rule{0.400pt}{0.723pt}}
\multiput(374.17,460.50)(1.000,-1.500){2}{\rule{0.400pt}{0.361pt}}
\put(375.67,456){\rule{0.400pt}{0.723pt}}
\multiput(375.17,457.50)(1.000,-1.500){2}{\rule{0.400pt}{0.361pt}}
\put(377.17,453){\rule{0.400pt}{0.700pt}}
\multiput(376.17,454.55)(2.000,-1.547){2}{\rule{0.400pt}{0.350pt}}
\put(378.67,450){\rule{0.400pt}{0.723pt}}
\multiput(378.17,451.50)(1.000,-1.500){2}{\rule{0.400pt}{0.361pt}}
\put(379.67,446){\rule{0.400pt}{0.964pt}}
\multiput(379.17,448.00)(1.000,-2.000){2}{\rule{0.400pt}{0.482pt}}
\put(380.67,443){\rule{0.400pt}{0.723pt}}
\multiput(380.17,444.50)(1.000,-1.500){2}{\rule{0.400pt}{0.361pt}}
\put(382.17,440){\rule{0.400pt}{0.700pt}}
\multiput(381.17,441.55)(2.000,-1.547){2}{\rule{0.400pt}{0.350pt}}
\put(383.67,437){\rule{0.400pt}{0.723pt}}
\multiput(383.17,438.50)(1.000,-1.500){2}{\rule{0.400pt}{0.361pt}}
\put(384.67,434){\rule{0.400pt}{0.723pt}}
\multiput(384.17,435.50)(1.000,-1.500){2}{\rule{0.400pt}{0.361pt}}
\put(386.17,431){\rule{0.400pt}{0.700pt}}
\multiput(385.17,432.55)(2.000,-1.547){2}{\rule{0.400pt}{0.350pt}}
\put(387.67,427){\rule{0.400pt}{0.964pt}}
\multiput(387.17,429.00)(1.000,-2.000){2}{\rule{0.400pt}{0.482pt}}
\put(388.67,424){\rule{0.400pt}{0.723pt}}
\multiput(388.17,425.50)(1.000,-1.500){2}{\rule{0.400pt}{0.361pt}}
\put(390.17,421){\rule{0.400pt}{0.700pt}}
\multiput(389.17,422.55)(2.000,-1.547){2}{\rule{0.400pt}{0.350pt}}
\put(391.67,418){\rule{0.400pt}{0.723pt}}
\multiput(391.17,419.50)(1.000,-1.500){2}{\rule{0.400pt}{0.361pt}}
\put(392.67,415){\rule{0.400pt}{0.723pt}}
\multiput(392.17,416.50)(1.000,-1.500){2}{\rule{0.400pt}{0.361pt}}
\put(393.67,411){\rule{0.400pt}{0.964pt}}
\multiput(393.17,413.00)(1.000,-2.000){2}{\rule{0.400pt}{0.482pt}}
\put(395.17,408){\rule{0.400pt}{0.700pt}}
\multiput(394.17,409.55)(2.000,-1.547){2}{\rule{0.400pt}{0.350pt}}
\put(396.67,405){\rule{0.400pt}{0.723pt}}
\multiput(396.17,406.50)(1.000,-1.500){2}{\rule{0.400pt}{0.361pt}}
\put(397.67,402){\rule{0.400pt}{0.723pt}}
\multiput(397.17,403.50)(1.000,-1.500){2}{\rule{0.400pt}{0.361pt}}
\put(399.17,399){\rule{0.400pt}{0.700pt}}
\multiput(398.17,400.55)(2.000,-1.547){2}{\rule{0.400pt}{0.350pt}}
\put(400.67,396){\rule{0.400pt}{0.723pt}}
\multiput(400.17,397.50)(1.000,-1.500){2}{\rule{0.400pt}{0.361pt}}
\put(401.67,392){\rule{0.400pt}{0.964pt}}
\multiput(401.17,394.00)(1.000,-2.000){2}{\rule{0.400pt}{0.482pt}}
\put(402.67,389){\rule{0.400pt}{0.723pt}}
\multiput(402.17,390.50)(1.000,-1.500){2}{\rule{0.400pt}{0.361pt}}
\put(404.17,386){\rule{0.400pt}{0.700pt}}
\multiput(403.17,387.55)(2.000,-1.547){2}{\rule{0.400pt}{0.350pt}}
\put(405.67,383){\rule{0.400pt}{0.723pt}}
\multiput(405.17,384.50)(1.000,-1.500){2}{\rule{0.400pt}{0.361pt}}
\put(406.67,380){\rule{0.400pt}{0.723pt}}
\multiput(406.17,381.50)(1.000,-1.500){2}{\rule{0.400pt}{0.361pt}}
\put(408.17,377){\rule{0.400pt}{0.700pt}}
\multiput(407.17,378.55)(2.000,-1.547){2}{\rule{0.400pt}{0.350pt}}
\put(409.67,373){\rule{0.400pt}{0.964pt}}
\multiput(409.17,375.00)(1.000,-2.000){2}{\rule{0.400pt}{0.482pt}}
\put(410.67,370){\rule{0.400pt}{0.723pt}}
\multiput(410.17,371.50)(1.000,-1.500){2}{\rule{0.400pt}{0.361pt}}
\put(412.17,367){\rule{0.400pt}{0.700pt}}
\multiput(411.17,368.55)(2.000,-1.547){2}{\rule{0.400pt}{0.350pt}}
\put(413.67,364){\rule{0.400pt}{0.723pt}}
\multiput(413.17,365.50)(1.000,-1.500){2}{\rule{0.400pt}{0.361pt}}
\put(414.67,361){\rule{0.400pt}{0.723pt}}
\multiput(414.17,362.50)(1.000,-1.500){2}{\rule{0.400pt}{0.361pt}}
\put(415.67,357){\rule{0.400pt}{0.964pt}}
\multiput(415.17,359.00)(1.000,-2.000){2}{\rule{0.400pt}{0.482pt}}
\put(417.17,354){\rule{0.400pt}{0.700pt}}
\multiput(416.17,355.55)(2.000,-1.547){2}{\rule{0.400pt}{0.350pt}}
\put(418.67,351){\rule{0.400pt}{0.723pt}}
\multiput(418.17,352.50)(1.000,-1.500){2}{\rule{0.400pt}{0.361pt}}
\put(419.67,348){\rule{0.400pt}{0.723pt}}
\multiput(419.17,349.50)(1.000,-1.500){2}{\rule{0.400pt}{0.361pt}}
\put(421.17,345){\rule{0.400pt}{0.700pt}}
\multiput(420.17,346.55)(2.000,-1.547){2}{\rule{0.400pt}{0.350pt}}
\put(422.67,342){\rule{0.400pt}{0.723pt}}
\multiput(422.17,343.50)(1.000,-1.500){2}{\rule{0.400pt}{0.361pt}}
\put(423.67,338){\rule{0.400pt}{0.964pt}}
\multiput(423.17,340.00)(1.000,-2.000){2}{\rule{0.400pt}{0.482pt}}
\put(424.67,335){\rule{0.400pt}{0.723pt}}
\multiput(424.17,336.50)(1.000,-1.500){2}{\rule{0.400pt}{0.361pt}}
\put(426.17,332){\rule{0.400pt}{0.700pt}}
\multiput(425.17,333.55)(2.000,-1.547){2}{\rule{0.400pt}{0.350pt}}
\put(427.67,329){\rule{0.400pt}{0.723pt}}
\multiput(427.17,330.50)(1.000,-1.500){2}{\rule{0.400pt}{0.361pt}}
\put(428.67,326){\rule{0.400pt}{0.723pt}}
\multiput(428.17,327.50)(1.000,-1.500){2}{\rule{0.400pt}{0.361pt}}
\put(430.17,322){\rule{0.400pt}{0.900pt}}
\multiput(429.17,324.13)(2.000,-2.132){2}{\rule{0.400pt}{0.450pt}}
\put(431.67,319){\rule{0.400pt}{0.723pt}}
\multiput(431.17,320.50)(1.000,-1.500){2}{\rule{0.400pt}{0.361pt}}
\put(432.67,316){\rule{0.400pt}{0.723pt}}
\multiput(432.17,317.50)(1.000,-1.500){2}{\rule{0.400pt}{0.361pt}}
\put(433.67,313){\rule{0.400pt}{0.723pt}}
\multiput(433.17,314.50)(1.000,-1.500){2}{\rule{0.400pt}{0.361pt}}
\put(435.17,310){\rule{0.400pt}{0.700pt}}
\multiput(434.17,311.55)(2.000,-1.547){2}{\rule{0.400pt}{0.350pt}}
\put(436.67,307){\rule{0.400pt}{0.723pt}}
\multiput(436.17,308.50)(1.000,-1.500){2}{\rule{0.400pt}{0.361pt}}
\put(437.67,303){\rule{0.400pt}{0.964pt}}
\multiput(437.17,305.00)(1.000,-2.000){2}{\rule{0.400pt}{0.482pt}}
\put(439.17,300){\rule{0.400pt}{0.700pt}}
\multiput(438.17,301.55)(2.000,-1.547){2}{\rule{0.400pt}{0.350pt}}
\put(440.67,297){\rule{0.400pt}{0.723pt}}
\multiput(440.17,298.50)(1.000,-1.500){2}{\rule{0.400pt}{0.361pt}}
\put(441.67,294){\rule{0.400pt}{0.723pt}}
\multiput(441.17,295.50)(1.000,-1.500){2}{\rule{0.400pt}{0.361pt}}
\put(443.17,291){\rule{0.400pt}{0.700pt}}
\multiput(442.17,292.55)(2.000,-1.547){2}{\rule{0.400pt}{0.350pt}}
\put(444.67,288){\rule{0.400pt}{0.723pt}}
\multiput(444.17,289.50)(1.000,-1.500){2}{\rule{0.400pt}{0.361pt}}
\put(445.67,284){\rule{0.400pt}{0.964pt}}
\multiput(445.17,286.00)(1.000,-2.000){2}{\rule{0.400pt}{0.482pt}}
\put(446.67,281){\rule{0.400pt}{0.723pt}}
\multiput(446.17,282.50)(1.000,-1.500){2}{\rule{0.400pt}{0.361pt}}
\put(448.17,278){\rule{0.400pt}{0.700pt}}
\multiput(447.17,279.55)(2.000,-1.547){2}{\rule{0.400pt}{0.350pt}}
\put(449.67,275){\rule{0.400pt}{0.723pt}}
\multiput(449.17,276.50)(1.000,-1.500){2}{\rule{0.400pt}{0.361pt}}
\put(450.67,272){\rule{0.400pt}{0.723pt}}
\multiput(450.17,273.50)(1.000,-1.500){2}{\rule{0.400pt}{0.361pt}}
\put(452.17,268){\rule{0.400pt}{0.900pt}}
\multiput(451.17,270.13)(2.000,-2.132){2}{\rule{0.400pt}{0.450pt}}
\put(453.67,265){\rule{0.400pt}{0.723pt}}
\multiput(453.17,266.50)(1.000,-1.500){2}{\rule{0.400pt}{0.361pt}}
\put(454.67,262){\rule{0.400pt}{0.723pt}}
\multiput(454.17,263.50)(1.000,-1.500){2}{\rule{0.400pt}{0.361pt}}
\put(455.67,259){\rule{0.400pt}{0.723pt}}
\multiput(455.17,260.50)(1.000,-1.500){2}{\rule{0.400pt}{0.361pt}}
\put(457.17,256){\rule{0.400pt}{0.700pt}}
\multiput(456.17,257.55)(2.000,-1.547){2}{\rule{0.400pt}{0.350pt}}
\put(458.67,253){\rule{0.400pt}{0.723pt}}
\multiput(458.17,254.50)(1.000,-1.500){2}{\rule{0.400pt}{0.361pt}}
\put(459.67,249){\rule{0.400pt}{0.964pt}}
\multiput(459.17,251.00)(1.000,-2.000){2}{\rule{0.400pt}{0.482pt}}
\put(461.17,246){\rule{0.400pt}{0.700pt}}
\multiput(460.17,247.55)(2.000,-1.547){2}{\rule{0.400pt}{0.350pt}}
\put(462.67,243){\rule{0.400pt}{0.723pt}}
\multiput(462.17,244.50)(1.000,-1.500){2}{\rule{0.400pt}{0.361pt}}
\put(463.67,240){\rule{0.400pt}{0.723pt}}
\multiput(463.17,241.50)(1.000,-1.500){2}{\rule{0.400pt}{0.361pt}}
\put(465.17,237){\rule{0.400pt}{0.700pt}}
\multiput(464.17,238.55)(2.000,-1.547){2}{\rule{0.400pt}{0.350pt}}
\put(466.67,233){\rule{0.400pt}{0.964pt}}
\multiput(466.17,235.00)(1.000,-2.000){2}{\rule{0.400pt}{0.482pt}}
\put(467.67,230){\rule{0.400pt}{0.723pt}}
\multiput(467.17,231.50)(1.000,-1.500){2}{\rule{0.400pt}{0.361pt}}
\put(468.67,227){\rule{0.400pt}{0.723pt}}
\multiput(468.17,228.50)(1.000,-1.500){2}{\rule{0.400pt}{0.361pt}}
\put(470.17,224){\rule{0.400pt}{0.700pt}}
\multiput(469.17,225.55)(2.000,-1.547){2}{\rule{0.400pt}{0.350pt}}
\put(471.67,221){\rule{0.400pt}{0.723pt}}
\multiput(471.17,222.50)(1.000,-1.500){2}{\rule{0.400pt}{0.361pt}}
\put(472.67,218){\rule{0.400pt}{0.723pt}}
\multiput(472.17,219.50)(1.000,-1.500){2}{\rule{0.400pt}{0.361pt}}
\put(474.17,214){\rule{0.400pt}{0.900pt}}
\multiput(473.17,216.13)(2.000,-2.132){2}{\rule{0.400pt}{0.450pt}}
\put(475.67,211){\rule{0.400pt}{0.723pt}}
\multiput(475.17,212.50)(1.000,-1.500){2}{\rule{0.400pt}{0.361pt}}
\put(476.67,208){\rule{0.400pt}{0.723pt}}
\multiput(476.17,209.50)(1.000,-1.500){2}{\rule{0.400pt}{0.361pt}}
\put(477.67,205){\rule{0.400pt}{0.723pt}}
\multiput(477.17,206.50)(1.000,-1.500){2}{\rule{0.400pt}{0.361pt}}
\put(479.17,202){\rule{0.400pt}{0.700pt}}
\multiput(478.17,203.55)(2.000,-1.547){2}{\rule{0.400pt}{0.350pt}}
\put(480.67,199){\rule{0.400pt}{0.723pt}}
\multiput(480.17,200.50)(1.000,-1.500){2}{\rule{0.400pt}{0.361pt}}
\put(481.67,197){\rule{0.400pt}{0.482pt}}
\multiput(481.17,198.00)(1.000,-1.000){2}{\rule{0.400pt}{0.241pt}}
\put(222.0,523.0){\rule[-0.200pt]{30.835pt}{0.400pt}}
\put(483.0,197.0){\rule[-0.200pt]{61.189pt}{0.400pt}}
\put(578,496){\makebox(0,0)[r]{$\hat{\mu}_W$}}
\multiput(598,496)(20.756,0.000){5}{\usebox{\plotpoint}}
\put(698,496){\usebox{\plotpoint}}
\put(222,577){\usebox{\plotpoint}}
\multiput(222,577)(0.768,-20.741){2}{\usebox{\plotpoint}}
\put(225.58,546.24){\usebox{\plotpoint}}
\put(227.67,533.33){\usebox{\plotpoint}}
\put(231.15,549.31){\usebox{\plotpoint}}
\put(235.97,541.16){\usebox{\plotpoint}}
\put(243.34,543.72){\usebox{\plotpoint}}
\put(248.13,542.62){\usebox{\plotpoint}}
\put(257.25,535.01){\usebox{\plotpoint}}
\put(269.59,526.00){\usebox{\plotpoint}}
\put(284.51,524.00){\usebox{\plotpoint}}
\put(299.64,522.18){\usebox{\plotpoint}}
\put(315.66,528.34){\usebox{\plotpoint}}
\put(334.02,529.49){\usebox{\plotpoint}}
\put(352.12,529.00){\usebox{\plotpoint}}
\put(370.11,528.00){\usebox{\plotpoint}}
\put(376.55,511.80){\usebox{\plotpoint}}
\put(377.99,491.10){\usebox{\plotpoint}}
\put(383.36,473.00){\usebox{\plotpoint}}
\put(389.55,454.17){\usebox{\plotpoint}}
\put(391.94,433.56){\usebox{\plotpoint}}
\put(398.50,430.49){\usebox{\plotpoint}}
\put(405.81,431.58){\usebox{\plotpoint}}
\put(416.90,419.20){\usebox{\plotpoint}}
\put(424.57,410.00){\usebox{\plotpoint}}
\put(437.60,395.20){\usebox{\plotpoint}}
\put(448.42,382.33){\usebox{\plotpoint}}
\put(454.50,363.50){\usebox{\plotpoint}}
\put(456.52,343.03){\usebox{\plotpoint}}
\put(460.30,323.76){\usebox{\plotpoint}}
\put(464.47,303.70){\usebox{\plotpoint}}
\put(472.68,292.27){\usebox{\plotpoint}}
\put(478.76,275.46){\usebox{\plotpoint}}
\put(485.01,255.75){\usebox{\plotpoint}}
\put(486.28,235.03){\usebox{\plotpoint}}
\put(493.88,220.00){\usebox{\plotpoint}}
\put(507.31,220.07){\usebox{\plotpoint}}
\put(519.97,214.00){\usebox{\plotpoint}}
\put(533.96,212.98){\usebox{\plotpoint}}
\put(550.17,211.00){\usebox{\plotpoint}}
\put(567.28,210.86){\usebox{\plotpoint}}
\put(585.47,205.76){\usebox{\plotpoint}}
\put(604.98,206.00){\usebox{\plotpoint}}
\put(622.42,199.58){\usebox{\plotpoint}}
\put(641.45,196.55){\usebox{\plotpoint}}
\put(660.50,194.00){\usebox{\plotpoint}}
\put(679.54,195.00){\usebox{\plotpoint}}
\put(698.82,195.00){\usebox{\plotpoint}}
\put(718.10,195.00){\usebox{\plotpoint}}
\put(737,194){\usebox{\plotpoint}}
\sbox{\plotpoint}{\rule[-0.400pt]{0.800pt}{0.800pt}}%
\sbox{\plotpoint}{\rule[-0.200pt]{0.400pt}{0.400pt}}%
\put(578,455){\makebox(0,0)[r]{$W$}}
\sbox{\plotpoint}{\rule[-0.400pt]{0.800pt}{0.800pt}}%
\put(598.0,455.0){\rule[-0.400pt]{24.090pt}{0.800pt}}
\put(222,145){\usebox{\plotpoint}}
\put(222,143.84){\rule{0.241pt}{0.800pt}}
\multiput(222.00,143.34)(0.500,1.000){2}{\rule{0.120pt}{0.800pt}}
\put(223,145.34){\rule{0.482pt}{0.800pt}}
\multiput(223.00,144.34)(1.000,2.000){2}{\rule{0.241pt}{0.800pt}}
\put(223.84,148){\rule{0.800pt}{0.482pt}}
\multiput(223.34,148.00)(1.000,1.000){2}{\rule{0.800pt}{0.241pt}}
\put(224.84,150){\rule{0.800pt}{0.482pt}}
\multiput(224.34,150.00)(1.000,1.000){2}{\rule{0.800pt}{0.241pt}}
\put(227,150.84){\rule{0.482pt}{0.800pt}}
\multiput(227.00,150.34)(1.000,1.000){2}{\rule{0.241pt}{0.800pt}}
\put(227.84,153){\rule{0.800pt}{0.482pt}}
\multiput(227.34,153.00)(1.000,1.000){2}{\rule{0.800pt}{0.241pt}}
\put(228.84,155){\rule{0.800pt}{0.482pt}}
\multiput(228.34,155.00)(1.000,1.000){2}{\rule{0.800pt}{0.241pt}}
\put(231,156.34){\rule{0.482pt}{0.800pt}}
\multiput(231.00,155.34)(1.000,2.000){2}{\rule{0.241pt}{0.800pt}}
\put(233,157.84){\rule{0.241pt}{0.800pt}}
\multiput(233.00,157.34)(0.500,1.000){2}{\rule{0.120pt}{0.800pt}}
\put(232.84,160){\rule{0.800pt}{0.482pt}}
\multiput(232.34,160.00)(1.000,1.000){2}{\rule{0.800pt}{0.241pt}}
\put(233.84,162){\rule{0.800pt}{0.482pt}}
\multiput(233.34,162.00)(1.000,1.000){2}{\rule{0.800pt}{0.241pt}}
\put(236,163.34){\rule{0.482pt}{0.800pt}}
\multiput(236.00,162.34)(1.000,2.000){2}{\rule{0.241pt}{0.800pt}}
\put(238,164.84){\rule{0.241pt}{0.800pt}}
\multiput(238.00,164.34)(0.500,1.000){2}{\rule{0.120pt}{0.800pt}}
\put(237.84,167){\rule{0.800pt}{0.482pt}}
\multiput(237.34,167.00)(1.000,1.000){2}{\rule{0.800pt}{0.241pt}}
\put(240,168.34){\rule{0.482pt}{0.800pt}}
\multiput(240.00,167.34)(1.000,2.000){2}{\rule{0.241pt}{0.800pt}}
\put(240.84,171){\rule{0.800pt}{0.482pt}}
\multiput(240.34,171.00)(1.000,1.000){2}{\rule{0.800pt}{0.241pt}}
\put(243,171.84){\rule{0.241pt}{0.800pt}}
\multiput(243.00,171.34)(0.500,1.000){2}{\rule{0.120pt}{0.800pt}}
\put(242.84,174){\rule{0.800pt}{0.482pt}}
\multiput(242.34,174.00)(1.000,1.000){2}{\rule{0.800pt}{0.241pt}}
\put(245,175.34){\rule{0.482pt}{0.800pt}}
\multiput(245.00,174.34)(1.000,2.000){2}{\rule{0.241pt}{0.800pt}}
\put(247,176.84){\rule{0.241pt}{0.800pt}}
\multiput(247.00,176.34)(0.500,1.000){2}{\rule{0.120pt}{0.800pt}}
\put(246.84,179){\rule{0.800pt}{0.482pt}}
\multiput(246.34,179.00)(1.000,1.000){2}{\rule{0.800pt}{0.241pt}}
\put(249,180.34){\rule{0.482pt}{0.800pt}}
\multiput(249.00,179.34)(1.000,2.000){2}{\rule{0.241pt}{0.800pt}}
\put(249.84,183){\rule{0.800pt}{0.482pt}}
\multiput(249.34,183.00)(1.000,1.000){2}{\rule{0.800pt}{0.241pt}}
\put(252,183.84){\rule{0.241pt}{0.800pt}}
\multiput(252.00,183.34)(0.500,1.000){2}{\rule{0.120pt}{0.800pt}}
\put(251.84,186){\rule{0.800pt}{0.482pt}}
\multiput(251.34,186.00)(1.000,1.000){2}{\rule{0.800pt}{0.241pt}}
\put(254,187.34){\rule{0.482pt}{0.800pt}}
\multiput(254.00,186.34)(1.000,2.000){2}{\rule{0.241pt}{0.800pt}}
\put(254.84,190){\rule{0.800pt}{0.482pt}}
\multiput(254.34,190.00)(1.000,1.000){2}{\rule{0.800pt}{0.241pt}}
\put(257,190.84){\rule{0.241pt}{0.800pt}}
\multiput(257.00,190.34)(0.500,1.000){2}{\rule{0.120pt}{0.800pt}}
\put(258,192.34){\rule{0.482pt}{0.800pt}}
\multiput(258.00,191.34)(1.000,2.000){2}{\rule{0.241pt}{0.800pt}}
\put(258.84,195){\rule{0.800pt}{0.482pt}}
\multiput(258.34,195.00)(1.000,1.000){2}{\rule{0.800pt}{0.241pt}}
\put(259.84,197){\rule{0.800pt}{0.482pt}}
\multiput(259.34,197.00)(1.000,1.000){2}{\rule{0.800pt}{0.241pt}}
\put(262,197.84){\rule{0.482pt}{0.800pt}}
\multiput(262.00,197.34)(1.000,1.000){2}{\rule{0.241pt}{0.800pt}}
\put(262.84,200){\rule{0.800pt}{0.482pt}}
\multiput(262.34,200.00)(1.000,1.000){2}{\rule{0.800pt}{0.241pt}}
\put(263.84,202){\rule{0.800pt}{0.482pt}}
\multiput(263.34,202.00)(1.000,1.000){2}{\rule{0.800pt}{0.241pt}}
\put(266,202.84){\rule{0.241pt}{0.800pt}}
\multiput(266.00,202.34)(0.500,1.000){2}{\rule{0.120pt}{0.800pt}}
\put(267,204.34){\rule{0.482pt}{0.800pt}}
\multiput(267.00,203.34)(1.000,2.000){2}{\rule{0.241pt}{0.800pt}}
\put(267.84,207){\rule{0.800pt}{0.482pt}}
\multiput(267.34,207.00)(1.000,1.000){2}{\rule{0.800pt}{0.241pt}}
\put(268.84,209){\rule{0.800pt}{0.482pt}}
\multiput(268.34,209.00)(1.000,1.000){2}{\rule{0.800pt}{0.241pt}}
\put(271,209.84){\rule{0.482pt}{0.800pt}}
\multiput(271.00,209.34)(1.000,1.000){2}{\rule{0.241pt}{0.800pt}}
\put(271.84,212){\rule{0.800pt}{0.482pt}}
\multiput(271.34,212.00)(1.000,1.000){2}{\rule{0.800pt}{0.241pt}}
\put(272.84,214){\rule{0.800pt}{0.482pt}}
\multiput(272.34,214.00)(1.000,1.000){2}{\rule{0.800pt}{0.241pt}}
\put(273.84,216){\rule{0.800pt}{0.482pt}}
\multiput(273.34,216.00)(1.000,1.000){2}{\rule{0.800pt}{0.241pt}}
\put(276,216.84){\rule{0.482pt}{0.800pt}}
\multiput(276.00,216.34)(1.000,1.000){2}{\rule{0.241pt}{0.800pt}}
\put(276.84,219){\rule{0.800pt}{0.482pt}}
\multiput(276.34,219.00)(1.000,1.000){2}{\rule{0.800pt}{0.241pt}}
\put(277.84,221){\rule{0.800pt}{0.482pt}}
\multiput(277.34,221.00)(1.000,1.000){2}{\rule{0.800pt}{0.241pt}}
\put(280,222.34){\rule{0.482pt}{0.800pt}}
\multiput(280.00,221.34)(1.000,2.000){2}{\rule{0.241pt}{0.800pt}}
\put(282,223.84){\rule{0.241pt}{0.800pt}}
\multiput(282.00,223.34)(0.500,1.000){2}{\rule{0.120pt}{0.800pt}}
\put(281.84,226){\rule{0.800pt}{0.482pt}}
\multiput(281.34,226.00)(1.000,1.000){2}{\rule{0.800pt}{0.241pt}}
\put(284,227.34){\rule{0.482pt}{0.800pt}}
\multiput(284.00,226.34)(1.000,2.000){2}{\rule{0.241pt}{0.800pt}}
\put(284.84,230){\rule{0.800pt}{0.482pt}}
\multiput(284.34,230.00)(1.000,1.000){2}{\rule{0.800pt}{0.241pt}}
\put(287,230.84){\rule{0.241pt}{0.800pt}}
\multiput(287.00,230.34)(0.500,1.000){2}{\rule{0.120pt}{0.800pt}}
\put(286.84,233){\rule{0.800pt}{0.482pt}}
\multiput(286.34,233.00)(1.000,1.000){2}{\rule{0.800pt}{0.241pt}}
\put(289,234.34){\rule{0.482pt}{0.800pt}}
\multiput(289.00,233.34)(1.000,2.000){2}{\rule{0.241pt}{0.800pt}}
\put(291,235.84){\rule{0.241pt}{0.800pt}}
\multiput(291.00,235.34)(0.500,1.000){2}{\rule{0.120pt}{0.800pt}}
\put(290.84,238){\rule{0.800pt}{0.482pt}}
\multiput(290.34,238.00)(1.000,1.000){2}{\rule{0.800pt}{0.241pt}}
\put(293,239.34){\rule{0.482pt}{0.800pt}}
\multiput(293.00,238.34)(1.000,2.000){2}{\rule{0.241pt}{0.800pt}}
\put(293.84,242){\rule{0.800pt}{0.482pt}}
\multiput(293.34,242.00)(1.000,1.000){2}{\rule{0.800pt}{0.241pt}}
\put(296,242.84){\rule{0.241pt}{0.800pt}}
\multiput(296.00,242.34)(0.500,1.000){2}{\rule{0.120pt}{0.800pt}}
\put(295.84,245){\rule{0.800pt}{0.482pt}}
\multiput(295.34,245.00)(1.000,1.000){2}{\rule{0.800pt}{0.241pt}}
\put(298,246.34){\rule{0.482pt}{0.800pt}}
\multiput(298.00,245.34)(1.000,2.000){2}{\rule{0.241pt}{0.800pt}}
\put(298.84,249){\rule{0.800pt}{0.482pt}}
\multiput(298.34,249.00)(1.000,1.000){2}{\rule{0.800pt}{0.241pt}}
\put(301,249.84){\rule{0.241pt}{0.800pt}}
\multiput(301.00,249.34)(0.500,1.000){2}{\rule{0.120pt}{0.800pt}}
\put(302,251.34){\rule{0.482pt}{0.800pt}}
\multiput(302.00,250.34)(1.000,2.000){2}{\rule{0.241pt}{0.800pt}}
\put(302.84,254){\rule{0.800pt}{0.482pt}}
\multiput(302.34,254.00)(1.000,1.000){2}{\rule{0.800pt}{0.241pt}}
\put(303.84,256){\rule{0.800pt}{0.482pt}}
\multiput(303.34,256.00)(1.000,1.000){2}{\rule{0.800pt}{0.241pt}}
\put(306,256.84){\rule{0.241pt}{0.800pt}}
\multiput(306.00,256.34)(0.500,1.000){2}{\rule{0.120pt}{0.800pt}}
\put(307,258.34){\rule{0.482pt}{0.800pt}}
\multiput(307.00,257.34)(1.000,2.000){2}{\rule{0.241pt}{0.800pt}}
\put(307.84,261){\rule{0.800pt}{0.482pt}}
\multiput(307.34,261.00)(1.000,1.000){2}{\rule{0.800pt}{0.241pt}}
\put(308.84,263){\rule{0.800pt}{0.482pt}}
\multiput(308.34,263.00)(1.000,1.000){2}{\rule{0.800pt}{0.241pt}}
\put(311,263.84){\rule{0.482pt}{0.800pt}}
\multiput(311.00,263.34)(1.000,1.000){2}{\rule{0.241pt}{0.800pt}}
\put(311.84,266){\rule{0.800pt}{0.482pt}}
\multiput(311.34,266.00)(1.000,1.000){2}{\rule{0.800pt}{0.241pt}}
\put(312.84,268){\rule{0.800pt}{0.482pt}}
\multiput(312.34,268.00)(1.000,1.000){2}{\rule{0.800pt}{0.241pt}}
\put(315,268.84){\rule{0.482pt}{0.800pt}}
\multiput(315.00,268.34)(1.000,1.000){2}{\rule{0.241pt}{0.800pt}}
\put(315.84,271){\rule{0.800pt}{0.482pt}}
\multiput(315.34,271.00)(1.000,1.000){2}{\rule{0.800pt}{0.241pt}}
\put(316.84,273){\rule{0.800pt}{0.482pt}}
\multiput(316.34,273.00)(1.000,1.000){2}{\rule{0.800pt}{0.241pt}}
\put(317.84,275){\rule{0.800pt}{0.482pt}}
\multiput(317.34,275.00)(1.000,1.000){2}{\rule{0.800pt}{0.241pt}}
\put(320,275.84){\rule{0.482pt}{0.800pt}}
\multiput(320.00,275.34)(1.000,1.000){2}{\rule{0.241pt}{0.800pt}}
\put(320.84,278){\rule{0.800pt}{0.482pt}}
\multiput(320.34,278.00)(1.000,1.000){2}{\rule{0.800pt}{0.241pt}}
\put(321.84,280){\rule{0.800pt}{0.482pt}}
\multiput(321.34,280.00)(1.000,1.000){2}{\rule{0.800pt}{0.241pt}}
\put(324,281.34){\rule{0.482pt}{0.800pt}}
\multiput(324.00,280.34)(1.000,2.000){2}{\rule{0.241pt}{0.800pt}}
\put(326,282.84){\rule{0.241pt}{0.800pt}}
\multiput(326.00,282.34)(0.500,1.000){2}{\rule{0.120pt}{0.800pt}}
\put(325.84,285){\rule{0.800pt}{0.482pt}}
\multiput(325.34,285.00)(1.000,1.000){2}{\rule{0.800pt}{0.241pt}}
\put(326.84,287){\rule{0.800pt}{0.482pt}}
\multiput(326.34,287.00)(1.000,1.000){2}{\rule{0.800pt}{0.241pt}}
\put(329,288.34){\rule{0.482pt}{0.800pt}}
\multiput(329.00,287.34)(1.000,2.000){2}{\rule{0.241pt}{0.800pt}}
\put(331,289.84){\rule{0.241pt}{0.800pt}}
\multiput(331.00,289.34)(0.500,1.000){2}{\rule{0.120pt}{0.800pt}}
\put(330.84,292){\rule{0.800pt}{0.482pt}}
\multiput(330.34,292.00)(1.000,1.000){2}{\rule{0.800pt}{0.241pt}}
\put(333,293.34){\rule{0.482pt}{0.800pt}}
\multiput(333.00,292.34)(1.000,2.000){2}{\rule{0.241pt}{0.800pt}}
\put(333.84,296){\rule{0.800pt}{0.482pt}}
\multiput(333.34,296.00)(1.000,1.000){2}{\rule{0.800pt}{0.241pt}}
\put(336,296.84){\rule{0.241pt}{0.800pt}}
\multiput(336.00,296.34)(0.500,1.000){2}{\rule{0.120pt}{0.800pt}}
\put(337,298.34){\rule{0.482pt}{0.800pt}}
\multiput(337.00,297.34)(1.000,2.000){2}{\rule{0.241pt}{0.800pt}}
\put(337.84,301){\rule{0.800pt}{0.482pt}}
\multiput(337.34,301.00)(1.000,1.000){2}{\rule{0.800pt}{0.241pt}}
\put(340,301.84){\rule{0.241pt}{0.800pt}}
\multiput(340.00,301.34)(0.500,1.000){2}{\rule{0.120pt}{0.800pt}}
\put(339.84,304){\rule{0.800pt}{0.482pt}}
\multiput(339.34,304.00)(1.000,1.000){2}{\rule{0.800pt}{0.241pt}}
\put(342,305.34){\rule{0.482pt}{0.800pt}}
\multiput(342.00,304.34)(1.000,2.000){2}{\rule{0.241pt}{0.800pt}}
\put(342.84,308){\rule{0.800pt}{0.482pt}}
\multiput(342.34,308.00)(1.000,1.000){2}{\rule{0.800pt}{0.241pt}}
\put(345,308.84){\rule{0.241pt}{0.800pt}}
\multiput(345.00,308.34)(0.500,1.000){2}{\rule{0.120pt}{0.800pt}}
\put(346,310.34){\rule{0.482pt}{0.800pt}}
\multiput(346.00,309.34)(1.000,2.000){2}{\rule{0.241pt}{0.800pt}}
\put(346.84,313){\rule{0.800pt}{0.482pt}}
\multiput(346.34,313.00)(1.000,1.000){2}{\rule{0.800pt}{0.241pt}}
\put(347.84,315){\rule{0.800pt}{0.482pt}}
\multiput(347.34,315.00)(1.000,1.000){2}{\rule{0.800pt}{0.241pt}}
\put(350,315.84){\rule{0.241pt}{0.800pt}}
\multiput(350.00,315.34)(0.500,1.000){2}{\rule{0.120pt}{0.800pt}}
\put(351,317.34){\rule{0.482pt}{0.800pt}}
\multiput(351.00,316.34)(1.000,2.000){2}{\rule{0.241pt}{0.800pt}}
\put(351.84,320){\rule{0.800pt}{0.482pt}}
\multiput(351.34,320.00)(1.000,1.000){2}{\rule{0.800pt}{0.241pt}}
\put(352.84,322){\rule{0.800pt}{0.482pt}}
\multiput(352.34,322.00)(1.000,1.000){2}{\rule{0.800pt}{0.241pt}}
\put(355,322.84){\rule{0.482pt}{0.800pt}}
\multiput(355.00,322.34)(1.000,1.000){2}{\rule{0.241pt}{0.800pt}}
\put(355.84,325){\rule{0.800pt}{0.482pt}}
\multiput(355.34,325.00)(1.000,1.000){2}{\rule{0.800pt}{0.241pt}}
\put(356.84,327){\rule{0.800pt}{0.482pt}}
\multiput(356.34,327.00)(1.000,1.000){2}{\rule{0.800pt}{0.241pt}}
\put(359,327.84){\rule{0.241pt}{0.800pt}}
\multiput(359.00,327.34)(0.500,1.000){2}{\rule{0.120pt}{0.800pt}}
\put(360,329.34){\rule{0.482pt}{0.800pt}}
\multiput(360.00,328.34)(1.000,2.000){2}{\rule{0.241pt}{0.800pt}}
\put(360.84,332){\rule{0.800pt}{0.482pt}}
\multiput(360.34,332.00)(1.000,1.000){2}{\rule{0.800pt}{0.241pt}}
\put(361.84,334){\rule{0.800pt}{0.482pt}}
\multiput(361.34,334.00)(1.000,1.000){2}{\rule{0.800pt}{0.241pt}}
\put(364,334.84){\rule{0.482pt}{0.800pt}}
\multiput(364.00,334.34)(1.000,1.000){2}{\rule{0.241pt}{0.800pt}}
\put(364.84,337){\rule{0.800pt}{0.482pt}}
\multiput(364.34,337.00)(1.000,1.000){2}{\rule{0.800pt}{0.241pt}}
\put(365.84,339){\rule{0.800pt}{0.482pt}}
\multiput(365.34,339.00)(1.000,1.000){2}{\rule{0.800pt}{0.241pt}}
\put(368,340.34){\rule{0.482pt}{0.800pt}}
\multiput(368.00,339.34)(1.000,2.000){2}{\rule{0.241pt}{0.800pt}}
\put(370,341.84){\rule{0.241pt}{0.800pt}}
\multiput(370.00,341.34)(0.500,1.000){2}{\rule{0.120pt}{0.800pt}}
\put(369.84,344){\rule{0.800pt}{0.482pt}}
\multiput(369.34,344.00)(1.000,1.000){2}{\rule{0.800pt}{0.241pt}}
\put(370.84,346){\rule{0.800pt}{0.482pt}}
\multiput(370.34,346.00)(1.000,1.000){2}{\rule{0.800pt}{0.241pt}}
\put(373,347.34){\rule{0.482pt}{0.800pt}}
\multiput(373.00,346.34)(1.000,2.000){2}{\rule{0.241pt}{0.800pt}}
\put(375,348.84){\rule{0.241pt}{0.800pt}}
\multiput(375.00,348.34)(0.500,1.000){2}{\rule{0.120pt}{0.800pt}}
\put(374.84,191){\rule{0.800pt}{38.544pt}}
\multiput(374.34,271.00)(1.000,-80.000){2}{\rule{0.800pt}{19.272pt}}
\put(376.34,169){\rule{0.800pt}{5.300pt}}
\multiput(375.34,180.00)(2.000,-11.000){2}{\rule{0.800pt}{2.650pt}}
\put(377.84,169){\rule{0.800pt}{0.482pt}}
\multiput(377.34,169.00)(1.000,1.000){2}{\rule{0.800pt}{0.241pt}}
\put(378.84,171){\rule{0.800pt}{0.482pt}}
\multiput(378.34,171.00)(1.000,1.000){2}{\rule{0.800pt}{0.241pt}}
\put(379.84,173){\rule{0.800pt}{0.482pt}}
\multiput(379.34,173.00)(1.000,1.000){2}{\rule{0.800pt}{0.241pt}}
\put(382,173.84){\rule{0.482pt}{0.800pt}}
\multiput(382.00,173.34)(1.000,1.000){2}{\rule{0.241pt}{0.800pt}}
\put(382.84,176){\rule{0.800pt}{0.482pt}}
\multiput(382.34,176.00)(1.000,1.000){2}{\rule{0.800pt}{0.241pt}}
\put(383.84,178){\rule{0.800pt}{0.482pt}}
\multiput(383.34,178.00)(1.000,1.000){2}{\rule{0.800pt}{0.241pt}}
\put(386,179.34){\rule{0.482pt}{0.800pt}}
\multiput(386.00,178.34)(1.000,2.000){2}{\rule{0.241pt}{0.800pt}}
\put(388,180.84){\rule{0.241pt}{0.800pt}}
\multiput(388.00,180.34)(0.500,1.000){2}{\rule{0.120pt}{0.800pt}}
\put(387.84,183){\rule{0.800pt}{0.482pt}}
\multiput(387.34,183.00)(1.000,1.000){2}{\rule{0.800pt}{0.241pt}}
\put(389.34,181){\rule{0.800pt}{0.964pt}}
\multiput(388.34,183.00)(2.000,-2.000){2}{\rule{0.800pt}{0.482pt}}
\put(390.84,181){\rule{0.800pt}{0.482pt}}
\multiput(390.34,181.00)(1.000,1.000){2}{\rule{0.800pt}{0.241pt}}
\put(393,181.84){\rule{0.241pt}{0.800pt}}
\multiput(393.00,181.34)(0.500,1.000){2}{\rule{0.120pt}{0.800pt}}
\put(392.84,184){\rule{0.800pt}{0.482pt}}
\multiput(392.34,184.00)(1.000,1.000){2}{\rule{0.800pt}{0.241pt}}
\put(395,185.34){\rule{0.482pt}{0.800pt}}
\multiput(395.00,184.34)(1.000,2.000){2}{\rule{0.241pt}{0.800pt}}
\put(395.84,188){\rule{0.800pt}{0.482pt}}
\multiput(395.34,188.00)(1.000,1.000){2}{\rule{0.800pt}{0.241pt}}
\put(398,188.84){\rule{0.241pt}{0.800pt}}
\multiput(398.00,188.34)(0.500,1.000){2}{\rule{0.120pt}{0.800pt}}
\put(399,190.34){\rule{0.482pt}{0.800pt}}
\multiput(399.00,189.34)(1.000,2.000){2}{\rule{0.241pt}{0.800pt}}
\put(399.84,193){\rule{0.800pt}{0.482pt}}
\multiput(399.34,193.00)(1.000,1.000){2}{\rule{0.800pt}{0.241pt}}
\put(402,193.84){\rule{0.241pt}{0.800pt}}
\multiput(402.00,193.34)(0.500,1.000){2}{\rule{0.120pt}{0.800pt}}
\put(401.84,196){\rule{0.800pt}{0.482pt}}
\multiput(401.34,196.00)(1.000,1.000){2}{\rule{0.800pt}{0.241pt}}
\put(404,197.34){\rule{0.482pt}{0.800pt}}
\multiput(404.00,196.34)(1.000,2.000){2}{\rule{0.241pt}{0.800pt}}
\put(404.84,200){\rule{0.800pt}{0.482pt}}
\multiput(404.34,200.00)(1.000,1.000){2}{\rule{0.800pt}{0.241pt}}
\put(407,200.84){\rule{0.241pt}{0.800pt}}
\multiput(407.00,200.34)(0.500,1.000){2}{\rule{0.120pt}{0.800pt}}
\put(408,202.34){\rule{0.482pt}{0.800pt}}
\multiput(408.00,201.34)(1.000,2.000){2}{\rule{0.241pt}{0.800pt}}
\put(408.84,205){\rule{0.800pt}{0.482pt}}
\multiput(408.34,205.00)(1.000,1.000){2}{\rule{0.800pt}{0.241pt}}
\put(409.84,207){\rule{0.800pt}{0.482pt}}
\multiput(409.34,207.00)(1.000,1.000){2}{\rule{0.800pt}{0.241pt}}
\put(412,207.84){\rule{0.482pt}{0.800pt}}
\multiput(412.00,207.34)(1.000,1.000){2}{\rule{0.241pt}{0.800pt}}
\put(412.84,210){\rule{0.800pt}{0.482pt}}
\multiput(412.34,210.00)(1.000,1.000){2}{\rule{0.800pt}{0.241pt}}
\put(413.84,212){\rule{0.800pt}{0.482pt}}
\multiput(413.34,212.00)(1.000,1.000){2}{\rule{0.800pt}{0.241pt}}
\put(416,212.84){\rule{0.241pt}{0.800pt}}
\multiput(416.00,212.34)(0.500,1.000){2}{\rule{0.120pt}{0.800pt}}
\put(417,214.34){\rule{0.482pt}{0.800pt}}
\multiput(417.00,213.34)(1.000,2.000){2}{\rule{0.241pt}{0.800pt}}
\put(417.84,217){\rule{0.800pt}{0.482pt}}
\multiput(417.34,217.00)(1.000,1.000){2}{\rule{0.800pt}{0.241pt}}
\put(420,217.84){\rule{0.241pt}{0.800pt}}
\multiput(420.00,217.34)(0.500,1.000){2}{\rule{0.120pt}{0.800pt}}
\put(421,219.34){\rule{0.482pt}{0.800pt}}
\multiput(421.00,218.34)(1.000,2.000){2}{\rule{0.241pt}{0.800pt}}
\put(422.84,222){\rule{0.800pt}{0.482pt}}
\multiput(422.34,222.00)(1.000,1.000){2}{\rule{0.800pt}{0.241pt}}
\put(425,222.84){\rule{0.241pt}{0.800pt}}
\multiput(425.00,222.34)(0.500,1.000){2}{\rule{0.120pt}{0.800pt}}
\put(426,224.34){\rule{0.482pt}{0.800pt}}
\multiput(426.00,223.34)(1.000,2.000){2}{\rule{0.241pt}{0.800pt}}
\put(426.84,227){\rule{0.800pt}{0.482pt}}
\multiput(426.34,227.00)(1.000,1.000){2}{\rule{0.800pt}{0.241pt}}
\put(429,227.84){\rule{0.241pt}{0.800pt}}
\multiput(429.00,227.34)(0.500,1.000){2}{\rule{0.120pt}{0.800pt}}
\put(430,229.34){\rule{0.482pt}{0.800pt}}
\multiput(430.00,228.34)(1.000,2.000){2}{\rule{0.241pt}{0.800pt}}
\put(430.84,232){\rule{0.800pt}{0.482pt}}
\multiput(430.34,232.00)(1.000,1.000){2}{\rule{0.800pt}{0.241pt}}
\put(433,232.84){\rule{0.241pt}{0.800pt}}
\multiput(433.00,232.34)(0.500,1.000){2}{\rule{0.120pt}{0.800pt}}
\put(432.84,235){\rule{0.800pt}{0.482pt}}
\multiput(432.34,235.00)(1.000,1.000){2}{\rule{0.800pt}{0.241pt}}
\put(435,236.34){\rule{0.482pt}{0.800pt}}
\multiput(435.00,235.34)(1.000,2.000){2}{\rule{0.241pt}{0.800pt}}
\put(437,236.84){\rule{0.241pt}{0.800pt}}
\multiput(437.00,237.34)(0.500,-1.000){2}{\rule{0.120pt}{0.800pt}}
\put(438,236.84){\rule{0.241pt}{0.800pt}}
\multiput(438.00,236.34)(0.500,1.000){2}{\rule{0.120pt}{0.800pt}}
\put(439,238.34){\rule{0.482pt}{0.800pt}}
\multiput(439.00,237.34)(1.000,2.000){2}{\rule{0.241pt}{0.800pt}}
\put(441,239.84){\rule{0.241pt}{0.800pt}}
\multiput(441.00,239.34)(0.500,1.000){2}{\rule{0.120pt}{0.800pt}}
\put(440.84,242){\rule{0.800pt}{0.482pt}}
\multiput(440.34,242.00)(1.000,1.000){2}{\rule{0.800pt}{0.241pt}}
\put(443,243.34){\rule{0.482pt}{0.800pt}}
\multiput(443.00,242.34)(1.000,2.000){2}{\rule{0.241pt}{0.800pt}}
\put(443.84,246){\rule{0.800pt}{0.482pt}}
\multiput(443.34,246.00)(1.000,1.000){2}{\rule{0.800pt}{0.241pt}}
\put(446,246.84){\rule{0.241pt}{0.800pt}}
\multiput(446.00,246.34)(0.500,1.000){2}{\rule{0.120pt}{0.800pt}}
\put(445.84,249){\rule{0.800pt}{0.482pt}}
\multiput(445.34,249.00)(1.000,1.000){2}{\rule{0.800pt}{0.241pt}}
\put(448,248.84){\rule{0.482pt}{0.800pt}}
\multiput(448.00,249.34)(1.000,-1.000){2}{\rule{0.241pt}{0.800pt}}
\put(448.84,234){\rule{0.800pt}{3.854pt}}
\multiput(448.34,242.00)(1.000,-8.000){2}{\rule{0.800pt}{1.927pt}}
\put(451,232.84){\rule{0.241pt}{0.800pt}}
\multiput(451.00,232.34)(0.500,1.000){2}{\rule{0.120pt}{0.800pt}}
\put(452,234.34){\rule{0.482pt}{0.800pt}}
\multiput(452.00,233.34)(1.000,2.000){2}{\rule{0.241pt}{0.800pt}}
\put(452.84,237){\rule{0.800pt}{0.482pt}}
\multiput(452.34,237.00)(1.000,1.000){2}{\rule{0.800pt}{0.241pt}}
\put(453.84,226){\rule{0.800pt}{3.132pt}}
\multiput(453.34,232.50)(1.000,-6.500){2}{\rule{0.800pt}{1.566pt}}
\put(454.84,214){\rule{0.800pt}{2.891pt}}
\multiput(454.34,220.00)(1.000,-6.000){2}{\rule{0.800pt}{1.445pt}}
\put(456.34,210){\rule{0.800pt}{0.964pt}}
\multiput(455.34,212.00)(2.000,-2.000){2}{\rule{0.800pt}{0.482pt}}
\put(457.84,210){\rule{0.800pt}{0.482pt}}
\multiput(457.34,210.00)(1.000,1.000){2}{\rule{0.800pt}{0.241pt}}
\put(458.84,200){\rule{0.800pt}{2.891pt}}
\multiput(458.34,206.00)(1.000,-6.000){2}{\rule{0.800pt}{1.445pt}}
\put(461,198.84){\rule{0.482pt}{0.800pt}}
\multiput(461.00,198.34)(1.000,1.000){2}{\rule{0.241pt}{0.800pt}}
\put(463,199.84){\rule{0.241pt}{0.800pt}}
\multiput(463.00,199.34)(0.500,1.000){2}{\rule{0.120pt}{0.800pt}}
\put(462.84,200){\rule{0.800pt}{0.482pt}}
\multiput(462.34,201.00)(1.000,-1.000){2}{\rule{0.800pt}{0.241pt}}
\put(465,199.34){\rule{0.482pt}{0.800pt}}
\multiput(465.00,198.34)(1.000,2.000){2}{\rule{0.241pt}{0.800pt}}
\put(465.84,202){\rule{0.800pt}{0.482pt}}
\multiput(465.34,202.00)(1.000,1.000){2}{\rule{0.800pt}{0.241pt}}
\put(468,202.84){\rule{0.241pt}{0.800pt}}
\multiput(468.00,202.34)(0.500,1.000){2}{\rule{0.120pt}{0.800pt}}
\put(467.84,203){\rule{0.800pt}{0.482pt}}
\multiput(467.34,204.00)(1.000,-1.000){2}{\rule{0.800pt}{0.241pt}}
\put(470,202.34){\rule{0.482pt}{0.800pt}}
\multiput(470.00,201.34)(1.000,2.000){2}{\rule{0.241pt}{0.800pt}}
\put(470.84,205){\rule{0.800pt}{0.482pt}}
\multiput(470.34,205.00)(1.000,1.000){2}{\rule{0.800pt}{0.241pt}}
\put(473,205.84){\rule{0.241pt}{0.800pt}}
\multiput(473.00,205.34)(0.500,1.000){2}{\rule{0.120pt}{0.800pt}}
\put(474,207.34){\rule{0.482pt}{0.800pt}}
\multiput(474.00,206.34)(1.000,2.000){2}{\rule{0.241pt}{0.800pt}}
\put(474.84,210){\rule{0.800pt}{0.482pt}}
\multiput(474.34,210.00)(1.000,1.000){2}{\rule{0.800pt}{0.241pt}}
\put(477,209.84){\rule{0.241pt}{0.800pt}}
\multiput(477.00,210.34)(0.500,-1.000){2}{\rule{0.120pt}{0.800pt}}
\put(478,208.84){\rule{0.241pt}{0.800pt}}
\multiput(478.00,209.34)(0.500,-1.000){2}{\rule{0.120pt}{0.800pt}}
\put(479,207.34){\rule{0.482pt}{0.800pt}}
\multiput(479.00,208.34)(1.000,-2.000){2}{\rule{0.241pt}{0.800pt}}
\put(423.0,222.0){\usebox{\plotpoint}}
\put(480.84,208){\rule{0.800pt}{0.482pt}}
\multiput(480.34,208.00)(1.000,1.000){2}{\rule{0.800pt}{0.241pt}}
\put(481.0,208.0){\usebox{\plotpoint}}
\put(483.84,200){\rule{0.800pt}{2.409pt}}
\multiput(483.34,205.00)(1.000,-5.000){2}{\rule{0.800pt}{1.204pt}}
\put(484.84,198){\rule{0.800pt}{0.482pt}}
\multiput(484.34,199.00)(1.000,-1.000){2}{\rule{0.800pt}{0.241pt}}
\put(485.84,198){\rule{0.800pt}{0.482pt}}
\multiput(485.34,198.00)(1.000,1.000){2}{\rule{0.800pt}{0.241pt}}
\put(488,199.34){\rule{0.482pt}{0.800pt}}
\multiput(488.00,198.34)(1.000,2.000){2}{\rule{0.241pt}{0.800pt}}
\put(490,200.84){\rule{0.241pt}{0.800pt}}
\multiput(490.00,200.34)(0.500,1.000){2}{\rule{0.120pt}{0.800pt}}
\put(489.84,203){\rule{0.800pt}{0.482pt}}
\multiput(489.34,203.00)(1.000,1.000){2}{\rule{0.800pt}{0.241pt}}
\put(492,203.84){\rule{0.482pt}{0.800pt}}
\multiput(492.00,203.34)(1.000,1.000){2}{\rule{0.241pt}{0.800pt}}
\put(492.84,206){\rule{0.800pt}{0.482pt}}
\multiput(492.34,206.00)(1.000,1.000){2}{\rule{0.800pt}{0.241pt}}
\put(493.84,208){\rule{0.800pt}{0.482pt}}
\multiput(493.34,208.00)(1.000,1.000){2}{\rule{0.800pt}{0.241pt}}
\put(496,209.34){\rule{0.482pt}{0.800pt}}
\multiput(496.00,208.34)(1.000,2.000){2}{\rule{0.241pt}{0.800pt}}
\put(498,210.84){\rule{0.241pt}{0.800pt}}
\multiput(498.00,210.34)(0.500,1.000){2}{\rule{0.120pt}{0.800pt}}
\put(497.84,213){\rule{0.800pt}{0.482pt}}
\multiput(497.34,213.00)(1.000,1.000){2}{\rule{0.800pt}{0.241pt}}
\put(498.84,215){\rule{0.800pt}{0.482pt}}
\multiput(498.34,215.00)(1.000,1.000){2}{\rule{0.800pt}{0.241pt}}
\put(501,216.34){\rule{0.482pt}{0.800pt}}
\multiput(501.00,215.34)(1.000,2.000){2}{\rule{0.241pt}{0.800pt}}
\put(503,217.84){\rule{0.241pt}{0.800pt}}
\multiput(503.00,217.34)(0.500,1.000){2}{\rule{0.120pt}{0.800pt}}
\put(502.84,220){\rule{0.800pt}{0.482pt}}
\multiput(502.34,220.00)(1.000,1.000){2}{\rule{0.800pt}{0.241pt}}
\put(505,221.34){\rule{0.482pt}{0.800pt}}
\multiput(505.00,220.34)(1.000,2.000){2}{\rule{0.241pt}{0.800pt}}
\put(507,222.84){\rule{0.241pt}{0.800pt}}
\multiput(507.00,222.34)(0.500,1.000){2}{\rule{0.120pt}{0.800pt}}
\put(506.84,225){\rule{0.800pt}{0.482pt}}
\multiput(506.34,225.00)(1.000,1.000){2}{\rule{0.800pt}{0.241pt}}
\put(507.84,227){\rule{0.800pt}{0.482pt}}
\multiput(507.34,227.00)(1.000,1.000){2}{\rule{0.800pt}{0.241pt}}
\put(510,228.34){\rule{0.482pt}{0.800pt}}
\multiput(510.00,227.34)(1.000,2.000){2}{\rule{0.241pt}{0.800pt}}
\put(512,229.84){\rule{0.241pt}{0.800pt}}
\multiput(512.00,229.34)(0.500,1.000){2}{\rule{0.120pt}{0.800pt}}
\put(511.84,232){\rule{0.800pt}{0.482pt}}
\multiput(511.34,232.00)(1.000,1.000){2}{\rule{0.800pt}{0.241pt}}
\put(514,233.34){\rule{0.482pt}{0.800pt}}
\multiput(514.00,232.34)(1.000,2.000){2}{\rule{0.241pt}{0.800pt}}
\put(514.84,236){\rule{0.800pt}{0.482pt}}
\multiput(514.34,236.00)(1.000,1.000){2}{\rule{0.800pt}{0.241pt}}
\put(517,236.84){\rule{0.241pt}{0.800pt}}
\multiput(517.00,236.34)(0.500,1.000){2}{\rule{0.120pt}{0.800pt}}
\put(518,238.34){\rule{0.482pt}{0.800pt}}
\multiput(518.00,237.34)(1.000,2.000){2}{\rule{0.241pt}{0.800pt}}
\put(518.84,241){\rule{0.800pt}{0.482pt}}
\multiput(518.34,241.00)(1.000,1.000){2}{\rule{0.800pt}{0.241pt}}
\put(519.84,243){\rule{0.800pt}{0.482pt}}
\multiput(519.34,243.00)(1.000,1.000){2}{\rule{0.800pt}{0.241pt}}
\put(522,243.84){\rule{0.241pt}{0.800pt}}
\multiput(522.00,243.34)(0.500,1.000){2}{\rule{0.120pt}{0.800pt}}
\put(523,245.34){\rule{0.482pt}{0.800pt}}
\multiput(523.00,244.34)(1.000,2.000){2}{\rule{0.241pt}{0.800pt}}
\put(523.84,248){\rule{0.800pt}{0.482pt}}
\multiput(523.34,248.00)(1.000,1.000){2}{\rule{0.800pt}{0.241pt}}
\put(524.84,250){\rule{0.800pt}{0.482pt}}
\multiput(524.34,250.00)(1.000,1.000){2}{\rule{0.800pt}{0.241pt}}
\put(527,250.84){\rule{0.482pt}{0.800pt}}
\multiput(527.00,250.34)(1.000,1.000){2}{\rule{0.241pt}{0.800pt}}
\put(527.84,253){\rule{0.800pt}{0.482pt}}
\multiput(527.34,253.00)(1.000,1.000){2}{\rule{0.800pt}{0.241pt}}
\put(528.84,255){\rule{0.800pt}{0.482pt}}
\multiput(528.34,255.00)(1.000,1.000){2}{\rule{0.800pt}{0.241pt}}
\put(531,255.84){\rule{0.241pt}{0.800pt}}
\multiput(531.00,255.34)(0.500,1.000){2}{\rule{0.120pt}{0.800pt}}
\put(532,257.34){\rule{0.482pt}{0.800pt}}
\multiput(532.00,256.34)(1.000,2.000){2}{\rule{0.241pt}{0.800pt}}
\put(532.84,260){\rule{0.800pt}{0.482pt}}
\multiput(532.34,260.00)(1.000,1.000){2}{\rule{0.800pt}{0.241pt}}
\put(533.84,262){\rule{0.800pt}{0.482pt}}
\multiput(533.34,262.00)(1.000,1.000){2}{\rule{0.800pt}{0.241pt}}
\put(536,262.84){\rule{0.482pt}{0.800pt}}
\multiput(536.00,262.34)(1.000,1.000){2}{\rule{0.241pt}{0.800pt}}
\put(536.84,265){\rule{0.800pt}{0.482pt}}
\multiput(536.34,265.00)(1.000,1.000){2}{\rule{0.800pt}{0.241pt}}
\put(537.84,267){\rule{0.800pt}{0.482pt}}
\multiput(537.34,267.00)(1.000,1.000){2}{\rule{0.800pt}{0.241pt}}
\put(538.84,269){\rule{0.800pt}{0.482pt}}
\multiput(538.34,269.00)(1.000,1.000){2}{\rule{0.800pt}{0.241pt}}
\put(541,269.84){\rule{0.482pt}{0.800pt}}
\multiput(541.00,269.34)(1.000,1.000){2}{\rule{0.241pt}{0.800pt}}
\put(541.84,272){\rule{0.800pt}{0.482pt}}
\multiput(541.34,272.00)(1.000,1.000){2}{\rule{0.800pt}{0.241pt}}
\put(542.84,274){\rule{0.800pt}{0.482pt}}
\multiput(542.34,274.00)(1.000,1.000){2}{\rule{0.800pt}{0.241pt}}
\put(545,275.34){\rule{0.482pt}{0.800pt}}
\multiput(545.00,274.34)(1.000,2.000){2}{\rule{0.241pt}{0.800pt}}
\put(547,276.84){\rule{0.241pt}{0.800pt}}
\multiput(547.00,276.34)(0.500,1.000){2}{\rule{0.120pt}{0.800pt}}
\put(546.84,279){\rule{0.800pt}{0.482pt}}
\multiput(546.34,279.00)(1.000,1.000){2}{\rule{0.800pt}{0.241pt}}
\put(549,280.34){\rule{0.482pt}{0.800pt}}
\multiput(549.00,279.34)(1.000,2.000){2}{\rule{0.241pt}{0.800pt}}
\put(551,281.84){\rule{0.241pt}{0.800pt}}
\multiput(551.00,281.34)(0.500,1.000){2}{\rule{0.120pt}{0.800pt}}
\put(550.84,284){\rule{0.800pt}{0.482pt}}
\multiput(550.34,284.00)(1.000,1.000){2}{\rule{0.800pt}{0.241pt}}
\put(551.84,286){\rule{0.800pt}{0.482pt}}
\multiput(551.34,286.00)(1.000,1.000){2}{\rule{0.800pt}{0.241pt}}
\put(554,287.34){\rule{0.482pt}{0.800pt}}
\multiput(554.00,286.34)(1.000,2.000){2}{\rule{0.241pt}{0.800pt}}
\put(556,288.84){\rule{0.241pt}{0.800pt}}
\multiput(556.00,288.34)(0.500,1.000){2}{\rule{0.120pt}{0.800pt}}
\put(555.84,291){\rule{0.800pt}{0.482pt}}
\multiput(555.34,291.00)(1.000,1.000){2}{\rule{0.800pt}{0.241pt}}
\put(558,292.34){\rule{0.482pt}{0.800pt}}
\multiput(558.00,291.34)(1.000,2.000){2}{\rule{0.241pt}{0.800pt}}
\put(558.84,295){\rule{0.800pt}{0.482pt}}
\multiput(558.34,295.00)(1.000,1.000){2}{\rule{0.800pt}{0.241pt}}
\put(561,295.84){\rule{0.241pt}{0.800pt}}
\multiput(561.00,295.34)(0.500,1.000){2}{\rule{0.120pt}{0.800pt}}
\put(560.84,298){\rule{0.800pt}{0.482pt}}
\multiput(560.34,298.00)(1.000,1.000){2}{\rule{0.800pt}{0.241pt}}
\put(563,299.34){\rule{0.482pt}{0.800pt}}
\multiput(563.00,298.34)(1.000,2.000){2}{\rule{0.241pt}{0.800pt}}
\put(563.84,302){\rule{0.800pt}{0.482pt}}
\multiput(563.34,302.00)(1.000,1.000){2}{\rule{0.800pt}{0.241pt}}
\put(566,302.84){\rule{0.241pt}{0.800pt}}
\multiput(566.00,302.34)(0.500,1.000){2}{\rule{0.120pt}{0.800pt}}
\put(567,304.34){\rule{0.482pt}{0.800pt}}
\multiput(567.00,303.34)(1.000,2.000){2}{\rule{0.241pt}{0.800pt}}
\put(567.84,307){\rule{0.800pt}{0.482pt}}
\multiput(567.34,307.00)(1.000,1.000){2}{\rule{0.800pt}{0.241pt}}
\put(568.84,309){\rule{0.800pt}{0.482pt}}
\multiput(568.34,309.00)(1.000,1.000){2}{\rule{0.800pt}{0.241pt}}
\put(571,309.84){\rule{0.482pt}{0.800pt}}
\multiput(571.00,309.34)(1.000,1.000){2}{\rule{0.241pt}{0.800pt}}
\put(571.84,312){\rule{0.800pt}{0.482pt}}
\multiput(571.34,312.00)(1.000,1.000){2}{\rule{0.800pt}{0.241pt}}
\put(572.84,314){\rule{0.800pt}{0.482pt}}
\multiput(572.34,314.00)(1.000,1.000){2}{\rule{0.800pt}{0.241pt}}
\put(575,314.84){\rule{0.241pt}{0.800pt}}
\multiput(575.00,314.34)(0.500,1.000){2}{\rule{0.120pt}{0.800pt}}
\put(576,316.34){\rule{0.482pt}{0.800pt}}
\multiput(576.00,315.34)(1.000,2.000){2}{\rule{0.241pt}{0.800pt}}
\put(576.84,319){\rule{0.800pt}{0.482pt}}
\multiput(576.34,319.00)(1.000,1.000){2}{\rule{0.800pt}{0.241pt}}
\put(577.84,321){\rule{0.800pt}{0.482pt}}
\multiput(577.34,321.00)(1.000,1.000){2}{\rule{0.800pt}{0.241pt}}
\put(580,321.84){\rule{0.482pt}{0.800pt}}
\multiput(580.00,321.34)(1.000,1.000){2}{\rule{0.241pt}{0.800pt}}
\put(580.84,324){\rule{0.800pt}{0.482pt}}
\multiput(580.34,324.00)(1.000,1.000){2}{\rule{0.800pt}{0.241pt}}
\put(581.84,326){\rule{0.800pt}{0.482pt}}
\multiput(581.34,326.00)(1.000,1.000){2}{\rule{0.800pt}{0.241pt}}
\put(582.84,328){\rule{0.800pt}{0.482pt}}
\multiput(582.34,328.00)(1.000,1.000){2}{\rule{0.800pt}{0.241pt}}
\put(585,328.84){\rule{0.482pt}{0.800pt}}
\multiput(585.00,328.34)(1.000,1.000){2}{\rule{0.241pt}{0.800pt}}
\put(585.84,331){\rule{0.800pt}{0.482pt}}
\multiput(585.34,331.00)(1.000,1.000){2}{\rule{0.800pt}{0.241pt}}
\put(586.84,333){\rule{0.800pt}{0.482pt}}
\multiput(586.34,333.00)(1.000,1.000){2}{\rule{0.800pt}{0.241pt}}
\put(589,334.34){\rule{0.482pt}{0.800pt}}
\multiput(589.00,333.34)(1.000,2.000){2}{\rule{0.241pt}{0.800pt}}
\put(591,335.84){\rule{0.241pt}{0.800pt}}
\multiput(591.00,335.34)(0.500,1.000){2}{\rule{0.120pt}{0.800pt}}
\put(590.84,338){\rule{0.800pt}{0.482pt}}
\multiput(590.34,338.00)(1.000,1.000){2}{\rule{0.800pt}{0.241pt}}
\put(591.84,340){\rule{0.800pt}{0.482pt}}
\multiput(591.34,340.00)(1.000,1.000){2}{\rule{0.800pt}{0.241pt}}
\put(594,341.34){\rule{0.482pt}{0.800pt}}
\multiput(594.00,340.34)(1.000,2.000){2}{\rule{0.241pt}{0.800pt}}
\put(596,342.84){\rule{0.241pt}{0.800pt}}
\multiput(596.00,342.34)(0.500,1.000){2}{\rule{0.120pt}{0.800pt}}
\put(595.84,345){\rule{0.800pt}{0.482pt}}
\multiput(595.34,345.00)(1.000,1.000){2}{\rule{0.800pt}{0.241pt}}
\put(598,346.34){\rule{0.482pt}{0.800pt}}
\multiput(598.00,345.34)(1.000,2.000){2}{\rule{0.241pt}{0.800pt}}
\put(600,347.84){\rule{0.241pt}{0.800pt}}
\multiput(600.00,347.34)(0.500,1.000){2}{\rule{0.120pt}{0.800pt}}
\put(599.84,350){\rule{0.800pt}{0.482pt}}
\multiput(599.34,350.00)(1.000,1.000){2}{\rule{0.800pt}{0.241pt}}
\put(602,351.34){\rule{0.482pt}{0.800pt}}
\multiput(602.00,350.34)(1.000,2.000){2}{\rule{0.241pt}{0.800pt}}
\put(602.84,354){\rule{0.800pt}{0.482pt}}
\multiput(602.34,354.00)(1.000,1.000){2}{\rule{0.800pt}{0.241pt}}
\put(605,354.84){\rule{0.241pt}{0.800pt}}
\multiput(605.00,354.34)(0.500,1.000){2}{\rule{0.120pt}{0.800pt}}
\put(604.84,357){\rule{0.800pt}{0.482pt}}
\multiput(604.34,357.00)(1.000,1.000){2}{\rule{0.800pt}{0.241pt}}
\put(607,358.34){\rule{0.482pt}{0.800pt}}
\multiput(607.00,357.34)(1.000,2.000){2}{\rule{0.241pt}{0.800pt}}
\put(607.84,361){\rule{0.800pt}{0.482pt}}
\multiput(607.34,361.00)(1.000,1.000){2}{\rule{0.800pt}{0.241pt}}
\put(610,361.84){\rule{0.241pt}{0.800pt}}
\multiput(610.00,361.34)(0.500,1.000){2}{\rule{0.120pt}{0.800pt}}
\put(611,363.34){\rule{0.482pt}{0.800pt}}
\multiput(611.00,362.34)(1.000,2.000){2}{\rule{0.241pt}{0.800pt}}
\put(611.84,366){\rule{0.800pt}{0.482pt}}
\multiput(611.34,366.00)(1.000,1.000){2}{\rule{0.800pt}{0.241pt}}
\put(612.84,368){\rule{0.800pt}{0.482pt}}
\multiput(612.34,368.00)(1.000,1.000){2}{\rule{0.800pt}{0.241pt}}
\put(615,368.84){\rule{0.241pt}{0.800pt}}
\multiput(615.00,368.34)(0.500,1.000){2}{\rule{0.120pt}{0.800pt}}
\put(616,370.34){\rule{0.482pt}{0.800pt}}
\multiput(616.00,369.34)(1.000,2.000){2}{\rule{0.241pt}{0.800pt}}
\put(616.84,373){\rule{0.800pt}{0.482pt}}
\multiput(616.34,373.00)(1.000,1.000){2}{\rule{0.800pt}{0.241pt}}
\put(619,373.84){\rule{0.241pt}{0.800pt}}
\multiput(619.00,373.34)(0.500,1.000){2}{\rule{0.120pt}{0.800pt}}
\put(620,375.34){\rule{0.482pt}{0.800pt}}
\multiput(620.00,374.34)(1.000,2.000){2}{\rule{0.241pt}{0.800pt}}
\put(620.84,378){\rule{0.800pt}{0.482pt}}
\multiput(620.34,378.00)(1.000,1.000){2}{\rule{0.800pt}{0.241pt}}
\put(623,378.84){\rule{0.241pt}{0.800pt}}
\multiput(623.00,378.34)(0.500,1.000){2}{\rule{0.120pt}{0.800pt}}
\put(624,380.34){\rule{0.482pt}{0.800pt}}
\multiput(624.00,379.34)(1.000,2.000){2}{\rule{0.241pt}{0.800pt}}
\put(624.84,383){\rule{0.800pt}{0.482pt}}
\multiput(624.34,383.00)(1.000,1.000){2}{\rule{0.800pt}{0.241pt}}
\put(625.84,385){\rule{0.800pt}{0.482pt}}
\multiput(625.34,385.00)(1.000,1.000){2}{\rule{0.800pt}{0.241pt}}
\put(628,385.84){\rule{0.241pt}{0.800pt}}
\multiput(628.00,385.34)(0.500,1.000){2}{\rule{0.120pt}{0.800pt}}
\put(629,387.34){\rule{0.482pt}{0.800pt}}
\multiput(629.00,386.34)(1.000,2.000){2}{\rule{0.241pt}{0.800pt}}
\put(629.84,390){\rule{0.800pt}{0.482pt}}
\multiput(629.34,390.00)(1.000,1.000){2}{\rule{0.800pt}{0.241pt}}
\put(632,390.84){\rule{0.241pt}{0.800pt}}
\multiput(632.00,390.34)(0.500,1.000){2}{\rule{0.120pt}{0.800pt}}
\put(633,392.34){\rule{0.482pt}{0.800pt}}
\multiput(633.00,391.34)(1.000,2.000){2}{\rule{0.241pt}{0.800pt}}
\put(633.84,395){\rule{0.800pt}{0.482pt}}
\multiput(633.34,395.00)(1.000,1.000){2}{\rule{0.800pt}{0.241pt}}
\put(636,395.84){\rule{0.241pt}{0.800pt}}
\multiput(636.00,395.34)(0.500,1.000){2}{\rule{0.120pt}{0.800pt}}
\put(635.84,398){\rule{0.800pt}{0.482pt}}
\multiput(635.34,398.00)(1.000,1.000){2}{\rule{0.800pt}{0.241pt}}
\put(638,399.34){\rule{0.482pt}{0.800pt}}
\multiput(638.00,398.34)(1.000,2.000){2}{\rule{0.241pt}{0.800pt}}
\put(638.84,402){\rule{0.800pt}{0.482pt}}
\multiput(638.34,402.00)(1.000,1.000){2}{\rule{0.800pt}{0.241pt}}
\put(641,402.84){\rule{0.241pt}{0.800pt}}
\multiput(641.00,402.34)(0.500,1.000){2}{\rule{0.120pt}{0.800pt}}
\put(642,404.34){\rule{0.482pt}{0.800pt}}
\multiput(642.00,403.34)(1.000,2.000){2}{\rule{0.241pt}{0.800pt}}
\put(642.84,407){\rule{0.800pt}{0.482pt}}
\multiput(642.34,407.00)(1.000,1.000){2}{\rule{0.800pt}{0.241pt}}
\put(645,407.84){\rule{0.241pt}{0.800pt}}
\multiput(645.00,407.34)(0.500,1.000){2}{\rule{0.120pt}{0.800pt}}
\put(646,409.34){\rule{0.482pt}{0.800pt}}
\multiput(646.00,408.34)(1.000,2.000){2}{\rule{0.241pt}{0.800pt}}
\put(646.84,412){\rule{0.800pt}{0.482pt}}
\multiput(646.34,412.00)(1.000,1.000){2}{\rule{0.800pt}{0.241pt}}
\put(647.84,414){\rule{0.800pt}{0.482pt}}
\multiput(647.34,414.00)(1.000,1.000){2}{\rule{0.800pt}{0.241pt}}
\put(650,414.84){\rule{0.241pt}{0.800pt}}
\multiput(650.00,414.34)(0.500,1.000){2}{\rule{0.120pt}{0.800pt}}
\put(651,416.34){\rule{0.482pt}{0.800pt}}
\multiput(651.00,415.34)(1.000,2.000){2}{\rule{0.241pt}{0.800pt}}
\put(651.84,419){\rule{0.800pt}{0.482pt}}
\multiput(651.34,419.00)(1.000,1.000){2}{\rule{0.800pt}{0.241pt}}
\put(654,419.84){\rule{0.241pt}{0.800pt}}
\multiput(654.00,419.34)(0.500,1.000){2}{\rule{0.120pt}{0.800pt}}
\put(655,421.34){\rule{0.482pt}{0.800pt}}
\multiput(655.00,420.34)(1.000,2.000){2}{\rule{0.241pt}{0.800pt}}
\put(655.84,421){\rule{0.800pt}{0.723pt}}
\multiput(655.34,422.50)(1.000,-1.500){2}{\rule{0.800pt}{0.361pt}}
\put(656.84,421){\rule{0.800pt}{0.482pt}}
\multiput(656.34,421.00)(1.000,1.000){2}{\rule{0.800pt}{0.241pt}}
\put(657.84,423){\rule{0.800pt}{0.482pt}}
\multiput(657.34,423.00)(1.000,1.000){2}{\rule{0.800pt}{0.241pt}}
\put(660,423.84){\rule{0.482pt}{0.800pt}}
\multiput(660.00,423.34)(1.000,1.000){2}{\rule{0.241pt}{0.800pt}}
\put(660.84,426){\rule{0.800pt}{0.482pt}}
\multiput(660.34,426.00)(1.000,1.000){2}{\rule{0.800pt}{0.241pt}}
\put(661.84,428){\rule{0.800pt}{0.482pt}}
\multiput(661.34,428.00)(1.000,1.000){2}{\rule{0.800pt}{0.241pt}}
\put(664,429.34){\rule{0.482pt}{0.800pt}}
\multiput(664.00,428.34)(1.000,2.000){2}{\rule{0.241pt}{0.800pt}}
\put(666,430.84){\rule{0.241pt}{0.800pt}}
\multiput(666.00,430.34)(0.500,1.000){2}{\rule{0.120pt}{0.800pt}}
\put(665.84,433){\rule{0.800pt}{0.482pt}}
\multiput(665.34,433.00)(1.000,1.000){2}{\rule{0.800pt}{0.241pt}}
\put(666.84,435){\rule{0.800pt}{0.482pt}}
\multiput(666.34,435.00)(1.000,1.000){2}{\rule{0.800pt}{0.241pt}}
\put(669,436.34){\rule{0.482pt}{0.800pt}}
\multiput(669.00,435.34)(1.000,2.000){2}{\rule{0.241pt}{0.800pt}}
\put(671,437.84){\rule{0.241pt}{0.800pt}}
\multiput(671.00,437.34)(0.500,1.000){2}{\rule{0.120pt}{0.800pt}}
\put(670.84,440){\rule{0.800pt}{0.482pt}}
\multiput(670.34,440.00)(1.000,1.000){2}{\rule{0.800pt}{0.241pt}}
\put(673,441.34){\rule{0.482pt}{0.800pt}}
\multiput(673.00,440.34)(1.000,2.000){2}{\rule{0.241pt}{0.800pt}}
\put(673.84,444){\rule{0.800pt}{0.482pt}}
\multiput(673.34,444.00)(1.000,1.000){2}{\rule{0.800pt}{0.241pt}}
\put(676,444.84){\rule{0.241pt}{0.800pt}}
\multiput(676.00,444.34)(0.500,1.000){2}{\rule{0.120pt}{0.800pt}}
\put(677,446.34){\rule{0.482pt}{0.800pt}}
\multiput(677.00,445.34)(1.000,2.000){2}{\rule{0.241pt}{0.800pt}}
\put(677.84,449){\rule{0.800pt}{0.482pt}}
\multiput(677.34,449.00)(1.000,1.000){2}{\rule{0.800pt}{0.241pt}}
\put(678.84,451){\rule{0.800pt}{0.482pt}}
\multiput(678.34,451.00)(1.000,1.000){2}{\rule{0.800pt}{0.241pt}}
\put(681,451.84){\rule{0.241pt}{0.800pt}}
\multiput(681.00,451.34)(0.500,1.000){2}{\rule{0.120pt}{0.800pt}}
\put(682,453.34){\rule{0.482pt}{0.800pt}}
\multiput(682.00,452.34)(1.000,2.000){2}{\rule{0.241pt}{0.800pt}}
\put(682.84,456){\rule{0.800pt}{0.482pt}}
\multiput(682.34,456.00)(1.000,1.000){2}{\rule{0.800pt}{0.241pt}}
\put(685,456.84){\rule{0.241pt}{0.800pt}}
\multiput(685.00,456.34)(0.500,1.000){2}{\rule{0.120pt}{0.800pt}}
\put(686,458.34){\rule{0.482pt}{0.800pt}}
\multiput(686.00,457.34)(1.000,2.000){2}{\rule{0.241pt}{0.800pt}}
\put(686.84,461){\rule{0.800pt}{0.482pt}}
\multiput(686.34,461.00)(1.000,1.000){2}{\rule{0.800pt}{0.241pt}}
\put(687.84,463){\rule{0.800pt}{0.482pt}}
\multiput(687.34,463.00)(1.000,1.000){2}{\rule{0.800pt}{0.241pt}}
\put(690,463.84){\rule{0.241pt}{0.800pt}}
\multiput(690.00,463.34)(0.500,1.000){2}{\rule{0.120pt}{0.800pt}}
\put(691,465.34){\rule{0.482pt}{0.800pt}}
\multiput(691.00,464.34)(1.000,2.000){2}{\rule{0.241pt}{0.800pt}}
\put(691.84,468){\rule{0.800pt}{0.482pt}}
\multiput(691.34,468.00)(1.000,1.000){2}{\rule{0.800pt}{0.241pt}}
\put(692.84,470){\rule{0.800pt}{0.482pt}}
\multiput(692.34,470.00)(1.000,1.000){2}{\rule{0.800pt}{0.241pt}}
\put(695,470.84){\rule{0.482pt}{0.800pt}}
\multiput(695.00,470.34)(1.000,1.000){2}{\rule{0.241pt}{0.800pt}}
\put(695.84,473){\rule{0.800pt}{0.482pt}}
\multiput(695.34,473.00)(1.000,1.000){2}{\rule{0.800pt}{0.241pt}}
\put(696.84,475){\rule{0.800pt}{0.482pt}}
\multiput(696.34,475.00)(1.000,1.000){2}{\rule{0.800pt}{0.241pt}}
\put(699,476.34){\rule{0.482pt}{0.800pt}}
\multiput(699.00,475.34)(1.000,2.000){2}{\rule{0.241pt}{0.800pt}}
\put(701,477.84){\rule{0.241pt}{0.800pt}}
\multiput(701.00,477.34)(0.500,1.000){2}{\rule{0.120pt}{0.800pt}}
\put(700.84,480){\rule{0.800pt}{0.482pt}}
\multiput(700.34,480.00)(1.000,1.000){2}{\rule{0.800pt}{0.241pt}}
\put(701.84,482){\rule{0.800pt}{0.482pt}}
\multiput(701.34,482.00)(1.000,1.000){2}{\rule{0.800pt}{0.241pt}}
\put(704,483.34){\rule{0.482pt}{0.800pt}}
\multiput(704.00,482.34)(1.000,2.000){2}{\rule{0.241pt}{0.800pt}}
\put(706,484.84){\rule{0.241pt}{0.800pt}}
\multiput(706.00,484.34)(0.500,1.000){2}{\rule{0.120pt}{0.800pt}}
\put(705.84,487){\rule{0.800pt}{0.482pt}}
\multiput(705.34,487.00)(1.000,1.000){2}{\rule{0.800pt}{0.241pt}}
\put(708,488.34){\rule{0.482pt}{0.800pt}}
\multiput(708.00,487.34)(1.000,2.000){2}{\rule{0.241pt}{0.800pt}}
\put(710,489.84){\rule{0.241pt}{0.800pt}}
\multiput(710.00,489.34)(0.500,1.000){2}{\rule{0.120pt}{0.800pt}}
\put(709.84,492){\rule{0.800pt}{0.482pt}}
\multiput(709.34,492.00)(1.000,1.000){2}{\rule{0.800pt}{0.241pt}}
\put(710.84,494){\rule{0.800pt}{0.482pt}}
\multiput(710.34,494.00)(1.000,1.000){2}{\rule{0.800pt}{0.241pt}}
\put(713,495.34){\rule{0.482pt}{0.800pt}}
\multiput(713.00,494.34)(1.000,2.000){2}{\rule{0.241pt}{0.800pt}}
\put(715,496.84){\rule{0.241pt}{0.800pt}}
\multiput(715.00,496.34)(0.500,1.000){2}{\rule{0.120pt}{0.800pt}}
\put(714.84,499){\rule{0.800pt}{0.482pt}}
\multiput(714.34,499.00)(1.000,1.000){2}{\rule{0.800pt}{0.241pt}}
\put(717,500.34){\rule{0.482pt}{0.800pt}}
\multiput(717.00,499.34)(1.000,2.000){2}{\rule{0.241pt}{0.800pt}}
\put(717.84,503){\rule{0.800pt}{0.482pt}}
\multiput(717.34,503.00)(1.000,1.000){2}{\rule{0.800pt}{0.241pt}}
\put(720,503.84){\rule{0.241pt}{0.800pt}}
\multiput(720.00,503.34)(0.500,1.000){2}{\rule{0.120pt}{0.800pt}}
\put(719.84,506){\rule{0.800pt}{0.482pt}}
\multiput(719.34,506.00)(1.000,1.000){2}{\rule{0.800pt}{0.241pt}}
\put(722,507.34){\rule{0.482pt}{0.800pt}}
\multiput(722.00,506.34)(1.000,2.000){2}{\rule{0.241pt}{0.800pt}}
\put(722.84,510){\rule{0.800pt}{0.482pt}}
\multiput(722.34,510.00)(1.000,1.000){2}{\rule{0.800pt}{0.241pt}}
\put(725,510.84){\rule{0.241pt}{0.800pt}}
\multiput(725.00,510.34)(0.500,1.000){2}{\rule{0.120pt}{0.800pt}}
\put(726,512.34){\rule{0.482pt}{0.800pt}}
\multiput(726.00,511.34)(1.000,2.000){2}{\rule{0.241pt}{0.800pt}}
\put(726.84,515){\rule{0.800pt}{0.482pt}}
\multiput(726.34,515.00)(1.000,1.000){2}{\rule{0.800pt}{0.241pt}}
\put(729,515.84){\rule{0.241pt}{0.800pt}}
\multiput(729.00,515.34)(0.500,1.000){2}{\rule{0.120pt}{0.800pt}}
\put(730,517.34){\rule{0.482pt}{0.800pt}}
\multiput(730.00,516.34)(1.000,2.000){2}{\rule{0.241pt}{0.800pt}}
\put(730.84,520){\rule{0.800pt}{0.482pt}}
\multiput(730.34,520.00)(1.000,1.000){2}{\rule{0.800pt}{0.241pt}}
\put(731.84,522){\rule{0.800pt}{0.482pt}}
\multiput(731.34,522.00)(1.000,1.000){2}{\rule{0.800pt}{0.241pt}}
\put(734,522.84){\rule{0.241pt}{0.800pt}}
\multiput(734.00,522.34)(0.500,1.000){2}{\rule{0.120pt}{0.800pt}}
\put(735,524.34){\rule{0.482pt}{0.800pt}}
\multiput(735.00,523.34)(1.000,2.000){2}{\rule{0.241pt}{0.800pt}}
\put(483.0,210.0){\usebox{\plotpoint}}
\sbox{\plotpoint}{\rule[-0.200pt]{0.400pt}{0.400pt}}%
\put(221.0,143.0){\rule[-0.200pt]{124.545pt}{0.400pt}}
\put(738.0,143.0){\rule[-0.200pt]{0.400pt}{104.551pt}}
\put(221.0,577.0){\rule[-0.200pt]{124.545pt}{0.400pt}}
\put(221.0,143.0){\rule[-0.200pt]{0.400pt}{104.551pt}}
\end{picture}
=======
% GNUPLOT: LaTeX picture
\setlength{\unitlength}{0.240900pt}
\ifx\plotpoint\undefined\newsavebox{\plotpoint}\fi
\begin{picture}(959,720)(0,0)
\sbox{\plotpoint}{\rule[-0.200pt]{0.400pt}{0.400pt}}%
\put(181,143){\makebox(0,0)[r]{$0.1$}}
\put(201.0,143.0){\rule[-0.200pt]{4.818pt}{0.400pt}}
\put(181,197){\makebox(0,0)[r]{$0.2$}}
\put(201.0,197.0){\rule[-0.200pt]{4.818pt}{0.400pt}}
\put(181,252){\makebox(0,0)[r]{$0.3$}}
\put(201.0,252.0){\rule[-0.200pt]{4.818pt}{0.400pt}}
\put(181,306){\makebox(0,0)[r]{$0.4$}}
\put(201.0,306.0){\rule[-0.200pt]{4.818pt}{0.400pt}}
\put(181,360){\makebox(0,0)[r]{$0.5$}}
\put(201.0,360.0){\rule[-0.200pt]{4.818pt}{0.400pt}}
\put(181,414){\makebox(0,0)[r]{$0.6$}}
\put(201.0,414.0){\rule[-0.200pt]{4.818pt}{0.400pt}}
\put(181,468){\makebox(0,0)[r]{$0.7$}}
\put(201.0,468.0){\rule[-0.200pt]{4.818pt}{0.400pt}}
\put(181,523){\makebox(0,0)[r]{$0.8$}}
\put(201.0,523.0){\rule[-0.200pt]{4.818pt}{0.400pt}}
\put(181,577){\makebox(0,0)[r]{$0.9$}}
\put(201.0,577.0){\rule[-0.200pt]{4.818pt}{0.400pt}}
\put(221.0,123.0){\rule[-0.200pt]{0.400pt}{4.818pt}}
\put(221,82){\makebox(0,0){$0$}}
\put(221.0,577.0){\rule[-0.200pt]{0.400pt}{4.818pt}}
\put(286.0,123.0){\rule[-0.200pt]{0.400pt}{4.818pt}}
%\put(286,82){\makebox(0,0){$500$}}
\put(286.0,577.0){\rule[-0.200pt]{0.400pt}{4.818pt}}
\put(350.0,123.0){\rule[-0.200pt]{0.400pt}{4.818pt}}
\put(350,82){\makebox(0,0){$1000$}}
\put(350.0,577.0){\rule[-0.200pt]{0.400pt}{4.818pt}}
\put(415.0,123.0){\rule[-0.200pt]{0.400pt}{4.818pt}}
%\put(415,82){\makebox(0,0){$1500$}}
\put(415.0,577.0){\rule[-0.200pt]{0.400pt}{4.818pt}}
\put(479.0,123.0){\rule[-0.200pt]{0.400pt}{4.818pt}}
\put(479,82){\makebox(0,0){$2000$}}
\put(479.0,577.0){\rule[-0.200pt]{0.400pt}{4.818pt}}
\put(544.0,123.0){\rule[-0.200pt]{0.400pt}{4.818pt}}
%\put(544,82){\makebox(0,0){$2500$}}
\put(544.0,577.0){\rule[-0.200pt]{0.400pt}{4.818pt}}
\put(609.0,123.0){\rule[-0.200pt]{0.400pt}{4.818pt}}
\put(609,82){\makebox(0,0){$3000$}}
\put(609.0,577.0){\rule[-0.200pt]{0.400pt}{4.818pt}}
\put(673.0,123.0){\rule[-0.200pt]{0.400pt}{4.818pt}}
%\put(673,82){\makebox(0,0){$3500$}}
\put(673.0,577.0){\rule[-0.200pt]{0.400pt}{4.818pt}}
\put(738.0,123.0){\rule[-0.200pt]{0.400pt}{4.818pt}}
\put(738,82){\makebox(0,0){$4000$}}
\put(738.0,577.0){\rule[-0.200pt]{0.400pt}{4.818pt}}
\put(778,143){\makebox(0,0)[l]{ 0}}
\put(738.0,143.0){\rule[-0.200pt]{4.818pt}{0.400pt}}
\put(778,230){\makebox(0,0)[l]{ 500}}
\put(738.0,230.0){\rule[-0.200pt]{4.818pt}{0.400pt}}
\put(778,317){\makebox(0,0)[l]{ 1000}}
\put(738.0,317.0){\rule[-0.200pt]{4.818pt}{0.400pt}}
\put(778,403){\makebox(0,0)[l]{ 1500}}
\put(738.0,403.0){\rule[-0.200pt]{4.818pt}{0.400pt}}
\put(778,490){\makebox(0,0)[l]{ 2000}}
\put(738.0,490.0){\rule[-0.200pt]{4.818pt}{0.400pt}}
\put(778,577){\makebox(0,0)[l]{ 2500}}
\put(738.0,577.0){\rule[-0.200pt]{4.818pt}{0.400pt}}
\put(221.0,143.0){\rule[-0.200pt]{124.545pt}{0.400pt}}
\put(738.0,143.0){\rule[-0.200pt]{0.400pt}{104.551pt}}
\put(221.0,577.0){\rule[-0.200pt]{124.545pt}{0.400pt}}
\put(221.0,143.0){\rule[-0.200pt]{0.400pt}{104.551pt}}
\put(40,360){\makebox(0,0){$\mu$ axis}}
\put(967,360){\makebox(0,0){Width}}
\put(479,21){\makebox(0,0){$t$ axis}}
\put(479,659){\makebox(0,0){\adwinz}}
\put(578,537){\makebox(0,0)[r]{$\mu_t$}}
\put(598.0,537.0){\rule[-0.200pt]{24.090pt}{0.400pt}}
\put(222,523){\usebox{\plotpoint}}
\put(349.67,520){\rule{0.400pt}{0.723pt}}
\multiput(349.17,521.50)(1.000,-1.500){2}{\rule{0.400pt}{0.361pt}}
\put(351.17,516){\rule{0.400pt}{0.900pt}}
\multiput(350.17,518.13)(2.000,-2.132){2}{\rule{0.400pt}{0.450pt}}
\put(352.67,513){\rule{0.400pt}{0.723pt}}
\multiput(352.17,514.50)(1.000,-1.500){2}{\rule{0.400pt}{0.361pt}}
\put(353.67,510){\rule{0.400pt}{0.723pt}}
\multiput(353.17,511.50)(1.000,-1.500){2}{\rule{0.400pt}{0.361pt}}
\put(355.17,507){\rule{0.400pt}{0.700pt}}
\multiput(354.17,508.55)(2.000,-1.547){2}{\rule{0.400pt}{0.350pt}}
\put(356.67,504){\rule{0.400pt}{0.723pt}}
\multiput(356.17,505.50)(1.000,-1.500){2}{\rule{0.400pt}{0.361pt}}
\put(357.67,500){\rule{0.400pt}{0.964pt}}
\multiput(357.17,502.00)(1.000,-2.000){2}{\rule{0.400pt}{0.482pt}}
\put(358.67,497){\rule{0.400pt}{0.723pt}}
\multiput(358.17,498.50)(1.000,-1.500){2}{\rule{0.400pt}{0.361pt}}
\put(360.17,494){\rule{0.400pt}{0.700pt}}
\multiput(359.17,495.55)(2.000,-1.547){2}{\rule{0.400pt}{0.350pt}}
\put(361.67,491){\rule{0.400pt}{0.723pt}}
\multiput(361.17,492.50)(1.000,-1.500){2}{\rule{0.400pt}{0.361pt}}
\put(362.67,488){\rule{0.400pt}{0.723pt}}
\multiput(362.17,489.50)(1.000,-1.500){2}{\rule{0.400pt}{0.361pt}}
\put(364.17,485){\rule{0.400pt}{0.700pt}}
\multiput(363.17,486.55)(2.000,-1.547){2}{\rule{0.400pt}{0.350pt}}
\put(365.67,481){\rule{0.400pt}{0.964pt}}
\multiput(365.17,483.00)(1.000,-2.000){2}{\rule{0.400pt}{0.482pt}}
\put(366.67,478){\rule{0.400pt}{0.723pt}}
\multiput(366.17,479.50)(1.000,-1.500){2}{\rule{0.400pt}{0.361pt}}
\put(368.17,475){\rule{0.400pt}{0.700pt}}
\multiput(367.17,476.55)(2.000,-1.547){2}{\rule{0.400pt}{0.350pt}}
\put(369.67,472){\rule{0.400pt}{0.723pt}}
\multiput(369.17,473.50)(1.000,-1.500){2}{\rule{0.400pt}{0.361pt}}
\put(370.67,469){\rule{0.400pt}{0.723pt}}
\multiput(370.17,470.50)(1.000,-1.500){2}{\rule{0.400pt}{0.361pt}}
\put(371.67,466){\rule{0.400pt}{0.723pt}}
\multiput(371.17,467.50)(1.000,-1.500){2}{\rule{0.400pt}{0.361pt}}
\put(373.17,462){\rule{0.400pt}{0.900pt}}
\multiput(372.17,464.13)(2.000,-2.132){2}{\rule{0.400pt}{0.450pt}}
\put(374.67,459){\rule{0.400pt}{0.723pt}}
\multiput(374.17,460.50)(1.000,-1.500){2}{\rule{0.400pt}{0.361pt}}
\put(375.67,456){\rule{0.400pt}{0.723pt}}
\multiput(375.17,457.50)(1.000,-1.500){2}{\rule{0.400pt}{0.361pt}}
\put(377.17,453){\rule{0.400pt}{0.700pt}}
\multiput(376.17,454.55)(2.000,-1.547){2}{\rule{0.400pt}{0.350pt}}
\put(378.67,450){\rule{0.400pt}{0.723pt}}
\multiput(378.17,451.50)(1.000,-1.500){2}{\rule{0.400pt}{0.361pt}}
\put(379.67,446){\rule{0.400pt}{0.964pt}}
\multiput(379.17,448.00)(1.000,-2.000){2}{\rule{0.400pt}{0.482pt}}
\put(380.67,443){\rule{0.400pt}{0.723pt}}
\multiput(380.17,444.50)(1.000,-1.500){2}{\rule{0.400pt}{0.361pt}}
\put(382.17,440){\rule{0.400pt}{0.700pt}}
\multiput(381.17,441.55)(2.000,-1.547){2}{\rule{0.400pt}{0.350pt}}
\put(383.67,437){\rule{0.400pt}{0.723pt}}
\multiput(383.17,438.50)(1.000,-1.500){2}{\rule{0.400pt}{0.361pt}}
\put(384.67,434){\rule{0.400pt}{0.723pt}}
\multiput(384.17,435.50)(1.000,-1.500){2}{\rule{0.400pt}{0.361pt}}
\put(386.17,431){\rule{0.400pt}{0.700pt}}
\multiput(385.17,432.55)(2.000,-1.547){2}{\rule{0.400pt}{0.350pt}}
\put(387.67,427){\rule{0.400pt}{0.964pt}}
\multiput(387.17,429.00)(1.000,-2.000){2}{\rule{0.400pt}{0.482pt}}
\put(388.67,424){\rule{0.400pt}{0.723pt}}
\multiput(388.17,425.50)(1.000,-1.500){2}{\rule{0.400pt}{0.361pt}}
\put(390.17,421){\rule{0.400pt}{0.700pt}}
\multiput(389.17,422.55)(2.000,-1.547){2}{\rule{0.400pt}{0.350pt}}
\put(391.67,418){\rule{0.400pt}{0.723pt}}
\multiput(391.17,419.50)(1.000,-1.500){2}{\rule{0.400pt}{0.361pt}}
\put(392.67,415){\rule{0.400pt}{0.723pt}}
\multiput(392.17,416.50)(1.000,-1.500){2}{\rule{0.400pt}{0.361pt}}
\put(393.67,411){\rule{0.400pt}{0.964pt}}
\multiput(393.17,413.00)(1.000,-2.000){2}{\rule{0.400pt}{0.482pt}}
\put(395.17,408){\rule{0.400pt}{0.700pt}}
\multiput(394.17,409.55)(2.000,-1.547){2}{\rule{0.400pt}{0.350pt}}
\put(396.67,405){\rule{0.400pt}{0.723pt}}
\multiput(396.17,406.50)(1.000,-1.500){2}{\rule{0.400pt}{0.361pt}}
\put(397.67,402){\rule{0.400pt}{0.723pt}}
\multiput(397.17,403.50)(1.000,-1.500){2}{\rule{0.400pt}{0.361pt}}
\put(399.17,399){\rule{0.400pt}{0.700pt}}
\multiput(398.17,400.55)(2.000,-1.547){2}{\rule{0.400pt}{0.350pt}}
\put(400.67,396){\rule{0.400pt}{0.723pt}}
\multiput(400.17,397.50)(1.000,-1.500){2}{\rule{0.400pt}{0.361pt}}
\put(401.67,392){\rule{0.400pt}{0.964pt}}
\multiput(401.17,394.00)(1.000,-2.000){2}{\rule{0.400pt}{0.482pt}}
\put(402.67,389){\rule{0.400pt}{0.723pt}}
\multiput(402.17,390.50)(1.000,-1.500){2}{\rule{0.400pt}{0.361pt}}
\put(404.17,386){\rule{0.400pt}{0.700pt}}
\multiput(403.17,387.55)(2.000,-1.547){2}{\rule{0.400pt}{0.350pt}}
\put(405.67,383){\rule{0.400pt}{0.723pt}}
\multiput(405.17,384.50)(1.000,-1.500){2}{\rule{0.400pt}{0.361pt}}
\put(406.67,380){\rule{0.400pt}{0.723pt}}
\multiput(406.17,381.50)(1.000,-1.500){2}{\rule{0.400pt}{0.361pt}}
\put(408.17,377){\rule{0.400pt}{0.700pt}}
\multiput(407.17,378.55)(2.000,-1.547){2}{\rule{0.400pt}{0.350pt}}
\put(409.67,373){\rule{0.400pt}{0.964pt}}
\multiput(409.17,375.00)(1.000,-2.000){2}{\rule{0.400pt}{0.482pt}}
\put(410.67,370){\rule{0.400pt}{0.723pt}}
\multiput(410.17,371.50)(1.000,-1.500){2}{\rule{0.400pt}{0.361pt}}
\put(412.17,367){\rule{0.400pt}{0.700pt}}
\multiput(411.17,368.55)(2.000,-1.547){2}{\rule{0.400pt}{0.350pt}}
\put(413.67,364){\rule{0.400pt}{0.723pt}}
\multiput(413.17,365.50)(1.000,-1.500){2}{\rule{0.400pt}{0.361pt}}
\put(414.67,361){\rule{0.400pt}{0.723pt}}
\multiput(414.17,362.50)(1.000,-1.500){2}{\rule{0.400pt}{0.361pt}}
\put(415.67,357){\rule{0.400pt}{0.964pt}}
\multiput(415.17,359.00)(1.000,-2.000){2}{\rule{0.400pt}{0.482pt}}
\put(417.17,354){\rule{0.400pt}{0.700pt}}
\multiput(416.17,355.55)(2.000,-1.547){2}{\rule{0.400pt}{0.350pt}}
\put(418.67,351){\rule{0.400pt}{0.723pt}}
\multiput(418.17,352.50)(1.000,-1.500){2}{\rule{0.400pt}{0.361pt}}
\put(419.67,348){\rule{0.400pt}{0.723pt}}
\multiput(419.17,349.50)(1.000,-1.500){2}{\rule{0.400pt}{0.361pt}}
\put(421.17,345){\rule{0.400pt}{0.700pt}}
\multiput(420.17,346.55)(2.000,-1.547){2}{\rule{0.400pt}{0.350pt}}
\put(422.67,342){\rule{0.400pt}{0.723pt}}
\multiput(422.17,343.50)(1.000,-1.500){2}{\rule{0.400pt}{0.361pt}}
\put(423.67,338){\rule{0.400pt}{0.964pt}}
\multiput(423.17,340.00)(1.000,-2.000){2}{\rule{0.400pt}{0.482pt}}
\put(424.67,335){\rule{0.400pt}{0.723pt}}
\multiput(424.17,336.50)(1.000,-1.500){2}{\rule{0.400pt}{0.361pt}}
\put(426.17,332){\rule{0.400pt}{0.700pt}}
\multiput(425.17,333.55)(2.000,-1.547){2}{\rule{0.400pt}{0.350pt}}
\put(427.67,329){\rule{0.400pt}{0.723pt}}
\multiput(427.17,330.50)(1.000,-1.500){2}{\rule{0.400pt}{0.361pt}}
\put(428.67,326){\rule{0.400pt}{0.723pt}}
\multiput(428.17,327.50)(1.000,-1.500){2}{\rule{0.400pt}{0.361pt}}
\put(430.17,322){\rule{0.400pt}{0.900pt}}
\multiput(429.17,324.13)(2.000,-2.132){2}{\rule{0.400pt}{0.450pt}}
\put(431.67,319){\rule{0.400pt}{0.723pt}}
\multiput(431.17,320.50)(1.000,-1.500){2}{\rule{0.400pt}{0.361pt}}
\put(432.67,316){\rule{0.400pt}{0.723pt}}
\multiput(432.17,317.50)(1.000,-1.500){2}{\rule{0.400pt}{0.361pt}}
\put(433.67,313){\rule{0.400pt}{0.723pt}}
\multiput(433.17,314.50)(1.000,-1.500){2}{\rule{0.400pt}{0.361pt}}
\put(435.17,310){\rule{0.400pt}{0.700pt}}
\multiput(434.17,311.55)(2.000,-1.547){2}{\rule{0.400pt}{0.350pt}}
\put(436.67,307){\rule{0.400pt}{0.723pt}}
\multiput(436.17,308.50)(1.000,-1.500){2}{\rule{0.400pt}{0.361pt}}
\put(437.67,303){\rule{0.400pt}{0.964pt}}
\multiput(437.17,305.00)(1.000,-2.000){2}{\rule{0.400pt}{0.482pt}}
\put(439.17,300){\rule{0.400pt}{0.700pt}}
\multiput(438.17,301.55)(2.000,-1.547){2}{\rule{0.400pt}{0.350pt}}
\put(440.67,297){\rule{0.400pt}{0.723pt}}
\multiput(440.17,298.50)(1.000,-1.500){2}{\rule{0.400pt}{0.361pt}}
\put(441.67,294){\rule{0.400pt}{0.723pt}}
\multiput(441.17,295.50)(1.000,-1.500){2}{\rule{0.400pt}{0.361pt}}
\put(443.17,291){\rule{0.400pt}{0.700pt}}
\multiput(442.17,292.55)(2.000,-1.547){2}{\rule{0.400pt}{0.350pt}}
\put(444.67,288){\rule{0.400pt}{0.723pt}}
\multiput(444.17,289.50)(1.000,-1.500){2}{\rule{0.400pt}{0.361pt}}
\put(445.67,284){\rule{0.400pt}{0.964pt}}
\multiput(445.17,286.00)(1.000,-2.000){2}{\rule{0.400pt}{0.482pt}}
\put(446.67,281){\rule{0.400pt}{0.723pt}}
\multiput(446.17,282.50)(1.000,-1.500){2}{\rule{0.400pt}{0.361pt}}
\put(448.17,278){\rule{0.400pt}{0.700pt}}
\multiput(447.17,279.55)(2.000,-1.547){2}{\rule{0.400pt}{0.350pt}}
\put(449.67,275){\rule{0.400pt}{0.723pt}}
\multiput(449.17,276.50)(1.000,-1.500){2}{\rule{0.400pt}{0.361pt}}
\put(450.67,272){\rule{0.400pt}{0.723pt}}
\multiput(450.17,273.50)(1.000,-1.500){2}{\rule{0.400pt}{0.361pt}}
\put(452.17,268){\rule{0.400pt}{0.900pt}}
\multiput(451.17,270.13)(2.000,-2.132){2}{\rule{0.400pt}{0.450pt}}
\put(453.67,265){\rule{0.400pt}{0.723pt}}
\multiput(453.17,266.50)(1.000,-1.500){2}{\rule{0.400pt}{0.361pt}}
\put(454.67,262){\rule{0.400pt}{0.723pt}}
\multiput(454.17,263.50)(1.000,-1.500){2}{\rule{0.400pt}{0.361pt}}
\put(455.67,259){\rule{0.400pt}{0.723pt}}
\multiput(455.17,260.50)(1.000,-1.500){2}{\rule{0.400pt}{0.361pt}}
\put(457.17,256){\rule{0.400pt}{0.700pt}}
\multiput(456.17,257.55)(2.000,-1.547){2}{\rule{0.400pt}{0.350pt}}
\put(458.67,253){\rule{0.400pt}{0.723pt}}
\multiput(458.17,254.50)(1.000,-1.500){2}{\rule{0.400pt}{0.361pt}}
\put(459.67,249){\rule{0.400pt}{0.964pt}}
\multiput(459.17,251.00)(1.000,-2.000){2}{\rule{0.400pt}{0.482pt}}
\put(461.17,246){\rule{0.400pt}{0.700pt}}
\multiput(460.17,247.55)(2.000,-1.547){2}{\rule{0.400pt}{0.350pt}}
\put(462.67,243){\rule{0.400pt}{0.723pt}}
\multiput(462.17,244.50)(1.000,-1.500){2}{\rule{0.400pt}{0.361pt}}
\put(463.67,240){\rule{0.400pt}{0.723pt}}
\multiput(463.17,241.50)(1.000,-1.500){2}{\rule{0.400pt}{0.361pt}}
\put(465.17,237){\rule{0.400pt}{0.700pt}}
\multiput(464.17,238.55)(2.000,-1.547){2}{\rule{0.400pt}{0.350pt}}
\put(466.67,233){\rule{0.400pt}{0.964pt}}
\multiput(466.17,235.00)(1.000,-2.000){2}{\rule{0.400pt}{0.482pt}}
\put(467.67,230){\rule{0.400pt}{0.723pt}}
\multiput(467.17,231.50)(1.000,-1.500){2}{\rule{0.400pt}{0.361pt}}
\put(468.67,227){\rule{0.400pt}{0.723pt}}
\multiput(468.17,228.50)(1.000,-1.500){2}{\rule{0.400pt}{0.361pt}}
\put(470.17,224){\rule{0.400pt}{0.700pt}}
\multiput(469.17,225.55)(2.000,-1.547){2}{\rule{0.400pt}{0.350pt}}
\put(471.67,221){\rule{0.400pt}{0.723pt}}
\multiput(471.17,222.50)(1.000,-1.500){2}{\rule{0.400pt}{0.361pt}}
\put(472.67,218){\rule{0.400pt}{0.723pt}}
\multiput(472.17,219.50)(1.000,-1.500){2}{\rule{0.400pt}{0.361pt}}
\put(474.17,214){\rule{0.400pt}{0.900pt}}
\multiput(473.17,216.13)(2.000,-2.132){2}{\rule{0.400pt}{0.450pt}}
\put(475.67,211){\rule{0.400pt}{0.723pt}}
\multiput(475.17,212.50)(1.000,-1.500){2}{\rule{0.400pt}{0.361pt}}
\put(476.67,208){\rule{0.400pt}{0.723pt}}
\multiput(476.17,209.50)(1.000,-1.500){2}{\rule{0.400pt}{0.361pt}}
\put(477.67,205){\rule{0.400pt}{0.723pt}}
\multiput(477.17,206.50)(1.000,-1.500){2}{\rule{0.400pt}{0.361pt}}
\put(479.17,202){\rule{0.400pt}{0.700pt}}
\multiput(478.17,203.55)(2.000,-1.547){2}{\rule{0.400pt}{0.350pt}}
\put(480.67,199){\rule{0.400pt}{0.723pt}}
\multiput(480.17,200.50)(1.000,-1.500){2}{\rule{0.400pt}{0.361pt}}
\put(481.67,197){\rule{0.400pt}{0.482pt}}
\multiput(481.17,198.00)(1.000,-1.000){2}{\rule{0.400pt}{0.241pt}}
\put(222.0,523.0){\rule[-0.200pt]{30.835pt}{0.400pt}}
\put(483.0,197.0){\rule[-0.200pt]{61.189pt}{0.400pt}}
\put(578,496){\makebox(0,0)[r]{$\hat{\mu}_W$}}
\multiput(598,496)(20.756,0.000){5}{\usebox{\plotpoint}}
\put(698,496){\usebox{\plotpoint}}
\put(222,577){\usebox{\plotpoint}}
\multiput(222,577)(0.768,-20.741){2}{\usebox{\plotpoint}}
\put(225.58,546.24){\usebox{\plotpoint}}
\put(227.67,533.33){\usebox{\plotpoint}}
\put(231.15,549.31){\usebox{\plotpoint}}
\put(235.97,541.16){\usebox{\plotpoint}}
\put(243.34,543.72){\usebox{\plotpoint}}
\put(248.13,542.62){\usebox{\plotpoint}}
\put(257.25,535.01){\usebox{\plotpoint}}
\put(269.59,526.00){\usebox{\plotpoint}}
\put(284.51,524.00){\usebox{\plotpoint}}
\put(299.64,522.18){\usebox{\plotpoint}}
\put(315.66,528.34){\usebox{\plotpoint}}
\put(334.02,529.49){\usebox{\plotpoint}}
\put(352.12,529.00){\usebox{\plotpoint}}
\put(370.11,528.00){\usebox{\plotpoint}}
\put(376.55,511.80){\usebox{\plotpoint}}
\put(377.99,491.10){\usebox{\plotpoint}}
\put(383.36,473.00){\usebox{\plotpoint}}
\put(389.55,454.17){\usebox{\plotpoint}}
\put(391.94,433.56){\usebox{\plotpoint}}
\put(398.50,430.49){\usebox{\plotpoint}}
\put(405.81,431.58){\usebox{\plotpoint}}
\put(416.90,419.20){\usebox{\plotpoint}}
\put(424.57,410.00){\usebox{\plotpoint}}
\put(437.60,395.20){\usebox{\plotpoint}}
\put(448.42,382.33){\usebox{\plotpoint}}
\put(454.50,363.50){\usebox{\plotpoint}}
\put(456.52,343.03){\usebox{\plotpoint}}
\put(460.30,323.76){\usebox{\plotpoint}}
\put(464.47,303.70){\usebox{\plotpoint}}
\put(472.68,292.27){\usebox{\plotpoint}}
\put(478.76,275.46){\usebox{\plotpoint}}
\put(485.01,255.75){\usebox{\plotpoint}}
\put(486.28,235.03){\usebox{\plotpoint}}
\put(493.88,220.00){\usebox{\plotpoint}}
\put(507.31,220.07){\usebox{\plotpoint}}
\put(519.97,214.00){\usebox{\plotpoint}}
\put(533.96,212.98){\usebox{\plotpoint}}
\put(550.17,211.00){\usebox{\plotpoint}}
\put(567.28,210.86){\usebox{\plotpoint}}
\put(585.47,205.76){\usebox{\plotpoint}}
\put(604.98,206.00){\usebox{\plotpoint}}
\put(622.42,199.58){\usebox{\plotpoint}}
\put(641.45,196.55){\usebox{\plotpoint}}
\put(660.50,194.00){\usebox{\plotpoint}}
\put(679.54,195.00){\usebox{\plotpoint}}
\put(698.82,195.00){\usebox{\plotpoint}}
\put(718.10,195.00){\usebox{\plotpoint}}
\put(737,194){\usebox{\plotpoint}}
\sbox{\plotpoint}{\rule[-0.400pt]{0.800pt}{0.800pt}}%
\sbox{\plotpoint}{\rule[-0.200pt]{0.400pt}{0.400pt}}%
\put(578,455){\makebox(0,0)[r]{$W$}}
\sbox{\plotpoint}{\rule[-0.400pt]{0.800pt}{0.800pt}}%
\put(598.0,455.0){\rule[-0.400pt]{24.090pt}{0.800pt}}
\put(222,145){\usebox{\plotpoint}}
\put(222,143.84){\rule{0.241pt}{0.800pt}}
\multiput(222.00,143.34)(0.500,1.000){2}{\rule{0.120pt}{0.800pt}}
\put(223,145.34){\rule{0.482pt}{0.800pt}}
\multiput(223.00,144.34)(1.000,2.000){2}{\rule{0.241pt}{0.800pt}}
\put(223.84,148){\rule{0.800pt}{0.482pt}}
\multiput(223.34,148.00)(1.000,1.000){2}{\rule{0.800pt}{0.241pt}}
\put(224.84,150){\rule{0.800pt}{0.482pt}}
\multiput(224.34,150.00)(1.000,1.000){2}{\rule{0.800pt}{0.241pt}}
\put(227,150.84){\rule{0.482pt}{0.800pt}}
\multiput(227.00,150.34)(1.000,1.000){2}{\rule{0.241pt}{0.800pt}}
\put(227.84,153){\rule{0.800pt}{0.482pt}}
\multiput(227.34,153.00)(1.000,1.000){2}{\rule{0.800pt}{0.241pt}}
\put(228.84,155){\rule{0.800pt}{0.482pt}}
\multiput(228.34,155.00)(1.000,1.000){2}{\rule{0.800pt}{0.241pt}}
\put(231,156.34){\rule{0.482pt}{0.800pt}}
\multiput(231.00,155.34)(1.000,2.000){2}{\rule{0.241pt}{0.800pt}}
\put(233,157.84){\rule{0.241pt}{0.800pt}}
\multiput(233.00,157.34)(0.500,1.000){2}{\rule{0.120pt}{0.800pt}}
\put(232.84,160){\rule{0.800pt}{0.482pt}}
\multiput(232.34,160.00)(1.000,1.000){2}{\rule{0.800pt}{0.241pt}}
\put(233.84,162){\rule{0.800pt}{0.482pt}}
\multiput(233.34,162.00)(1.000,1.000){2}{\rule{0.800pt}{0.241pt}}
\put(236,163.34){\rule{0.482pt}{0.800pt}}
\multiput(236.00,162.34)(1.000,2.000){2}{\rule{0.241pt}{0.800pt}}
\put(238,164.84){\rule{0.241pt}{0.800pt}}
\multiput(238.00,164.34)(0.500,1.000){2}{\rule{0.120pt}{0.800pt}}
\put(237.84,167){\rule{0.800pt}{0.482pt}}
\multiput(237.34,167.00)(1.000,1.000){2}{\rule{0.800pt}{0.241pt}}
\put(240,168.34){\rule{0.482pt}{0.800pt}}
\multiput(240.00,167.34)(1.000,2.000){2}{\rule{0.241pt}{0.800pt}}
\put(240.84,171){\rule{0.800pt}{0.482pt}}
\multiput(240.34,171.00)(1.000,1.000){2}{\rule{0.800pt}{0.241pt}}
\put(243,171.84){\rule{0.241pt}{0.800pt}}
\multiput(243.00,171.34)(0.500,1.000){2}{\rule{0.120pt}{0.800pt}}
\put(242.84,174){\rule{0.800pt}{0.482pt}}
\multiput(242.34,174.00)(1.000,1.000){2}{\rule{0.800pt}{0.241pt}}
\put(245,175.34){\rule{0.482pt}{0.800pt}}
\multiput(245.00,174.34)(1.000,2.000){2}{\rule{0.241pt}{0.800pt}}
\put(247,176.84){\rule{0.241pt}{0.800pt}}
\multiput(247.00,176.34)(0.500,1.000){2}{\rule{0.120pt}{0.800pt}}
\put(246.84,179){\rule{0.800pt}{0.482pt}}
\multiput(246.34,179.00)(1.000,1.000){2}{\rule{0.800pt}{0.241pt}}
\put(249,180.34){\rule{0.482pt}{0.800pt}}
\multiput(249.00,179.34)(1.000,2.000){2}{\rule{0.241pt}{0.800pt}}
\put(249.84,183){\rule{0.800pt}{0.482pt}}
\multiput(249.34,183.00)(1.000,1.000){2}{\rule{0.800pt}{0.241pt}}
\put(252,183.84){\rule{0.241pt}{0.800pt}}
\multiput(252.00,183.34)(0.500,1.000){2}{\rule{0.120pt}{0.800pt}}
\put(251.84,186){\rule{0.800pt}{0.482pt}}
\multiput(251.34,186.00)(1.000,1.000){2}{\rule{0.800pt}{0.241pt}}
\put(254,187.34){\rule{0.482pt}{0.800pt}}
\multiput(254.00,186.34)(1.000,2.000){2}{\rule{0.241pt}{0.800pt}}
\put(254.84,190){\rule{0.800pt}{0.482pt}}
\multiput(254.34,190.00)(1.000,1.000){2}{\rule{0.800pt}{0.241pt}}
\put(257,190.84){\rule{0.241pt}{0.800pt}}
\multiput(257.00,190.34)(0.500,1.000){2}{\rule{0.120pt}{0.800pt}}
\put(258,192.34){\rule{0.482pt}{0.800pt}}
\multiput(258.00,191.34)(1.000,2.000){2}{\rule{0.241pt}{0.800pt}}
\put(258.84,195){\rule{0.800pt}{0.482pt}}
\multiput(258.34,195.00)(1.000,1.000){2}{\rule{0.800pt}{0.241pt}}
\put(259.84,197){\rule{0.800pt}{0.482pt}}
\multiput(259.34,197.00)(1.000,1.000){2}{\rule{0.800pt}{0.241pt}}
\put(262,197.84){\rule{0.482pt}{0.800pt}}
\multiput(262.00,197.34)(1.000,1.000){2}{\rule{0.241pt}{0.800pt}}
\put(262.84,200){\rule{0.800pt}{0.482pt}}
\multiput(262.34,200.00)(1.000,1.000){2}{\rule{0.800pt}{0.241pt}}
\put(263.84,202){\rule{0.800pt}{0.482pt}}
\multiput(263.34,202.00)(1.000,1.000){2}{\rule{0.800pt}{0.241pt}}
\put(266,202.84){\rule{0.241pt}{0.800pt}}
\multiput(266.00,202.34)(0.500,1.000){2}{\rule{0.120pt}{0.800pt}}
\put(267,204.34){\rule{0.482pt}{0.800pt}}
\multiput(267.00,203.34)(1.000,2.000){2}{\rule{0.241pt}{0.800pt}}
\put(267.84,207){\rule{0.800pt}{0.482pt}}
\multiput(267.34,207.00)(1.000,1.000){2}{\rule{0.800pt}{0.241pt}}
\put(268.84,209){\rule{0.800pt}{0.482pt}}
\multiput(268.34,209.00)(1.000,1.000){2}{\rule{0.800pt}{0.241pt}}
\put(271,209.84){\rule{0.482pt}{0.800pt}}
\multiput(271.00,209.34)(1.000,1.000){2}{\rule{0.241pt}{0.800pt}}
\put(271.84,212){\rule{0.800pt}{0.482pt}}
\multiput(271.34,212.00)(1.000,1.000){2}{\rule{0.800pt}{0.241pt}}
\put(272.84,214){\rule{0.800pt}{0.482pt}}
\multiput(272.34,214.00)(1.000,1.000){2}{\rule{0.800pt}{0.241pt}}
\put(273.84,216){\rule{0.800pt}{0.482pt}}
\multiput(273.34,216.00)(1.000,1.000){2}{\rule{0.800pt}{0.241pt}}
\put(276,216.84){\rule{0.482pt}{0.800pt}}
\multiput(276.00,216.34)(1.000,1.000){2}{\rule{0.241pt}{0.800pt}}
\put(276.84,219){\rule{0.800pt}{0.482pt}}
\multiput(276.34,219.00)(1.000,1.000){2}{\rule{0.800pt}{0.241pt}}
\put(277.84,221){\rule{0.800pt}{0.482pt}}
\multiput(277.34,221.00)(1.000,1.000){2}{\rule{0.800pt}{0.241pt}}
\put(280,222.34){\rule{0.482pt}{0.800pt}}
\multiput(280.00,221.34)(1.000,2.000){2}{\rule{0.241pt}{0.800pt}}
\put(282,223.84){\rule{0.241pt}{0.800pt}}
\multiput(282.00,223.34)(0.500,1.000){2}{\rule{0.120pt}{0.800pt}}
\put(281.84,226){\rule{0.800pt}{0.482pt}}
\multiput(281.34,226.00)(1.000,1.000){2}{\rule{0.800pt}{0.241pt}}
\put(284,227.34){\rule{0.482pt}{0.800pt}}
\multiput(284.00,226.34)(1.000,2.000){2}{\rule{0.241pt}{0.800pt}}
\put(284.84,230){\rule{0.800pt}{0.482pt}}
\multiput(284.34,230.00)(1.000,1.000){2}{\rule{0.800pt}{0.241pt}}
\put(287,230.84){\rule{0.241pt}{0.800pt}}
\multiput(287.00,230.34)(0.500,1.000){2}{\rule{0.120pt}{0.800pt}}
\put(286.84,233){\rule{0.800pt}{0.482pt}}
\multiput(286.34,233.00)(1.000,1.000){2}{\rule{0.800pt}{0.241pt}}
\put(289,234.34){\rule{0.482pt}{0.800pt}}
\multiput(289.00,233.34)(1.000,2.000){2}{\rule{0.241pt}{0.800pt}}
\put(291,235.84){\rule{0.241pt}{0.800pt}}
\multiput(291.00,235.34)(0.500,1.000){2}{\rule{0.120pt}{0.800pt}}
\put(290.84,238){\rule{0.800pt}{0.482pt}}
\multiput(290.34,238.00)(1.000,1.000){2}{\rule{0.800pt}{0.241pt}}
\put(293,239.34){\rule{0.482pt}{0.800pt}}
\multiput(293.00,238.34)(1.000,2.000){2}{\rule{0.241pt}{0.800pt}}
\put(293.84,242){\rule{0.800pt}{0.482pt}}
\multiput(293.34,242.00)(1.000,1.000){2}{\rule{0.800pt}{0.241pt}}
\put(296,242.84){\rule{0.241pt}{0.800pt}}
\multiput(296.00,242.34)(0.500,1.000){2}{\rule{0.120pt}{0.800pt}}
\put(295.84,245){\rule{0.800pt}{0.482pt}}
\multiput(295.34,245.00)(1.000,1.000){2}{\rule{0.800pt}{0.241pt}}
\put(298,246.34){\rule{0.482pt}{0.800pt}}
\multiput(298.00,245.34)(1.000,2.000){2}{\rule{0.241pt}{0.800pt}}
\put(298.84,249){\rule{0.800pt}{0.482pt}}
\multiput(298.34,249.00)(1.000,1.000){2}{\rule{0.800pt}{0.241pt}}
\put(301,249.84){\rule{0.241pt}{0.800pt}}
\multiput(301.00,249.34)(0.500,1.000){2}{\rule{0.120pt}{0.800pt}}
\put(302,251.34){\rule{0.482pt}{0.800pt}}
\multiput(302.00,250.34)(1.000,2.000){2}{\rule{0.241pt}{0.800pt}}
\put(302.84,254){\rule{0.800pt}{0.482pt}}
\multiput(302.34,254.00)(1.000,1.000){2}{\rule{0.800pt}{0.241pt}}
\put(303.84,256){\rule{0.800pt}{0.482pt}}
\multiput(303.34,256.00)(1.000,1.000){2}{\rule{0.800pt}{0.241pt}}
\put(306,256.84){\rule{0.241pt}{0.800pt}}
\multiput(306.00,256.34)(0.500,1.000){2}{\rule{0.120pt}{0.800pt}}
\put(307,258.34){\rule{0.482pt}{0.800pt}}
\multiput(307.00,257.34)(1.000,2.000){2}{\rule{0.241pt}{0.800pt}}
\put(307.84,261){\rule{0.800pt}{0.482pt}}
\multiput(307.34,261.00)(1.000,1.000){2}{\rule{0.800pt}{0.241pt}}
\put(308.84,263){\rule{0.800pt}{0.482pt}}
\multiput(308.34,263.00)(1.000,1.000){2}{\rule{0.800pt}{0.241pt}}
\put(311,263.84){\rule{0.482pt}{0.800pt}}
\multiput(311.00,263.34)(1.000,1.000){2}{\rule{0.241pt}{0.800pt}}
\put(311.84,266){\rule{0.800pt}{0.482pt}}
\multiput(311.34,266.00)(1.000,1.000){2}{\rule{0.800pt}{0.241pt}}
\put(312.84,268){\rule{0.800pt}{0.482pt}}
\multiput(312.34,268.00)(1.000,1.000){2}{\rule{0.800pt}{0.241pt}}
\put(315,268.84){\rule{0.482pt}{0.800pt}}
\multiput(315.00,268.34)(1.000,1.000){2}{\rule{0.241pt}{0.800pt}}
\put(315.84,271){\rule{0.800pt}{0.482pt}}
\multiput(315.34,271.00)(1.000,1.000){2}{\rule{0.800pt}{0.241pt}}
\put(316.84,273){\rule{0.800pt}{0.482pt}}
\multiput(316.34,273.00)(1.000,1.000){2}{\rule{0.800pt}{0.241pt}}
\put(317.84,275){\rule{0.800pt}{0.482pt}}
\multiput(317.34,275.00)(1.000,1.000){2}{\rule{0.800pt}{0.241pt}}
\put(320,275.84){\rule{0.482pt}{0.800pt}}
\multiput(320.00,275.34)(1.000,1.000){2}{\rule{0.241pt}{0.800pt}}
\put(320.84,278){\rule{0.800pt}{0.482pt}}
\multiput(320.34,278.00)(1.000,1.000){2}{\rule{0.800pt}{0.241pt}}
\put(321.84,280){\rule{0.800pt}{0.482pt}}
\multiput(321.34,280.00)(1.000,1.000){2}{\rule{0.800pt}{0.241pt}}
\put(324,281.34){\rule{0.482pt}{0.800pt}}
\multiput(324.00,280.34)(1.000,2.000){2}{\rule{0.241pt}{0.800pt}}
\put(326,282.84){\rule{0.241pt}{0.800pt}}
\multiput(326.00,282.34)(0.500,1.000){2}{\rule{0.120pt}{0.800pt}}
\put(325.84,285){\rule{0.800pt}{0.482pt}}
\multiput(325.34,285.00)(1.000,1.000){2}{\rule{0.800pt}{0.241pt}}
\put(326.84,287){\rule{0.800pt}{0.482pt}}
\multiput(326.34,287.00)(1.000,1.000){2}{\rule{0.800pt}{0.241pt}}
\put(329,288.34){\rule{0.482pt}{0.800pt}}
\multiput(329.00,287.34)(1.000,2.000){2}{\rule{0.241pt}{0.800pt}}
\put(331,289.84){\rule{0.241pt}{0.800pt}}
\multiput(331.00,289.34)(0.500,1.000){2}{\rule{0.120pt}{0.800pt}}
\put(330.84,292){\rule{0.800pt}{0.482pt}}
\multiput(330.34,292.00)(1.000,1.000){2}{\rule{0.800pt}{0.241pt}}
\put(333,293.34){\rule{0.482pt}{0.800pt}}
\multiput(333.00,292.34)(1.000,2.000){2}{\rule{0.241pt}{0.800pt}}
\put(333.84,296){\rule{0.800pt}{0.482pt}}
\multiput(333.34,296.00)(1.000,1.000){2}{\rule{0.800pt}{0.241pt}}
\put(336,296.84){\rule{0.241pt}{0.800pt}}
\multiput(336.00,296.34)(0.500,1.000){2}{\rule{0.120pt}{0.800pt}}
\put(337,298.34){\rule{0.482pt}{0.800pt}}
\multiput(337.00,297.34)(1.000,2.000){2}{\rule{0.241pt}{0.800pt}}
\put(337.84,301){\rule{0.800pt}{0.482pt}}
\multiput(337.34,301.00)(1.000,1.000){2}{\rule{0.800pt}{0.241pt}}
\put(340,301.84){\rule{0.241pt}{0.800pt}}
\multiput(340.00,301.34)(0.500,1.000){2}{\rule{0.120pt}{0.800pt}}
\put(339.84,304){\rule{0.800pt}{0.482pt}}
\multiput(339.34,304.00)(1.000,1.000){2}{\rule{0.800pt}{0.241pt}}
\put(342,305.34){\rule{0.482pt}{0.800pt}}
\multiput(342.00,304.34)(1.000,2.000){2}{\rule{0.241pt}{0.800pt}}
\put(342.84,308){\rule{0.800pt}{0.482pt}}
\multiput(342.34,308.00)(1.000,1.000){2}{\rule{0.800pt}{0.241pt}}
\put(345,308.84){\rule{0.241pt}{0.800pt}}
\multiput(345.00,308.34)(0.500,1.000){2}{\rule{0.120pt}{0.800pt}}
\put(346,310.34){\rule{0.482pt}{0.800pt}}
\multiput(346.00,309.34)(1.000,2.000){2}{\rule{0.241pt}{0.800pt}}
\put(346.84,313){\rule{0.800pt}{0.482pt}}
\multiput(346.34,313.00)(1.000,1.000){2}{\rule{0.800pt}{0.241pt}}
\put(347.84,315){\rule{0.800pt}{0.482pt}}
\multiput(347.34,315.00)(1.000,1.000){2}{\rule{0.800pt}{0.241pt}}
\put(350,315.84){\rule{0.241pt}{0.800pt}}
\multiput(350.00,315.34)(0.500,1.000){2}{\rule{0.120pt}{0.800pt}}
\put(351,317.34){\rule{0.482pt}{0.800pt}}
\multiput(351.00,316.34)(1.000,2.000){2}{\rule{0.241pt}{0.800pt}}
\put(351.84,320){\rule{0.800pt}{0.482pt}}
\multiput(351.34,320.00)(1.000,1.000){2}{\rule{0.800pt}{0.241pt}}
\put(352.84,322){\rule{0.800pt}{0.482pt}}
\multiput(352.34,322.00)(1.000,1.000){2}{\rule{0.800pt}{0.241pt}}
\put(355,322.84){\rule{0.482pt}{0.800pt}}
\multiput(355.00,322.34)(1.000,1.000){2}{\rule{0.241pt}{0.800pt}}
\put(355.84,325){\rule{0.800pt}{0.482pt}}
\multiput(355.34,325.00)(1.000,1.000){2}{\rule{0.800pt}{0.241pt}}
\put(356.84,327){\rule{0.800pt}{0.482pt}}
\multiput(356.34,327.00)(1.000,1.000){2}{\rule{0.800pt}{0.241pt}}
\put(359,327.84){\rule{0.241pt}{0.800pt}}
\multiput(359.00,327.34)(0.500,1.000){2}{\rule{0.120pt}{0.800pt}}
\put(360,329.34){\rule{0.482pt}{0.800pt}}
\multiput(360.00,328.34)(1.000,2.000){2}{\rule{0.241pt}{0.800pt}}
\put(360.84,332){\rule{0.800pt}{0.482pt}}
\multiput(360.34,332.00)(1.000,1.000){2}{\rule{0.800pt}{0.241pt}}
\put(361.84,334){\rule{0.800pt}{0.482pt}}
\multiput(361.34,334.00)(1.000,1.000){2}{\rule{0.800pt}{0.241pt}}
\put(364,334.84){\rule{0.482pt}{0.800pt}}
\multiput(364.00,334.34)(1.000,1.000){2}{\rule{0.241pt}{0.800pt}}
\put(364.84,337){\rule{0.800pt}{0.482pt}}
\multiput(364.34,337.00)(1.000,1.000){2}{\rule{0.800pt}{0.241pt}}
\put(365.84,339){\rule{0.800pt}{0.482pt}}
\multiput(365.34,339.00)(1.000,1.000){2}{\rule{0.800pt}{0.241pt}}
\put(368,340.34){\rule{0.482pt}{0.800pt}}
\multiput(368.00,339.34)(1.000,2.000){2}{\rule{0.241pt}{0.800pt}}
\put(370,341.84){\rule{0.241pt}{0.800pt}}
\multiput(370.00,341.34)(0.500,1.000){2}{\rule{0.120pt}{0.800pt}}
\put(369.84,344){\rule{0.800pt}{0.482pt}}
\multiput(369.34,344.00)(1.000,1.000){2}{\rule{0.800pt}{0.241pt}}
\put(370.84,346){\rule{0.800pt}{0.482pt}}
\multiput(370.34,346.00)(1.000,1.000){2}{\rule{0.800pt}{0.241pt}}
\put(373,347.34){\rule{0.482pt}{0.800pt}}
\multiput(373.00,346.34)(1.000,2.000){2}{\rule{0.241pt}{0.800pt}}
\put(375,348.84){\rule{0.241pt}{0.800pt}}
\multiput(375.00,348.34)(0.500,1.000){2}{\rule{0.120pt}{0.800pt}}
\put(374.84,191){\rule{0.800pt}{38.544pt}}
\multiput(374.34,271.00)(1.000,-80.000){2}{\rule{0.800pt}{19.272pt}}
\put(376.34,169){\rule{0.800pt}{5.300pt}}
\multiput(375.34,180.00)(2.000,-11.000){2}{\rule{0.800pt}{2.650pt}}
\put(377.84,169){\rule{0.800pt}{0.482pt}}
\multiput(377.34,169.00)(1.000,1.000){2}{\rule{0.800pt}{0.241pt}}
\put(378.84,171){\rule{0.800pt}{0.482pt}}
\multiput(378.34,171.00)(1.000,1.000){2}{\rule{0.800pt}{0.241pt}}
\put(379.84,173){\rule{0.800pt}{0.482pt}}
\multiput(379.34,173.00)(1.000,1.000){2}{\rule{0.800pt}{0.241pt}}
\put(382,173.84){\rule{0.482pt}{0.800pt}}
\multiput(382.00,173.34)(1.000,1.000){2}{\rule{0.241pt}{0.800pt}}
\put(382.84,176){\rule{0.800pt}{0.482pt}}
\multiput(382.34,176.00)(1.000,1.000){2}{\rule{0.800pt}{0.241pt}}
\put(383.84,178){\rule{0.800pt}{0.482pt}}
\multiput(383.34,178.00)(1.000,1.000){2}{\rule{0.800pt}{0.241pt}}
\put(386,179.34){\rule{0.482pt}{0.800pt}}
\multiput(386.00,178.34)(1.000,2.000){2}{\rule{0.241pt}{0.800pt}}
\put(388,180.84){\rule{0.241pt}{0.800pt}}
\multiput(388.00,180.34)(0.500,1.000){2}{\rule{0.120pt}{0.800pt}}
\put(387.84,183){\rule{0.800pt}{0.482pt}}
\multiput(387.34,183.00)(1.000,1.000){2}{\rule{0.800pt}{0.241pt}}
\put(389.34,181){\rule{0.800pt}{0.964pt}}
\multiput(388.34,183.00)(2.000,-2.000){2}{\rule{0.800pt}{0.482pt}}
\put(390.84,181){\rule{0.800pt}{0.482pt}}
\multiput(390.34,181.00)(1.000,1.000){2}{\rule{0.800pt}{0.241pt}}
\put(393,181.84){\rule{0.241pt}{0.800pt}}
\multiput(393.00,181.34)(0.500,1.000){2}{\rule{0.120pt}{0.800pt}}
\put(392.84,184){\rule{0.800pt}{0.482pt}}
\multiput(392.34,184.00)(1.000,1.000){2}{\rule{0.800pt}{0.241pt}}
\put(395,185.34){\rule{0.482pt}{0.800pt}}
\multiput(395.00,184.34)(1.000,2.000){2}{\rule{0.241pt}{0.800pt}}
\put(395.84,188){\rule{0.800pt}{0.482pt}}
\multiput(395.34,188.00)(1.000,1.000){2}{\rule{0.800pt}{0.241pt}}
\put(398,188.84){\rule{0.241pt}{0.800pt}}
\multiput(398.00,188.34)(0.500,1.000){2}{\rule{0.120pt}{0.800pt}}
\put(399,190.34){\rule{0.482pt}{0.800pt}}
\multiput(399.00,189.34)(1.000,2.000){2}{\rule{0.241pt}{0.800pt}}
\put(399.84,193){\rule{0.800pt}{0.482pt}}
\multiput(399.34,193.00)(1.000,1.000){2}{\rule{0.800pt}{0.241pt}}
\put(402,193.84){\rule{0.241pt}{0.800pt}}
\multiput(402.00,193.34)(0.500,1.000){2}{\rule{0.120pt}{0.800pt}}
\put(401.84,196){\rule{0.800pt}{0.482pt}}
\multiput(401.34,196.00)(1.000,1.000){2}{\rule{0.800pt}{0.241pt}}
\put(404,197.34){\rule{0.482pt}{0.800pt}}
\multiput(404.00,196.34)(1.000,2.000){2}{\rule{0.241pt}{0.800pt}}
\put(404.84,200){\rule{0.800pt}{0.482pt}}
\multiput(404.34,200.00)(1.000,1.000){2}{\rule{0.800pt}{0.241pt}}
\put(407,200.84){\rule{0.241pt}{0.800pt}}
\multiput(407.00,200.34)(0.500,1.000){2}{\rule{0.120pt}{0.800pt}}
\put(408,202.34){\rule{0.482pt}{0.800pt}}
\multiput(408.00,201.34)(1.000,2.000){2}{\rule{0.241pt}{0.800pt}}
\put(408.84,205){\rule{0.800pt}{0.482pt}}
\multiput(408.34,205.00)(1.000,1.000){2}{\rule{0.800pt}{0.241pt}}
\put(409.84,207){\rule{0.800pt}{0.482pt}}
\multiput(409.34,207.00)(1.000,1.000){2}{\rule{0.800pt}{0.241pt}}
\put(412,207.84){\rule{0.482pt}{0.800pt}}
\multiput(412.00,207.34)(1.000,1.000){2}{\rule{0.241pt}{0.800pt}}
\put(412.84,210){\rule{0.800pt}{0.482pt}}
\multiput(412.34,210.00)(1.000,1.000){2}{\rule{0.800pt}{0.241pt}}
\put(413.84,212){\rule{0.800pt}{0.482pt}}
\multiput(413.34,212.00)(1.000,1.000){2}{\rule{0.800pt}{0.241pt}}
\put(416,212.84){\rule{0.241pt}{0.800pt}}
\multiput(416.00,212.34)(0.500,1.000){2}{\rule{0.120pt}{0.800pt}}
\put(417,214.34){\rule{0.482pt}{0.800pt}}
\multiput(417.00,213.34)(1.000,2.000){2}{\rule{0.241pt}{0.800pt}}
\put(417.84,217){\rule{0.800pt}{0.482pt}}
\multiput(417.34,217.00)(1.000,1.000){2}{\rule{0.800pt}{0.241pt}}
\put(420,217.84){\rule{0.241pt}{0.800pt}}
\multiput(420.00,217.34)(0.500,1.000){2}{\rule{0.120pt}{0.800pt}}
\put(421,219.34){\rule{0.482pt}{0.800pt}}
\multiput(421.00,218.34)(1.000,2.000){2}{\rule{0.241pt}{0.800pt}}
\put(422.84,222){\rule{0.800pt}{0.482pt}}
\multiput(422.34,222.00)(1.000,1.000){2}{\rule{0.800pt}{0.241pt}}
\put(425,222.84){\rule{0.241pt}{0.800pt}}
\multiput(425.00,222.34)(0.500,1.000){2}{\rule{0.120pt}{0.800pt}}
\put(426,224.34){\rule{0.482pt}{0.800pt}}
\multiput(426.00,223.34)(1.000,2.000){2}{\rule{0.241pt}{0.800pt}}
\put(426.84,227){\rule{0.800pt}{0.482pt}}
\multiput(426.34,227.00)(1.000,1.000){2}{\rule{0.800pt}{0.241pt}}
\put(429,227.84){\rule{0.241pt}{0.800pt}}
\multiput(429.00,227.34)(0.500,1.000){2}{\rule{0.120pt}{0.800pt}}
\put(430,229.34){\rule{0.482pt}{0.800pt}}
\multiput(430.00,228.34)(1.000,2.000){2}{\rule{0.241pt}{0.800pt}}
\put(430.84,232){\rule{0.800pt}{0.482pt}}
\multiput(430.34,232.00)(1.000,1.000){2}{\rule{0.800pt}{0.241pt}}
\put(433,232.84){\rule{0.241pt}{0.800pt}}
\multiput(433.00,232.34)(0.500,1.000){2}{\rule{0.120pt}{0.800pt}}
\put(432.84,235){\rule{0.800pt}{0.482pt}}
\multiput(432.34,235.00)(1.000,1.000){2}{\rule{0.800pt}{0.241pt}}
\put(435,236.34){\rule{0.482pt}{0.800pt}}
\multiput(435.00,235.34)(1.000,2.000){2}{\rule{0.241pt}{0.800pt}}
\put(437,236.84){\rule{0.241pt}{0.800pt}}
\multiput(437.00,237.34)(0.500,-1.000){2}{\rule{0.120pt}{0.800pt}}
\put(438,236.84){\rule{0.241pt}{0.800pt}}
\multiput(438.00,236.34)(0.500,1.000){2}{\rule{0.120pt}{0.800pt}}
\put(439,238.34){\rule{0.482pt}{0.800pt}}
\multiput(439.00,237.34)(1.000,2.000){2}{\rule{0.241pt}{0.800pt}}
\put(441,239.84){\rule{0.241pt}{0.800pt}}
\multiput(441.00,239.34)(0.500,1.000){2}{\rule{0.120pt}{0.800pt}}
\put(440.84,242){\rule{0.800pt}{0.482pt}}
\multiput(440.34,242.00)(1.000,1.000){2}{\rule{0.800pt}{0.241pt}}
\put(443,243.34){\rule{0.482pt}{0.800pt}}
\multiput(443.00,242.34)(1.000,2.000){2}{\rule{0.241pt}{0.800pt}}
\put(443.84,246){\rule{0.800pt}{0.482pt}}
\multiput(443.34,246.00)(1.000,1.000){2}{\rule{0.800pt}{0.241pt}}
\put(446,246.84){\rule{0.241pt}{0.800pt}}
\multiput(446.00,246.34)(0.500,1.000){2}{\rule{0.120pt}{0.800pt}}
\put(445.84,249){\rule{0.800pt}{0.482pt}}
\multiput(445.34,249.00)(1.000,1.000){2}{\rule{0.800pt}{0.241pt}}
\put(448,248.84){\rule{0.482pt}{0.800pt}}
\multiput(448.00,249.34)(1.000,-1.000){2}{\rule{0.241pt}{0.800pt}}
\put(448.84,234){\rule{0.800pt}{3.854pt}}
\multiput(448.34,242.00)(1.000,-8.000){2}{\rule{0.800pt}{1.927pt}}
\put(451,232.84){\rule{0.241pt}{0.800pt}}
\multiput(451.00,232.34)(0.500,1.000){2}{\rule{0.120pt}{0.800pt}}
\put(452,234.34){\rule{0.482pt}{0.800pt}}
\multiput(452.00,233.34)(1.000,2.000){2}{\rule{0.241pt}{0.800pt}}
\put(452.84,237){\rule{0.800pt}{0.482pt}}
\multiput(452.34,237.00)(1.000,1.000){2}{\rule{0.800pt}{0.241pt}}
\put(453.84,226){\rule{0.800pt}{3.132pt}}
\multiput(453.34,232.50)(1.000,-6.500){2}{\rule{0.800pt}{1.566pt}}
\put(454.84,214){\rule{0.800pt}{2.891pt}}
\multiput(454.34,220.00)(1.000,-6.000){2}{\rule{0.800pt}{1.445pt}}
\put(456.34,210){\rule{0.800pt}{0.964pt}}
\multiput(455.34,212.00)(2.000,-2.000){2}{\rule{0.800pt}{0.482pt}}
\put(457.84,210){\rule{0.800pt}{0.482pt}}
\multiput(457.34,210.00)(1.000,1.000){2}{\rule{0.800pt}{0.241pt}}
\put(458.84,200){\rule{0.800pt}{2.891pt}}
\multiput(458.34,206.00)(1.000,-6.000){2}{\rule{0.800pt}{1.445pt}}
\put(461,198.84){\rule{0.482pt}{0.800pt}}
\multiput(461.00,198.34)(1.000,1.000){2}{\rule{0.241pt}{0.800pt}}
\put(463,199.84){\rule{0.241pt}{0.800pt}}
\multiput(463.00,199.34)(0.500,1.000){2}{\rule{0.120pt}{0.800pt}}
\put(462.84,200){\rule{0.800pt}{0.482pt}}
\multiput(462.34,201.00)(1.000,-1.000){2}{\rule{0.800pt}{0.241pt}}
\put(465,199.34){\rule{0.482pt}{0.800pt}}
\multiput(465.00,198.34)(1.000,2.000){2}{\rule{0.241pt}{0.800pt}}
\put(465.84,202){\rule{0.800pt}{0.482pt}}
\multiput(465.34,202.00)(1.000,1.000){2}{\rule{0.800pt}{0.241pt}}
\put(468,202.84){\rule{0.241pt}{0.800pt}}
\multiput(468.00,202.34)(0.500,1.000){2}{\rule{0.120pt}{0.800pt}}
\put(467.84,203){\rule{0.800pt}{0.482pt}}
\multiput(467.34,204.00)(1.000,-1.000){2}{\rule{0.800pt}{0.241pt}}
\put(470,202.34){\rule{0.482pt}{0.800pt}}
\multiput(470.00,201.34)(1.000,2.000){2}{\rule{0.241pt}{0.800pt}}
\put(470.84,205){\rule{0.800pt}{0.482pt}}
\multiput(470.34,205.00)(1.000,1.000){2}{\rule{0.800pt}{0.241pt}}
\put(473,205.84){\rule{0.241pt}{0.800pt}}
\multiput(473.00,205.34)(0.500,1.000){2}{\rule{0.120pt}{0.800pt}}
\put(474,207.34){\rule{0.482pt}{0.800pt}}
\multiput(474.00,206.34)(1.000,2.000){2}{\rule{0.241pt}{0.800pt}}
\put(474.84,210){\rule{0.800pt}{0.482pt}}
\multiput(474.34,210.00)(1.000,1.000){2}{\rule{0.800pt}{0.241pt}}
\put(477,209.84){\rule{0.241pt}{0.800pt}}
\multiput(477.00,210.34)(0.500,-1.000){2}{\rule{0.120pt}{0.800pt}}
\put(478,208.84){\rule{0.241pt}{0.800pt}}
\multiput(478.00,209.34)(0.500,-1.000){2}{\rule{0.120pt}{0.800pt}}
\put(479,207.34){\rule{0.482pt}{0.800pt}}
\multiput(479.00,208.34)(1.000,-2.000){2}{\rule{0.241pt}{0.800pt}}
\put(423.0,222.0){\usebox{\plotpoint}}
\put(480.84,208){\rule{0.800pt}{0.482pt}}
\multiput(480.34,208.00)(1.000,1.000){2}{\rule{0.800pt}{0.241pt}}
\put(481.0,208.0){\usebox{\plotpoint}}
\put(483.84,200){\rule{0.800pt}{2.409pt}}
\multiput(483.34,205.00)(1.000,-5.000){2}{\rule{0.800pt}{1.204pt}}
\put(484.84,198){\rule{0.800pt}{0.482pt}}
\multiput(484.34,199.00)(1.000,-1.000){2}{\rule{0.800pt}{0.241pt}}
\put(485.84,198){\rule{0.800pt}{0.482pt}}
\multiput(485.34,198.00)(1.000,1.000){2}{\rule{0.800pt}{0.241pt}}
\put(488,199.34){\rule{0.482pt}{0.800pt}}
\multiput(488.00,198.34)(1.000,2.000){2}{\rule{0.241pt}{0.800pt}}
\put(490,200.84){\rule{0.241pt}{0.800pt}}
\multiput(490.00,200.34)(0.500,1.000){2}{\rule{0.120pt}{0.800pt}}
\put(489.84,203){\rule{0.800pt}{0.482pt}}
\multiput(489.34,203.00)(1.000,1.000){2}{\rule{0.800pt}{0.241pt}}
\put(492,203.84){\rule{0.482pt}{0.800pt}}
\multiput(492.00,203.34)(1.000,1.000){2}{\rule{0.241pt}{0.800pt}}
\put(492.84,206){\rule{0.800pt}{0.482pt}}
\multiput(492.34,206.00)(1.000,1.000){2}{\rule{0.800pt}{0.241pt}}
\put(493.84,208){\rule{0.800pt}{0.482pt}}
\multiput(493.34,208.00)(1.000,1.000){2}{\rule{0.800pt}{0.241pt}}
\put(496,209.34){\rule{0.482pt}{0.800pt}}
\multiput(496.00,208.34)(1.000,2.000){2}{\rule{0.241pt}{0.800pt}}
\put(498,210.84){\rule{0.241pt}{0.800pt}}
\multiput(498.00,210.34)(0.500,1.000){2}{\rule{0.120pt}{0.800pt}}
\put(497.84,213){\rule{0.800pt}{0.482pt}}
\multiput(497.34,213.00)(1.000,1.000){2}{\rule{0.800pt}{0.241pt}}
\put(498.84,215){\rule{0.800pt}{0.482pt}}
\multiput(498.34,215.00)(1.000,1.000){2}{\rule{0.800pt}{0.241pt}}
\put(501,216.34){\rule{0.482pt}{0.800pt}}
\multiput(501.00,215.34)(1.000,2.000){2}{\rule{0.241pt}{0.800pt}}
\put(503,217.84){\rule{0.241pt}{0.800pt}}
\multiput(503.00,217.34)(0.500,1.000){2}{\rule{0.120pt}{0.800pt}}
\put(502.84,220){\rule{0.800pt}{0.482pt}}
\multiput(502.34,220.00)(1.000,1.000){2}{\rule{0.800pt}{0.241pt}}
\put(505,221.34){\rule{0.482pt}{0.800pt}}
\multiput(505.00,220.34)(1.000,2.000){2}{\rule{0.241pt}{0.800pt}}
\put(507,222.84){\rule{0.241pt}{0.800pt}}
\multiput(507.00,222.34)(0.500,1.000){2}{\rule{0.120pt}{0.800pt}}
\put(506.84,225){\rule{0.800pt}{0.482pt}}
\multiput(506.34,225.00)(1.000,1.000){2}{\rule{0.800pt}{0.241pt}}
\put(507.84,227){\rule{0.800pt}{0.482pt}}
\multiput(507.34,227.00)(1.000,1.000){2}{\rule{0.800pt}{0.241pt}}
\put(510,228.34){\rule{0.482pt}{0.800pt}}
\multiput(510.00,227.34)(1.000,2.000){2}{\rule{0.241pt}{0.800pt}}
\put(512,229.84){\rule{0.241pt}{0.800pt}}
\multiput(512.00,229.34)(0.500,1.000){2}{\rule{0.120pt}{0.800pt}}
\put(511.84,232){\rule{0.800pt}{0.482pt}}
\multiput(511.34,232.00)(1.000,1.000){2}{\rule{0.800pt}{0.241pt}}
\put(514,233.34){\rule{0.482pt}{0.800pt}}
\multiput(514.00,232.34)(1.000,2.000){2}{\rule{0.241pt}{0.800pt}}
\put(514.84,236){\rule{0.800pt}{0.482pt}}
\multiput(514.34,236.00)(1.000,1.000){2}{\rule{0.800pt}{0.241pt}}
\put(517,236.84){\rule{0.241pt}{0.800pt}}
\multiput(517.00,236.34)(0.500,1.000){2}{\rule{0.120pt}{0.800pt}}
\put(518,238.34){\rule{0.482pt}{0.800pt}}
\multiput(518.00,237.34)(1.000,2.000){2}{\rule{0.241pt}{0.800pt}}
\put(518.84,241){\rule{0.800pt}{0.482pt}}
\multiput(518.34,241.00)(1.000,1.000){2}{\rule{0.800pt}{0.241pt}}
\put(519.84,243){\rule{0.800pt}{0.482pt}}
\multiput(519.34,243.00)(1.000,1.000){2}{\rule{0.800pt}{0.241pt}}
\put(522,243.84){\rule{0.241pt}{0.800pt}}
\multiput(522.00,243.34)(0.500,1.000){2}{\rule{0.120pt}{0.800pt}}
\put(523,245.34){\rule{0.482pt}{0.800pt}}
\multiput(523.00,244.34)(1.000,2.000){2}{\rule{0.241pt}{0.800pt}}
\put(523.84,248){\rule{0.800pt}{0.482pt}}
\multiput(523.34,248.00)(1.000,1.000){2}{\rule{0.800pt}{0.241pt}}
\put(524.84,250){\rule{0.800pt}{0.482pt}}
\multiput(524.34,250.00)(1.000,1.000){2}{\rule{0.800pt}{0.241pt}}
\put(527,250.84){\rule{0.482pt}{0.800pt}}
\multiput(527.00,250.34)(1.000,1.000){2}{\rule{0.241pt}{0.800pt}}
\put(527.84,253){\rule{0.800pt}{0.482pt}}
\multiput(527.34,253.00)(1.000,1.000){2}{\rule{0.800pt}{0.241pt}}
\put(528.84,255){\rule{0.800pt}{0.482pt}}
\multiput(528.34,255.00)(1.000,1.000){2}{\rule{0.800pt}{0.241pt}}
\put(531,255.84){\rule{0.241pt}{0.800pt}}
\multiput(531.00,255.34)(0.500,1.000){2}{\rule{0.120pt}{0.800pt}}
\put(532,257.34){\rule{0.482pt}{0.800pt}}
\multiput(532.00,256.34)(1.000,2.000){2}{\rule{0.241pt}{0.800pt}}
\put(532.84,260){\rule{0.800pt}{0.482pt}}
\multiput(532.34,260.00)(1.000,1.000){2}{\rule{0.800pt}{0.241pt}}
\put(533.84,262){\rule{0.800pt}{0.482pt}}
\multiput(533.34,262.00)(1.000,1.000){2}{\rule{0.800pt}{0.241pt}}
\put(536,262.84){\rule{0.482pt}{0.800pt}}
\multiput(536.00,262.34)(1.000,1.000){2}{\rule{0.241pt}{0.800pt}}
\put(536.84,265){\rule{0.800pt}{0.482pt}}
\multiput(536.34,265.00)(1.000,1.000){2}{\rule{0.800pt}{0.241pt}}
\put(537.84,267){\rule{0.800pt}{0.482pt}}
\multiput(537.34,267.00)(1.000,1.000){2}{\rule{0.800pt}{0.241pt}}
\put(538.84,269){\rule{0.800pt}{0.482pt}}
\multiput(538.34,269.00)(1.000,1.000){2}{\rule{0.800pt}{0.241pt}}
\put(541,269.84){\rule{0.482pt}{0.800pt}}
\multiput(541.00,269.34)(1.000,1.000){2}{\rule{0.241pt}{0.800pt}}
\put(541.84,272){\rule{0.800pt}{0.482pt}}
\multiput(541.34,272.00)(1.000,1.000){2}{\rule{0.800pt}{0.241pt}}
\put(542.84,274){\rule{0.800pt}{0.482pt}}
\multiput(542.34,274.00)(1.000,1.000){2}{\rule{0.800pt}{0.241pt}}
\put(545,275.34){\rule{0.482pt}{0.800pt}}
\multiput(545.00,274.34)(1.000,2.000){2}{\rule{0.241pt}{0.800pt}}
\put(547,276.84){\rule{0.241pt}{0.800pt}}
\multiput(547.00,276.34)(0.500,1.000){2}{\rule{0.120pt}{0.800pt}}
\put(546.84,279){\rule{0.800pt}{0.482pt}}
\multiput(546.34,279.00)(1.000,1.000){2}{\rule{0.800pt}{0.241pt}}
\put(549,280.34){\rule{0.482pt}{0.800pt}}
\multiput(549.00,279.34)(1.000,2.000){2}{\rule{0.241pt}{0.800pt}}
\put(551,281.84){\rule{0.241pt}{0.800pt}}
\multiput(551.00,281.34)(0.500,1.000){2}{\rule{0.120pt}{0.800pt}}
\put(550.84,284){\rule{0.800pt}{0.482pt}}
\multiput(550.34,284.00)(1.000,1.000){2}{\rule{0.800pt}{0.241pt}}
\put(551.84,286){\rule{0.800pt}{0.482pt}}
\multiput(551.34,286.00)(1.000,1.000){2}{\rule{0.800pt}{0.241pt}}
\put(554,287.34){\rule{0.482pt}{0.800pt}}
\multiput(554.00,286.34)(1.000,2.000){2}{\rule{0.241pt}{0.800pt}}
\put(556,288.84){\rule{0.241pt}{0.800pt}}
\multiput(556.00,288.34)(0.500,1.000){2}{\rule{0.120pt}{0.800pt}}
\put(555.84,291){\rule{0.800pt}{0.482pt}}
\multiput(555.34,291.00)(1.000,1.000){2}{\rule{0.800pt}{0.241pt}}
\put(558,292.34){\rule{0.482pt}{0.800pt}}
\multiput(558.00,291.34)(1.000,2.000){2}{\rule{0.241pt}{0.800pt}}
\put(558.84,295){\rule{0.800pt}{0.482pt}}
\multiput(558.34,295.00)(1.000,1.000){2}{\rule{0.800pt}{0.241pt}}
\put(561,295.84){\rule{0.241pt}{0.800pt}}
\multiput(561.00,295.34)(0.500,1.000){2}{\rule{0.120pt}{0.800pt}}
\put(560.84,298){\rule{0.800pt}{0.482pt}}
\multiput(560.34,298.00)(1.000,1.000){2}{\rule{0.800pt}{0.241pt}}
\put(563,299.34){\rule{0.482pt}{0.800pt}}
\multiput(563.00,298.34)(1.000,2.000){2}{\rule{0.241pt}{0.800pt}}
\put(563.84,302){\rule{0.800pt}{0.482pt}}
\multiput(563.34,302.00)(1.000,1.000){2}{\rule{0.800pt}{0.241pt}}
\put(566,302.84){\rule{0.241pt}{0.800pt}}
\multiput(566.00,302.34)(0.500,1.000){2}{\rule{0.120pt}{0.800pt}}
\put(567,304.34){\rule{0.482pt}{0.800pt}}
\multiput(567.00,303.34)(1.000,2.000){2}{\rule{0.241pt}{0.800pt}}
\put(567.84,307){\rule{0.800pt}{0.482pt}}
\multiput(567.34,307.00)(1.000,1.000){2}{\rule{0.800pt}{0.241pt}}
\put(568.84,309){\rule{0.800pt}{0.482pt}}
\multiput(568.34,309.00)(1.000,1.000){2}{\rule{0.800pt}{0.241pt}}
\put(571,309.84){\rule{0.482pt}{0.800pt}}
\multiput(571.00,309.34)(1.000,1.000){2}{\rule{0.241pt}{0.800pt}}
\put(571.84,312){\rule{0.800pt}{0.482pt}}
\multiput(571.34,312.00)(1.000,1.000){2}{\rule{0.800pt}{0.241pt}}
\put(572.84,314){\rule{0.800pt}{0.482pt}}
\multiput(572.34,314.00)(1.000,1.000){2}{\rule{0.800pt}{0.241pt}}
\put(575,314.84){\rule{0.241pt}{0.800pt}}
\multiput(575.00,314.34)(0.500,1.000){2}{\rule{0.120pt}{0.800pt}}
\put(576,316.34){\rule{0.482pt}{0.800pt}}
\multiput(576.00,315.34)(1.000,2.000){2}{\rule{0.241pt}{0.800pt}}
\put(576.84,319){\rule{0.800pt}{0.482pt}}
\multiput(576.34,319.00)(1.000,1.000){2}{\rule{0.800pt}{0.241pt}}
\put(577.84,321){\rule{0.800pt}{0.482pt}}
\multiput(577.34,321.00)(1.000,1.000){2}{\rule{0.800pt}{0.241pt}}
\put(580,321.84){\rule{0.482pt}{0.800pt}}
\multiput(580.00,321.34)(1.000,1.000){2}{\rule{0.241pt}{0.800pt}}
\put(580.84,324){\rule{0.800pt}{0.482pt}}
\multiput(580.34,324.00)(1.000,1.000){2}{\rule{0.800pt}{0.241pt}}
\put(581.84,326){\rule{0.800pt}{0.482pt}}
\multiput(581.34,326.00)(1.000,1.000){2}{\rule{0.800pt}{0.241pt}}
\put(582.84,328){\rule{0.800pt}{0.482pt}}
\multiput(582.34,328.00)(1.000,1.000){2}{\rule{0.800pt}{0.241pt}}
\put(585,328.84){\rule{0.482pt}{0.800pt}}
\multiput(585.00,328.34)(1.000,1.000){2}{\rule{0.241pt}{0.800pt}}
\put(585.84,331){\rule{0.800pt}{0.482pt}}
\multiput(585.34,331.00)(1.000,1.000){2}{\rule{0.800pt}{0.241pt}}
\put(586.84,333){\rule{0.800pt}{0.482pt}}
\multiput(586.34,333.00)(1.000,1.000){2}{\rule{0.800pt}{0.241pt}}
\put(589,334.34){\rule{0.482pt}{0.800pt}}
\multiput(589.00,333.34)(1.000,2.000){2}{\rule{0.241pt}{0.800pt}}
\put(591,335.84){\rule{0.241pt}{0.800pt}}
\multiput(591.00,335.34)(0.500,1.000){2}{\rule{0.120pt}{0.800pt}}
\put(590.84,338){\rule{0.800pt}{0.482pt}}
\multiput(590.34,338.00)(1.000,1.000){2}{\rule{0.800pt}{0.241pt}}
\put(591.84,340){\rule{0.800pt}{0.482pt}}
\multiput(591.34,340.00)(1.000,1.000){2}{\rule{0.800pt}{0.241pt}}
\put(594,341.34){\rule{0.482pt}{0.800pt}}
\multiput(594.00,340.34)(1.000,2.000){2}{\rule{0.241pt}{0.800pt}}
\put(596,342.84){\rule{0.241pt}{0.800pt}}
\multiput(596.00,342.34)(0.500,1.000){2}{\rule{0.120pt}{0.800pt}}
\put(595.84,345){\rule{0.800pt}{0.482pt}}
\multiput(595.34,345.00)(1.000,1.000){2}{\rule{0.800pt}{0.241pt}}
\put(598,346.34){\rule{0.482pt}{0.800pt}}
\multiput(598.00,345.34)(1.000,2.000){2}{\rule{0.241pt}{0.800pt}}
\put(600,347.84){\rule{0.241pt}{0.800pt}}
\multiput(600.00,347.34)(0.500,1.000){2}{\rule{0.120pt}{0.800pt}}
\put(599.84,350){\rule{0.800pt}{0.482pt}}
\multiput(599.34,350.00)(1.000,1.000){2}{\rule{0.800pt}{0.241pt}}
\put(602,351.34){\rule{0.482pt}{0.800pt}}
\multiput(602.00,350.34)(1.000,2.000){2}{\rule{0.241pt}{0.800pt}}
\put(602.84,354){\rule{0.800pt}{0.482pt}}
\multiput(602.34,354.00)(1.000,1.000){2}{\rule{0.800pt}{0.241pt}}
\put(605,354.84){\rule{0.241pt}{0.800pt}}
\multiput(605.00,354.34)(0.500,1.000){2}{\rule{0.120pt}{0.800pt}}
\put(604.84,357){\rule{0.800pt}{0.482pt}}
\multiput(604.34,357.00)(1.000,1.000){2}{\rule{0.800pt}{0.241pt}}
\put(607,358.34){\rule{0.482pt}{0.800pt}}
\multiput(607.00,357.34)(1.000,2.000){2}{\rule{0.241pt}{0.800pt}}
\put(607.84,361){\rule{0.800pt}{0.482pt}}
\multiput(607.34,361.00)(1.000,1.000){2}{\rule{0.800pt}{0.241pt}}
\put(610,361.84){\rule{0.241pt}{0.800pt}}
\multiput(610.00,361.34)(0.500,1.000){2}{\rule{0.120pt}{0.800pt}}
\put(611,363.34){\rule{0.482pt}{0.800pt}}
\multiput(611.00,362.34)(1.000,2.000){2}{\rule{0.241pt}{0.800pt}}
\put(611.84,366){\rule{0.800pt}{0.482pt}}
\multiput(611.34,366.00)(1.000,1.000){2}{\rule{0.800pt}{0.241pt}}
\put(612.84,368){\rule{0.800pt}{0.482pt}}
\multiput(612.34,368.00)(1.000,1.000){2}{\rule{0.800pt}{0.241pt}}
\put(615,368.84){\rule{0.241pt}{0.800pt}}
\multiput(615.00,368.34)(0.500,1.000){2}{\rule{0.120pt}{0.800pt}}
\put(616,370.34){\rule{0.482pt}{0.800pt}}
\multiput(616.00,369.34)(1.000,2.000){2}{\rule{0.241pt}{0.800pt}}
\put(616.84,373){\rule{0.800pt}{0.482pt}}
\multiput(616.34,373.00)(1.000,1.000){2}{\rule{0.800pt}{0.241pt}}
\put(619,373.84){\rule{0.241pt}{0.800pt}}
\multiput(619.00,373.34)(0.500,1.000){2}{\rule{0.120pt}{0.800pt}}
\put(620,375.34){\rule{0.482pt}{0.800pt}}
\multiput(620.00,374.34)(1.000,2.000){2}{\rule{0.241pt}{0.800pt}}
\put(620.84,378){\rule{0.800pt}{0.482pt}}
\multiput(620.34,378.00)(1.000,1.000){2}{\rule{0.800pt}{0.241pt}}
\put(623,378.84){\rule{0.241pt}{0.800pt}}
\multiput(623.00,378.34)(0.500,1.000){2}{\rule{0.120pt}{0.800pt}}
\put(624,380.34){\rule{0.482pt}{0.800pt}}
\multiput(624.00,379.34)(1.000,2.000){2}{\rule{0.241pt}{0.800pt}}
\put(624.84,383){\rule{0.800pt}{0.482pt}}
\multiput(624.34,383.00)(1.000,1.000){2}{\rule{0.800pt}{0.241pt}}
\put(625.84,385){\rule{0.800pt}{0.482pt}}
\multiput(625.34,385.00)(1.000,1.000){2}{\rule{0.800pt}{0.241pt}}
\put(628,385.84){\rule{0.241pt}{0.800pt}}
\multiput(628.00,385.34)(0.500,1.000){2}{\rule{0.120pt}{0.800pt}}
\put(629,387.34){\rule{0.482pt}{0.800pt}}
\multiput(629.00,386.34)(1.000,2.000){2}{\rule{0.241pt}{0.800pt}}
\put(629.84,390){\rule{0.800pt}{0.482pt}}
\multiput(629.34,390.00)(1.000,1.000){2}{\rule{0.800pt}{0.241pt}}
\put(632,390.84){\rule{0.241pt}{0.800pt}}
\multiput(632.00,390.34)(0.500,1.000){2}{\rule{0.120pt}{0.800pt}}
\put(633,392.34){\rule{0.482pt}{0.800pt}}
\multiput(633.00,391.34)(1.000,2.000){2}{\rule{0.241pt}{0.800pt}}
\put(633.84,395){\rule{0.800pt}{0.482pt}}
\multiput(633.34,395.00)(1.000,1.000){2}{\rule{0.800pt}{0.241pt}}
\put(636,395.84){\rule{0.241pt}{0.800pt}}
\multiput(636.00,395.34)(0.500,1.000){2}{\rule{0.120pt}{0.800pt}}
\put(635.84,398){\rule{0.800pt}{0.482pt}}
\multiput(635.34,398.00)(1.000,1.000){2}{\rule{0.800pt}{0.241pt}}
\put(638,399.34){\rule{0.482pt}{0.800pt}}
\multiput(638.00,398.34)(1.000,2.000){2}{\rule{0.241pt}{0.800pt}}
\put(638.84,402){\rule{0.800pt}{0.482pt}}
\multiput(638.34,402.00)(1.000,1.000){2}{\rule{0.800pt}{0.241pt}}
\put(641,402.84){\rule{0.241pt}{0.800pt}}
\multiput(641.00,402.34)(0.500,1.000){2}{\rule{0.120pt}{0.800pt}}
\put(642,404.34){\rule{0.482pt}{0.800pt}}
\multiput(642.00,403.34)(1.000,2.000){2}{\rule{0.241pt}{0.800pt}}
\put(642.84,407){\rule{0.800pt}{0.482pt}}
\multiput(642.34,407.00)(1.000,1.000){2}{\rule{0.800pt}{0.241pt}}
\put(645,407.84){\rule{0.241pt}{0.800pt}}
\multiput(645.00,407.34)(0.500,1.000){2}{\rule{0.120pt}{0.800pt}}
\put(646,409.34){\rule{0.482pt}{0.800pt}}
\multiput(646.00,408.34)(1.000,2.000){2}{\rule{0.241pt}{0.800pt}}
\put(646.84,412){\rule{0.800pt}{0.482pt}}
\multiput(646.34,412.00)(1.000,1.000){2}{\rule{0.800pt}{0.241pt}}
\put(647.84,414){\rule{0.800pt}{0.482pt}}
\multiput(647.34,414.00)(1.000,1.000){2}{\rule{0.800pt}{0.241pt}}
\put(650,414.84){\rule{0.241pt}{0.800pt}}
\multiput(650.00,414.34)(0.500,1.000){2}{\rule{0.120pt}{0.800pt}}
\put(651,416.34){\rule{0.482pt}{0.800pt}}
\multiput(651.00,415.34)(1.000,2.000){2}{\rule{0.241pt}{0.800pt}}
\put(651.84,419){\rule{0.800pt}{0.482pt}}
\multiput(651.34,419.00)(1.000,1.000){2}{\rule{0.800pt}{0.241pt}}
\put(654,419.84){\rule{0.241pt}{0.800pt}}
\multiput(654.00,419.34)(0.500,1.000){2}{\rule{0.120pt}{0.800pt}}
\put(655,421.34){\rule{0.482pt}{0.800pt}}
\multiput(655.00,420.34)(1.000,2.000){2}{\rule{0.241pt}{0.800pt}}
\put(655.84,421){\rule{0.800pt}{0.723pt}}
\multiput(655.34,422.50)(1.000,-1.500){2}{\rule{0.800pt}{0.361pt}}
\put(656.84,421){\rule{0.800pt}{0.482pt}}
\multiput(656.34,421.00)(1.000,1.000){2}{\rule{0.800pt}{0.241pt}}
\put(657.84,423){\rule{0.800pt}{0.482pt}}
\multiput(657.34,423.00)(1.000,1.000){2}{\rule{0.800pt}{0.241pt}}
\put(660,423.84){\rule{0.482pt}{0.800pt}}
\multiput(660.00,423.34)(1.000,1.000){2}{\rule{0.241pt}{0.800pt}}
\put(660.84,426){\rule{0.800pt}{0.482pt}}
\multiput(660.34,426.00)(1.000,1.000){2}{\rule{0.800pt}{0.241pt}}
\put(661.84,428){\rule{0.800pt}{0.482pt}}
\multiput(661.34,428.00)(1.000,1.000){2}{\rule{0.800pt}{0.241pt}}
\put(664,429.34){\rule{0.482pt}{0.800pt}}
\multiput(664.00,428.34)(1.000,2.000){2}{\rule{0.241pt}{0.800pt}}
\put(666,430.84){\rule{0.241pt}{0.800pt}}
\multiput(666.00,430.34)(0.500,1.000){2}{\rule{0.120pt}{0.800pt}}
\put(665.84,433){\rule{0.800pt}{0.482pt}}
\multiput(665.34,433.00)(1.000,1.000){2}{\rule{0.800pt}{0.241pt}}
\put(666.84,435){\rule{0.800pt}{0.482pt}}
\multiput(666.34,435.00)(1.000,1.000){2}{\rule{0.800pt}{0.241pt}}
\put(669,436.34){\rule{0.482pt}{0.800pt}}
\multiput(669.00,435.34)(1.000,2.000){2}{\rule{0.241pt}{0.800pt}}
\put(671,437.84){\rule{0.241pt}{0.800pt}}
\multiput(671.00,437.34)(0.500,1.000){2}{\rule{0.120pt}{0.800pt}}
\put(670.84,440){\rule{0.800pt}{0.482pt}}
\multiput(670.34,440.00)(1.000,1.000){2}{\rule{0.800pt}{0.241pt}}
\put(673,441.34){\rule{0.482pt}{0.800pt}}
\multiput(673.00,440.34)(1.000,2.000){2}{\rule{0.241pt}{0.800pt}}
\put(673.84,444){\rule{0.800pt}{0.482pt}}
\multiput(673.34,444.00)(1.000,1.000){2}{\rule{0.800pt}{0.241pt}}
\put(676,444.84){\rule{0.241pt}{0.800pt}}
\multiput(676.00,444.34)(0.500,1.000){2}{\rule{0.120pt}{0.800pt}}
\put(677,446.34){\rule{0.482pt}{0.800pt}}
\multiput(677.00,445.34)(1.000,2.000){2}{\rule{0.241pt}{0.800pt}}
\put(677.84,449){\rule{0.800pt}{0.482pt}}
\multiput(677.34,449.00)(1.000,1.000){2}{\rule{0.800pt}{0.241pt}}
\put(678.84,451){\rule{0.800pt}{0.482pt}}
\multiput(678.34,451.00)(1.000,1.000){2}{\rule{0.800pt}{0.241pt}}
\put(681,451.84){\rule{0.241pt}{0.800pt}}
\multiput(681.00,451.34)(0.500,1.000){2}{\rule{0.120pt}{0.800pt}}
\put(682,453.34){\rule{0.482pt}{0.800pt}}
\multiput(682.00,452.34)(1.000,2.000){2}{\rule{0.241pt}{0.800pt}}
\put(682.84,456){\rule{0.800pt}{0.482pt}}
\multiput(682.34,456.00)(1.000,1.000){2}{\rule{0.800pt}{0.241pt}}
\put(685,456.84){\rule{0.241pt}{0.800pt}}
\multiput(685.00,456.34)(0.500,1.000){2}{\rule{0.120pt}{0.800pt}}
\put(686,458.34){\rule{0.482pt}{0.800pt}}
\multiput(686.00,457.34)(1.000,2.000){2}{\rule{0.241pt}{0.800pt}}
\put(686.84,461){\rule{0.800pt}{0.482pt}}
\multiput(686.34,461.00)(1.000,1.000){2}{\rule{0.800pt}{0.241pt}}
\put(687.84,463){\rule{0.800pt}{0.482pt}}
\multiput(687.34,463.00)(1.000,1.000){2}{\rule{0.800pt}{0.241pt}}
\put(690,463.84){\rule{0.241pt}{0.800pt}}
\multiput(690.00,463.34)(0.500,1.000){2}{\rule{0.120pt}{0.800pt}}
\put(691,465.34){\rule{0.482pt}{0.800pt}}
\multiput(691.00,464.34)(1.000,2.000){2}{\rule{0.241pt}{0.800pt}}
\put(691.84,468){\rule{0.800pt}{0.482pt}}
\multiput(691.34,468.00)(1.000,1.000){2}{\rule{0.800pt}{0.241pt}}
\put(692.84,470){\rule{0.800pt}{0.482pt}}
\multiput(692.34,470.00)(1.000,1.000){2}{\rule{0.800pt}{0.241pt}}
\put(695,470.84){\rule{0.482pt}{0.800pt}}
\multiput(695.00,470.34)(1.000,1.000){2}{\rule{0.241pt}{0.800pt}}
\put(695.84,473){\rule{0.800pt}{0.482pt}}
\multiput(695.34,473.00)(1.000,1.000){2}{\rule{0.800pt}{0.241pt}}
\put(696.84,475){\rule{0.800pt}{0.482pt}}
\multiput(696.34,475.00)(1.000,1.000){2}{\rule{0.800pt}{0.241pt}}
\put(699,476.34){\rule{0.482pt}{0.800pt}}
\multiput(699.00,475.34)(1.000,2.000){2}{\rule{0.241pt}{0.800pt}}
\put(701,477.84){\rule{0.241pt}{0.800pt}}
\multiput(701.00,477.34)(0.500,1.000){2}{\rule{0.120pt}{0.800pt}}
\put(700.84,480){\rule{0.800pt}{0.482pt}}
\multiput(700.34,480.00)(1.000,1.000){2}{\rule{0.800pt}{0.241pt}}
\put(701.84,482){\rule{0.800pt}{0.482pt}}
\multiput(701.34,482.00)(1.000,1.000){2}{\rule{0.800pt}{0.241pt}}
\put(704,483.34){\rule{0.482pt}{0.800pt}}
\multiput(704.00,482.34)(1.000,2.000){2}{\rule{0.241pt}{0.800pt}}
\put(706,484.84){\rule{0.241pt}{0.800pt}}
\multiput(706.00,484.34)(0.500,1.000){2}{\rule{0.120pt}{0.800pt}}
\put(705.84,487){\rule{0.800pt}{0.482pt}}
\multiput(705.34,487.00)(1.000,1.000){2}{\rule{0.800pt}{0.241pt}}
\put(708,488.34){\rule{0.482pt}{0.800pt}}
\multiput(708.00,487.34)(1.000,2.000){2}{\rule{0.241pt}{0.800pt}}
\put(710,489.84){\rule{0.241pt}{0.800pt}}
\multiput(710.00,489.34)(0.500,1.000){2}{\rule{0.120pt}{0.800pt}}
\put(709.84,492){\rule{0.800pt}{0.482pt}}
\multiput(709.34,492.00)(1.000,1.000){2}{\rule{0.800pt}{0.241pt}}
\put(710.84,494){\rule{0.800pt}{0.482pt}}
\multiput(710.34,494.00)(1.000,1.000){2}{\rule{0.800pt}{0.241pt}}
\put(713,495.34){\rule{0.482pt}{0.800pt}}
\multiput(713.00,494.34)(1.000,2.000){2}{\rule{0.241pt}{0.800pt}}
\put(715,496.84){\rule{0.241pt}{0.800pt}}
\multiput(715.00,496.34)(0.500,1.000){2}{\rule{0.120pt}{0.800pt}}
\put(714.84,499){\rule{0.800pt}{0.482pt}}
\multiput(714.34,499.00)(1.000,1.000){2}{\rule{0.800pt}{0.241pt}}
\put(717,500.34){\rule{0.482pt}{0.800pt}}
\multiput(717.00,499.34)(1.000,2.000){2}{\rule{0.241pt}{0.800pt}}
\put(717.84,503){\rule{0.800pt}{0.482pt}}
\multiput(717.34,503.00)(1.000,1.000){2}{\rule{0.800pt}{0.241pt}}
\put(720,503.84){\rule{0.241pt}{0.800pt}}
\multiput(720.00,503.34)(0.500,1.000){2}{\rule{0.120pt}{0.800pt}}
\put(719.84,506){\rule{0.800pt}{0.482pt}}
\multiput(719.34,506.00)(1.000,1.000){2}{\rule{0.800pt}{0.241pt}}
\put(722,507.34){\rule{0.482pt}{0.800pt}}
\multiput(722.00,506.34)(1.000,2.000){2}{\rule{0.241pt}{0.800pt}}
\put(722.84,510){\rule{0.800pt}{0.482pt}}
\multiput(722.34,510.00)(1.000,1.000){2}{\rule{0.800pt}{0.241pt}}
\put(725,510.84){\rule{0.241pt}{0.800pt}}
\multiput(725.00,510.34)(0.500,1.000){2}{\rule{0.120pt}{0.800pt}}
\put(726,512.34){\rule{0.482pt}{0.800pt}}
\multiput(726.00,511.34)(1.000,2.000){2}{\rule{0.241pt}{0.800pt}}
\put(726.84,515){\rule{0.800pt}{0.482pt}}
\multiput(726.34,515.00)(1.000,1.000){2}{\rule{0.800pt}{0.241pt}}
\put(729,515.84){\rule{0.241pt}{0.800pt}}
\multiput(729.00,515.34)(0.500,1.000){2}{\rule{0.120pt}{0.800pt}}
\put(730,517.34){\rule{0.482pt}{0.800pt}}
\multiput(730.00,516.34)(1.000,2.000){2}{\rule{0.241pt}{0.800pt}}
\put(730.84,520){\rule{0.800pt}{0.482pt}}
\multiput(730.34,520.00)(1.000,1.000){2}{\rule{0.800pt}{0.241pt}}
\put(731.84,522){\rule{0.800pt}{0.482pt}}
\multiput(731.34,522.00)(1.000,1.000){2}{\rule{0.800pt}{0.241pt}}
\put(734,522.84){\rule{0.241pt}{0.800pt}}
\multiput(734.00,522.34)(0.500,1.000){2}{\rule{0.120pt}{0.800pt}}
\put(735,524.34){\rule{0.482pt}{0.800pt}}
\multiput(735.00,523.34)(1.000,2.000){2}{\rule{0.241pt}{0.800pt}}
\put(483.0,210.0){\usebox{\plotpoint}}
\sbox{\plotpoint}{\rule[-0.200pt]{0.400pt}{0.400pt}}%
\put(221.0,143.0){\rule[-0.200pt]{124.545pt}{0.400pt}}
\put(738.0,143.0){\rule[-0.200pt]{0.400pt}{104.551pt}}
\put(221.0,577.0){\rule[-0.200pt]{124.545pt}{0.400pt}}
\put(221.0,143.0){\rule[-0.200pt]{0.400pt}{104.551pt}}
\end{picture}
>>>>>>> 11d381b22515b9114312bca4f8718025eae5b72f

	\end{center}
\caption{Output of algorithm \adwinz with slow gradual changes.}
  \label{fig:ADWIN-R} 
\end{figure}
observe that the average of $W_1$ after $n_1$ steps
is $\mu + \alpha n_1 /2$; we have $\epsilon (=\alpha n_1/2) \ge O(\sqrt{\mu \ln(1/\delta) /n_1})$
iff $n_1 = O(\mu \ln(1/\delta) / \alpha^{2})^{1/3}$. So $n_1$ steps after the change
the window will start shrinking, and will remain at approximately size $n_1$ from then on. 
A dependence on $\alpha$ of the form $O(\alpha^{-2/3})$ may seem odd at first, but one can show
that this window length is actually optimal in this setting, even if $\alpha$ is known: 
it minimizes the sum of variance error (due to short window) and 
error due to out-of-date data (due to long windows in the presence of change). 
Thus, in this setting, \adwinz provably adjusts automatically
the window setting to its optimal value, up to multiplicative constants. 

Figures \ref{fig:ADWIN-E} and \ref{fig:ADWIN-R} illustrate these behaviors. 
In Figure~\ref{fig:ADWIN-E}, a sudden change from $\mu_{t-1}=0.8$ to $\mu_t=0.4$ occurs, 
at $t=1000$. 
One can see that the window size grows linearly up to $t=1000$, 
that \adwinz cuts the window severely $10$ steps later (at $t=1010$), and
that the window expands again linearly after time $t=1010$. 
In Figure \ref{fig:ADWIN-R}, $\mu_t$ gradually descends from $0.8$ to $0.2$
in the range $t\in [1000..2000]$. In this case, \adwinz cuts the window sharply 
at $t$ around $1200$ (i.e., 200 steps after the slope starts), keeps the window length
bounded (with some random fluctuations) while the slope lasts, and starts growing
it linearly again after that. As predicted by theory, detecting the
change is harder in slopes than in abrupt changes. 

%%%%%%%%%%%%%%%%%%%%%%%%%%%%%%%%%%%%%%%%%%%%%%%%%%%%%%%

\subsection{\adwinz for Poisson processes}
\label{Spoisson}

A Poisson process %, named after the French mathematician Siméon-Denis Poisson (1781 -- 1840), 
is the stochastic process in which events occur continuously and independently of one another. A well-known example is radioactive decay of atoms. Many processes are not exactly Poisson processes, but similar enough that for certain types of analysis they can be regarded as such; e.g., telephone calls arriving at a switchboard, webpage requests to a search engine, or rainfall.

Using the Chernoff bound for Poisson processes~\cite{mayers02} 
$$
\Pr\{ X \geq cE[X]\} \leq \exp( -( c \ln(c) +1-c) E[X])
$$
we find a similar $\epsilon_{cut}$ for Poisson processes.

First, we look for a simpler form of this bound. Let $c=1 +\epsilon$ then

$$c \ln(c) -c + 1 = ( 1 +\epsilon) \cdot \ln(1+\epsilon) -\epsilon$$

Using the Taylor expansion of $\ln(x)$
   $$\ln(1+x)= \sum (-1)^{n+1} \cdot \frac{x^n}{n} = x -\frac{x^2}{2} + \frac{x^3}{3} - \cdots $$
we get the following simpler expression:

$$\Pr\{ X \geq (1+\epsilon)E[X]\} \leq \exp( -\epsilon^2 E[X] /2)$$

Now, let $S_n$ be the sum of n Poisson processes. As $S_n$ is also a Poisson process

$$E[S_n] =\lambda_{S_n} = n E[X] = n \cdot \lambda_X$$

and then we obtain

$$\Pr\{ S_n \geq (1+\epsilon)E[S_n]\} \leq \exp( -\epsilon^2 E[S_n] /2)$$

In order to obtain a formula for $\epsilon_{cut}$, let $Y=S_n/n$

$$\Pr\{ Y \geq (1+\epsilon)E[Y]\} \leq \exp( -\epsilon^2 \cdot n \cdot E[Y] /2)$$

And finally, with this bound we get the following $\epsilon_{cut}$ for \adwinz

$$\epsilon_{cut} = \sqrt{ \frac{2 \lambda}{m} \ln \frac{2}{\delta}}$$

where $1/m=1/n_0 + 1/n_1$, and $\lambda$ is the mean of the window data.


%%%%%%%%%%%%%%%%%%%%%%%%%%%%%%%%%%%%%%%%%%%%%%%%%%%%%%%

\subsection{Improving time and memory requirements}
\label{Sadwin2}
%
The first version of \adwinz is computationally expensive, 
because it checks exhaustively all ``large enough'' subwindows
of the current window for possible cuts. 
Furthermore, the contents of the window is kept explicitly, 
with the corresponding memory cost as the window grows. 
To reduce these costs we present a new version \adwintwo
using ideas developed in data stream algorithmics
\cite{BBD02,MUT03,babcock-sampling,datar02}
to find a good cutpoint quickly.
Figure~\ref{blnAdwin2} shows the \adwintwo algorithm.
 We next 
provide a sketch of how this algorithm and these data structures work. 


\begin{figure}

\centering

\begin{codebox}
\Procname{$\proc{\adwintwoz: Adaptive Windowing Algorithm}$}
\li Initialize $W$ as an empty list of buckets
\li Initialize WIDTH, VARIANCE and TOTAL
\li \For each $t >0$ %
\li \Do $\proc{setInput}( x_t, W)$ %\End 
%\li output $\hmu_{W}$ as TOTAL/WIDTH % the pair $\hmu_{W}$ and $\eps_t$ (as in the Theorem)
\li output $\hmu_{W}$ as TOTAL/WIDTH and ChangeAlarm % the pair $\hmu_{W}$ and $\eps_t$ (as in the Theorem)

\end{codebox}

\begin{codebox}
\Procname{$\proc{setInput}$(item e, List W)}
\li  $\proc{insertElement}(e,W)$
\li \Repeat $\proc{deleteElement}(W)$ 
\li \Until $|\hmu_{W_0}-\hmu_{W_1}|  < \epsc$ holds %\geq
\li \quad\quad for every split of $W$ into $W=W_0 \cdot W_1$ 
%\li output $\hmu_{W}$ and ChangeAlarm % the pair $\hmu_{W}$ and $\eps_t$ (as in the Theorem)
\end{codebox}

\begin{codebox}
\Procname{$\proc{insertElement}$(item e, List W)}
\li create a new bucket $b$ with content $e$ and capacity $1$
\li $W \gets W \cup \{b\}$ (i.e., add $e$ to the head of $W$)
\li update WIDTH, VARIANCE and TOTAL
\li $\proc{compressBuckets}(W)$
\end{codebox}

\begin{codebox}
\Procname{$\proc{deleteElement}$(List W)}
\li remove a bucket from tail of List W
\li update WIDTH, VARIANCE and TOTAL
\li ChangeAlarm $ \gets $ {\bf true}	
\end{codebox}

\begin{codebox}
\Procname{$\proc{compressBuckets}$(List W)}
 \li   Traverse the list of buckets in increasing order
 \li    \Do If there are more than $M$ buckets of the same capacity %size
\li      \Do merge buckets 	
 \li    $\proc{compressBuckets}$(sublist of W not traversed)
\end{codebox}

\caption{Algorithm \adwintwoz.}\label{blnAdwin2}
\end{figure}


Our data structure is a variation of exponential histograms \cite{datar02},
a data structure that maintains an approximation of the number of
$1$'s in a sliding window of length $W$ with logarithmic memory 
and update time. 
We adapt this data structure in a way that can provide this approximation
simultaneously for about $O(\log W)$ subwindows 
whose lengths follow a geometric law, {\em with no memory overhead}
with respect to keeping the count for a single window. 
That is, our data structure will be able to give 
the number of $1$s among the most recently $t-1$, $t-\lfloor c\rfloor$, 
$t-\lfloor c^2\rfloor$ ,\dots, $t-\lfloor c^i\rfloor$, \dots read bits, 
with the same amount of memory required to keep an approximation 
for the whole $W$. Note that keeping exact counts for a fixed-window
size is provably impossible in sublinear memory. 
We go around this problem by shrinking or enlarging the window strategically 
so that what would otherwise be an approximate count happens to be exact. 

More precisely, to design the algorithm one chooses a parameter $M$. 
This parameter controls both 
1) the amount of memory used (it will be $O(M \log W/M$) words, and 2)
the closeness of the cutpoints checked (the basis $c$ 
of the geometric series above, which will be about $c=1+1/M$).
Note that the choice of $M$ does {\em not} reflect 
any assumption about the time-scale of change: Since points
are checked at a geometric rate anyway, this policy is essentially
scale-independent. 

More precisely, in the boolean case, the information on the number of $1$'s 
is kept as a series of buckets whose size is always a power of $2$. We keep 
at most $M$ buckets of each size $2^i$, where $M$ is a design parameter. 
For each bucket we record two (integer) elements: {\em capacity} and {\em content }(size, 
or number of $1$s it contains). 

Thus, we use about $M \cdot \log (W/M)$ buckets to maintain our data stream sliding window.
\adwintwo checks as a possible cut every border of a bucket, i.e., 
window lengths of the form $M (1+2+\dots+2^{i-1})+ j \cdot 2^i$, 
for $0 \le j \le M$. It can be seen that these $M \cdot \log (W/M)$ 
points follow approximately a geometric law of basis $\cong 1+1/M$. 

Let's look at an example: a sliding window with 14 elements. We register it as:

\medskip
\hspace{10pt} \fbox{$1010101$}  \fbox{$101$} \fbox{$11$} \fbox{$1$} \fbox{$1$}
\smallskip

\nopagebreak
Content: \hspace{.2pt} $4$ \hspace{10pt} $2$ \hspace{10pt} $2$ \hspace{6pt} $1$ \hspace{3pt} $1$
\smallskip

\nopagebreak
Capacity:  $7$ \hspace{10pt} $3$ \hspace{10pt} $2$ \hspace{6pt} $1$ \hspace{3pt} $1$
\medskip

\noindent
Each time a new element arrives, 
%the timestamp increases and 
if the element is "1", we create a new bucket of {\em content} 1 and {\em capacity}
the number of elements arrived since the last "1". 
After that we compress the rest of buckets: When there are $M+1$ buckets of size $2^i$, 
we merge the two oldest ones (adding its capacity) into a bucket of size $2^{i+1}$. 
So, we use %$O(\2(M-1) \log^2 W/M)$ bits of memory -- i.e. 
$O(M \cdot\log W/M)$ memory words if we assume that a word can contain a number up to $W$. 
In \cite{datar02}, the window is kept at a fixed size $W$. The information missing
about the last bucket is responsible for the approximation error. 
Here, each time we detect change, we reduce the window's length deleting the last bucket, 
instead of (conceptually) dropping a single element as in a typical sliding window framework.
This lets us keep an exact counting, since when throwing away a whole bucket 
we know that we are dropping exactly $2^i$ "1"s.

We summarize these results with the following theorem.

\begin{theorem}
The \adwintwo algorithm maintains a data structure with the following properties: 
\label{ThAdwin2}
\begin{itemize}
\item 
It uses $O(M \cdot \log (W/M))$ memory words
(assuming a memory word can contain numbers up to $W$). 
\item 
It can process the arrival of a new element in $O(1)$ amortized time 
and $O(\log W)$ worst-case time.
\item 
It can provide the exact counts of $1$'s
for all the subwindows whose lengths are of the form 
$\lfloor(1+1/M)^i\rfloor$, in $O(1)$ time per query.
\end{itemize}
\end{theorem}
%
Since \adwintwo tries $O(\log W)$ cutpoints,
the total processing time per example is $O(\log W)$ (amortized) 
and $O(\log W)$ (worst-case). 

In our example, suppose $M=2$, if a new element "$1$" arrives then

\medskip
\hspace{10pt} \fbox{$1010101$}  \fbox{$101$} \fbox{$11$} \fbox{$1$} \fbox{$1$} \fbox{$1$}
\smallskip

\nopagebreak
Content: \hspace{.2pt} $4$ \hspace{10pt} $2$ \hspace{10pt} $2$ \hspace{6pt} $1$ \hspace{3pt} $1$ \hspace{3pt} $1$
\smallskip

\nopagebreak
Capacity: $7$ \hspace{10pt} $3$ \hspace{10pt} $2$ \hspace{6pt} $1$  \hspace{3pt} $1$ \hspace{3pt} $1$
\medskip

\noindent
There are $3$ buckets of $1$, so we compress it:

\medskip
\hspace{10pt} \fbox{$1010101$}  \fbox{$101$} \fbox{$11$} \fbox{$11$} \fbox{$1$} 
\smallskip

\nopagebreak
Content: \hspace{.2pt} $4$ \hspace{10pt} $2$ \hspace{12pt} $2$ \hspace{6pt} $2$ \hspace{5pt} $1$
\smallskip

\nopagebreak
Capacity:  $7$ \hspace{10pt} $3$ \hspace{12pt} $2$ \hspace{6pt} $2$  \hspace{5pt} $1$ 
\medskip

\noindent
and now as we have $3$ buckets of size $2$, we compress it again

\medskip
\hspace{10pt} \fbox{$1010101$}  \fbox{$10111$}  \fbox{$11$} \fbox{$1$} 
\smallskip

\nopagebreak
Content: \hspace{.2pt} $4$ \hspace{14pt} $4$  \hspace{16pt} $2$ \hspace{6pt} $1$
\smallskip

\nopagebreak
Capacity:  $7$ \hspace{14pt} $5$  \hspace{16pt} $2$  \hspace{6pt} $1$ 
\medskip

\noindent
And finally, if we detect change, we reduce the size of our sliding window deleting the last bucket:

\medskip
\hspace{30pt}  \fbox{$10111$}  \fbox{$11$} \fbox{$1$} 
\smallskip

\nopagebreak
Content: \hspace{.2pt} $4$  \hspace{16pt} $2$ \hspace{6pt} $1$
\smallskip

\nopagebreak
Capacity:   $5$  \hspace{16pt} $2$  \hspace{6pt} $1$ 
\medskip

\noindent
In the case of real values, we also maintain buckets of two elements: 
{\em capacity} and {\em content}.
We store at {\em content} the sum of the real numbers we want to summarize.
We restrict {\em capacity} to be a power of two. As in the boolean case,
we use $O(\log W)$ buckets, and check $O(\log W)$ possible cuts. The memory requirement for each
bucket is $ \log W + R + \log \log W$ bits per bucket, where $R$ is number of bits used to store
a real number. 

Figure \ref{fig:ADWIN2-E} shows  the output of \adwintwo to a sudden change, 
and Figure \ref{fig:ADWIN2-R} to a slow gradual change.
The main difference with \adwintwo output is that as \adwinz reduces 
one element by one each time it detects changes, 
\adwintwo deletes an entire bucket, which yields a slightly more jagged
graph in the case of a gradual change. The difference in approximation 
power between \adwinz and \adwintwo is almost negligible, so we 
use \adwintwo exclusively for our experiments. 


\begin{figure}[t]
	\begin{center}
		% GNUPLOT: LaTeX picture
\setlength{\unitlength}{0.240900pt}
\ifx\plotpoint\undefined\newsavebox{\plotpoint}\fi
\begin{picture}(959,720)(0,0)
\sbox{\plotpoint}{\rule[-0.200pt]{0.400pt}{0.400pt}}%
\put(181,143){\makebox(0,0)[r]{$0.1$}}
\put(201.0,143.0){\rule[-0.200pt]{4.818pt}{0.400pt}}
\put(181,197){\makebox(0,0)[r]{$0.2$}}
\put(201.0,197.0){\rule[-0.200pt]{4.818pt}{0.400pt}}
\put(181,252){\makebox(0,0)[r]{$0.3$}}
\put(201.0,252.0){\rule[-0.200pt]{4.818pt}{0.400pt}}
\put(181,306){\makebox(0,0)[r]{$0.4$}}
\put(201.0,306.0){\rule[-0.200pt]{4.818pt}{0.400pt}}
\put(181,360){\makebox(0,0)[r]{$0.5$}}
\put(201.0,360.0){\rule[-0.200pt]{4.818pt}{0.400pt}}
\put(181,414){\makebox(0,0)[r]{$0.6$}}
\put(201.0,414.0){\rule[-0.200pt]{4.818pt}{0.400pt}}
\put(181,468){\makebox(0,0)[r]{$0.7$}}
\put(201.0,468.0){\rule[-0.200pt]{4.818pt}{0.400pt}}
\put(181,523){\makebox(0,0)[r]{$0.8$}}
\put(201.0,523.0){\rule[-0.200pt]{4.818pt}{0.400pt}}
\put(181,577){\makebox(0,0)[r]{$0.9$}}
\put(201.0,577.0){\rule[-0.200pt]{4.818pt}{0.400pt}}
\put(221.0,123.0){\rule[-0.200pt]{0.400pt}{4.818pt}}
\put(221,82){\makebox(0,0){$0$}}
\put(221.0,577.0){\rule[-0.200pt]{0.400pt}{4.818pt}}
\put(307.0,123.0){\rule[-0.200pt]{0.400pt}{4.818pt}}
%\put(307,82){\makebox(0,0){$500$}}
\put(307.0,577.0){\rule[-0.200pt]{0.400pt}{4.818pt}}
\put(393.0,123.0){\rule[-0.200pt]{0.400pt}{4.818pt}}
\put(393,82){\makebox(0,0){$1000$}}
\put(393.0,577.0){\rule[-0.200pt]{0.400pt}{4.818pt}}
\put(479.0,123.0){\rule[-0.200pt]{0.400pt}{4.818pt}}
%\put(479,82){\makebox(0,0){$1500$}}
\put(479.0,577.0){\rule[-0.200pt]{0.400pt}{4.818pt}}
\put(566.0,123.0){\rule[-0.200pt]{0.400pt}{4.818pt}}
\put(566,82){\makebox(0,0){$2000$}}
\put(566.0,577.0){\rule[-0.200pt]{0.400pt}{4.818pt}}
\put(652.0,123.0){\rule[-0.200pt]{0.400pt}{4.818pt}}
%\put(652,82){\makebox(0,0){$2500$}}
\put(652.0,577.0){\rule[-0.200pt]{0.400pt}{4.818pt}}
\put(738.0,123.0){\rule[-0.200pt]{0.400pt}{4.818pt}}
\put(738,82){\makebox(0,0){$3000$}}
\put(738.0,577.0){\rule[-0.200pt]{0.400pt}{4.818pt}}
\put(778,143){\makebox(0,0)[l]{ 0}}
\put(738.0,143.0){\rule[-0.200pt]{4.818pt}{0.400pt}}
\put(778,230){\makebox(0,0)[l]{ 500}}
\put(738.0,230.0){\rule[-0.200pt]{4.818pt}{0.400pt}}
\put(778,317){\makebox(0,0)[l]{ 1000}}
\put(738.0,317.0){\rule[-0.200pt]{4.818pt}{0.400pt}}
\put(778,403){\makebox(0,0)[l]{ 1500}}
\put(738.0,403.0){\rule[-0.200pt]{4.818pt}{0.400pt}}
\put(778,490){\makebox(0,0)[l]{ 2000}}
\put(738.0,490.0){\rule[-0.200pt]{4.818pt}{0.400pt}}
\put(778,577){\makebox(0,0)[l]{ 2500}}
\put(738.0,577.0){\rule[-0.200pt]{4.818pt}{0.400pt}}
\put(221.0,143.0){\rule[-0.200pt]{124.545pt}{0.400pt}}
\put(738.0,143.0){\rule[-0.200pt]{0.400pt}{104.551pt}}
\put(221.0,577.0){\rule[-0.200pt]{124.545pt}{0.400pt}}
\put(221.0,143.0){\rule[-0.200pt]{0.400pt}{104.551pt}}
\put(40,360){\makebox(0,0){$\mu$ axis}}
\put(967,360){\makebox(0,0){Width}}
\put(479,21){\makebox(0,0){$t$ axis}}
\put(479,659){\makebox(0,0){\adwintwo}}
\put(578,537){\makebox(0,0)[r]{$\mu_t$}}
\put(598.0,537.0){\rule[-0.200pt]{24.090pt}{0.400pt}}
\put(223,523){\usebox{\plotpoint}}
\put(391.17,306){\rule{0.400pt}{43.500pt}}
\multiput(390.17,432.71)(2.000,-126.714){2}{\rule{0.400pt}{21.750pt}}
\put(223.0,523.0){\rule[-0.200pt]{40.471pt}{0.400pt}}
\put(393.0,306.0){\rule[-0.200pt]{82.629pt}{0.400pt}}
\put(578,496){\makebox(0,0)[r]{$\hat{\mu}_W$}}
\multiput(598,496)(20.756,0.000){5}{\usebox{\plotpoint}}
\put(698,496){\usebox{\plotpoint}}
\put(223,577){\usebox{\plotpoint}}
\put(223.00,577.00){\usebox{\plotpoint}}
\put(226.74,560.47){\usebox{\plotpoint}}
\put(228.94,545.10){\usebox{\plotpoint}}
\put(235.26,532.86){\usebox{\plotpoint}}
\put(242.16,548.24){\usebox{\plotpoint}}
\put(247.23,528.40){\usebox{\plotpoint}}
\put(254.56,518.68){\usebox{\plotpoint}}
\put(265.93,524.85){\usebox{\plotpoint}}
\put(280.52,532.00){\usebox{\plotpoint}}
\put(295.73,523.73){\usebox{\plotpoint}}
\put(313.67,525.33){\usebox{\plotpoint}}
\put(329.99,526.01){\usebox{\plotpoint}}
\put(347.23,529.23){\usebox{\plotpoint}}
\put(365.54,529.00){\usebox{\plotpoint}}
\put(382.90,528.90){\usebox{\plotpoint}}
\multiput(393,528)(1.151,-20.724){2}{\usebox{\plotpoint}}
\put(396.26,476.21){\usebox{\plotpoint}}
\put(397.96,455.53){\usebox{\plotpoint}}
\multiput(398,455)(0.768,-20.741){2}{\usebox{\plotpoint}}
\put(400.90,393.35){\usebox{\plotpoint}}
\put(406.12,373.83){\usebox{\plotpoint}}
\put(409.85,354.41){\usebox{\plotpoint}}
\put(414.43,338.70){\usebox{\plotpoint}}
\put(421.76,343.34){\usebox{\plotpoint}}
\put(433.70,346.61){\usebox{\plotpoint}}
\put(440.82,329.89){\usebox{\plotpoint}}
\put(453.37,327.00){\usebox{\plotpoint}}
\put(466.10,325.21){\usebox{\plotpoint}}
\put(481.91,323.55){\usebox{\plotpoint}}
\put(497.11,317.89){\usebox{\plotpoint}}
\put(514.86,312.57){\usebox{\plotpoint}}
\put(533.94,311.00){\usebox{\plotpoint}}
\put(552.88,310.44){\usebox{\plotpoint}}
\put(572.14,314.00){\usebox{\plotpoint}}
\put(590.59,309.59){\usebox{\plotpoint}}
\put(609.28,311.72){\usebox{\plotpoint}}
\put(628.62,311.00){\usebox{\plotpoint}}
\put(648.72,311.00){\usebox{\plotpoint}}
\put(668.59,312.00){\usebox{\plotpoint}}
\put(687.25,314.62){\usebox{\plotpoint}}
\put(706.79,315.00){\usebox{\plotpoint}}
\put(726.25,315.00){\usebox{\plotpoint}}
\put(736,314){\usebox{\plotpoint}}
\sbox{\plotpoint}{\rule[-0.400pt]{0.800pt}{0.800pt}}%
\sbox{\plotpoint}{\rule[-0.200pt]{0.400pt}{0.400pt}}%
\put(578,455){\makebox(0,0)[r]{$W$}}
\sbox{\plotpoint}{\rule[-0.400pt]{0.800pt}{0.800pt}}%
\put(598.0,455.0){\rule[-0.400pt]{24.090pt}{0.800pt}}
\put(223,145){\usebox{\plotpoint}}
\put(223,143.84){\rule{0.241pt}{0.800pt}}
\multiput(223.00,143.34)(0.500,1.000){2}{\rule{0.120pt}{0.800pt}}
\put(224,145.34){\rule{0.482pt}{0.800pt}}
\multiput(224.00,144.34)(1.000,2.000){2}{\rule{0.241pt}{0.800pt}}
\put(226,147.34){\rule{0.482pt}{0.800pt}}
\multiput(226.00,146.34)(1.000,2.000){2}{\rule{0.241pt}{0.800pt}}
\put(226.84,150){\rule{0.800pt}{0.482pt}}
\multiput(226.34,150.00)(1.000,1.000){2}{\rule{0.800pt}{0.241pt}}
\put(229,150.84){\rule{0.482pt}{0.800pt}}
\multiput(229.00,150.34)(1.000,1.000){2}{\rule{0.241pt}{0.800pt}}
\put(231,152.34){\rule{0.482pt}{0.800pt}}
\multiput(231.00,151.34)(1.000,2.000){2}{\rule{0.241pt}{0.800pt}}
\put(233,154.34){\rule{0.482pt}{0.800pt}}
\multiput(233.00,153.34)(1.000,2.000){2}{\rule{0.241pt}{0.800pt}}
\put(233.84,157){\rule{0.800pt}{0.482pt}}
\multiput(233.34,157.00)(1.000,1.000){2}{\rule{0.800pt}{0.241pt}}
\put(236,157.84){\rule{0.482pt}{0.800pt}}
\multiput(236.00,157.34)(1.000,1.000){2}{\rule{0.241pt}{0.800pt}}
\put(238,159.34){\rule{0.482pt}{0.800pt}}
\multiput(238.00,158.34)(1.000,2.000){2}{\rule{0.241pt}{0.800pt}}
\put(240,161.34){\rule{0.482pt}{0.800pt}}
\multiput(240.00,160.34)(1.000,2.000){2}{\rule{0.241pt}{0.800pt}}
\put(240.84,164){\rule{0.800pt}{0.482pt}}
\multiput(240.34,164.00)(1.000,1.000){2}{\rule{0.800pt}{0.241pt}}
\put(243,164.84){\rule{0.482pt}{0.800pt}}
\multiput(243.00,164.34)(1.000,1.000){2}{\rule{0.241pt}{0.800pt}}
\put(245,166.34){\rule{0.482pt}{0.800pt}}
\multiput(245.00,165.34)(1.000,2.000){2}{\rule{0.241pt}{0.800pt}}
\put(245.84,169){\rule{0.800pt}{0.482pt}}
\multiput(245.34,169.00)(1.000,1.000){2}{\rule{0.800pt}{0.241pt}}
\put(248,170.34){\rule{0.482pt}{0.800pt}}
\multiput(248.00,169.34)(1.000,2.000){2}{\rule{0.241pt}{0.800pt}}
\put(250,171.84){\rule{0.482pt}{0.800pt}}
\multiput(250.00,171.34)(1.000,1.000){2}{\rule{0.241pt}{0.800pt}}
\put(252,173.34){\rule{0.482pt}{0.800pt}}
\multiput(252.00,172.34)(1.000,2.000){2}{\rule{0.241pt}{0.800pt}}
\put(252.84,176){\rule{0.800pt}{0.482pt}}
\multiput(252.34,176.00)(1.000,1.000){2}{\rule{0.800pt}{0.241pt}}
\put(255,176.84){\rule{0.482pt}{0.800pt}}
\multiput(255.00,176.34)(1.000,1.000){2}{\rule{0.241pt}{0.800pt}}
\put(257,178.34){\rule{0.482pt}{0.800pt}}
\multiput(257.00,177.34)(1.000,2.000){2}{\rule{0.241pt}{0.800pt}}
\put(257.84,181){\rule{0.800pt}{0.482pt}}
\multiput(257.34,181.00)(1.000,1.000){2}{\rule{0.800pt}{0.241pt}}
\put(260,182.34){\rule{0.482pt}{0.800pt}}
\multiput(260.00,181.34)(1.000,2.000){2}{\rule{0.241pt}{0.800pt}}
\put(262,183.84){\rule{0.482pt}{0.800pt}}
\multiput(262.00,183.34)(1.000,1.000){2}{\rule{0.241pt}{0.800pt}}
\put(264,185.34){\rule{0.482pt}{0.800pt}}
\multiput(264.00,184.34)(1.000,2.000){2}{\rule{0.241pt}{0.800pt}}
\put(264.84,188){\rule{0.800pt}{0.482pt}}
\multiput(264.34,188.00)(1.000,1.000){2}{\rule{0.800pt}{0.241pt}}
\put(267,189.34){\rule{0.482pt}{0.800pt}}
\multiput(267.00,188.34)(1.000,2.000){2}{\rule{0.241pt}{0.800pt}}
\put(269,190.84){\rule{0.482pt}{0.800pt}}
\multiput(269.00,190.34)(1.000,1.000){2}{\rule{0.241pt}{0.800pt}}
\put(271,192.34){\rule{0.482pt}{0.800pt}}
\multiput(271.00,191.34)(1.000,2.000){2}{\rule{0.241pt}{0.800pt}}
\put(271.84,195){\rule{0.800pt}{0.482pt}}
\multiput(271.34,195.00)(1.000,1.000){2}{\rule{0.800pt}{0.241pt}}
\put(274,196.34){\rule{0.482pt}{0.800pt}}
\multiput(274.00,195.34)(1.000,2.000){2}{\rule{0.241pt}{0.800pt}}
\put(276,197.84){\rule{0.482pt}{0.800pt}}
\multiput(276.00,197.34)(1.000,1.000){2}{\rule{0.241pt}{0.800pt}}
\put(276.84,200){\rule{0.800pt}{0.482pt}}
\multiput(276.34,200.00)(1.000,1.000){2}{\rule{0.800pt}{0.241pt}}
\put(279,201.34){\rule{0.482pt}{0.800pt}}
\multiput(279.00,200.34)(1.000,2.000){2}{\rule{0.241pt}{0.800pt}}
\put(281,202.84){\rule{0.482pt}{0.800pt}}
\multiput(281.00,202.34)(1.000,1.000){2}{\rule{0.241pt}{0.800pt}}
\put(283,204.34){\rule{0.482pt}{0.800pt}}
\multiput(283.00,203.34)(1.000,2.000){2}{\rule{0.241pt}{0.800pt}}
\put(283.84,207){\rule{0.800pt}{0.482pt}}
\multiput(283.34,207.00)(1.000,1.000){2}{\rule{0.800pt}{0.241pt}}
\put(286,208.34){\rule{0.482pt}{0.800pt}}
\multiput(286.00,207.34)(1.000,2.000){2}{\rule{0.241pt}{0.800pt}}
\put(288,209.84){\rule{0.482pt}{0.800pt}}
\multiput(288.00,209.34)(1.000,1.000){2}{\rule{0.241pt}{0.800pt}}
\put(290,211.34){\rule{0.482pt}{0.800pt}}
\multiput(290.00,210.34)(1.000,2.000){2}{\rule{0.241pt}{0.800pt}}
\put(290.84,214){\rule{0.800pt}{0.482pt}}
\multiput(290.34,214.00)(1.000,1.000){2}{\rule{0.800pt}{0.241pt}}
\put(293,215.34){\rule{0.482pt}{0.800pt}}
\multiput(293.00,214.34)(1.000,2.000){2}{\rule{0.241pt}{0.800pt}}
\put(295,216.84){\rule{0.482pt}{0.800pt}}
\multiput(295.00,216.34)(1.000,1.000){2}{\rule{0.241pt}{0.800pt}}
\put(295.84,219){\rule{0.800pt}{0.482pt}}
\multiput(295.34,219.00)(1.000,1.000){2}{\rule{0.800pt}{0.241pt}}
\put(298,220.34){\rule{0.482pt}{0.800pt}}
\multiput(298.00,219.34)(1.000,2.000){2}{\rule{0.241pt}{0.800pt}}
\put(300,222.34){\rule{0.482pt}{0.800pt}}
\multiput(300.00,221.34)(1.000,2.000){2}{\rule{0.241pt}{0.800pt}}
\put(302,223.84){\rule{0.482pt}{0.800pt}}
\multiput(302.00,223.34)(1.000,1.000){2}{\rule{0.241pt}{0.800pt}}
\put(302.84,226){\rule{0.800pt}{0.482pt}}
\multiput(302.34,226.00)(1.000,1.000){2}{\rule{0.800pt}{0.241pt}}
\put(305,227.34){\rule{0.482pt}{0.800pt}}
\multiput(305.00,226.34)(1.000,2.000){2}{\rule{0.241pt}{0.800pt}}
\put(307,229.34){\rule{0.482pt}{0.800pt}}
\multiput(307.00,228.34)(1.000,2.000){2}{\rule{0.241pt}{0.800pt}}
\put(309,230.84){\rule{0.241pt}{0.800pt}}
\multiput(309.00,230.34)(0.500,1.000){2}{\rule{0.120pt}{0.800pt}}
\put(310,232.34){\rule{0.482pt}{0.800pt}}
\multiput(310.00,231.34)(1.000,2.000){2}{\rule{0.241pt}{0.800pt}}
\put(312,234.34){\rule{0.482pt}{0.800pt}}
\multiput(312.00,233.34)(1.000,2.000){2}{\rule{0.241pt}{0.800pt}}
\put(314,235.84){\rule{0.482pt}{0.800pt}}
\multiput(314.00,235.34)(1.000,1.000){2}{\rule{0.241pt}{0.800pt}}
\put(314.84,238){\rule{0.800pt}{0.482pt}}
\multiput(314.34,238.00)(1.000,1.000){2}{\rule{0.800pt}{0.241pt}}
\put(317,239.34){\rule{0.482pt}{0.800pt}}
\multiput(317.00,238.34)(1.000,2.000){2}{\rule{0.241pt}{0.800pt}}
\put(319,241.34){\rule{0.482pt}{0.800pt}}
\multiput(319.00,240.34)(1.000,2.000){2}{\rule{0.241pt}{0.800pt}}
\put(321,242.84){\rule{0.482pt}{0.800pt}}
\multiput(321.00,242.34)(1.000,1.000){2}{\rule{0.241pt}{0.800pt}}
\put(321.84,245){\rule{0.800pt}{0.482pt}}
\multiput(321.34,245.00)(1.000,1.000){2}{\rule{0.800pt}{0.241pt}}
\put(324,246.34){\rule{0.482pt}{0.800pt}}
\multiput(324.00,245.34)(1.000,2.000){2}{\rule{0.241pt}{0.800pt}}
\put(326,248.34){\rule{0.482pt}{0.800pt}}
\multiput(326.00,247.34)(1.000,2.000){2}{\rule{0.241pt}{0.800pt}}
\put(328,249.84){\rule{0.241pt}{0.800pt}}
\multiput(328.00,249.34)(0.500,1.000){2}{\rule{0.120pt}{0.800pt}}
\put(329,251.34){\rule{0.482pt}{0.800pt}}
\multiput(329.00,250.34)(1.000,2.000){2}{\rule{0.241pt}{0.800pt}}
\put(331,253.34){\rule{0.482pt}{0.800pt}}
\multiput(331.00,252.34)(1.000,2.000){2}{\rule{0.241pt}{0.800pt}}
\put(333,255.34){\rule{0.482pt}{0.800pt}}
\multiput(333.00,254.34)(1.000,2.000){2}{\rule{0.241pt}{0.800pt}}
\put(335,256.84){\rule{0.241pt}{0.800pt}}
\multiput(335.00,256.34)(0.500,1.000){2}{\rule{0.120pt}{0.800pt}}
\put(336,258.34){\rule{0.482pt}{0.800pt}}
\multiput(336.00,257.34)(1.000,2.000){2}{\rule{0.241pt}{0.800pt}}
\put(338,260.34){\rule{0.482pt}{0.800pt}}
\multiput(338.00,259.34)(1.000,2.000){2}{\rule{0.241pt}{0.800pt}}
\put(340,262.34){\rule{0.482pt}{0.800pt}}
\multiput(340.00,261.34)(1.000,2.000){2}{\rule{0.241pt}{0.800pt}}
\put(342,263.84){\rule{0.241pt}{0.800pt}}
\multiput(342.00,263.34)(0.500,1.000){2}{\rule{0.120pt}{0.800pt}}
\put(343,265.34){\rule{0.482pt}{0.800pt}}
\multiput(343.00,264.34)(1.000,2.000){2}{\rule{0.241pt}{0.800pt}}
\put(345,267.34){\rule{0.482pt}{0.800pt}}
\multiput(345.00,266.34)(1.000,2.000){2}{\rule{0.241pt}{0.800pt}}
\put(347,268.84){\rule{0.241pt}{0.800pt}}
\multiput(347.00,268.34)(0.500,1.000){2}{\rule{0.120pt}{0.800pt}}
\put(348,270.34){\rule{0.482pt}{0.800pt}}
\multiput(348.00,269.34)(1.000,2.000){2}{\rule{0.241pt}{0.800pt}}
\put(350,272.34){\rule{0.482pt}{0.800pt}}
\multiput(350.00,271.34)(1.000,2.000){2}{\rule{0.241pt}{0.800pt}}
\put(352,274.34){\rule{0.482pt}{0.800pt}}
\multiput(352.00,273.34)(1.000,2.000){2}{\rule{0.241pt}{0.800pt}}
\put(354,275.84){\rule{0.241pt}{0.800pt}}
\multiput(354.00,275.34)(0.500,1.000){2}{\rule{0.120pt}{0.800pt}}
\put(355,277.34){\rule{0.482pt}{0.800pt}}
\multiput(355.00,276.34)(1.000,2.000){2}{\rule{0.241pt}{0.800pt}}
\put(357,279.34){\rule{0.482pt}{0.800pt}}
\multiput(357.00,278.34)(1.000,2.000){2}{\rule{0.241pt}{0.800pt}}
\put(357.84,282){\rule{0.800pt}{0.482pt}}
\multiput(357.34,282.00)(1.000,1.000){2}{\rule{0.800pt}{0.241pt}}
\put(360,282.84){\rule{0.482pt}{0.800pt}}
\multiput(360.00,282.34)(1.000,1.000){2}{\rule{0.241pt}{0.800pt}}
\put(362,284.34){\rule{0.482pt}{0.800pt}}
\multiput(362.00,283.34)(1.000,2.000){2}{\rule{0.241pt}{0.800pt}}
\put(364,286.34){\rule{0.482pt}{0.800pt}}
\multiput(364.00,285.34)(1.000,2.000){2}{\rule{0.241pt}{0.800pt}}
\put(364.84,289){\rule{0.800pt}{0.482pt}}
\multiput(364.34,289.00)(1.000,1.000){2}{\rule{0.800pt}{0.241pt}}
\put(367,289.84){\rule{0.482pt}{0.800pt}}
\multiput(367.00,289.34)(1.000,1.000){2}{\rule{0.241pt}{0.800pt}}
\put(369,291.34){\rule{0.482pt}{0.800pt}}
\multiput(369.00,290.34)(1.000,2.000){2}{\rule{0.241pt}{0.800pt}}
\put(371,293.34){\rule{0.482pt}{0.800pt}}
\multiput(371.00,292.34)(1.000,2.000){2}{\rule{0.241pt}{0.800pt}}
\put(371.84,296){\rule{0.800pt}{0.482pt}}
\multiput(371.34,296.00)(1.000,1.000){2}{\rule{0.800pt}{0.241pt}}
\put(374,296.84){\rule{0.482pt}{0.800pt}}
\multiput(374.00,296.34)(1.000,1.000){2}{\rule{0.241pt}{0.800pt}}
\put(376,298.34){\rule{0.482pt}{0.800pt}}
\multiput(376.00,297.34)(1.000,2.000){2}{\rule{0.241pt}{0.800pt}}
\put(376.84,301){\rule{0.800pt}{0.482pt}}
\multiput(376.34,301.00)(1.000,1.000){2}{\rule{0.800pt}{0.241pt}}
\put(379,301.84){\rule{0.482pt}{0.800pt}}
\multiput(379.00,301.34)(1.000,1.000){2}{\rule{0.241pt}{0.800pt}}
\put(381,303.34){\rule{0.482pt}{0.800pt}}
\multiput(381.00,302.34)(1.000,2.000){2}{\rule{0.241pt}{0.800pt}}
\put(383,305.34){\rule{0.482pt}{0.800pt}}
\multiput(383.00,304.34)(1.000,2.000){2}{\rule{0.241pt}{0.800pt}}
\put(383.84,308){\rule{0.800pt}{0.482pt}}
\multiput(383.34,308.00)(1.000,1.000){2}{\rule{0.800pt}{0.241pt}}
\put(386,308.84){\rule{0.482pt}{0.800pt}}
\multiput(386.00,308.34)(1.000,1.000){2}{\rule{0.241pt}{0.800pt}}
\put(388,310.34){\rule{0.482pt}{0.800pt}}
\multiput(388.00,309.34)(1.000,2.000){2}{\rule{0.241pt}{0.800pt}}
\put(388.84,313){\rule{0.800pt}{0.482pt}}
\multiput(388.34,313.00)(1.000,1.000){2}{\rule{0.800pt}{0.241pt}}
\put(391,314.34){\rule{0.482pt}{0.800pt}}
\multiput(391.00,313.34)(1.000,2.000){2}{\rule{0.241pt}{0.800pt}}
\put(392.34,154){\rule{0.800pt}{39.267pt}}
\multiput(391.34,235.50)(2.000,-81.500){2}{\rule{0.800pt}{19.633pt}}
\put(395,153.34){\rule{0.482pt}{0.800pt}}
\multiput(395.00,152.34)(1.000,2.000){2}{\rule{0.241pt}{0.800pt}}
\put(395.84,156){\rule{0.800pt}{0.482pt}}
\multiput(395.34,156.00)(1.000,1.000){2}{\rule{0.800pt}{0.241pt}}
\put(398,155.84){\rule{0.482pt}{0.800pt}}
\multiput(398.00,156.34)(1.000,-1.000){2}{\rule{0.241pt}{0.800pt}}
\put(400,156.34){\rule{0.482pt}{0.800pt}}
\multiput(400.00,155.34)(1.000,2.000){2}{\rule{0.241pt}{0.800pt}}
\put(402,157.84){\rule{0.482pt}{0.800pt}}
\multiput(402.00,157.34)(1.000,1.000){2}{\rule{0.241pt}{0.800pt}}
\put(402.84,160){\rule{0.800pt}{0.482pt}}
\multiput(402.34,160.00)(1.000,1.000){2}{\rule{0.800pt}{0.241pt}}
\put(405,161.34){\rule{0.482pt}{0.800pt}}
\multiput(405.00,160.34)(1.000,2.000){2}{\rule{0.241pt}{0.800pt}}
\put(407,163.34){\rule{0.482pt}{0.800pt}}
\multiput(407.00,162.34)(1.000,2.000){2}{\rule{0.241pt}{0.800pt}}
\put(409,164.84){\rule{0.241pt}{0.800pt}}
\multiput(409.00,164.34)(0.500,1.000){2}{\rule{0.120pt}{0.800pt}}
\put(410,166.34){\rule{0.482pt}{0.800pt}}
\multiput(410.00,165.34)(1.000,2.000){2}{\rule{0.241pt}{0.800pt}}
\put(412,168.34){\rule{0.482pt}{0.800pt}}
\multiput(412.00,167.34)(1.000,2.000){2}{\rule{0.241pt}{0.800pt}}
\put(414,170.34){\rule{0.482pt}{0.800pt}}
\multiput(414.00,169.34)(1.000,2.000){2}{\rule{0.241pt}{0.800pt}}
\put(416,171.84){\rule{0.241pt}{0.800pt}}
\multiput(416.00,171.34)(0.500,1.000){2}{\rule{0.120pt}{0.800pt}}
\put(417,173.34){\rule{0.482pt}{0.800pt}}
\multiput(417.00,172.34)(1.000,2.000){2}{\rule{0.241pt}{0.800pt}}
\put(419,175.34){\rule{0.482pt}{0.800pt}}
\multiput(419.00,174.34)(1.000,2.000){2}{\rule{0.241pt}{0.800pt}}
\put(421,176.84){\rule{0.482pt}{0.800pt}}
\multiput(421.00,176.34)(1.000,1.000){2}{\rule{0.241pt}{0.800pt}}
\put(421.84,179){\rule{0.800pt}{0.482pt}}
\multiput(421.34,179.00)(1.000,1.000){2}{\rule{0.800pt}{0.241pt}}
\put(424,180.34){\rule{0.482pt}{0.800pt}}
\multiput(424.00,179.34)(1.000,2.000){2}{\rule{0.241pt}{0.800pt}}
\put(426,182.34){\rule{0.482pt}{0.800pt}}
\multiput(426.00,181.34)(1.000,2.000){2}{\rule{0.241pt}{0.800pt}}
\put(428,183.84){\rule{0.241pt}{0.800pt}}
\multiput(428.00,183.34)(0.500,1.000){2}{\rule{0.120pt}{0.800pt}}
\put(429,185.34){\rule{0.482pt}{0.800pt}}
\multiput(429.00,184.34)(1.000,2.000){2}{\rule{0.241pt}{0.800pt}}
\put(431,187.34){\rule{0.482pt}{0.800pt}}
\multiput(431.00,186.34)(1.000,2.000){2}{\rule{0.241pt}{0.800pt}}
\put(433,189.34){\rule{0.482pt}{0.800pt}}
\multiput(433.00,188.34)(1.000,2.000){2}{\rule{0.241pt}{0.800pt}}
\put(435,190.84){\rule{0.241pt}{0.800pt}}
\multiput(435.00,190.34)(0.500,1.000){2}{\rule{0.120pt}{0.800pt}}
\put(436,192.34){\rule{0.482pt}{0.800pt}}
\multiput(436.00,191.34)(1.000,2.000){2}{\rule{0.241pt}{0.800pt}}
\put(438,194.34){\rule{0.482pt}{0.800pt}}
\multiput(438.00,193.34)(1.000,2.000){2}{\rule{0.241pt}{0.800pt}}
\put(438.84,197){\rule{0.800pt}{0.482pt}}
\multiput(438.34,197.00)(1.000,1.000){2}{\rule{0.800pt}{0.241pt}}
\put(441,197.84){\rule{0.482pt}{0.800pt}}
\multiput(441.00,197.34)(1.000,1.000){2}{\rule{0.241pt}{0.800pt}}
\put(443,199.34){\rule{0.482pt}{0.800pt}}
\multiput(443.00,198.34)(1.000,2.000){2}{\rule{0.241pt}{0.800pt}}
\put(445,201.34){\rule{0.482pt}{0.800pt}}
\multiput(445.00,200.34)(1.000,2.000){2}{\rule{0.241pt}{0.800pt}}
\put(447,202.84){\rule{0.241pt}{0.800pt}}
\multiput(447.00,202.34)(0.500,1.000){2}{\rule{0.120pt}{0.800pt}}
\put(448,204.34){\rule{0.482pt}{0.800pt}}
\multiput(448.00,203.34)(1.000,2.000){2}{\rule{0.241pt}{0.800pt}}
\put(450,206.34){\rule{0.482pt}{0.800pt}}
\multiput(450.00,205.34)(1.000,2.000){2}{\rule{0.241pt}{0.800pt}}
\put(452,208.34){\rule{0.482pt}{0.800pt}}
\multiput(452.00,207.34)(1.000,2.000){2}{\rule{0.241pt}{0.800pt}}
\put(454,209.84){\rule{0.241pt}{0.800pt}}
\multiput(454.00,209.34)(0.500,1.000){2}{\rule{0.120pt}{0.800pt}}
\put(455,211.34){\rule{0.482pt}{0.800pt}}
\multiput(455.00,210.34)(1.000,2.000){2}{\rule{0.241pt}{0.800pt}}
\put(457,213.34){\rule{0.482pt}{0.800pt}}
\multiput(457.00,212.34)(1.000,2.000){2}{\rule{0.241pt}{0.800pt}}
\put(457.84,216){\rule{0.800pt}{0.482pt}}
\multiput(457.34,216.00)(1.000,1.000){2}{\rule{0.800pt}{0.241pt}}
\put(460,216.84){\rule{0.482pt}{0.800pt}}
\multiput(460.00,216.34)(1.000,1.000){2}{\rule{0.241pt}{0.800pt}}
\put(462,218.34){\rule{0.482pt}{0.800pt}}
\multiput(462.00,217.34)(1.000,2.000){2}{\rule{0.241pt}{0.800pt}}
\put(464,220.34){\rule{0.482pt}{0.800pt}}
\multiput(464.00,219.34)(1.000,2.000){2}{\rule{0.241pt}{0.800pt}}
\put(464.84,223){\rule{0.800pt}{0.482pt}}
\multiput(464.34,223.00)(1.000,1.000){2}{\rule{0.800pt}{0.241pt}}
\put(467,223.84){\rule{0.482pt}{0.800pt}}
\multiput(467.00,223.34)(1.000,1.000){2}{\rule{0.241pt}{0.800pt}}
\put(469,225.34){\rule{0.482pt}{0.800pt}}
\multiput(469.00,224.34)(1.000,2.000){2}{\rule{0.241pt}{0.800pt}}
\put(471,227.34){\rule{0.482pt}{0.800pt}}
\multiput(471.00,226.34)(1.000,2.000){2}{\rule{0.241pt}{0.800pt}}
\put(471.84,230){\rule{0.800pt}{0.482pt}}
\multiput(471.34,230.00)(1.000,1.000){2}{\rule{0.800pt}{0.241pt}}
\put(474,230.84){\rule{0.482pt}{0.800pt}}
\multiput(474.00,230.34)(1.000,1.000){2}{\rule{0.241pt}{0.800pt}}
\put(476,232.34){\rule{0.482pt}{0.800pt}}
\multiput(476.00,231.34)(1.000,2.000){2}{\rule{0.241pt}{0.800pt}}
\put(476.84,235){\rule{0.800pt}{0.482pt}}
\multiput(476.34,235.00)(1.000,1.000){2}{\rule{0.800pt}{0.241pt}}
\put(479,235.84){\rule{0.482pt}{0.800pt}}
\multiput(479.00,235.34)(1.000,1.000){2}{\rule{0.241pt}{0.800pt}}
\put(481,237.34){\rule{0.482pt}{0.800pt}}
\multiput(481.00,236.34)(1.000,2.000){2}{\rule{0.241pt}{0.800pt}}
\put(483,239.34){\rule{0.482pt}{0.800pt}}
\multiput(483.00,238.34)(1.000,2.000){2}{\rule{0.241pt}{0.800pt}}
\put(483.84,242){\rule{0.800pt}{0.482pt}}
\multiput(483.34,242.00)(1.000,1.000){2}{\rule{0.800pt}{0.241pt}}
\put(486,242.84){\rule{0.482pt}{0.800pt}}
\multiput(486.00,242.34)(1.000,1.000){2}{\rule{0.241pt}{0.800pt}}
\put(488,244.34){\rule{0.482pt}{0.800pt}}
\multiput(488.00,243.34)(1.000,2.000){2}{\rule{0.241pt}{0.800pt}}
\put(488.84,247){\rule{0.800pt}{0.482pt}}
\multiput(488.34,247.00)(1.000,1.000){2}{\rule{0.800pt}{0.241pt}}
\put(491,248.34){\rule{0.482pt}{0.800pt}}
\multiput(491.00,247.34)(1.000,2.000){2}{\rule{0.241pt}{0.800pt}}
\put(493,249.84){\rule{0.482pt}{0.800pt}}
\multiput(493.00,249.34)(1.000,1.000){2}{\rule{0.241pt}{0.800pt}}
\put(495,251.34){\rule{0.482pt}{0.800pt}}
\multiput(495.00,250.34)(1.000,2.000){2}{\rule{0.241pt}{0.800pt}}
\put(495.84,254){\rule{0.800pt}{0.482pt}}
\multiput(495.34,254.00)(1.000,1.000){2}{\rule{0.800pt}{0.241pt}}
\put(498,255.34){\rule{0.482pt}{0.800pt}}
\multiput(498.00,254.34)(1.000,2.000){2}{\rule{0.241pt}{0.800pt}}
\put(500,256.84){\rule{0.482pt}{0.800pt}}
\multiput(500.00,256.34)(1.000,1.000){2}{\rule{0.241pt}{0.800pt}}
\put(502,258.34){\rule{0.482pt}{0.800pt}}
\multiput(502.00,257.34)(1.000,2.000){2}{\rule{0.241pt}{0.800pt}}
\put(502.84,261){\rule{0.800pt}{0.482pt}}
\multiput(502.34,261.00)(1.000,1.000){2}{\rule{0.800pt}{0.241pt}}
\put(505,262.34){\rule{0.482pt}{0.800pt}}
\multiput(505.00,261.34)(1.000,2.000){2}{\rule{0.241pt}{0.800pt}}
\put(507,263.84){\rule{0.482pt}{0.800pt}}
\multiput(507.00,263.34)(1.000,1.000){2}{\rule{0.241pt}{0.800pt}}
\put(507.84,266){\rule{0.800pt}{0.482pt}}
\multiput(507.34,266.00)(1.000,1.000){2}{\rule{0.800pt}{0.241pt}}
\put(510,267.34){\rule{0.482pt}{0.800pt}}
\multiput(510.00,266.34)(1.000,2.000){2}{\rule{0.241pt}{0.800pt}}
\put(512,268.84){\rule{0.482pt}{0.800pt}}
\multiput(512.00,268.34)(1.000,1.000){2}{\rule{0.241pt}{0.800pt}}
\put(514,270.34){\rule{0.482pt}{0.800pt}}
\multiput(514.00,269.34)(1.000,2.000){2}{\rule{0.241pt}{0.800pt}}
\put(514.84,273){\rule{0.800pt}{0.482pt}}
\multiput(514.34,273.00)(1.000,1.000){2}{\rule{0.800pt}{0.241pt}}
\put(517,274.34){\rule{0.482pt}{0.800pt}}
\multiput(517.00,273.34)(1.000,2.000){2}{\rule{0.241pt}{0.800pt}}
\put(519,275.84){\rule{0.482pt}{0.800pt}}
\multiput(519.00,275.34)(1.000,1.000){2}{\rule{0.241pt}{0.800pt}}
\put(521,277.34){\rule{0.482pt}{0.800pt}}
\multiput(521.00,276.34)(1.000,2.000){2}{\rule{0.241pt}{0.800pt}}
\put(521.84,280){\rule{0.800pt}{0.482pt}}
\multiput(521.34,280.00)(1.000,1.000){2}{\rule{0.800pt}{0.241pt}}
\put(524,281.34){\rule{0.482pt}{0.800pt}}
\multiput(524.00,280.34)(1.000,2.000){2}{\rule{0.241pt}{0.800pt}}
\put(526,282.84){\rule{0.482pt}{0.800pt}}
\multiput(526.00,282.34)(1.000,1.000){2}{\rule{0.241pt}{0.800pt}}
\put(526.84,285){\rule{0.800pt}{0.482pt}}
\multiput(526.34,285.00)(1.000,1.000){2}{\rule{0.800pt}{0.241pt}}
\put(529,286.34){\rule{0.482pt}{0.800pt}}
\multiput(529.00,285.34)(1.000,2.000){2}{\rule{0.241pt}{0.800pt}}
\put(531,288.34){\rule{0.482pt}{0.800pt}}
\multiput(531.00,287.34)(1.000,2.000){2}{\rule{0.241pt}{0.800pt}}
\put(533,289.84){\rule{0.482pt}{0.800pt}}
\multiput(533.00,289.34)(1.000,1.000){2}{\rule{0.241pt}{0.800pt}}
\put(533.84,292){\rule{0.800pt}{0.482pt}}
\multiput(533.34,292.00)(1.000,1.000){2}{\rule{0.800pt}{0.241pt}}
\put(536,293.34){\rule{0.482pt}{0.800pt}}
\multiput(536.00,292.34)(1.000,2.000){2}{\rule{0.241pt}{0.800pt}}
\put(538,295.34){\rule{0.482pt}{0.800pt}}
\multiput(538.00,294.34)(1.000,2.000){2}{\rule{0.241pt}{0.800pt}}
\put(540,296.84){\rule{0.241pt}{0.800pt}}
\multiput(540.00,296.34)(0.500,1.000){2}{\rule{0.120pt}{0.800pt}}
\put(541,298.34){\rule{0.482pt}{0.800pt}}
\multiput(541.00,297.34)(1.000,2.000){2}{\rule{0.241pt}{0.800pt}}
\put(543,300.34){\rule{0.482pt}{0.800pt}}
\multiput(543.00,299.34)(1.000,2.000){2}{\rule{0.241pt}{0.800pt}}
\put(545,301.84){\rule{0.482pt}{0.800pt}}
\multiput(545.00,301.34)(1.000,1.000){2}{\rule{0.241pt}{0.800pt}}
\put(545.84,304){\rule{0.800pt}{0.482pt}}
\multiput(545.34,304.00)(1.000,1.000){2}{\rule{0.800pt}{0.241pt}}
\put(548,305.34){\rule{0.482pt}{0.800pt}}
\multiput(548.00,304.34)(1.000,2.000){2}{\rule{0.241pt}{0.800pt}}
\put(550,307.34){\rule{0.482pt}{0.800pt}}
\multiput(550.00,306.34)(1.000,2.000){2}{\rule{0.241pt}{0.800pt}}
\put(552,308.84){\rule{0.482pt}{0.800pt}}
\multiput(552.00,308.34)(1.000,1.000){2}{\rule{0.241pt}{0.800pt}}
\put(552.84,311){\rule{0.800pt}{0.482pt}}
\multiput(552.34,311.00)(1.000,1.000){2}{\rule{0.800pt}{0.241pt}}
\put(555,312.34){\rule{0.482pt}{0.800pt}}
\multiput(555.00,311.34)(1.000,2.000){2}{\rule{0.241pt}{0.800pt}}
\put(557,314.34){\rule{0.482pt}{0.800pt}}
\multiput(557.00,313.34)(1.000,2.000){2}{\rule{0.241pt}{0.800pt}}
\put(559,315.84){\rule{0.241pt}{0.800pt}}
\multiput(559.00,315.34)(0.500,1.000){2}{\rule{0.120pt}{0.800pt}}
\put(560,317.34){\rule{0.482pt}{0.800pt}}
\multiput(560.00,316.34)(1.000,2.000){2}{\rule{0.241pt}{0.800pt}}
\put(562,319.34){\rule{0.482pt}{0.800pt}}
\multiput(562.00,318.34)(1.000,2.000){2}{\rule{0.241pt}{0.800pt}}
\put(564,321.34){\rule{0.482pt}{0.800pt}}
\multiput(564.00,320.34)(1.000,2.000){2}{\rule{0.241pt}{0.800pt}}
\put(566,322.84){\rule{0.241pt}{0.800pt}}
\multiput(566.00,322.34)(0.500,1.000){2}{\rule{0.120pt}{0.800pt}}
\put(567,324.34){\rule{0.482pt}{0.800pt}}
\multiput(567.00,323.34)(1.000,2.000){2}{\rule{0.241pt}{0.800pt}}
\put(569,326.34){\rule{0.482pt}{0.800pt}}
\multiput(569.00,325.34)(1.000,2.000){2}{\rule{0.241pt}{0.800pt}}
\put(571,327.84){\rule{0.482pt}{0.800pt}}
\multiput(571.00,327.34)(1.000,1.000){2}{\rule{0.241pt}{0.800pt}}
\put(571.84,330){\rule{0.800pt}{0.482pt}}
\multiput(571.34,330.00)(1.000,1.000){2}{\rule{0.800pt}{0.241pt}}
\put(574,331.34){\rule{0.482pt}{0.800pt}}
\multiput(574.00,330.34)(1.000,2.000){2}{\rule{0.241pt}{0.800pt}}
\put(576,333.34){\rule{0.482pt}{0.800pt}}
\multiput(576.00,332.34)(1.000,2.000){2}{\rule{0.241pt}{0.800pt}}
\put(578,334.84){\rule{0.241pt}{0.800pt}}
\multiput(578.00,334.34)(0.500,1.000){2}{\rule{0.120pt}{0.800pt}}
\put(579,336.34){\rule{0.482pt}{0.800pt}}
\multiput(579.00,335.34)(1.000,2.000){2}{\rule{0.241pt}{0.800pt}}
\put(581,338.34){\rule{0.482pt}{0.800pt}}
\multiput(581.00,337.34)(1.000,2.000){2}{\rule{0.241pt}{0.800pt}}
\put(583,340.34){\rule{0.482pt}{0.800pt}}
\multiput(583.00,339.34)(1.000,2.000){2}{\rule{0.241pt}{0.800pt}}
\put(585,341.84){\rule{0.241pt}{0.800pt}}
\multiput(585.00,341.34)(0.500,1.000){2}{\rule{0.120pt}{0.800pt}}
\put(586,343.34){\rule{0.482pt}{0.800pt}}
\multiput(586.00,342.34)(1.000,2.000){2}{\rule{0.241pt}{0.800pt}}
\put(588,345.34){\rule{0.482pt}{0.800pt}}
\multiput(588.00,344.34)(1.000,2.000){2}{\rule{0.241pt}{0.800pt}}
\put(588.84,348){\rule{0.800pt}{0.482pt}}
\multiput(588.34,348.00)(1.000,1.000){2}{\rule{0.800pt}{0.241pt}}
\put(591,348.84){\rule{0.482pt}{0.800pt}}
\multiput(591.00,348.34)(1.000,1.000){2}{\rule{0.241pt}{0.800pt}}
\put(593,350.34){\rule{0.482pt}{0.800pt}}
\multiput(593.00,349.34)(1.000,2.000){2}{\rule{0.241pt}{0.800pt}}
\put(595,352.34){\rule{0.482pt}{0.800pt}}
\multiput(595.00,351.34)(1.000,2.000){2}{\rule{0.241pt}{0.800pt}}
\put(595.84,355){\rule{0.800pt}{0.482pt}}
\multiput(595.34,355.00)(1.000,1.000){2}{\rule{0.800pt}{0.241pt}}
\put(598,355.84){\rule{0.482pt}{0.800pt}}
\multiput(598.00,355.34)(1.000,1.000){2}{\rule{0.241pt}{0.800pt}}
\put(600,357.34){\rule{0.482pt}{0.800pt}}
\multiput(600.00,356.34)(1.000,2.000){2}{\rule{0.241pt}{0.800pt}}
\put(602,359.34){\rule{0.482pt}{0.800pt}}
\multiput(602.00,358.34)(1.000,2.000){2}{\rule{0.241pt}{0.800pt}}
\put(604,360.84){\rule{0.241pt}{0.800pt}}
\multiput(604.00,360.34)(0.500,1.000){2}{\rule{0.120pt}{0.800pt}}
\put(605,362.34){\rule{0.482pt}{0.800pt}}
\multiput(605.00,361.34)(1.000,2.000){2}{\rule{0.241pt}{0.800pt}}
\put(607,364.34){\rule{0.482pt}{0.800pt}}
\multiput(607.00,363.34)(1.000,2.000){2}{\rule{0.241pt}{0.800pt}}
\put(607.84,367){\rule{0.800pt}{0.482pt}}
\multiput(607.34,367.00)(1.000,1.000){2}{\rule{0.800pt}{0.241pt}}
\put(610,367.84){\rule{0.482pt}{0.800pt}}
\multiput(610.00,367.34)(1.000,1.000){2}{\rule{0.241pt}{0.800pt}}
\put(612,369.34){\rule{0.482pt}{0.800pt}}
\multiput(612.00,368.34)(1.000,2.000){2}{\rule{0.241pt}{0.800pt}}
\put(614,371.34){\rule{0.482pt}{0.800pt}}
\multiput(614.00,370.34)(1.000,2.000){2}{\rule{0.241pt}{0.800pt}}
\put(614.84,374){\rule{0.800pt}{0.482pt}}
\multiput(614.34,374.00)(1.000,1.000){2}{\rule{0.800pt}{0.241pt}}
\put(617,374.84){\rule{0.482pt}{0.800pt}}
\multiput(617.00,374.34)(1.000,1.000){2}{\rule{0.241pt}{0.800pt}}
\put(619,376.34){\rule{0.482pt}{0.800pt}}
\multiput(619.00,375.34)(1.000,2.000){2}{\rule{0.241pt}{0.800pt}}
\put(619.84,379){\rule{0.800pt}{0.482pt}}
\multiput(619.34,379.00)(1.000,1.000){2}{\rule{0.800pt}{0.241pt}}
\put(622,380.34){\rule{0.482pt}{0.800pt}}
\multiput(622.00,379.34)(1.000,2.000){2}{\rule{0.241pt}{0.800pt}}
\put(624,381.84){\rule{0.482pt}{0.800pt}}
\multiput(624.00,381.34)(1.000,1.000){2}{\rule{0.241pt}{0.800pt}}
\put(626,383.34){\rule{0.482pt}{0.800pt}}
\multiput(626.00,382.34)(1.000,2.000){2}{\rule{0.241pt}{0.800pt}}
\put(626.84,386){\rule{0.800pt}{0.482pt}}
\multiput(626.34,386.00)(1.000,1.000){2}{\rule{0.800pt}{0.241pt}}
\put(629,387.34){\rule{0.482pt}{0.800pt}}
\multiput(629.00,386.34)(1.000,2.000){2}{\rule{0.241pt}{0.800pt}}
\put(631,388.84){\rule{0.482pt}{0.800pt}}
\multiput(631.00,388.34)(1.000,1.000){2}{\rule{0.241pt}{0.800pt}}
\put(633,390.34){\rule{0.482pt}{0.800pt}}
\multiput(633.00,389.34)(1.000,2.000){2}{\rule{0.241pt}{0.800pt}}
\put(633.84,393){\rule{0.800pt}{0.482pt}}
\multiput(633.34,393.00)(1.000,1.000){2}{\rule{0.800pt}{0.241pt}}
\put(636,393.84){\rule{0.482pt}{0.800pt}}
\multiput(636.00,393.34)(1.000,1.000){2}{\rule{0.241pt}{0.800pt}}
\put(638,395.34){\rule{0.482pt}{0.800pt}}
\multiput(638.00,394.34)(1.000,2.000){2}{\rule{0.241pt}{0.800pt}}
\put(638.84,398){\rule{0.800pt}{0.482pt}}
\multiput(638.34,398.00)(1.000,1.000){2}{\rule{0.800pt}{0.241pt}}
\put(641,399.34){\rule{0.482pt}{0.800pt}}
\multiput(641.00,398.34)(1.000,2.000){2}{\rule{0.241pt}{0.800pt}}
\put(643,400.84){\rule{0.482pt}{0.800pt}}
\multiput(643.00,400.34)(1.000,1.000){2}{\rule{0.241pt}{0.800pt}}
\put(645,402.34){\rule{0.482pt}{0.800pt}}
\multiput(645.00,401.34)(1.000,2.000){2}{\rule{0.241pt}{0.800pt}}
\put(645.84,405){\rule{0.800pt}{0.482pt}}
\multiput(645.34,405.00)(1.000,1.000){2}{\rule{0.800pt}{0.241pt}}
\put(648,406.34){\rule{0.482pt}{0.800pt}}
\multiput(648.00,405.34)(1.000,2.000){2}{\rule{0.241pt}{0.800pt}}
\put(650,407.84){\rule{0.482pt}{0.800pt}}
\multiput(650.00,407.34)(1.000,1.000){2}{\rule{0.241pt}{0.800pt}}
\put(652,409.34){\rule{0.482pt}{0.800pt}}
\multiput(652.00,408.34)(1.000,2.000){2}{\rule{0.241pt}{0.800pt}}
\put(652.84,412){\rule{0.800pt}{0.482pt}}
\multiput(652.34,412.00)(1.000,1.000){2}{\rule{0.800pt}{0.241pt}}
\put(655,413.34){\rule{0.482pt}{0.800pt}}
\multiput(655.00,412.34)(1.000,2.000){2}{\rule{0.241pt}{0.800pt}}
\put(657,414.84){\rule{0.482pt}{0.800pt}}
\multiput(657.00,414.34)(1.000,1.000){2}{\rule{0.241pt}{0.800pt}}
\put(657.84,417){\rule{0.800pt}{0.482pt}}
\multiput(657.34,417.00)(1.000,1.000){2}{\rule{0.800pt}{0.241pt}}
\put(660,418.34){\rule{0.482pt}{0.800pt}}
\multiput(660.00,417.34)(1.000,2.000){2}{\rule{0.241pt}{0.800pt}}
\put(662,419.84){\rule{0.482pt}{0.800pt}}
\multiput(662.00,419.34)(1.000,1.000){2}{\rule{0.241pt}{0.800pt}}
\put(664,421.34){\rule{0.482pt}{0.800pt}}
\multiput(664.00,420.34)(1.000,2.000){2}{\rule{0.241pt}{0.800pt}}
\put(664.84,424){\rule{0.800pt}{0.482pt}}
\multiput(664.34,424.00)(1.000,1.000){2}{\rule{0.800pt}{0.241pt}}
\put(667,425.34){\rule{0.482pt}{0.800pt}}
\multiput(667.00,424.34)(1.000,2.000){2}{\rule{0.241pt}{0.800pt}}
\put(669,426.84){\rule{0.482pt}{0.800pt}}
\multiput(669.00,426.34)(1.000,1.000){2}{\rule{0.241pt}{0.800pt}}
\put(669.84,429){\rule{0.800pt}{0.482pt}}
\multiput(669.34,429.00)(1.000,1.000){2}{\rule{0.800pt}{0.241pt}}
\put(672,430.34){\rule{0.482pt}{0.800pt}}
\multiput(672.00,429.34)(1.000,2.000){2}{\rule{0.241pt}{0.800pt}}
\put(674,432.34){\rule{0.482pt}{0.800pt}}
\multiput(674.00,431.34)(1.000,2.000){2}{\rule{0.241pt}{0.800pt}}
\put(676,433.84){\rule{0.482pt}{0.800pt}}
\multiput(676.00,433.34)(1.000,1.000){2}{\rule{0.241pt}{0.800pt}}
\put(676.84,436){\rule{0.800pt}{0.482pt}}
\multiput(676.34,436.00)(1.000,1.000){2}{\rule{0.800pt}{0.241pt}}
\put(679,437.34){\rule{0.482pt}{0.800pt}}
\multiput(679.00,436.34)(1.000,2.000){2}{\rule{0.241pt}{0.800pt}}
\put(681,439.34){\rule{0.482pt}{0.800pt}}
\multiput(681.00,438.34)(1.000,2.000){2}{\rule{0.241pt}{0.800pt}}
\put(683,440.84){\rule{0.482pt}{0.800pt}}
\multiput(683.00,440.34)(1.000,1.000){2}{\rule{0.241pt}{0.800pt}}
\put(683.84,443){\rule{0.800pt}{0.482pt}}
\multiput(683.34,443.00)(1.000,1.000){2}{\rule{0.800pt}{0.241pt}}
\put(686,444.34){\rule{0.482pt}{0.800pt}}
\multiput(686.00,443.34)(1.000,2.000){2}{\rule{0.241pt}{0.800pt}}
\put(688,446.34){\rule{0.482pt}{0.800pt}}
\multiput(688.00,445.34)(1.000,2.000){2}{\rule{0.241pt}{0.800pt}}
\put(690,447.84){\rule{0.241pt}{0.800pt}}
\multiput(690.00,447.34)(0.500,1.000){2}{\rule{0.120pt}{0.800pt}}
\put(691,449.34){\rule{0.482pt}{0.800pt}}
\multiput(691.00,448.34)(1.000,2.000){2}{\rule{0.241pt}{0.800pt}}
\put(693,451.34){\rule{0.482pt}{0.800pt}}
\multiput(693.00,450.34)(1.000,2.000){2}{\rule{0.241pt}{0.800pt}}
\put(695,452.84){\rule{0.482pt}{0.800pt}}
\multiput(695.00,452.34)(1.000,1.000){2}{\rule{0.241pt}{0.800pt}}
\put(695.84,455){\rule{0.800pt}{0.482pt}}
\multiput(695.34,455.00)(1.000,1.000){2}{\rule{0.800pt}{0.241pt}}
\put(698,456.34){\rule{0.482pt}{0.800pt}}
\multiput(698.00,455.34)(1.000,2.000){2}{\rule{0.241pt}{0.800pt}}
\put(700,458.34){\rule{0.482pt}{0.800pt}}
\multiput(700.00,457.34)(1.000,2.000){2}{\rule{0.241pt}{0.800pt}}
\put(702,459.84){\rule{0.482pt}{0.800pt}}
\multiput(702.00,459.34)(1.000,1.000){2}{\rule{0.241pt}{0.800pt}}
\put(702.84,462){\rule{0.800pt}{0.482pt}}
\multiput(702.34,462.00)(1.000,1.000){2}{\rule{0.800pt}{0.241pt}}
\put(705,463.34){\rule{0.482pt}{0.800pt}}
\multiput(705.00,462.34)(1.000,2.000){2}{\rule{0.241pt}{0.800pt}}
\put(707,465.34){\rule{0.482pt}{0.800pt}}
\multiput(707.00,464.34)(1.000,2.000){2}{\rule{0.241pt}{0.800pt}}
\put(709,466.84){\rule{0.241pt}{0.800pt}}
\multiput(709.00,466.34)(0.500,1.000){2}{\rule{0.120pt}{0.800pt}}
\put(710,468.34){\rule{0.482pt}{0.800pt}}
\multiput(710.00,467.34)(1.000,2.000){2}{\rule{0.241pt}{0.800pt}}
\put(712,470.34){\rule{0.482pt}{0.800pt}}
\multiput(712.00,469.34)(1.000,2.000){2}{\rule{0.241pt}{0.800pt}}
\put(714,472.34){\rule{0.482pt}{0.800pt}}
\multiput(714.00,471.34)(1.000,2.000){2}{\rule{0.241pt}{0.800pt}}
\put(716,473.84){\rule{0.241pt}{0.800pt}}
\multiput(716.00,473.34)(0.500,1.000){2}{\rule{0.120pt}{0.800pt}}
\put(717,475.34){\rule{0.482pt}{0.800pt}}
\multiput(717.00,474.34)(1.000,2.000){2}{\rule{0.241pt}{0.800pt}}
\put(719,477.34){\rule{0.482pt}{0.800pt}}
\multiput(719.00,476.34)(1.000,2.000){2}{\rule{0.241pt}{0.800pt}}
\put(719.84,480){\rule{0.800pt}{0.482pt}}
\multiput(719.34,480.00)(1.000,1.000){2}{\rule{0.800pt}{0.241pt}}
\put(722,480.84){\rule{0.482pt}{0.800pt}}
\multiput(722.00,480.34)(1.000,1.000){2}{\rule{0.241pt}{0.800pt}}
\put(724,482.34){\rule{0.482pt}{0.800pt}}
\multiput(724.00,481.34)(1.000,2.000){2}{\rule{0.241pt}{0.800pt}}
\put(726,484.34){\rule{0.482pt}{0.800pt}}
\multiput(726.00,483.34)(1.000,2.000){2}{\rule{0.241pt}{0.800pt}}
\put(728,485.84){\rule{0.241pt}{0.800pt}}
\multiput(728.00,485.34)(0.500,1.000){2}{\rule{0.120pt}{0.800pt}}
\put(729,487.34){\rule{0.482pt}{0.800pt}}
\multiput(729.00,486.34)(1.000,2.000){2}{\rule{0.241pt}{0.800pt}}
\put(731,489.34){\rule{0.482pt}{0.800pt}}
\multiput(731.00,488.34)(1.000,2.000){2}{\rule{0.241pt}{0.800pt}}
\put(733,491.34){\rule{0.482pt}{0.800pt}}
\multiput(733.00,490.34)(1.000,2.000){2}{\rule{0.241pt}{0.800pt}}
\put(735,492.84){\rule{0.241pt}{0.800pt}}
\multiput(735.00,492.34)(0.500,1.000){2}{\rule{0.120pt}{0.800pt}}
\sbox{\plotpoint}{\rule[-0.200pt]{0.400pt}{0.400pt}}%
\put(221.0,143.0){\rule[-0.200pt]{124.545pt}{0.400pt}}
\put(738.0,143.0){\rule[-0.200pt]{0.400pt}{104.551pt}}
\put(221.0,577.0){\rule[-0.200pt]{124.545pt}{0.400pt}}
\put(221.0,143.0){\rule[-0.200pt]{0.400pt}{104.551pt}}
\end{picture}

	\end{center}
	\caption{Output of algorithm \adwintwo with abrupt change} 
\label{fig:ADWIN2-E}
\end{figure}

\begin{figure}[h]
	\begin{center}
		% GNUPLOT: LaTeX picture
\setlength{\unitlength}{0.240900pt}
\ifx\plotpoint\undefined\newsavebox{\plotpoint}\fi
\begin{picture}(959,720)(0,0)
\sbox{\plotpoint}{\rule[-0.200pt]{0.400pt}{0.400pt}}%
\put(181,143){\makebox(0,0)[r]{$0.1$}}
\put(201.0,143.0){\rule[-0.200pt]{4.818pt}{0.400pt}}
\put(181,197){\makebox(0,0)[r]{$0.2$}}
\put(201.0,197.0){\rule[-0.200pt]{4.818pt}{0.400pt}}
\put(181,252){\makebox(0,0)[r]{$0.3$}}
\put(201.0,252.0){\rule[-0.200pt]{4.818pt}{0.400pt}}
\put(181,306){\makebox(0,0)[r]{$0.4$}}
\put(201.0,306.0){\rule[-0.200pt]{4.818pt}{0.400pt}}
\put(181,360){\makebox(0,0)[r]{$0.5$}}
\put(201.0,360.0){\rule[-0.200pt]{4.818pt}{0.400pt}}
\put(181,414){\makebox(0,0)[r]{$0.6$}}
\put(201.0,414.0){\rule[-0.200pt]{4.818pt}{0.400pt}}
\put(181,468){\makebox(0,0)[r]{$0.7$}}
\put(201.0,468.0){\rule[-0.200pt]{4.818pt}{0.400pt}}
\put(181,523){\makebox(0,0)[r]{$0.8$}}
\put(201.0,523.0){\rule[-0.200pt]{4.818pt}{0.400pt}}
\put(181,577){\makebox(0,0)[r]{$0.9$}}
\put(201.0,577.0){\rule[-0.200pt]{4.818pt}{0.400pt}}
\put(221.0,123.0){\rule[-0.200pt]{0.400pt}{4.818pt}}
\put(221,82){\makebox(0,0){$0$}}
\put(221.0,577.0){\rule[-0.200pt]{0.400pt}{4.818pt}}
\put(286.0,123.0){\rule[-0.200pt]{0.400pt}{4.818pt}}
%\put(286,82){\makebox(0,0){$500$}}
\put(286.0,577.0){\rule[-0.200pt]{0.400pt}{4.818pt}}
\put(350.0,123.0){\rule[-0.200pt]{0.400pt}{4.818pt}}
\put(350,82){\makebox(0,0){$1000$}}
\put(350.0,577.0){\rule[-0.200pt]{0.400pt}{4.818pt}}
\put(415.0,123.0){\rule[-0.200pt]{0.400pt}{4.818pt}}
%\put(415,82){\makebox(0,0){$1500$}}
\put(415.0,577.0){\rule[-0.200pt]{0.400pt}{4.818pt}}
\put(479.0,123.0){\rule[-0.200pt]{0.400pt}{4.818pt}}
\put(479,82){\makebox(0,0){$2000$}}
\put(479.0,577.0){\rule[-0.200pt]{0.400pt}{4.818pt}}
\put(544.0,123.0){\rule[-0.200pt]{0.400pt}{4.818pt}}
%\put(544,82){\makebox(0,0){$2500$}}
\put(544.0,577.0){\rule[-0.200pt]{0.400pt}{4.818pt}}
\put(609.0,123.0){\rule[-0.200pt]{0.400pt}{4.818pt}}
\put(609,82){\makebox(0,0){$3000$}}
\put(609.0,577.0){\rule[-0.200pt]{0.400pt}{4.818pt}}
\put(673.0,123.0){\rule[-0.200pt]{0.400pt}{4.818pt}}
%\put(673,82){\makebox(0,0){$3500$}}
\put(673.0,577.0){\rule[-0.200pt]{0.400pt}{4.818pt}}
\put(738.0,123.0){\rule[-0.200pt]{0.400pt}{4.818pt}}
\put(738,82){\makebox(0,0){$4000$}}
\put(738.0,577.0){\rule[-0.200pt]{0.400pt}{4.818pt}}
\put(778,143){\makebox(0,0)[l]{ 0}}
\put(738.0,143.0){\rule[-0.200pt]{4.818pt}{0.400pt}}
\put(778,230){\makebox(0,0)[l]{ 500}}
\put(738.0,230.0){\rule[-0.200pt]{4.818pt}{0.400pt}}
\put(778,317){\makebox(0,0)[l]{ 1000}}
\put(738.0,317.0){\rule[-0.200pt]{4.818pt}{0.400pt}}
\put(778,403){\makebox(0,0)[l]{ 1500}}
\put(738.0,403.0){\rule[-0.200pt]{4.818pt}{0.400pt}}
\put(778,490){\makebox(0,0)[l]{ 2000}}
\put(738.0,490.0){\rule[-0.200pt]{4.818pt}{0.400pt}}
\put(778,577){\makebox(0,0)[l]{ 2500}}
\put(738.0,577.0){\rule[-0.200pt]{4.818pt}{0.400pt}}
\put(221.0,143.0){\rule[-0.200pt]{124.545pt}{0.400pt}}
\put(738.0,143.0){\rule[-0.200pt]{0.400pt}{104.551pt}}
\put(221.0,577.0){\rule[-0.200pt]{124.545pt}{0.400pt}}
\put(221.0,143.0){\rule[-0.200pt]{0.400pt}{104.551pt}}
\put(40,360){\makebox(0,0){$\mu$ axis}}
\put(967,360){\makebox(0,0){Width}}
\put(479,21){\makebox(0,0){$t$ axis}}
\put(479,659){\makebox(0,0){\adwintwo}}
\put(578,537){\makebox(0,0)[r]{$\mu_t$}}
\put(598.0,537.0){\rule[-0.200pt]{24.090pt}{0.400pt}}
\put(222,523){\usebox{\plotpoint}}
\put(349.67,520){\rule{0.400pt}{0.723pt}}
\multiput(349.17,521.50)(1.000,-1.500){2}{\rule{0.400pt}{0.361pt}}
\put(351.17,516){\rule{0.400pt}{0.900pt}}
\multiput(350.17,518.13)(2.000,-2.132){2}{\rule{0.400pt}{0.450pt}}
\put(352.67,513){\rule{0.400pt}{0.723pt}}
\multiput(352.17,514.50)(1.000,-1.500){2}{\rule{0.400pt}{0.361pt}}
\put(353.67,510){\rule{0.400pt}{0.723pt}}
\multiput(353.17,511.50)(1.000,-1.500){2}{\rule{0.400pt}{0.361pt}}
\put(355.17,507){\rule{0.400pt}{0.700pt}}
\multiput(354.17,508.55)(2.000,-1.547){2}{\rule{0.400pt}{0.350pt}}
\put(356.67,504){\rule{0.400pt}{0.723pt}}
\multiput(356.17,505.50)(1.000,-1.500){2}{\rule{0.400pt}{0.361pt}}
\put(357.67,500){\rule{0.400pt}{0.964pt}}
\multiput(357.17,502.00)(1.000,-2.000){2}{\rule{0.400pt}{0.482pt}}
\put(358.67,497){\rule{0.400pt}{0.723pt}}
\multiput(358.17,498.50)(1.000,-1.500){2}{\rule{0.400pt}{0.361pt}}
\put(360.17,494){\rule{0.400pt}{0.700pt}}
\multiput(359.17,495.55)(2.000,-1.547){2}{\rule{0.400pt}{0.350pt}}
\put(361.67,491){\rule{0.400pt}{0.723pt}}
\multiput(361.17,492.50)(1.000,-1.500){2}{\rule{0.400pt}{0.361pt}}
\put(362.67,488){\rule{0.400pt}{0.723pt}}
\multiput(362.17,489.50)(1.000,-1.500){2}{\rule{0.400pt}{0.361pt}}
\put(364.17,485){\rule{0.400pt}{0.700pt}}
\multiput(363.17,486.55)(2.000,-1.547){2}{\rule{0.400pt}{0.350pt}}
\put(365.67,481){\rule{0.400pt}{0.964pt}}
\multiput(365.17,483.00)(1.000,-2.000){2}{\rule{0.400pt}{0.482pt}}
\put(366.67,478){\rule{0.400pt}{0.723pt}}
\multiput(366.17,479.50)(1.000,-1.500){2}{\rule{0.400pt}{0.361pt}}
\put(368.17,475){\rule{0.400pt}{0.700pt}}
\multiput(367.17,476.55)(2.000,-1.547){2}{\rule{0.400pt}{0.350pt}}
\put(369.67,472){\rule{0.400pt}{0.723pt}}
\multiput(369.17,473.50)(1.000,-1.500){2}{\rule{0.400pt}{0.361pt}}
\put(370.67,469){\rule{0.400pt}{0.723pt}}
\multiput(370.17,470.50)(1.000,-1.500){2}{\rule{0.400pt}{0.361pt}}
\put(371.67,466){\rule{0.400pt}{0.723pt}}
\multiput(371.17,467.50)(1.000,-1.500){2}{\rule{0.400pt}{0.361pt}}
\put(373.17,462){\rule{0.400pt}{0.900pt}}
\multiput(372.17,464.13)(2.000,-2.132){2}{\rule{0.400pt}{0.450pt}}
\put(374.67,459){\rule{0.400pt}{0.723pt}}
\multiput(374.17,460.50)(1.000,-1.500){2}{\rule{0.400pt}{0.361pt}}
\put(375.67,456){\rule{0.400pt}{0.723pt}}
\multiput(375.17,457.50)(1.000,-1.500){2}{\rule{0.400pt}{0.361pt}}
\put(377.17,453){\rule{0.400pt}{0.700pt}}
\multiput(376.17,454.55)(2.000,-1.547){2}{\rule{0.400pt}{0.350pt}}
\put(378.67,450){\rule{0.400pt}{0.723pt}}
\multiput(378.17,451.50)(1.000,-1.500){2}{\rule{0.400pt}{0.361pt}}
\put(379.67,446){\rule{0.400pt}{0.964pt}}
\multiput(379.17,448.00)(1.000,-2.000){2}{\rule{0.400pt}{0.482pt}}
\put(380.67,443){\rule{0.400pt}{0.723pt}}
\multiput(380.17,444.50)(1.000,-1.500){2}{\rule{0.400pt}{0.361pt}}
\put(382.17,440){\rule{0.400pt}{0.700pt}}
\multiput(381.17,441.55)(2.000,-1.547){2}{\rule{0.400pt}{0.350pt}}
\put(383.67,437){\rule{0.400pt}{0.723pt}}
\multiput(383.17,438.50)(1.000,-1.500){2}{\rule{0.400pt}{0.361pt}}
\put(384.67,434){\rule{0.400pt}{0.723pt}}
\multiput(384.17,435.50)(1.000,-1.500){2}{\rule{0.400pt}{0.361pt}}
\put(386.17,431){\rule{0.400pt}{0.700pt}}
\multiput(385.17,432.55)(2.000,-1.547){2}{\rule{0.400pt}{0.350pt}}
\put(387.67,427){\rule{0.400pt}{0.964pt}}
\multiput(387.17,429.00)(1.000,-2.000){2}{\rule{0.400pt}{0.482pt}}
\put(388.67,424){\rule{0.400pt}{0.723pt}}
\multiput(388.17,425.50)(1.000,-1.500){2}{\rule{0.400pt}{0.361pt}}
\put(390.17,421){\rule{0.400pt}{0.700pt}}
\multiput(389.17,422.55)(2.000,-1.547){2}{\rule{0.400pt}{0.350pt}}
\put(391.67,418){\rule{0.400pt}{0.723pt}}
\multiput(391.17,419.50)(1.000,-1.500){2}{\rule{0.400pt}{0.361pt}}
\put(392.67,415){\rule{0.400pt}{0.723pt}}
\multiput(392.17,416.50)(1.000,-1.500){2}{\rule{0.400pt}{0.361pt}}
\put(393.67,411){\rule{0.400pt}{0.964pt}}
\multiput(393.17,413.00)(1.000,-2.000){2}{\rule{0.400pt}{0.482pt}}
\put(395.17,408){\rule{0.400pt}{0.700pt}}
\multiput(394.17,409.55)(2.000,-1.547){2}{\rule{0.400pt}{0.350pt}}
\put(396.67,405){\rule{0.400pt}{0.723pt}}
\multiput(396.17,406.50)(1.000,-1.500){2}{\rule{0.400pt}{0.361pt}}
\put(397.67,402){\rule{0.400pt}{0.723pt}}
\multiput(397.17,403.50)(1.000,-1.500){2}{\rule{0.400pt}{0.361pt}}
\put(399.17,399){\rule{0.400pt}{0.700pt}}
\multiput(398.17,400.55)(2.000,-1.547){2}{\rule{0.400pt}{0.350pt}}
\put(400.67,396){\rule{0.400pt}{0.723pt}}
\multiput(400.17,397.50)(1.000,-1.500){2}{\rule{0.400pt}{0.361pt}}
\put(401.67,392){\rule{0.400pt}{0.964pt}}
\multiput(401.17,394.00)(1.000,-2.000){2}{\rule{0.400pt}{0.482pt}}
\put(402.67,389){\rule{0.400pt}{0.723pt}}
\multiput(402.17,390.50)(1.000,-1.500){2}{\rule{0.400pt}{0.361pt}}
\put(404.17,386){\rule{0.400pt}{0.700pt}}
\multiput(403.17,387.55)(2.000,-1.547){2}{\rule{0.400pt}{0.350pt}}
\put(405.67,383){\rule{0.400pt}{0.723pt}}
\multiput(405.17,384.50)(1.000,-1.500){2}{\rule{0.400pt}{0.361pt}}
\put(406.67,380){\rule{0.400pt}{0.723pt}}
\multiput(406.17,381.50)(1.000,-1.500){2}{\rule{0.400pt}{0.361pt}}
\put(408.17,377){\rule{0.400pt}{0.700pt}}
\multiput(407.17,378.55)(2.000,-1.547){2}{\rule{0.400pt}{0.350pt}}
\put(409.67,373){\rule{0.400pt}{0.964pt}}
\multiput(409.17,375.00)(1.000,-2.000){2}{\rule{0.400pt}{0.482pt}}
\put(410.67,370){\rule{0.400pt}{0.723pt}}
\multiput(410.17,371.50)(1.000,-1.500){2}{\rule{0.400pt}{0.361pt}}
\put(412.17,367){\rule{0.400pt}{0.700pt}}
\multiput(411.17,368.55)(2.000,-1.547){2}{\rule{0.400pt}{0.350pt}}
\put(413.67,364){\rule{0.400pt}{0.723pt}}
\multiput(413.17,365.50)(1.000,-1.500){2}{\rule{0.400pt}{0.361pt}}
\put(414.67,361){\rule{0.400pt}{0.723pt}}
\multiput(414.17,362.50)(1.000,-1.500){2}{\rule{0.400pt}{0.361pt}}
\put(415.67,357){\rule{0.400pt}{0.964pt}}
\multiput(415.17,359.00)(1.000,-2.000){2}{\rule{0.400pt}{0.482pt}}
\put(417.17,354){\rule{0.400pt}{0.700pt}}
\multiput(416.17,355.55)(2.000,-1.547){2}{\rule{0.400pt}{0.350pt}}
\put(418.67,351){\rule{0.400pt}{0.723pt}}
\multiput(418.17,352.50)(1.000,-1.500){2}{\rule{0.400pt}{0.361pt}}
\put(419.67,348){\rule{0.400pt}{0.723pt}}
\multiput(419.17,349.50)(1.000,-1.500){2}{\rule{0.400pt}{0.361pt}}
\put(421.17,345){\rule{0.400pt}{0.700pt}}
\multiput(420.17,346.55)(2.000,-1.547){2}{\rule{0.400pt}{0.350pt}}
\put(422.67,342){\rule{0.400pt}{0.723pt}}
\multiput(422.17,343.50)(1.000,-1.500){2}{\rule{0.400pt}{0.361pt}}
\put(423.67,338){\rule{0.400pt}{0.964pt}}
\multiput(423.17,340.00)(1.000,-2.000){2}{\rule{0.400pt}{0.482pt}}
\put(424.67,335){\rule{0.400pt}{0.723pt}}
\multiput(424.17,336.50)(1.000,-1.500){2}{\rule{0.400pt}{0.361pt}}
\put(426.17,332){\rule{0.400pt}{0.700pt}}
\multiput(425.17,333.55)(2.000,-1.547){2}{\rule{0.400pt}{0.350pt}}
\put(427.67,329){\rule{0.400pt}{0.723pt}}
\multiput(427.17,330.50)(1.000,-1.500){2}{\rule{0.400pt}{0.361pt}}
\put(428.67,326){\rule{0.400pt}{0.723pt}}
\multiput(428.17,327.50)(1.000,-1.500){2}{\rule{0.400pt}{0.361pt}}
\put(430.17,322){\rule{0.400pt}{0.900pt}}
\multiput(429.17,324.13)(2.000,-2.132){2}{\rule{0.400pt}{0.450pt}}
\put(431.67,319){\rule{0.400pt}{0.723pt}}
\multiput(431.17,320.50)(1.000,-1.500){2}{\rule{0.400pt}{0.361pt}}
\put(432.67,316){\rule{0.400pt}{0.723pt}}
\multiput(432.17,317.50)(1.000,-1.500){2}{\rule{0.400pt}{0.361pt}}
\put(433.67,313){\rule{0.400pt}{0.723pt}}
\multiput(433.17,314.50)(1.000,-1.500){2}{\rule{0.400pt}{0.361pt}}
\put(435.17,310){\rule{0.400pt}{0.700pt}}
\multiput(434.17,311.55)(2.000,-1.547){2}{\rule{0.400pt}{0.350pt}}
\put(436.67,307){\rule{0.400pt}{0.723pt}}
\multiput(436.17,308.50)(1.000,-1.500){2}{\rule{0.400pt}{0.361pt}}
\put(437.67,303){\rule{0.400pt}{0.964pt}}
\multiput(437.17,305.00)(1.000,-2.000){2}{\rule{0.400pt}{0.482pt}}
\put(439.17,300){\rule{0.400pt}{0.700pt}}
\multiput(438.17,301.55)(2.000,-1.547){2}{\rule{0.400pt}{0.350pt}}
\put(440.67,297){\rule{0.400pt}{0.723pt}}
\multiput(440.17,298.50)(1.000,-1.500){2}{\rule{0.400pt}{0.361pt}}
\put(441.67,294){\rule{0.400pt}{0.723pt}}
\multiput(441.17,295.50)(1.000,-1.500){2}{\rule{0.400pt}{0.361pt}}
\put(443.17,291){\rule{0.400pt}{0.700pt}}
\multiput(442.17,292.55)(2.000,-1.547){2}{\rule{0.400pt}{0.350pt}}
\put(444.67,288){\rule{0.400pt}{0.723pt}}
\multiput(444.17,289.50)(1.000,-1.500){2}{\rule{0.400pt}{0.361pt}}
\put(445.67,284){\rule{0.400pt}{0.964pt}}
\multiput(445.17,286.00)(1.000,-2.000){2}{\rule{0.400pt}{0.482pt}}
\put(446.67,281){\rule{0.400pt}{0.723pt}}
\multiput(446.17,282.50)(1.000,-1.500){2}{\rule{0.400pt}{0.361pt}}
\put(448.17,278){\rule{0.400pt}{0.700pt}}
\multiput(447.17,279.55)(2.000,-1.547){2}{\rule{0.400pt}{0.350pt}}
\put(449.67,275){\rule{0.400pt}{0.723pt}}
\multiput(449.17,276.50)(1.000,-1.500){2}{\rule{0.400pt}{0.361pt}}
\put(450.67,272){\rule{0.400pt}{0.723pt}}
\multiput(450.17,273.50)(1.000,-1.500){2}{\rule{0.400pt}{0.361pt}}
\put(452.17,268){\rule{0.400pt}{0.900pt}}
\multiput(451.17,270.13)(2.000,-2.132){2}{\rule{0.400pt}{0.450pt}}
\put(453.67,265){\rule{0.400pt}{0.723pt}}
\multiput(453.17,266.50)(1.000,-1.500){2}{\rule{0.400pt}{0.361pt}}
\put(454.67,262){\rule{0.400pt}{0.723pt}}
\multiput(454.17,263.50)(1.000,-1.500){2}{\rule{0.400pt}{0.361pt}}
\put(455.67,259){\rule{0.400pt}{0.723pt}}
\multiput(455.17,260.50)(1.000,-1.500){2}{\rule{0.400pt}{0.361pt}}
\put(457.17,256){\rule{0.400pt}{0.700pt}}
\multiput(456.17,257.55)(2.000,-1.547){2}{\rule{0.400pt}{0.350pt}}
\put(458.67,253){\rule{0.400pt}{0.723pt}}
\multiput(458.17,254.50)(1.000,-1.500){2}{\rule{0.400pt}{0.361pt}}
\put(459.67,249){\rule{0.400pt}{0.964pt}}
\multiput(459.17,251.00)(1.000,-2.000){2}{\rule{0.400pt}{0.482pt}}
\put(461.17,246){\rule{0.400pt}{0.700pt}}
\multiput(460.17,247.55)(2.000,-1.547){2}{\rule{0.400pt}{0.350pt}}
\put(462.67,243){\rule{0.400pt}{0.723pt}}
\multiput(462.17,244.50)(1.000,-1.500){2}{\rule{0.400pt}{0.361pt}}
\put(463.67,240){\rule{0.400pt}{0.723pt}}
\multiput(463.17,241.50)(1.000,-1.500){2}{\rule{0.400pt}{0.361pt}}
\put(465.17,237){\rule{0.400pt}{0.700pt}}
\multiput(464.17,238.55)(2.000,-1.547){2}{\rule{0.400pt}{0.350pt}}
\put(466.67,233){\rule{0.400pt}{0.964pt}}
\multiput(466.17,235.00)(1.000,-2.000){2}{\rule{0.400pt}{0.482pt}}
\put(467.67,230){\rule{0.400pt}{0.723pt}}
\multiput(467.17,231.50)(1.000,-1.500){2}{\rule{0.400pt}{0.361pt}}
\put(468.67,227){\rule{0.400pt}{0.723pt}}
\multiput(468.17,228.50)(1.000,-1.500){2}{\rule{0.400pt}{0.361pt}}
\put(470.17,224){\rule{0.400pt}{0.700pt}}
\multiput(469.17,225.55)(2.000,-1.547){2}{\rule{0.400pt}{0.350pt}}
\put(471.67,221){\rule{0.400pt}{0.723pt}}
\multiput(471.17,222.50)(1.000,-1.500){2}{\rule{0.400pt}{0.361pt}}
\put(472.67,218){\rule{0.400pt}{0.723pt}}
\multiput(472.17,219.50)(1.000,-1.500){2}{\rule{0.400pt}{0.361pt}}
\put(474.17,214){\rule{0.400pt}{0.900pt}}
\multiput(473.17,216.13)(2.000,-2.132){2}{\rule{0.400pt}{0.450pt}}
\put(475.67,211){\rule{0.400pt}{0.723pt}}
\multiput(475.17,212.50)(1.000,-1.500){2}{\rule{0.400pt}{0.361pt}}
\put(476.67,208){\rule{0.400pt}{0.723pt}}
\multiput(476.17,209.50)(1.000,-1.500){2}{\rule{0.400pt}{0.361pt}}
\put(477.67,205){\rule{0.400pt}{0.723pt}}
\multiput(477.17,206.50)(1.000,-1.500){2}{\rule{0.400pt}{0.361pt}}
\put(479.17,202){\rule{0.400pt}{0.700pt}}
\multiput(478.17,203.55)(2.000,-1.547){2}{\rule{0.400pt}{0.350pt}}
\put(480.67,199){\rule{0.400pt}{0.723pt}}
\multiput(480.17,200.50)(1.000,-1.500){2}{\rule{0.400pt}{0.361pt}}
\put(481.67,197){\rule{0.400pt}{0.482pt}}
\multiput(481.17,198.00)(1.000,-1.000){2}{\rule{0.400pt}{0.241pt}}
\put(222.0,523.0){\rule[-0.200pt]{30.835pt}{0.400pt}}
\put(483.0,197.0){\rule[-0.200pt]{61.189pt}{0.400pt}}
\put(578,496){\makebox(0,0)[r]{$\hat{\mu}_W$}}
\multiput(598,496)(20.756,0.000){5}{\usebox{\plotpoint}}
\put(698,496){\usebox{\plotpoint}}
\put(222,577){\usebox{\plotpoint}}
\multiput(222,577)(0.768,-20.741){2}{\usebox{\plotpoint}}
\put(225.58,546.24){\usebox{\plotpoint}}
\put(227.67,533.33){\usebox{\plotpoint}}
\put(231.15,549.31){\usebox{\plotpoint}}
\put(235.97,541.16){\usebox{\plotpoint}}
\put(243.34,543.72){\usebox{\plotpoint}}
\put(248.13,542.62){\usebox{\plotpoint}}
\put(257.25,535.01){\usebox{\plotpoint}}
\put(269.59,526.00){\usebox{\plotpoint}}
\put(284.51,524.00){\usebox{\plotpoint}}
\put(299.64,522.18){\usebox{\plotpoint}}
\put(315.66,528.34){\usebox{\plotpoint}}
\put(334.02,529.49){\usebox{\plotpoint}}
\put(352.12,529.00){\usebox{\plotpoint}}
\put(370.11,528.00){\usebox{\plotpoint}}
\put(376.07,511.41){\usebox{\plotpoint}}
\put(379.79,491.21){\usebox{\plotpoint}}
\put(385.80,472.19){\usebox{\plotpoint}}
\put(390.74,452.62){\usebox{\plotpoint}}
\put(396.42,444.55){\usebox{\plotpoint}}
\put(403.15,438.61){\usebox{\plotpoint}}
\put(412.32,431.52){\usebox{\plotpoint}}
\put(421.44,422.00){\usebox{\plotpoint}}
\put(434.39,408.61){\usebox{\plotpoint}}
\put(447.61,397.17){\usebox{\plotpoint}}
\put(451.69,377.51){\usebox{\plotpoint}}
\put(456.43,359.14){\usebox{\plotpoint}}
\put(459.99,339.01){\usebox{\plotpoint}}
\put(464.64,319.46){\usebox{\plotpoint}}
\put(472.11,308.68){\usebox{\plotpoint}}
\put(477.79,292.63){\usebox{\plotpoint}}
\put(482.12,272.64){\usebox{\plotpoint}}
\put(485.94,252.88){\usebox{\plotpoint}}
\put(486.95,232.15){\usebox{\plotpoint}}
\put(500.20,228.60){\usebox{\plotpoint}}
\put(512.45,222.56){\usebox{\plotpoint}}
\put(525.48,216.03){\usebox{\plotpoint}}
\put(539.25,217.00){\usebox{\plotpoint}}
\put(555.83,213.08){\usebox{\plotpoint}}
\put(573.27,211.73){\usebox{\plotpoint}}
\put(592.48,208.00){\usebox{\plotpoint}}
\put(612.58,206.00){\usebox{\plotpoint}}
\put(630.28,200.36){\usebox{\plotpoint}}
\put(649.63,197.63){\usebox{\plotpoint}}
\put(668.53,197.47){\usebox{\plotpoint}}
\put(687.98,197.01){\usebox{\plotpoint}}
\put(707.30,197.30){\usebox{\plotpoint}}
\put(727.30,196.00){\usebox{\plotpoint}}
\put(737,196){\usebox{\plotpoint}}
\sbox{\plotpoint}{\rule[-0.400pt]{0.800pt}{0.800pt}}%
\sbox{\plotpoint}{\rule[-0.200pt]{0.400pt}{0.400pt}}%
\put(578,455){\makebox(0,0)[r]{$W$}}
\sbox{\plotpoint}{\rule[-0.400pt]{0.800pt}{0.800pt}}%
\put(598.0,455.0){\rule[-0.400pt]{24.090pt}{0.800pt}}
\put(222,145){\usebox{\plotpoint}}
\put(222,143.84){\rule{0.241pt}{0.800pt}}
\multiput(222.00,143.34)(0.500,1.000){2}{\rule{0.120pt}{0.800pt}}
\put(223,145.34){\rule{0.482pt}{0.800pt}}
\multiput(223.00,144.34)(1.000,2.000){2}{\rule{0.241pt}{0.800pt}}
\put(223.84,148){\rule{0.800pt}{0.482pt}}
\multiput(223.34,148.00)(1.000,1.000){2}{\rule{0.800pt}{0.241pt}}
\put(224.84,150){\rule{0.800pt}{0.482pt}}
\multiput(224.34,150.00)(1.000,1.000){2}{\rule{0.800pt}{0.241pt}}
\put(227,150.84){\rule{0.482pt}{0.800pt}}
\multiput(227.00,150.34)(1.000,1.000){2}{\rule{0.241pt}{0.800pt}}
\put(227.84,153){\rule{0.800pt}{0.482pt}}
\multiput(227.34,153.00)(1.000,1.000){2}{\rule{0.800pt}{0.241pt}}
\put(228.84,155){\rule{0.800pt}{0.482pt}}
\multiput(228.34,155.00)(1.000,1.000){2}{\rule{0.800pt}{0.241pt}}
\put(231,156.34){\rule{0.482pt}{0.800pt}}
\multiput(231.00,155.34)(1.000,2.000){2}{\rule{0.241pt}{0.800pt}}
\put(233,157.84){\rule{0.241pt}{0.800pt}}
\multiput(233.00,157.34)(0.500,1.000){2}{\rule{0.120pt}{0.800pt}}
\put(232.84,160){\rule{0.800pt}{0.482pt}}
\multiput(232.34,160.00)(1.000,1.000){2}{\rule{0.800pt}{0.241pt}}
\put(233.84,162){\rule{0.800pt}{0.482pt}}
\multiput(233.34,162.00)(1.000,1.000){2}{\rule{0.800pt}{0.241pt}}
\put(236,163.34){\rule{0.482pt}{0.800pt}}
\multiput(236.00,162.34)(1.000,2.000){2}{\rule{0.241pt}{0.800pt}}
\put(238,164.84){\rule{0.241pt}{0.800pt}}
\multiput(238.00,164.34)(0.500,1.000){2}{\rule{0.120pt}{0.800pt}}
\put(237.84,167){\rule{0.800pt}{0.482pt}}
\multiput(237.34,167.00)(1.000,1.000){2}{\rule{0.800pt}{0.241pt}}
\put(240,168.34){\rule{0.482pt}{0.800pt}}
\multiput(240.00,167.34)(1.000,2.000){2}{\rule{0.241pt}{0.800pt}}
\put(240.84,171){\rule{0.800pt}{0.482pt}}
\multiput(240.34,171.00)(1.000,1.000){2}{\rule{0.800pt}{0.241pt}}
\put(243,171.84){\rule{0.241pt}{0.800pt}}
\multiput(243.00,171.34)(0.500,1.000){2}{\rule{0.120pt}{0.800pt}}
\put(242.84,174){\rule{0.800pt}{0.482pt}}
\multiput(242.34,174.00)(1.000,1.000){2}{\rule{0.800pt}{0.241pt}}
\put(245,175.34){\rule{0.482pt}{0.800pt}}
\multiput(245.00,174.34)(1.000,2.000){2}{\rule{0.241pt}{0.800pt}}
\put(247,176.84){\rule{0.241pt}{0.800pt}}
\multiput(247.00,176.34)(0.500,1.000){2}{\rule{0.120pt}{0.800pt}}
\put(246.84,179){\rule{0.800pt}{0.482pt}}
\multiput(246.34,179.00)(1.000,1.000){2}{\rule{0.800pt}{0.241pt}}
\put(249,180.34){\rule{0.482pt}{0.800pt}}
\multiput(249.00,179.34)(1.000,2.000){2}{\rule{0.241pt}{0.800pt}}
\put(249.84,183){\rule{0.800pt}{0.482pt}}
\multiput(249.34,183.00)(1.000,1.000){2}{\rule{0.800pt}{0.241pt}}
\put(252,183.84){\rule{0.241pt}{0.800pt}}
\multiput(252.00,183.34)(0.500,1.000){2}{\rule{0.120pt}{0.800pt}}
\put(251.84,186){\rule{0.800pt}{0.482pt}}
\multiput(251.34,186.00)(1.000,1.000){2}{\rule{0.800pt}{0.241pt}}
\put(254,187.34){\rule{0.482pt}{0.800pt}}
\multiput(254.00,186.34)(1.000,2.000){2}{\rule{0.241pt}{0.800pt}}
\put(254.84,190){\rule{0.800pt}{0.482pt}}
\multiput(254.34,190.00)(1.000,1.000){2}{\rule{0.800pt}{0.241pt}}
\put(257,190.84){\rule{0.241pt}{0.800pt}}
\multiput(257.00,190.34)(0.500,1.000){2}{\rule{0.120pt}{0.800pt}}
\put(258,192.34){\rule{0.482pt}{0.800pt}}
\multiput(258.00,191.34)(1.000,2.000){2}{\rule{0.241pt}{0.800pt}}
\put(258.84,195){\rule{0.800pt}{0.482pt}}
\multiput(258.34,195.00)(1.000,1.000){2}{\rule{0.800pt}{0.241pt}}
\put(259.84,197){\rule{0.800pt}{0.482pt}}
\multiput(259.34,197.00)(1.000,1.000){2}{\rule{0.800pt}{0.241pt}}
\put(262,197.84){\rule{0.482pt}{0.800pt}}
\multiput(262.00,197.34)(1.000,1.000){2}{\rule{0.241pt}{0.800pt}}
\put(262.84,200){\rule{0.800pt}{0.482pt}}
\multiput(262.34,200.00)(1.000,1.000){2}{\rule{0.800pt}{0.241pt}}
\put(263.84,202){\rule{0.800pt}{0.482pt}}
\multiput(263.34,202.00)(1.000,1.000){2}{\rule{0.800pt}{0.241pt}}
\put(266,202.84){\rule{0.241pt}{0.800pt}}
\multiput(266.00,202.34)(0.500,1.000){2}{\rule{0.120pt}{0.800pt}}
\put(267,204.34){\rule{0.482pt}{0.800pt}}
\multiput(267.00,203.34)(1.000,2.000){2}{\rule{0.241pt}{0.800pt}}
\put(267.84,207){\rule{0.800pt}{0.482pt}}
\multiput(267.34,207.00)(1.000,1.000){2}{\rule{0.800pt}{0.241pt}}
\put(268.84,209){\rule{0.800pt}{0.482pt}}
\multiput(268.34,209.00)(1.000,1.000){2}{\rule{0.800pt}{0.241pt}}
\put(271,209.84){\rule{0.482pt}{0.800pt}}
\multiput(271.00,209.34)(1.000,1.000){2}{\rule{0.241pt}{0.800pt}}
\put(271.84,212){\rule{0.800pt}{0.482pt}}
\multiput(271.34,212.00)(1.000,1.000){2}{\rule{0.800pt}{0.241pt}}
\put(272.84,214){\rule{0.800pt}{0.482pt}}
\multiput(272.34,214.00)(1.000,1.000){2}{\rule{0.800pt}{0.241pt}}
\put(273.84,216){\rule{0.800pt}{0.482pt}}
\multiput(273.34,216.00)(1.000,1.000){2}{\rule{0.800pt}{0.241pt}}
\put(276,216.84){\rule{0.482pt}{0.800pt}}
\multiput(276.00,216.34)(1.000,1.000){2}{\rule{0.241pt}{0.800pt}}
\put(276.84,219){\rule{0.800pt}{0.482pt}}
\multiput(276.34,219.00)(1.000,1.000){2}{\rule{0.800pt}{0.241pt}}
\put(277.84,221){\rule{0.800pt}{0.482pt}}
\multiput(277.34,221.00)(1.000,1.000){2}{\rule{0.800pt}{0.241pt}}
\put(280,222.34){\rule{0.482pt}{0.800pt}}
\multiput(280.00,221.34)(1.000,2.000){2}{\rule{0.241pt}{0.800pt}}
\put(282,223.84){\rule{0.241pt}{0.800pt}}
\multiput(282.00,223.34)(0.500,1.000){2}{\rule{0.120pt}{0.800pt}}
\put(281.84,226){\rule{0.800pt}{0.482pt}}
\multiput(281.34,226.00)(1.000,1.000){2}{\rule{0.800pt}{0.241pt}}
\put(284,227.34){\rule{0.482pt}{0.800pt}}
\multiput(284.00,226.34)(1.000,2.000){2}{\rule{0.241pt}{0.800pt}}
\put(284.84,230){\rule{0.800pt}{0.482pt}}
\multiput(284.34,230.00)(1.000,1.000){2}{\rule{0.800pt}{0.241pt}}
\put(287,230.84){\rule{0.241pt}{0.800pt}}
\multiput(287.00,230.34)(0.500,1.000){2}{\rule{0.120pt}{0.800pt}}
\put(286.84,233){\rule{0.800pt}{0.482pt}}
\multiput(286.34,233.00)(1.000,1.000){2}{\rule{0.800pt}{0.241pt}}
\put(289,234.34){\rule{0.482pt}{0.800pt}}
\multiput(289.00,233.34)(1.000,2.000){2}{\rule{0.241pt}{0.800pt}}
\put(291,235.84){\rule{0.241pt}{0.800pt}}
\multiput(291.00,235.34)(0.500,1.000){2}{\rule{0.120pt}{0.800pt}}
\put(290.84,238){\rule{0.800pt}{0.482pt}}
\multiput(290.34,238.00)(1.000,1.000){2}{\rule{0.800pt}{0.241pt}}
\put(293,239.34){\rule{0.482pt}{0.800pt}}
\multiput(293.00,238.34)(1.000,2.000){2}{\rule{0.241pt}{0.800pt}}
\put(293.84,242){\rule{0.800pt}{0.482pt}}
\multiput(293.34,242.00)(1.000,1.000){2}{\rule{0.800pt}{0.241pt}}
\put(296,242.84){\rule{0.241pt}{0.800pt}}
\multiput(296.00,242.34)(0.500,1.000){2}{\rule{0.120pt}{0.800pt}}
\put(295.84,245){\rule{0.800pt}{0.482pt}}
\multiput(295.34,245.00)(1.000,1.000){2}{\rule{0.800pt}{0.241pt}}
\put(298,246.34){\rule{0.482pt}{0.800pt}}
\multiput(298.00,245.34)(1.000,2.000){2}{\rule{0.241pt}{0.800pt}}
\put(298.84,249){\rule{0.800pt}{0.482pt}}
\multiput(298.34,249.00)(1.000,1.000){2}{\rule{0.800pt}{0.241pt}}
\put(301,249.84){\rule{0.241pt}{0.800pt}}
\multiput(301.00,249.34)(0.500,1.000){2}{\rule{0.120pt}{0.800pt}}
\put(302,251.34){\rule{0.482pt}{0.800pt}}
\multiput(302.00,250.34)(1.000,2.000){2}{\rule{0.241pt}{0.800pt}}
\put(302.84,254){\rule{0.800pt}{0.482pt}}
\multiput(302.34,254.00)(1.000,1.000){2}{\rule{0.800pt}{0.241pt}}
\put(303.84,256){\rule{0.800pt}{0.482pt}}
\multiput(303.34,256.00)(1.000,1.000){2}{\rule{0.800pt}{0.241pt}}
\put(306,256.84){\rule{0.241pt}{0.800pt}}
\multiput(306.00,256.34)(0.500,1.000){2}{\rule{0.120pt}{0.800pt}}
\put(307,258.34){\rule{0.482pt}{0.800pt}}
\multiput(307.00,257.34)(1.000,2.000){2}{\rule{0.241pt}{0.800pt}}
\put(307.84,261){\rule{0.800pt}{0.482pt}}
\multiput(307.34,261.00)(1.000,1.000){2}{\rule{0.800pt}{0.241pt}}
\put(308.84,263){\rule{0.800pt}{0.482pt}}
\multiput(308.34,263.00)(1.000,1.000){2}{\rule{0.800pt}{0.241pt}}
\put(311,263.84){\rule{0.482pt}{0.800pt}}
\multiput(311.00,263.34)(1.000,1.000){2}{\rule{0.241pt}{0.800pt}}
\put(311.84,266){\rule{0.800pt}{0.482pt}}
\multiput(311.34,266.00)(1.000,1.000){2}{\rule{0.800pt}{0.241pt}}
\put(312.84,268){\rule{0.800pt}{0.482pt}}
\multiput(312.34,268.00)(1.000,1.000){2}{\rule{0.800pt}{0.241pt}}
\put(315,268.84){\rule{0.482pt}{0.800pt}}
\multiput(315.00,268.34)(1.000,1.000){2}{\rule{0.241pt}{0.800pt}}
\put(315.84,271){\rule{0.800pt}{0.482pt}}
\multiput(315.34,271.00)(1.000,1.000){2}{\rule{0.800pt}{0.241pt}}
\put(316.84,273){\rule{0.800pt}{0.482pt}}
\multiput(316.34,273.00)(1.000,1.000){2}{\rule{0.800pt}{0.241pt}}
\put(317.84,275){\rule{0.800pt}{0.482pt}}
\multiput(317.34,275.00)(1.000,1.000){2}{\rule{0.800pt}{0.241pt}}
\put(320,275.84){\rule{0.482pt}{0.800pt}}
\multiput(320.00,275.34)(1.000,1.000){2}{\rule{0.241pt}{0.800pt}}
\put(320.84,278){\rule{0.800pt}{0.482pt}}
\multiput(320.34,278.00)(1.000,1.000){2}{\rule{0.800pt}{0.241pt}}
\put(321.84,280){\rule{0.800pt}{0.482pt}}
\multiput(321.34,280.00)(1.000,1.000){2}{\rule{0.800pt}{0.241pt}}
\put(324,281.34){\rule{0.482pt}{0.800pt}}
\multiput(324.00,280.34)(1.000,2.000){2}{\rule{0.241pt}{0.800pt}}
\put(326,282.84){\rule{0.241pt}{0.800pt}}
\multiput(326.00,282.34)(0.500,1.000){2}{\rule{0.120pt}{0.800pt}}
\put(325.84,285){\rule{0.800pt}{0.482pt}}
\multiput(325.34,285.00)(1.000,1.000){2}{\rule{0.800pt}{0.241pt}}
\put(326.84,287){\rule{0.800pt}{0.482pt}}
\multiput(326.34,287.00)(1.000,1.000){2}{\rule{0.800pt}{0.241pt}}
\put(329,288.34){\rule{0.482pt}{0.800pt}}
\multiput(329.00,287.34)(1.000,2.000){2}{\rule{0.241pt}{0.800pt}}
\put(331,289.84){\rule{0.241pt}{0.800pt}}
\multiput(331.00,289.34)(0.500,1.000){2}{\rule{0.120pt}{0.800pt}}
\put(330.84,292){\rule{0.800pt}{0.482pt}}
\multiput(330.34,292.00)(1.000,1.000){2}{\rule{0.800pt}{0.241pt}}
\put(333,293.34){\rule{0.482pt}{0.800pt}}
\multiput(333.00,292.34)(1.000,2.000){2}{\rule{0.241pt}{0.800pt}}
\put(333.84,296){\rule{0.800pt}{0.482pt}}
\multiput(333.34,296.00)(1.000,1.000){2}{\rule{0.800pt}{0.241pt}}
\put(336,296.84){\rule{0.241pt}{0.800pt}}
\multiput(336.00,296.34)(0.500,1.000){2}{\rule{0.120pt}{0.800pt}}
\put(337,298.34){\rule{0.482pt}{0.800pt}}
\multiput(337.00,297.34)(1.000,2.000){2}{\rule{0.241pt}{0.800pt}}
\put(337.84,301){\rule{0.800pt}{0.482pt}}
\multiput(337.34,301.00)(1.000,1.000){2}{\rule{0.800pt}{0.241pt}}
\put(340,301.84){\rule{0.241pt}{0.800pt}}
\multiput(340.00,301.34)(0.500,1.000){2}{\rule{0.120pt}{0.800pt}}
\put(339.84,304){\rule{0.800pt}{0.482pt}}
\multiput(339.34,304.00)(1.000,1.000){2}{\rule{0.800pt}{0.241pt}}
\put(342,305.34){\rule{0.482pt}{0.800pt}}
\multiput(342.00,304.34)(1.000,2.000){2}{\rule{0.241pt}{0.800pt}}
\put(342.84,308){\rule{0.800pt}{0.482pt}}
\multiput(342.34,308.00)(1.000,1.000){2}{\rule{0.800pt}{0.241pt}}
\put(345,308.84){\rule{0.241pt}{0.800pt}}
\multiput(345.00,308.34)(0.500,1.000){2}{\rule{0.120pt}{0.800pt}}
\put(346,310.34){\rule{0.482pt}{0.800pt}}
\multiput(346.00,309.34)(1.000,2.000){2}{\rule{0.241pt}{0.800pt}}
\put(346.84,313){\rule{0.800pt}{0.482pt}}
\multiput(346.34,313.00)(1.000,1.000){2}{\rule{0.800pt}{0.241pt}}
\put(347.84,315){\rule{0.800pt}{0.482pt}}
\multiput(347.34,315.00)(1.000,1.000){2}{\rule{0.800pt}{0.241pt}}
\put(350,315.84){\rule{0.241pt}{0.800pt}}
\multiput(350.00,315.34)(0.500,1.000){2}{\rule{0.120pt}{0.800pt}}
\put(351,317.34){\rule{0.482pt}{0.800pt}}
\multiput(351.00,316.34)(1.000,2.000){2}{\rule{0.241pt}{0.800pt}}
\put(351.84,320){\rule{0.800pt}{0.482pt}}
\multiput(351.34,320.00)(1.000,1.000){2}{\rule{0.800pt}{0.241pt}}
\put(352.84,322){\rule{0.800pt}{0.482pt}}
\multiput(352.34,322.00)(1.000,1.000){2}{\rule{0.800pt}{0.241pt}}
\put(355,322.84){\rule{0.482pt}{0.800pt}}
\multiput(355.00,322.34)(1.000,1.000){2}{\rule{0.241pt}{0.800pt}}
\put(355.84,325){\rule{0.800pt}{0.482pt}}
\multiput(355.34,325.00)(1.000,1.000){2}{\rule{0.800pt}{0.241pt}}
\put(356.84,327){\rule{0.800pt}{0.482pt}}
\multiput(356.34,327.00)(1.000,1.000){2}{\rule{0.800pt}{0.241pt}}
\put(359,327.84){\rule{0.241pt}{0.800pt}}
\multiput(359.00,327.34)(0.500,1.000){2}{\rule{0.120pt}{0.800pt}}
\put(360,329.34){\rule{0.482pt}{0.800pt}}
\multiput(360.00,328.34)(1.000,2.000){2}{\rule{0.241pt}{0.800pt}}
\put(360.84,332){\rule{0.800pt}{0.482pt}}
\multiput(360.34,332.00)(1.000,1.000){2}{\rule{0.800pt}{0.241pt}}
\put(361.84,334){\rule{0.800pt}{0.482pt}}
\multiput(361.34,334.00)(1.000,1.000){2}{\rule{0.800pt}{0.241pt}}
\put(364,334.84){\rule{0.482pt}{0.800pt}}
\multiput(364.00,334.34)(1.000,1.000){2}{\rule{0.241pt}{0.800pt}}
\put(364.84,337){\rule{0.800pt}{0.482pt}}
\multiput(364.34,337.00)(1.000,1.000){2}{\rule{0.800pt}{0.241pt}}
\put(365.84,339){\rule{0.800pt}{0.482pt}}
\multiput(365.34,339.00)(1.000,1.000){2}{\rule{0.800pt}{0.241pt}}
\put(368,340.34){\rule{0.482pt}{0.800pt}}
\multiput(368.00,339.34)(1.000,2.000){2}{\rule{0.241pt}{0.800pt}}
\put(370,341.84){\rule{0.241pt}{0.800pt}}
\multiput(370.00,341.34)(0.500,1.000){2}{\rule{0.120pt}{0.800pt}}
\put(369.84,344){\rule{0.800pt}{0.482pt}}
\multiput(369.34,344.00)(1.000,1.000){2}{\rule{0.800pt}{0.241pt}}
\put(370.84,346){\rule{0.800pt}{0.482pt}}
\multiput(370.34,346.00)(1.000,1.000){2}{\rule{0.800pt}{0.241pt}}
\put(373,347.34){\rule{0.482pt}{0.800pt}}
\multiput(373.00,346.34)(1.000,2.000){2}{\rule{0.241pt}{0.800pt}}
\put(373.84,190){\rule{0.800pt}{38.544pt}}
\multiput(373.34,270.00)(1.000,-80.000){2}{\rule{0.800pt}{19.272pt}}
\put(374.84,190){\rule{0.800pt}{0.482pt}}
\multiput(374.34,190.00)(1.000,1.000){2}{\rule{0.800pt}{0.241pt}}
\put(376.34,171){\rule{0.800pt}{5.059pt}}
\multiput(375.34,181.50)(2.000,-10.500){2}{\rule{0.800pt}{2.529pt}}
\put(377.84,171){\rule{0.800pt}{0.482pt}}
\multiput(377.34,171.00)(1.000,1.000){2}{\rule{0.800pt}{0.241pt}}
\put(378.84,173){\rule{0.800pt}{0.482pt}}
\multiput(378.34,173.00)(1.000,1.000){2}{\rule{0.800pt}{0.241pt}}
\put(379.84,175){\rule{0.800pt}{0.482pt}}
\multiput(379.34,175.00)(1.000,1.000){2}{\rule{0.800pt}{0.241pt}}
\put(382,175.84){\rule{0.482pt}{0.800pt}}
\multiput(382.00,175.34)(1.000,1.000){2}{\rule{0.241pt}{0.800pt}}
\put(382.84,178){\rule{0.800pt}{0.482pt}}
\multiput(382.34,178.00)(1.000,1.000){2}{\rule{0.800pt}{0.241pt}}
\put(383.84,180){\rule{0.800pt}{0.482pt}}
\multiput(383.34,180.00)(1.000,1.000){2}{\rule{0.800pt}{0.241pt}}
\put(386,181.34){\rule{0.482pt}{0.800pt}}
\multiput(386.00,180.34)(1.000,2.000){2}{\rule{0.241pt}{0.800pt}}
\put(388,182.84){\rule{0.241pt}{0.800pt}}
\multiput(388.00,182.34)(0.500,1.000){2}{\rule{0.120pt}{0.800pt}}
\put(387.84,185){\rule{0.800pt}{0.482pt}}
\multiput(387.34,185.00)(1.000,1.000){2}{\rule{0.800pt}{0.241pt}}
\put(389.34,183){\rule{0.800pt}{0.964pt}}
\multiput(388.34,185.00)(2.000,-2.000){2}{\rule{0.800pt}{0.482pt}}
\put(390.84,183){\rule{0.800pt}{0.482pt}}
\multiput(390.34,183.00)(1.000,1.000){2}{\rule{0.800pt}{0.241pt}}
\put(391.84,185){\rule{0.800pt}{0.482pt}}
\multiput(391.34,185.00)(1.000,1.000){2}{\rule{0.800pt}{0.241pt}}
\put(394,185.84){\rule{0.241pt}{0.800pt}}
\multiput(394.00,185.34)(0.500,1.000){2}{\rule{0.120pt}{0.800pt}}
\put(395,187.34){\rule{0.482pt}{0.800pt}}
\multiput(395.00,186.34)(1.000,2.000){2}{\rule{0.241pt}{0.800pt}}
\put(395.84,190){\rule{0.800pt}{0.482pt}}
\multiput(395.34,190.00)(1.000,1.000){2}{\rule{0.800pt}{0.241pt}}
\put(396.84,192){\rule{0.800pt}{0.482pt}}
\multiput(396.34,192.00)(1.000,1.000){2}{\rule{0.800pt}{0.241pt}}
\put(399,192.84){\rule{0.482pt}{0.800pt}}
\multiput(399.00,192.34)(1.000,1.000){2}{\rule{0.241pt}{0.800pt}}
\put(399.84,195){\rule{0.800pt}{0.482pt}}
\multiput(399.34,195.00)(1.000,1.000){2}{\rule{0.800pt}{0.241pt}}
\put(400.84,197){\rule{0.800pt}{0.482pt}}
\multiput(400.34,197.00)(1.000,1.000){2}{\rule{0.800pt}{0.241pt}}
\put(401.84,199){\rule{0.800pt}{0.482pt}}
\multiput(401.34,199.00)(1.000,1.000){2}{\rule{0.800pt}{0.241pt}}
\put(404,199.84){\rule{0.482pt}{0.800pt}}
\multiput(404.00,199.34)(1.000,1.000){2}{\rule{0.241pt}{0.800pt}}
\put(404.84,202){\rule{0.800pt}{0.482pt}}
\multiput(404.34,202.00)(1.000,1.000){2}{\rule{0.800pt}{0.241pt}}
\put(405.84,204){\rule{0.800pt}{0.482pt}}
\multiput(405.34,204.00)(1.000,1.000){2}{\rule{0.800pt}{0.241pt}}
\put(408,205.34){\rule{0.482pt}{0.800pt}}
\multiput(408.00,204.34)(1.000,2.000){2}{\rule{0.241pt}{0.800pt}}
\put(410,206.84){\rule{0.241pt}{0.800pt}}
\multiput(410.00,206.34)(0.500,1.000){2}{\rule{0.120pt}{0.800pt}}
\put(409.84,209){\rule{0.800pt}{0.482pt}}
\multiput(409.34,209.00)(1.000,1.000){2}{\rule{0.800pt}{0.241pt}}
\put(412,210.34){\rule{0.482pt}{0.800pt}}
\multiput(412.00,209.34)(1.000,2.000){2}{\rule{0.241pt}{0.800pt}}
\put(412.84,213){\rule{0.800pt}{0.482pt}}
\multiput(412.34,213.00)(1.000,1.000){2}{\rule{0.800pt}{0.241pt}}
\put(415,213.84){\rule{0.241pt}{0.800pt}}
\multiput(415.00,213.34)(0.500,1.000){2}{\rule{0.120pt}{0.800pt}}
\put(414.84,216){\rule{0.800pt}{0.482pt}}
\multiput(414.34,216.00)(1.000,1.000){2}{\rule{0.800pt}{0.241pt}}
\put(417,217.34){\rule{0.482pt}{0.800pt}}
\multiput(417.00,216.34)(1.000,2.000){2}{\rule{0.241pt}{0.800pt}}
\put(419,218.84){\rule{0.241pt}{0.800pt}}
\multiput(419.00,218.34)(0.500,1.000){2}{\rule{0.120pt}{0.800pt}}
\put(418.84,221){\rule{0.800pt}{0.482pt}}
\multiput(418.34,221.00)(1.000,1.000){2}{\rule{0.800pt}{0.241pt}}
\put(421,222.34){\rule{0.482pt}{0.800pt}}
\multiput(421.00,221.34)(1.000,2.000){2}{\rule{0.241pt}{0.800pt}}
\put(421.84,225){\rule{0.800pt}{0.482pt}}
\multiput(421.34,225.00)(1.000,1.000){2}{\rule{0.800pt}{0.241pt}}
\put(424,225.84){\rule{0.241pt}{0.800pt}}
\multiput(424.00,225.34)(0.500,1.000){2}{\rule{0.120pt}{0.800pt}}
\put(423.84,228){\rule{0.800pt}{0.482pt}}
\multiput(423.34,228.00)(1.000,1.000){2}{\rule{0.800pt}{0.241pt}}
\put(426,229.34){\rule{0.482pt}{0.800pt}}
\multiput(426.00,228.34)(1.000,2.000){2}{\rule{0.241pt}{0.800pt}}
\put(426.84,232){\rule{0.800pt}{0.482pt}}
\multiput(426.34,232.00)(1.000,1.000){2}{\rule{0.800pt}{0.241pt}}
\put(429,232.84){\rule{0.241pt}{0.800pt}}
\multiput(429.00,232.34)(0.500,1.000){2}{\rule{0.120pt}{0.800pt}}
\put(430,234.34){\rule{0.482pt}{0.800pt}}
\multiput(430.00,233.34)(1.000,2.000){2}{\rule{0.241pt}{0.800pt}}
\put(430.84,237){\rule{0.800pt}{0.482pt}}
\multiput(430.34,237.00)(1.000,1.000){2}{\rule{0.800pt}{0.241pt}}
\put(431.84,239){\rule{0.800pt}{0.482pt}}
\multiput(431.34,239.00)(1.000,1.000){2}{\rule{0.800pt}{0.241pt}}
\put(434,239.84){\rule{0.241pt}{0.800pt}}
\multiput(434.00,239.34)(0.500,1.000){2}{\rule{0.120pt}{0.800pt}}
\put(435,241.34){\rule{0.482pt}{0.800pt}}
\multiput(435.00,240.34)(1.000,2.000){2}{\rule{0.241pt}{0.800pt}}
\put(435.84,244){\rule{0.800pt}{0.482pt}}
\multiput(435.34,244.00)(1.000,1.000){2}{\rule{0.800pt}{0.241pt}}
\put(436.84,246){\rule{0.800pt}{0.482pt}}
\multiput(436.34,246.00)(1.000,1.000){2}{\rule{0.800pt}{0.241pt}}
\put(439,246.84){\rule{0.482pt}{0.800pt}}
\multiput(439.00,246.34)(1.000,1.000){2}{\rule{0.241pt}{0.800pt}}
\put(439.84,249){\rule{0.800pt}{0.482pt}}
\multiput(439.34,249.00)(1.000,1.000){2}{\rule{0.800pt}{0.241pt}}
\put(440.84,251){\rule{0.800pt}{0.482pt}}
\multiput(440.34,251.00)(1.000,1.000){2}{\rule{0.800pt}{0.241pt}}
\put(443,251.84){\rule{0.482pt}{0.800pt}}
\multiput(443.00,251.34)(1.000,1.000){2}{\rule{0.241pt}{0.800pt}}
\put(443.84,254){\rule{0.800pt}{0.482pt}}
\multiput(443.34,254.00)(1.000,1.000){2}{\rule{0.800pt}{0.241pt}}
\put(444.84,256){\rule{0.800pt}{0.482pt}}
\multiput(444.34,256.00)(1.000,1.000){2}{\rule{0.800pt}{0.241pt}}
\put(445.84,258){\rule{0.800pt}{0.482pt}}
\multiput(445.34,258.00)(1.000,1.000){2}{\rule{0.800pt}{0.241pt}}
\put(448,258.84){\rule{0.482pt}{0.800pt}}
\multiput(448.00,258.34)(1.000,1.000){2}{\rule{0.241pt}{0.800pt}}
\put(448.84,261){\rule{0.800pt}{0.482pt}}
\multiput(448.34,261.00)(1.000,1.000){2}{\rule{0.800pt}{0.241pt}}
\put(449.84,243){\rule{0.800pt}{4.818pt}}
\multiput(449.34,253.00)(1.000,-10.000){2}{\rule{0.800pt}{2.409pt}}
\put(452,241.84){\rule{0.482pt}{0.800pt}}
\multiput(452.00,241.34)(1.000,1.000){2}{\rule{0.241pt}{0.800pt}}
\put(452.84,244){\rule{0.800pt}{0.482pt}}
\multiput(452.34,244.00)(1.000,1.000){2}{\rule{0.800pt}{0.241pt}}
\put(453.84,246){\rule{0.800pt}{0.482pt}}
\multiput(453.34,246.00)(1.000,1.000){2}{\rule{0.800pt}{0.241pt}}
\put(454.84,227){\rule{0.800pt}{5.059pt}}
\multiput(454.34,237.50)(1.000,-10.500){2}{\rule{0.800pt}{2.529pt}}
\put(456.34,218){\rule{0.800pt}{2.168pt}}
\multiput(455.34,222.50)(2.000,-4.500){2}{\rule{0.800pt}{1.084pt}}
\put(457.84,218){\rule{0.800pt}{0.482pt}}
\multiput(457.34,218.00)(1.000,1.000){2}{\rule{0.800pt}{0.241pt}}
\put(460,218.84){\rule{0.241pt}{0.800pt}}
\multiput(460.00,218.34)(0.500,1.000){2}{\rule{0.120pt}{0.800pt}}
\put(461,220.34){\rule{0.482pt}{0.800pt}}
\multiput(461.00,219.34)(1.000,2.000){2}{\rule{0.241pt}{0.800pt}}
\put(461.84,223){\rule{0.800pt}{0.482pt}}
\multiput(461.34,223.00)(1.000,1.000){2}{\rule{0.800pt}{0.241pt}}
\put(462.84,216){\rule{0.800pt}{2.168pt}}
\multiput(462.34,220.50)(1.000,-4.500){2}{\rule{0.800pt}{1.084pt}}
\put(465,214.84){\rule{0.482pt}{0.800pt}}
\multiput(465.00,214.34)(1.000,1.000){2}{\rule{0.241pt}{0.800pt}}
\put(465.84,217){\rule{0.800pt}{0.482pt}}
\multiput(465.34,217.00)(1.000,1.000){2}{\rule{0.800pt}{0.241pt}}
\put(466.84,219){\rule{0.800pt}{0.482pt}}
\multiput(466.34,219.00)(1.000,1.000){2}{\rule{0.800pt}{0.241pt}}
\put(467.84,221){\rule{0.800pt}{0.482pt}}
\multiput(467.34,221.00)(1.000,1.000){2}{\rule{0.800pt}{0.241pt}}
\put(470,221.84){\rule{0.482pt}{0.800pt}}
\multiput(470.00,221.34)(1.000,1.000){2}{\rule{0.241pt}{0.800pt}}
\put(470.84,215){\rule{0.800pt}{2.168pt}}
\multiput(470.34,219.50)(1.000,-4.500){2}{\rule{0.800pt}{1.084pt}}
\put(471.84,215){\rule{0.800pt}{0.482pt}}
\multiput(471.34,215.00)(1.000,1.000){2}{\rule{0.800pt}{0.241pt}}
\put(474,215.84){\rule{0.482pt}{0.800pt}}
\multiput(474.00,215.34)(1.000,1.000){2}{\rule{0.241pt}{0.800pt}}
\put(474.84,218){\rule{0.800pt}{0.482pt}}
\multiput(474.34,218.00)(1.000,1.000){2}{\rule{0.800pt}{0.241pt}}
\put(475.84,220){\rule{0.800pt}{0.482pt}}
\multiput(475.34,220.00)(1.000,1.000){2}{\rule{0.800pt}{0.241pt}}
\put(476.84,222){\rule{0.800pt}{0.482pt}}
\multiput(476.34,222.00)(1.000,1.000){2}{\rule{0.800pt}{0.241pt}}
\put(478.34,214){\rule{0.800pt}{2.409pt}}
\multiput(477.34,219.00)(2.000,-5.000){2}{\rule{0.800pt}{1.204pt}}
\put(479.84,214){\rule{0.800pt}{0.482pt}}
\multiput(479.34,214.00)(1.000,1.000){2}{\rule{0.800pt}{0.241pt}}
\put(480.84,216){\rule{0.800pt}{0.482pt}}
\multiput(480.34,216.00)(1.000,1.000){2}{\rule{0.800pt}{0.241pt}}
\put(483,216.84){\rule{0.482pt}{0.800pt}}
\multiput(483.00,216.34)(1.000,1.000){2}{\rule{0.241pt}{0.800pt}}
\put(483.84,210){\rule{0.800pt}{2.168pt}}
\multiput(483.34,214.50)(1.000,-4.500){2}{\rule{0.800pt}{1.084pt}}
\put(484.84,201){\rule{0.800pt}{2.168pt}}
\multiput(484.34,205.50)(1.000,-4.500){2}{\rule{0.800pt}{1.084pt}}
\put(487,199.84){\rule{0.241pt}{0.800pt}}
\multiput(487.00,199.34)(0.500,1.000){2}{\rule{0.120pt}{0.800pt}}
\put(488,201.34){\rule{0.482pt}{0.800pt}}
\multiput(488.00,200.34)(1.000,2.000){2}{\rule{0.241pt}{0.800pt}}
\put(488.84,204){\rule{0.800pt}{0.482pt}}
\multiput(488.34,204.00)(1.000,1.000){2}{\rule{0.800pt}{0.241pt}}
\put(489.84,206){\rule{0.800pt}{0.482pt}}
\multiput(489.34,206.00)(1.000,1.000){2}{\rule{0.800pt}{0.241pt}}
\put(492,206.84){\rule{0.482pt}{0.800pt}}
\multiput(492.00,206.34)(1.000,1.000){2}{\rule{0.241pt}{0.800pt}}
\put(492.84,209){\rule{0.800pt}{0.482pt}}
\multiput(492.34,209.00)(1.000,1.000){2}{\rule{0.800pt}{0.241pt}}
\put(493.84,211){\rule{0.800pt}{0.482pt}}
\multiput(493.34,211.00)(1.000,1.000){2}{\rule{0.800pt}{0.241pt}}
\put(496,212.34){\rule{0.482pt}{0.800pt}}
\multiput(496.00,211.34)(1.000,2.000){2}{\rule{0.241pt}{0.800pt}}
\put(498,213.84){\rule{0.241pt}{0.800pt}}
\multiput(498.00,213.34)(0.500,1.000){2}{\rule{0.120pt}{0.800pt}}
\put(497.84,216){\rule{0.800pt}{0.482pt}}
\multiput(497.34,216.00)(1.000,1.000){2}{\rule{0.800pt}{0.241pt}}
\put(498.84,218){\rule{0.800pt}{0.482pt}}
\multiput(498.34,218.00)(1.000,1.000){2}{\rule{0.800pt}{0.241pt}}
\put(501,218.84){\rule{0.482pt}{0.800pt}}
\multiput(501.00,218.34)(1.000,1.000){2}{\rule{0.241pt}{0.800pt}}
\put(501.84,221){\rule{0.800pt}{0.482pt}}
\multiput(501.34,221.00)(1.000,1.000){2}{\rule{0.800pt}{0.241pt}}
\put(502.84,223){\rule{0.800pt}{0.482pt}}
\multiput(502.34,223.00)(1.000,1.000){2}{\rule{0.800pt}{0.241pt}}
\put(505,224.34){\rule{0.482pt}{0.800pt}}
\multiput(505.00,223.34)(1.000,2.000){2}{\rule{0.241pt}{0.800pt}}
\put(507,225.84){\rule{0.241pt}{0.800pt}}
\multiput(507.00,225.34)(0.500,1.000){2}{\rule{0.120pt}{0.800pt}}
\put(506.84,228){\rule{0.800pt}{0.482pt}}
\multiput(506.34,228.00)(1.000,1.000){2}{\rule{0.800pt}{0.241pt}}
\put(507.84,230){\rule{0.800pt}{0.482pt}}
\multiput(507.34,230.00)(1.000,1.000){2}{\rule{0.800pt}{0.241pt}}
\put(510,231.34){\rule{0.482pt}{0.800pt}}
\multiput(510.00,230.34)(1.000,2.000){2}{\rule{0.241pt}{0.800pt}}
\put(512,232.84){\rule{0.241pt}{0.800pt}}
\multiput(512.00,232.34)(0.500,1.000){2}{\rule{0.120pt}{0.800pt}}
\put(511.84,235){\rule{0.800pt}{0.482pt}}
\multiput(511.34,235.00)(1.000,1.000){2}{\rule{0.800pt}{0.241pt}}
\put(514,236.34){\rule{0.482pt}{0.800pt}}
\multiput(514.00,235.34)(1.000,2.000){2}{\rule{0.241pt}{0.800pt}}
\put(514.84,239){\rule{0.800pt}{0.482pt}}
\multiput(514.34,239.00)(1.000,1.000){2}{\rule{0.800pt}{0.241pt}}
\put(517,239.84){\rule{0.241pt}{0.800pt}}
\multiput(517.00,239.34)(0.500,1.000){2}{\rule{0.120pt}{0.800pt}}
\put(518,241.34){\rule{0.482pt}{0.800pt}}
\multiput(518.00,240.34)(1.000,2.000){2}{\rule{0.241pt}{0.800pt}}
\put(518.84,244){\rule{0.800pt}{0.482pt}}
\multiput(518.34,244.00)(1.000,1.000){2}{\rule{0.800pt}{0.241pt}}
\put(519.84,246){\rule{0.800pt}{0.482pt}}
\multiput(519.34,246.00)(1.000,1.000){2}{\rule{0.800pt}{0.241pt}}
\put(522,246.84){\rule{0.241pt}{0.800pt}}
\multiput(522.00,246.34)(0.500,1.000){2}{\rule{0.120pt}{0.800pt}}
\put(523,248.34){\rule{0.482pt}{0.800pt}}
\multiput(523.00,247.34)(1.000,2.000){2}{\rule{0.241pt}{0.800pt}}
\put(523.84,251){\rule{0.800pt}{0.482pt}}
\multiput(523.34,251.00)(1.000,1.000){2}{\rule{0.800pt}{0.241pt}}
\put(526,251.84){\rule{0.241pt}{0.800pt}}
\multiput(526.00,251.34)(0.500,1.000){2}{\rule{0.120pt}{0.800pt}}
\put(527,253.34){\rule{0.482pt}{0.800pt}}
\multiput(527.00,252.34)(1.000,2.000){2}{\rule{0.241pt}{0.800pt}}
\put(527.84,256){\rule{0.800pt}{0.482pt}}
\multiput(527.34,256.00)(1.000,1.000){2}{\rule{0.800pt}{0.241pt}}
\put(528.84,258){\rule{0.800pt}{0.482pt}}
\multiput(528.34,258.00)(1.000,1.000){2}{\rule{0.800pt}{0.241pt}}
\put(531,258.84){\rule{0.241pt}{0.800pt}}
\multiput(531.00,258.34)(0.500,1.000){2}{\rule{0.120pt}{0.800pt}}
\put(532,260.34){\rule{0.482pt}{0.800pt}}
\multiput(532.00,259.34)(1.000,2.000){2}{\rule{0.241pt}{0.800pt}}
\put(532.84,263){\rule{0.800pt}{0.482pt}}
\multiput(532.34,263.00)(1.000,1.000){2}{\rule{0.800pt}{0.241pt}}
\put(533.84,265){\rule{0.800pt}{0.482pt}}
\multiput(533.34,265.00)(1.000,1.000){2}{\rule{0.800pt}{0.241pt}}
\put(536,265.84){\rule{0.482pt}{0.800pt}}
\multiput(536.00,265.34)(1.000,1.000){2}{\rule{0.241pt}{0.800pt}}
\put(536.84,268){\rule{0.800pt}{0.482pt}}
\multiput(536.34,268.00)(1.000,1.000){2}{\rule{0.800pt}{0.241pt}}
\put(537.84,270){\rule{0.800pt}{0.482pt}}
\multiput(537.34,270.00)(1.000,1.000){2}{\rule{0.800pt}{0.241pt}}
\put(538.84,272){\rule{0.800pt}{0.482pt}}
\multiput(538.34,272.00)(1.000,1.000){2}{\rule{0.800pt}{0.241pt}}
\put(541,272.84){\rule{0.482pt}{0.800pt}}
\multiput(541.00,272.34)(1.000,1.000){2}{\rule{0.241pt}{0.800pt}}
\put(541.84,275){\rule{0.800pt}{0.482pt}}
\multiput(541.34,275.00)(1.000,1.000){2}{\rule{0.800pt}{0.241pt}}
\put(542.84,277){\rule{0.800pt}{0.482pt}}
\multiput(542.34,277.00)(1.000,1.000){2}{\rule{0.800pt}{0.241pt}}
\put(545,277.84){\rule{0.482pt}{0.800pt}}
\multiput(545.00,277.34)(1.000,1.000){2}{\rule{0.241pt}{0.800pt}}
\put(545.84,280){\rule{0.800pt}{0.482pt}}
\multiput(545.34,280.00)(1.000,1.000){2}{\rule{0.800pt}{0.241pt}}
\put(546.84,282){\rule{0.800pt}{0.482pt}}
\multiput(546.34,282.00)(1.000,1.000){2}{\rule{0.800pt}{0.241pt}}
\put(549,283.34){\rule{0.482pt}{0.800pt}}
\multiput(549.00,282.34)(1.000,2.000){2}{\rule{0.241pt}{0.800pt}}
\put(551,284.84){\rule{0.241pt}{0.800pt}}
\multiput(551.00,284.34)(0.500,1.000){2}{\rule{0.120pt}{0.800pt}}
\put(550.84,287){\rule{0.800pt}{0.482pt}}
\multiput(550.34,287.00)(1.000,1.000){2}{\rule{0.800pt}{0.241pt}}
\put(551.84,289){\rule{0.800pt}{0.482pt}}
\multiput(551.34,289.00)(1.000,1.000){2}{\rule{0.800pt}{0.241pt}}
\put(554,290.34){\rule{0.482pt}{0.800pt}}
\multiput(554.00,289.34)(1.000,2.000){2}{\rule{0.241pt}{0.800pt}}
\put(556,291.84){\rule{0.241pt}{0.800pt}}
\multiput(556.00,291.34)(0.500,1.000){2}{\rule{0.120pt}{0.800pt}}
\put(555.84,294){\rule{0.800pt}{0.482pt}}
\multiput(555.34,294.00)(1.000,1.000){2}{\rule{0.800pt}{0.241pt}}
\put(558,295.34){\rule{0.482pt}{0.800pt}}
\multiput(558.00,294.34)(1.000,2.000){2}{\rule{0.241pt}{0.800pt}}
\put(558.84,298){\rule{0.800pt}{0.482pt}}
\multiput(558.34,298.00)(1.000,1.000){2}{\rule{0.800pt}{0.241pt}}
\put(561,298.84){\rule{0.241pt}{0.800pt}}
\multiput(561.00,298.34)(0.500,1.000){2}{\rule{0.120pt}{0.800pt}}
\put(560.84,301){\rule{0.800pt}{0.482pt}}
\multiput(560.34,301.00)(1.000,1.000){2}{\rule{0.800pt}{0.241pt}}
\put(563,302.34){\rule{0.482pt}{0.800pt}}
\multiput(563.00,301.34)(1.000,2.000){2}{\rule{0.241pt}{0.800pt}}
\put(563.84,305){\rule{0.800pt}{0.482pt}}
\multiput(563.34,305.00)(1.000,1.000){2}{\rule{0.800pt}{0.241pt}}
\put(566,305.84){\rule{0.241pt}{0.800pt}}
\multiput(566.00,305.34)(0.500,1.000){2}{\rule{0.120pt}{0.800pt}}
\put(567,307.34){\rule{0.482pt}{0.800pt}}
\multiput(567.00,306.34)(1.000,2.000){2}{\rule{0.241pt}{0.800pt}}
\put(567.84,310){\rule{0.800pt}{0.482pt}}
\multiput(567.34,310.00)(1.000,1.000){2}{\rule{0.800pt}{0.241pt}}
\put(570,310.84){\rule{0.241pt}{0.800pt}}
\multiput(570.00,310.34)(0.500,1.000){2}{\rule{0.120pt}{0.800pt}}
\put(571,312.34){\rule{0.482pt}{0.800pt}}
\multiput(571.00,311.34)(1.000,2.000){2}{\rule{0.241pt}{0.800pt}}
\put(571.84,315){\rule{0.800pt}{0.482pt}}
\multiput(571.34,315.00)(1.000,1.000){2}{\rule{0.800pt}{0.241pt}}
\put(572.84,317){\rule{0.800pt}{0.482pt}}
\multiput(572.34,317.00)(1.000,1.000){2}{\rule{0.800pt}{0.241pt}}
\put(575,317.84){\rule{0.241pt}{0.800pt}}
\multiput(575.00,317.34)(0.500,1.000){2}{\rule{0.120pt}{0.800pt}}
\put(576,319.34){\rule{0.482pt}{0.800pt}}
\multiput(576.00,318.34)(1.000,2.000){2}{\rule{0.241pt}{0.800pt}}
\put(576.84,322){\rule{0.800pt}{0.482pt}}
\multiput(576.34,322.00)(1.000,1.000){2}{\rule{0.800pt}{0.241pt}}
\put(577.84,324){\rule{0.800pt}{0.482pt}}
\multiput(577.34,324.00)(1.000,1.000){2}{\rule{0.800pt}{0.241pt}}
\put(580,324.84){\rule{0.482pt}{0.800pt}}
\multiput(580.00,324.34)(1.000,1.000){2}{\rule{0.241pt}{0.800pt}}
\put(580.84,327){\rule{0.800pt}{0.482pt}}
\multiput(580.34,327.00)(1.000,1.000){2}{\rule{0.800pt}{0.241pt}}
\put(581.84,329){\rule{0.800pt}{0.482pt}}
\multiput(581.34,329.00)(1.000,1.000){2}{\rule{0.800pt}{0.241pt}}
\put(582.84,331){\rule{0.800pt}{0.482pt}}
\multiput(582.34,331.00)(1.000,1.000){2}{\rule{0.800pt}{0.241pt}}
\put(585,331.84){\rule{0.482pt}{0.800pt}}
\multiput(585.00,331.34)(1.000,1.000){2}{\rule{0.241pt}{0.800pt}}
\put(585.84,334){\rule{0.800pt}{0.482pt}}
\multiput(585.34,334.00)(1.000,1.000){2}{\rule{0.800pt}{0.241pt}}
\put(586.84,336){\rule{0.800pt}{0.482pt}}
\multiput(586.34,336.00)(1.000,1.000){2}{\rule{0.800pt}{0.241pt}}
\put(589,337.34){\rule{0.482pt}{0.800pt}}
\multiput(589.00,336.34)(1.000,2.000){2}{\rule{0.241pt}{0.800pt}}
\put(591,338.84){\rule{0.241pt}{0.800pt}}
\multiput(591.00,338.34)(0.500,1.000){2}{\rule{0.120pt}{0.800pt}}
\put(590.84,341){\rule{0.800pt}{0.482pt}}
\multiput(590.34,341.00)(1.000,1.000){2}{\rule{0.800pt}{0.241pt}}
\put(591.84,343){\rule{0.800pt}{0.482pt}}
\multiput(591.34,343.00)(1.000,1.000){2}{\rule{0.800pt}{0.241pt}}
\put(594,343.84){\rule{0.482pt}{0.800pt}}
\multiput(594.00,343.34)(1.000,1.000){2}{\rule{0.241pt}{0.800pt}}
\put(594.84,346){\rule{0.800pt}{0.482pt}}
\multiput(594.34,346.00)(1.000,1.000){2}{\rule{0.800pt}{0.241pt}}
\put(595.84,348){\rule{0.800pt}{0.482pt}}
\multiput(595.34,348.00)(1.000,1.000){2}{\rule{0.800pt}{0.241pt}}
\put(598,349.34){\rule{0.482pt}{0.800pt}}
\multiput(598.00,348.34)(1.000,2.000){2}{\rule{0.241pt}{0.800pt}}
\put(600,350.84){\rule{0.241pt}{0.800pt}}
\multiput(600.00,350.34)(0.500,1.000){2}{\rule{0.120pt}{0.800pt}}
\put(599.84,353){\rule{0.800pt}{0.482pt}}
\multiput(599.34,353.00)(1.000,1.000){2}{\rule{0.800pt}{0.241pt}}
\put(602,354.34){\rule{0.482pt}{0.800pt}}
\multiput(602.00,353.34)(1.000,2.000){2}{\rule{0.241pt}{0.800pt}}
\put(602.84,357){\rule{0.800pt}{0.482pt}}
\multiput(602.34,357.00)(1.000,1.000){2}{\rule{0.800pt}{0.241pt}}
\put(605,357.84){\rule{0.241pt}{0.800pt}}
\multiput(605.00,357.34)(0.500,1.000){2}{\rule{0.120pt}{0.800pt}}
\put(604.84,360){\rule{0.800pt}{0.482pt}}
\multiput(604.34,360.00)(1.000,1.000){2}{\rule{0.800pt}{0.241pt}}
\put(607,361.34){\rule{0.482pt}{0.800pt}}
\multiput(607.00,360.34)(1.000,2.000){2}{\rule{0.241pt}{0.800pt}}
\put(607.84,364){\rule{0.800pt}{0.482pt}}
\multiput(607.34,364.00)(1.000,1.000){2}{\rule{0.800pt}{0.241pt}}
\put(610,364.84){\rule{0.241pt}{0.800pt}}
\multiput(610.00,364.34)(0.500,1.000){2}{\rule{0.120pt}{0.800pt}}
\put(611,366.34){\rule{0.482pt}{0.800pt}}
\multiput(611.00,365.34)(1.000,2.000){2}{\rule{0.241pt}{0.800pt}}
\put(611.84,369){\rule{0.800pt}{0.482pt}}
\multiput(611.34,369.00)(1.000,1.000){2}{\rule{0.800pt}{0.241pt}}
\put(614,369.84){\rule{0.241pt}{0.800pt}}
\multiput(614.00,369.34)(0.500,1.000){2}{\rule{0.120pt}{0.800pt}}
\put(613.84,372){\rule{0.800pt}{0.482pt}}
\multiput(613.34,372.00)(1.000,1.000){2}{\rule{0.800pt}{0.241pt}}
\put(616,373.34){\rule{0.482pt}{0.800pt}}
\multiput(616.00,372.34)(1.000,2.000){2}{\rule{0.241pt}{0.800pt}}
\put(616.84,376){\rule{0.800pt}{0.482pt}}
\multiput(616.34,376.00)(1.000,1.000){2}{\rule{0.800pt}{0.241pt}}
\put(619,376.84){\rule{0.241pt}{0.800pt}}
\multiput(619.00,376.34)(0.500,1.000){2}{\rule{0.120pt}{0.800pt}}
\put(620,378.34){\rule{0.482pt}{0.800pt}}
\multiput(620.00,377.34)(1.000,2.000){2}{\rule{0.241pt}{0.800pt}}
\put(620.84,381){\rule{0.800pt}{0.482pt}}
\multiput(620.34,381.00)(1.000,1.000){2}{\rule{0.800pt}{0.241pt}}
\put(621.84,383){\rule{0.800pt}{0.482pt}}
\multiput(621.34,383.00)(1.000,1.000){2}{\rule{0.800pt}{0.241pt}}
\put(624,383.84){\rule{0.482pt}{0.800pt}}
\multiput(624.00,383.34)(1.000,1.000){2}{\rule{0.241pt}{0.800pt}}
\put(624.84,386){\rule{0.800pt}{0.482pt}}
\multiput(624.34,386.00)(1.000,1.000){2}{\rule{0.800pt}{0.241pt}}
\put(625.84,388){\rule{0.800pt}{0.482pt}}
\multiput(625.34,388.00)(1.000,1.000){2}{\rule{0.800pt}{0.241pt}}
\put(626.84,390){\rule{0.800pt}{0.482pt}}
\multiput(626.34,390.00)(1.000,1.000){2}{\rule{0.800pt}{0.241pt}}
\put(629,390.84){\rule{0.482pt}{0.800pt}}
\multiput(629.00,390.34)(1.000,1.000){2}{\rule{0.241pt}{0.800pt}}
\put(629.84,393){\rule{0.800pt}{0.482pt}}
\multiput(629.34,393.00)(1.000,1.000){2}{\rule{0.800pt}{0.241pt}}
\put(630.84,395){\rule{0.800pt}{0.482pt}}
\multiput(630.34,395.00)(1.000,1.000){2}{\rule{0.800pt}{0.241pt}}
\put(633,396.34){\rule{0.482pt}{0.800pt}}
\multiput(633.00,395.34)(1.000,2.000){2}{\rule{0.241pt}{0.800pt}}
\put(635,397.84){\rule{0.241pt}{0.800pt}}
\multiput(635.00,397.34)(0.500,1.000){2}{\rule{0.120pt}{0.800pt}}
\put(634.84,400){\rule{0.800pt}{0.482pt}}
\multiput(634.34,400.00)(1.000,1.000){2}{\rule{0.800pt}{0.241pt}}
\put(635.84,402){\rule{0.800pt}{0.482pt}}
\multiput(635.34,402.00)(1.000,1.000){2}{\rule{0.800pt}{0.241pt}}
\put(638,402.84){\rule{0.482pt}{0.800pt}}
\multiput(638.00,402.34)(1.000,1.000){2}{\rule{0.241pt}{0.800pt}}
\put(638.84,405){\rule{0.800pt}{0.482pt}}
\multiput(638.34,405.00)(1.000,1.000){2}{\rule{0.800pt}{0.241pt}}
\put(639.84,407){\rule{0.800pt}{0.482pt}}
\multiput(639.34,407.00)(1.000,1.000){2}{\rule{0.800pt}{0.241pt}}
\put(642,408.34){\rule{0.482pt}{0.800pt}}
\multiput(642.00,407.34)(1.000,2.000){2}{\rule{0.241pt}{0.800pt}}
\put(644,409.84){\rule{0.241pt}{0.800pt}}
\multiput(644.00,409.34)(0.500,1.000){2}{\rule{0.120pt}{0.800pt}}
\put(643.84,412){\rule{0.800pt}{0.482pt}}
\multiput(643.34,412.00)(1.000,1.000){2}{\rule{0.800pt}{0.241pt}}
\put(646,413.34){\rule{0.482pt}{0.800pt}}
\multiput(646.00,412.34)(1.000,2.000){2}{\rule{0.241pt}{0.800pt}}
\put(646.84,416){\rule{0.800pt}{0.482pt}}
\multiput(646.34,416.00)(1.000,1.000){2}{\rule{0.800pt}{0.241pt}}
\put(649,416.84){\rule{0.241pt}{0.800pt}}
\multiput(649.00,416.34)(0.500,1.000){2}{\rule{0.120pt}{0.800pt}}
\put(648.84,419){\rule{0.800pt}{0.482pt}}
\multiput(648.34,419.00)(1.000,1.000){2}{\rule{0.800pt}{0.241pt}}
\put(651,420.34){\rule{0.482pt}{0.800pt}}
\multiput(651.00,419.34)(1.000,2.000){2}{\rule{0.241pt}{0.800pt}}
\put(651.84,423){\rule{0.800pt}{0.482pt}}
\multiput(651.34,423.00)(1.000,1.000){2}{\rule{0.800pt}{0.241pt}}
\put(654,423.84){\rule{0.241pt}{0.800pt}}
\multiput(654.00,423.34)(0.500,1.000){2}{\rule{0.120pt}{0.800pt}}
\put(655,425.34){\rule{0.482pt}{0.800pt}}
\multiput(655.00,424.34)(1.000,2.000){2}{\rule{0.241pt}{0.800pt}}
\put(655.84,428){\rule{0.800pt}{0.482pt}}
\multiput(655.34,428.00)(1.000,1.000){2}{\rule{0.800pt}{0.241pt}}
\put(656.84,430){\rule{0.800pt}{0.482pt}}
\multiput(656.34,430.00)(1.000,1.000){2}{\rule{0.800pt}{0.241pt}}
\put(659,430.84){\rule{0.241pt}{0.800pt}}
\multiput(659.00,430.34)(0.500,1.000){2}{\rule{0.120pt}{0.800pt}}
\put(660,432.34){\rule{0.482pt}{0.800pt}}
\multiput(660.00,431.34)(1.000,2.000){2}{\rule{0.241pt}{0.800pt}}
\put(660.84,435){\rule{0.800pt}{0.482pt}}
\multiput(660.34,435.00)(1.000,1.000){2}{\rule{0.800pt}{0.241pt}}
\put(663,435.84){\rule{0.241pt}{0.800pt}}
\multiput(663.00,435.34)(0.500,1.000){2}{\rule{0.120pt}{0.800pt}}
\put(664,437.34){\rule{0.482pt}{0.800pt}}
\multiput(664.00,436.34)(1.000,2.000){2}{\rule{0.241pt}{0.800pt}}
\put(664.84,440){\rule{0.800pt}{0.482pt}}
\multiput(664.34,440.00)(1.000,1.000){2}{\rule{0.800pt}{0.241pt}}
\put(665.84,442){\rule{0.800pt}{0.482pt}}
\multiput(665.34,442.00)(1.000,1.000){2}{\rule{0.800pt}{0.241pt}}
\put(668,442.84){\rule{0.241pt}{0.800pt}}
\multiput(668.00,442.34)(0.500,1.000){2}{\rule{0.120pt}{0.800pt}}
\put(669,444.34){\rule{0.482pt}{0.800pt}}
\multiput(669.00,443.34)(1.000,2.000){2}{\rule{0.241pt}{0.800pt}}
\put(669.84,447){\rule{0.800pt}{0.482pt}}
\multiput(669.34,447.00)(1.000,1.000){2}{\rule{0.800pt}{0.241pt}}
\put(670.84,449){\rule{0.800pt}{0.482pt}}
\multiput(670.34,449.00)(1.000,1.000){2}{\rule{0.800pt}{0.241pt}}
\put(673,449.84){\rule{0.482pt}{0.800pt}}
\multiput(673.00,449.34)(1.000,1.000){2}{\rule{0.241pt}{0.800pt}}
\put(673.84,452){\rule{0.800pt}{0.482pt}}
\multiput(673.34,452.00)(1.000,1.000){2}{\rule{0.800pt}{0.241pt}}
\put(674.84,454){\rule{0.800pt}{0.482pt}}
\multiput(674.34,454.00)(1.000,1.000){2}{\rule{0.800pt}{0.241pt}}
\put(677,455.34){\rule{0.482pt}{0.800pt}}
\multiput(677.00,454.34)(1.000,2.000){2}{\rule{0.241pt}{0.800pt}}
\put(679,456.84){\rule{0.241pt}{0.800pt}}
\multiput(679.00,456.34)(0.500,1.000){2}{\rule{0.120pt}{0.800pt}}
\put(678.84,459){\rule{0.800pt}{0.482pt}}
\multiput(678.34,459.00)(1.000,1.000){2}{\rule{0.800pt}{0.241pt}}
\put(679.84,461){\rule{0.800pt}{0.482pt}}
\multiput(679.34,461.00)(1.000,1.000){2}{\rule{0.800pt}{0.241pt}}
\put(682,462.34){\rule{0.482pt}{0.800pt}}
\multiput(682.00,461.34)(1.000,2.000){2}{\rule{0.241pt}{0.800pt}}
\put(684,463.84){\rule{0.241pt}{0.800pt}}
\multiput(684.00,463.34)(0.500,1.000){2}{\rule{0.120pt}{0.800pt}}
\put(683.84,466){\rule{0.800pt}{0.482pt}}
\multiput(683.34,466.00)(1.000,1.000){2}{\rule{0.800pt}{0.241pt}}
\put(686,467.34){\rule{0.482pt}{0.800pt}}
\multiput(686.00,466.34)(1.000,2.000){2}{\rule{0.241pt}{0.800pt}}
\put(688,468.84){\rule{0.241pt}{0.800pt}}
\multiput(688.00,468.34)(0.500,1.000){2}{\rule{0.120pt}{0.800pt}}
\put(687.84,471){\rule{0.800pt}{0.482pt}}
\multiput(687.34,471.00)(1.000,1.000){2}{\rule{0.800pt}{0.241pt}}
\put(688.84,473){\rule{0.800pt}{0.482pt}}
\multiput(688.34,473.00)(1.000,1.000){2}{\rule{0.800pt}{0.241pt}}
\put(691,474.34){\rule{0.482pt}{0.800pt}}
\multiput(691.00,473.34)(1.000,2.000){2}{\rule{0.241pt}{0.800pt}}
\put(693,475.84){\rule{0.241pt}{0.800pt}}
\multiput(693.00,475.34)(0.500,1.000){2}{\rule{0.120pt}{0.800pt}}
\put(692.84,478){\rule{0.800pt}{0.482pt}}
\multiput(692.34,478.00)(1.000,1.000){2}{\rule{0.800pt}{0.241pt}}
\put(695,479.34){\rule{0.482pt}{0.800pt}}
\multiput(695.00,478.34)(1.000,2.000){2}{\rule{0.241pt}{0.800pt}}
\put(695.84,482){\rule{0.800pt}{0.482pt}}
\multiput(695.34,482.00)(1.000,1.000){2}{\rule{0.800pt}{0.241pt}}
\put(698,482.84){\rule{0.241pt}{0.800pt}}
\multiput(698.00,482.34)(0.500,1.000){2}{\rule{0.120pt}{0.800pt}}
\put(699,484.34){\rule{0.482pt}{0.800pt}}
\multiput(699.00,483.34)(1.000,2.000){2}{\rule{0.241pt}{0.800pt}}
\put(699.84,487){\rule{0.800pt}{0.482pt}}
\multiput(699.34,487.00)(1.000,1.000){2}{\rule{0.800pt}{0.241pt}}
\put(700.84,489){\rule{0.800pt}{0.482pt}}
\multiput(700.34,489.00)(1.000,1.000){2}{\rule{0.800pt}{0.241pt}}
\put(703,489.84){\rule{0.241pt}{0.800pt}}
\multiput(703.00,489.34)(0.500,1.000){2}{\rule{0.120pt}{0.800pt}}
\put(704,491.34){\rule{0.482pt}{0.800pt}}
\multiput(704.00,490.34)(1.000,2.000){2}{\rule{0.241pt}{0.800pt}}
\put(704.84,494){\rule{0.800pt}{0.482pt}}
\multiput(704.34,494.00)(1.000,1.000){2}{\rule{0.800pt}{0.241pt}}
\put(707,494.84){\rule{0.241pt}{0.800pt}}
\multiput(707.00,494.34)(0.500,1.000){2}{\rule{0.120pt}{0.800pt}}
\put(708,496.34){\rule{0.482pt}{0.800pt}}
\multiput(708.00,495.34)(1.000,2.000){2}{\rule{0.241pt}{0.800pt}}
\put(708.84,499){\rule{0.800pt}{0.482pt}}
\multiput(708.34,499.00)(1.000,1.000){2}{\rule{0.800pt}{0.241pt}}
\put(709.84,501){\rule{0.800pt}{0.482pt}}
\multiput(709.34,501.00)(1.000,1.000){2}{\rule{0.800pt}{0.241pt}}
\put(712,501.84){\rule{0.241pt}{0.800pt}}
\multiput(712.00,501.34)(0.500,1.000){2}{\rule{0.120pt}{0.800pt}}
\put(713,503.34){\rule{0.482pt}{0.800pt}}
\multiput(713.00,502.34)(1.000,2.000){2}{\rule{0.241pt}{0.800pt}}
\put(713.84,506){\rule{0.800pt}{0.482pt}}
\multiput(713.34,506.00)(1.000,1.000){2}{\rule{0.800pt}{0.241pt}}
\put(714.84,508){\rule{0.800pt}{0.482pt}}
\multiput(714.34,508.00)(1.000,1.000){2}{\rule{0.800pt}{0.241pt}}
\put(717,508.84){\rule{0.482pt}{0.800pt}}
\multiput(717.00,508.34)(1.000,1.000){2}{\rule{0.241pt}{0.800pt}}
\put(717.84,511){\rule{0.800pt}{0.482pt}}
\multiput(717.34,511.00)(1.000,1.000){2}{\rule{0.800pt}{0.241pt}}
\put(718.84,513){\rule{0.800pt}{0.482pt}}
\multiput(718.34,513.00)(1.000,1.000){2}{\rule{0.800pt}{0.241pt}}
\put(719.84,515){\rule{0.800pt}{0.482pt}}
\multiput(719.34,515.00)(1.000,1.000){2}{\rule{0.800pt}{0.241pt}}
\put(722,515.84){\rule{0.482pt}{0.800pt}}
\multiput(722.00,515.34)(1.000,1.000){2}{\rule{0.241pt}{0.800pt}}
\put(722.84,518){\rule{0.800pt}{0.482pt}}
\multiput(722.34,518.00)(1.000,1.000){2}{\rule{0.800pt}{0.241pt}}
\put(723.84,520){\rule{0.800pt}{0.482pt}}
\multiput(723.34,520.00)(1.000,1.000){2}{\rule{0.800pt}{0.241pt}}
\put(726,521.34){\rule{0.482pt}{0.800pt}}
\multiput(726.00,520.34)(1.000,2.000){2}{\rule{0.241pt}{0.800pt}}
\put(728,522.84){\rule{0.241pt}{0.800pt}}
\multiput(728.00,522.34)(0.500,1.000){2}{\rule{0.120pt}{0.800pt}}
\put(727.84,525){\rule{0.800pt}{0.482pt}}
\multiput(727.34,525.00)(1.000,1.000){2}{\rule{0.800pt}{0.241pt}}
\put(730,526.34){\rule{0.482pt}{0.800pt}}
\multiput(730.00,525.34)(1.000,2.000){2}{\rule{0.241pt}{0.800pt}}
\put(732,527.84){\rule{0.241pt}{0.800pt}}
\multiput(732.00,527.34)(0.500,1.000){2}{\rule{0.120pt}{0.800pt}}
\put(731.84,530){\rule{0.800pt}{0.482pt}}
\multiput(731.34,530.00)(1.000,1.000){2}{\rule{0.800pt}{0.241pt}}
\put(732.84,532){\rule{0.800pt}{0.482pt}}
\multiput(732.34,532.00)(1.000,1.000){2}{\rule{0.800pt}{0.241pt}}
\put(735,533.34){\rule{0.482pt}{0.800pt}}
\multiput(735.00,532.34)(1.000,2.000){2}{\rule{0.241pt}{0.800pt}}
\sbox{\plotpoint}{\rule[-0.200pt]{0.400pt}{0.400pt}}%
\put(221.0,143.0){\rule[-0.200pt]{124.545pt}{0.400pt}}
\put(738.0,143.0){\rule[-0.200pt]{0.400pt}{104.551pt}}
\put(221.0,577.0){\rule[-0.200pt]{124.545pt}{0.400pt}}
\put(221.0,143.0){\rule[-0.200pt]{0.400pt}{104.551pt}}
\end{picture}

	\end{center}
\caption{Output of algorithm \adwintwo with slow gradual changes} 
\label{fig:ADWIN2-R}
\end{figure}

Finally, we state our main technical result about the performance of \adwintwoz, in a similar way to the Theorem~\ref{ThBV}:

\begin{theorem}
\label{ThBV2}
At every time step we have 

\begin{enumerate}
\item {\em (False positive rate bound).} If $\mu_t$ remains constant within $W$, 
the probability that \adwintwo shrinks the window 
at this step is at most $ M/n  \cdot \log (n/M) \cdot \delta$.

\item {\em (False negative rate bound).} 
Suppose that for {\em some} partition of $W$ in two parts $W_0W_1$ 
(where $W_1$ contains the most recent items) 
we have $|\mu_{W_0}-\mu_{W_1}| > 2\epsc$. 
Then with probability $1-\delta$ \adwintwo
shrinks $W$ to $W_1$, or shorter.
\end{enumerate}
\end{theorem}

\begin{proof}

{\bf{Part 1)}}
Assume $\muz=\muu=\mu_W$ as null hypothesis. We have shown in the proof of Theorem~\ref{ThBV} that for any partition
$W$ as $W_0W_1$ we have probability at most $\delta/n$ that \adwinz
decides to shrink $W$ to $W_1$, or equivalently,
\begin{eqnarray*}
%\label{Edecomposeprobs0}
\Pr [\, |\hmuu - \hmuz| \ge  \epsc\, ] \le \delta/n.
\end{eqnarray*}
%
Since \adwintwo checks at most  $M \log (n/M)$ partitions $W_0 W_1$, the claim follows.

{\bf{Part 2)}}
The proof is similar to the proof of Part 2 of Theorem~\ref{ThBV}.
\end{proof}
